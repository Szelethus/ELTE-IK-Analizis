\documentclass[a4paper,11.5pt]{article}
\usepackage[textwidth=170mm, textheight=230mm, inner=20mm, top=20mm, bottom=30mm]{geometry}
\usepackage[normalem]{ulem}
\usepackage[utf8]{inputenc}
\usepackage[T1]{fontenc}
\PassOptionsToPackage{defaults=hu-min}{magyar.ldf}
\usepackage[magyar]{babel}
\usepackage{amsmath, amsthm,amssymb,paralist,array, ellipsis, graphicx,float}
%\usepackage{marvosym}

\makeatletter
\renewcommand*{\mathellipsis}{%
	\mathinner{%
		\kern\ellipsisbeforegap%
		{\ldotp}\kern\ellipsisgap%
		{\ldotp}\kern\ellipsisgap%
		{\ldotp}\kern\ellipsisaftergap%
	}%
}
\renewcommand*{\dotsb@}{%
	\mathinner{%
		\kern\ellipsisbeforegap%
		{\cdotp}\kern\ellipsisgap%
		{\cdotp}\kern\ellipsisgap%
		{\cdotp}\kern\ellipsisaftergap%
	}%
}
\renewcommand*{\@cdots}{%
	\mathinner{%
		\kern\ellipsisbeforegap%
		{\cdotp}\kern\ellipsisgap%
		{\cdotp}\kern\ellipsisgap%
		{\cdotp}\kern\ellipsisaftergap%
	}%
}
\renewcommand*{\ellipsis@default}{%
	\ellipsis@before
	\kern\ellipsisbeforegap
	.\kern\ellipsisgap
	.\kern\ellipsisgap
	.\kern\ellipsisgap
	\ellipsis@after\relax}
\renewcommand*{\ellipsis@centered}{%
	\ellipsis@before
	\kern\ellipsisbeforegap
	.\kern\ellipsisgap
	.\kern\ellipsisgap
	.\kern\ellipsisaftergap
	\ellipsis@after\relax}
\AtBeginDocument{%
	\DeclareRobustCommand*{\dots}{%
		\ifmmode\@xp\mdots@\else\@xp\textellipsis\fi}}
\def\ellipsisgap{.1em}
\def\ellipsisbeforegap{.05em}
\def\ellipsisaftergap{.05em}
\makeatother

\usepackage{hyperref}
\hypersetup{
	colorlinks = true	
}

\DeclareMathOperator{\Int}{int}
\DeclareMathOperator{\tg}{tg}
\DeclareMathOperator{\Th}{th}
\DeclareMathOperator{\sh}{sh}
\DeclareMathOperator{\ch}{ch}

\begin{document}
	%%%%%%%%%%%RÖVIDÍTÉSEK%%%%%%%%%%
	\setlength\parindent{0pt}
	\def\s{\hspace{0.2mm}\vphantom{\beta}}
	\def\Z{\mathbb{Z}}
	\def\Q{\mathbb{Q}}
	\def\R{\mathbb{R}}
	\def\C{\mathbb{C}}
	\def\N{\mathbb{N}}
	\def\Ra{\overline{\mathbb{R}}}
	
	\def\sume{\displaystyle\sum_{n=1}^{+\infty}}
	\def\sumn{\displaystyle\sum_{n=0}^{+\infty}}
	
	\def\narrow{\underset{n\rightarrow+\infty}{\longrightarrow}}
	\def\limn{\displaystyle\lim_{n\to +\infty}}
	\def\limx{\displaystyle\lim_{x\to +\infty}}
	
	
	\theoremstyle{definition}
	\newtheorem{theorem}{Tétel}[subsection] 
	
	\theoremstyle{definition}
	\newtheorem{definition}[theorem]{Definíció} 
	\newtheorem{example}[theorem]{Példa} 
	\newtheorem{task}[theorem]{Feladat} 
	\newtheorem{note}[theorem]{Megjegyzés}
	\newtheorem{revision}[theorem]{Emlékeztető}
	%%%%%%%%%%%%%%%%%%%%%%%%%%%%%%%%%%%%%%%%%%%%%%%%%%%%%%%%%%%%%%%%%%%%%
	\begin{center}
		{\LARGE\textbf{Analízis II.}}
		
		{\Large Első ZH tételkidolgozás}
		
		%6. óra.
	\end{center}
	A jegyzetet \textsc{Umann} Kristóf készítette Dr. \textsc{Szili} László  előadása alapján. (\today)
	\bigskip
	
	Tantárgyi honlap: \url{http://numanal.inf.elte.hu/~szili/Oktatas/An2_BSc_2016/index_An2_2016.htm}
	
	Általános tudnivalók: \url{http://numanal.inf.elte.hu/~szili/Oktatas/An2_BSc_2016/Zh1-tudni.pdf}
	
	Követelményrendszer: \url{http://numanal.inf.elte.hu/~szili/Oktatas/An2_BSc_2016/Kov_An2_2016.pdf}
	
	ZH témakörei: \url{http://numanal.inf.elte.hu/~szili/Oktatas/An2_BSc_2016/An2_1_zh_temakork_2016.pdf}
	\begin{enumerate}
		\item \textbf{Korlátos és zárt intervallumon folytonos függvény korlátos.}
		
		Tegyük fel, hogy 
		$\left.
		\begin{gathered}
		f:[a,b]\to\R \\
		\text{folytonos } [a,b]\text{-n}
		\end{gathered}
		\right\}\quad  \Rightarrow \quad f $ korlátos.
		\bigskip
		
		\textbf{Bizonyítás:} $f$ korlátos, ha
		\[ \exists K>0, \quad \forall x \in [a,b]: \quad |f(x)|\leq K \]
		Indirekt: Tegyük fel, hogy nem korlátos, azaz, 
		
		\begin{gather}
		\forall K>0, \quad \exists x \in [a,b]: \quad |f(x)|> K \notag\\
		\Rightarrow \forall n = 1,2\ldots \quad \exists x_n \in [a,b], \quad |f(x_n)|\geq n\label{reference} 
		\end{gather} 
		
		Tehát: $(x_n)\subset[a,b]$ korlátos sorozat $\overset{\text{B-W kiv.}}{\underset{\text{tétel}}{\Longrightarrow}} \exists (x_{n_k})$ konv. részsorozat. 
		
		Legyen
		\[ \alpha:=\lim(x_{n_k}) \]
		Ekkor: $\alpha\in[a,b]$.\quad
		
		(Indirekt: Tegyük fel, hogy $\alpha \notin [a,b] \quad \Rightarrow \quad \exists K (\alpha) \cap[a,b] = \emptyset$.
		
		$\alpha:=\lim(x_{n_k})\quad  \Rightarrow \quad \exists k_0\in\N,\quad \forall k\geqq k_0, \quad x_{n_k}\in K(\alpha).$ 
		Ez ellentmondás, ui. $x_{n_k}\in[a,b])$.
		\medskip
		
		Az $f$ folytonos $[a,b]$-n $\Rightarrow $
		\[ f\in C \{\alpha\}\quad \overset{\text{átviteli elv}}{\Rightarrow} \quad x_{n_k} \to \alpha \quad \Rightarrow\quad  f(x_{n_k}) \to f(\alpha)\quad  \Rightarrow \quad \left(f(x_{n_k})\right) \quad \text{korlátos, mert konv.}\]
		
		Ez ellentmond \ref{reference}-nek.\quad $\blacksquare$
		\item \textbf{A Weierstrass-tétel.}
		
		Tegyük fel, hogy:
		\[ \left.
		\begin{gathered} 
		f: [a,b]\to\R \\
		\text{folytonos } [a,b] 
		\end{gathered}
		\right\} \Rightarrow
		\begin{gathered}
		\text{$f$-nek $\exists$ absz. szélsőértéke, azaz\quad }\exists \alpha, \beta, \in [a,b]: \\
		f(x) \leq f(\alpha) \\
		f(\beta) \leq f(x)
		\end{gathered}\quad (x\in[a,b])\]
		
		\textbf{Bizonyítás:} $f$ folytonos $[a,b]$-n $\Rightarrow f$ korlátos.
		
		Ekkor:\[
		\begin{gathered}
		\exists \sup\{f(x)\ |\ x\in [a,b]\} =: M \in \R\\
		\exists \inf\{f(x)\ |\ x\in [a,b]\} =: m \in \R\\
		\end{gathered}\]
		Igazoljuk: $\exists \alpha \in [a,b]:\quad  f(\alpha) = M$.
		
		\[\begin{gathered}
		M \sup\quad  \Rightarrow \quad \forall n \in \N,\quad  \exists y_n\in\mathcal{R}_f:\quad  M - \frac{1}{n} < y_n \leq M 
		%\label{weierstrass-label}
		\end{gathered}\]
		Viszont: \[y_n\in\mathcal{R}_f \quad \Rightarrow\quad  \exists x_n\in [a,b]:\quad  f(x_n) = y_n,\quad (\forall n\in\N)\]
		
		Az $(x_n): \N\to[a,b]$ korlátos sorozat $\quad \overset{\text{B-W kiv.}}{\underset{\text{tétel}}{\Longrightarrow}} \quad \exists (x_{n_k})$ konvergens részsorozata.
		
		Legyen $\lim(x_{n_k}) =: \alpha \in [a,b]$ (indirekt belátható)
		
		\[f\text{ folyt. }[a,b]\text{-n }\quad \Rightarrow \quad f\in C\{\alpha\} \quad \overset{\text{átviteli elv}}{\Rightarrow}\quad 
		\lim_{k\to+\infty}(x_{n_k})=\alpha,\quad \lim_{k\to+\infty}\underbrace{f (x_{n_k})}_{y_{n_k}} = f(\alpha)\]
		\[ \lim_{k\to+\infty}(y_{n_k}) = f(\alpha) \quad \Rightarrow\quad M = f(\alpha) \]
		Hasonlóan bizonyítható az abszolút minimum létezése.\quad $\blacksquare$
		\item \textbf{A Bolzano-tétel.}
		
		Tegyük fel, hogy $f:[a,b]\to\R$, továbbá
		
		\[\left.\begin{gathered}
		\text{folytonos } [a,b]\text{-n}\\
		f(a)\cdot f(b)<0\\
		\end{gathered}\right\}\quad \Rightarrow\quad \exists\xi\in[a,b]:\quad f(\xi)=0. \]
		\textbf{Bizonyítás:} (Bolzano-féle felezési eljárás)
		
		Tegyük fel, hogy $f(a)<0,\quad  f(b)>0.$ \quad Legyen $[x_0, y_0]:=[a,b]$.
		
		\medskip
		Felezzük meg az intervallumot! Legyen $z_0:=\frac{a+b}{2}$. 3 eset lehetséges:
		\begin{enumerate}
			\item $f(z_0)=0 \checkmark$
			\item $f(z_0)>0$ esetén $[x_1,y_1]:=[a,z_0]$.
			\item $f(z_0)<0$ esetén $[x_1,y_1]:=[z_0,b].$
		\end{enumerate}
		Megfelezzük $[x_1,y_1]$-et. Itt is 3 eset lehetséges. (\ldots) Folytatjuk az eljárást.
		
		\medskip
		Az eljárás közben vagy találunk véges sok lépésben olyan  $\xi$-t melyre $f(\xi)=0$, vagy nem. Amennyiben nem,
		$\exists[x_n,y_n]\quad (n\in\N) \quad \text{intervallumsorozat, melyre teljesül hogy}$
		\begin{enumerate}
			\item $[x_{n+1}, y_{n+1}]\subset[x_n,y_n]\quad (\forall n\in\N)$
			\item $f(x_n)<0,\quad f(y_n)>0\quad (\forall n\in\N)$
			\item $y_n-x_n=\displaystyle \frac{b-a}{2^n}$
		\end{enumerate}
		Cantor-féle közösrész tételből következik hogy ezeknek az intervallumoknak van közös pontja, ha $n\in\N$, azaz:
		\[ \overset{\text{Cantor}}{\underset{\text{tétel}}{\Longrightarrow}}\quad \exists\xi\in\bigcap_{n\in\N}[x_n,y_n],\quad x_n\nearrow\xi, \quad y_n\searrow\xi. \quad (\text{monoton tartanak $\xi$-hez})\]
		$f$ folytonos $[a,b]$-n\quad $\Rightarrow$ \quad $f\in C\{\xi \} \quad \overset{\text{átviteli}}{\underset{\text{elv}}{\Longrightarrow}}\quad \lim(f(x_n))=f(\xi)=\lim(f(y_n))$
		Ha
		\begin{enumerate}
			\item $f(x_n)< 0\quad \Rightarrow\quad \lim(f(x_n))\leq0$
			\item $f(y_n)>0\quad \Rightarrow\quad \lim(f(y_n))\geq 0$
		\end{enumerate}
		Tehát:
		\[\underbrace{f(\xi)\leq 0 \quad \text{és}\quad f(\xi)\geq0}_{\substack{\big\Downarrow\\\displaystyle f(\xi)=0}}\]
		Ezzel a tételt bebizonyítottuk. \quad $\blacksquare$
		\item \textbf{Az inverz függvény folytonosságára vonatkozó tétel}
		
		
		Tegyük fel, hogy $f:[a,b]\to\R$,
		\[\left.\begin{gathered}
		\text{folytonos } [a,b]\text{-n}\\
		\exists f^{-1}
		\end{gathered}\right\}\quad \Rightarrow\quad \text{ az }f^{-1}\text{ függvény folytonos }\mathcal{D}_{f^{-1}}=\mathcal{R}_f\text{-en.}\]
		
		\textbf{Bizonyítás:} Indirekt, tegyük fel hogy $f^{-1}: \mathcal{R}_f\to[a,b]$ nem folytonos, azaz
		\[ \exists y_0\in\mathcal{R}_f:\quad f^{-1}\notin C\{y_0\}. \]
		$\text{Átviteli elv}\quad \Rightarrow\quad \exists(y_n)\subset\mathcal{R}_f\quad  \lim(y_n)=y_0,\quad \text{DE}\quad \limn f^{-1}(y_n)\not= f^{-1}(y_0).$
		
		Legyen \[x_n:=f^{-1}(y_n),\quad \text{ azaz}\quad  f(x_n)=y_n \quad \forall n\in\N,\]
		\[ x_0:=f^{-1}(y_0),\quad \text{azaz }\quad f(x_0)=y_0.\]
		
		Így: \begin{gather}
		\displaystyle \lim(x_n)\not=x_0.\label{second_lecture_reference}
		\end{gather} Ez azt jelenti, hogy:
		\[ \exists\delta>0:\quad \{ n\in\N\ :\ |x_n-x_0|\geq \delta \}\quad \text{végtelen halmaz.} \]
		
		Az $(x_n)\subset[a,b]$ korlátos sorozat$\quad \overset{\text{Bolz-Weier}}{\underset{\text{tétel}}{\Longrightarrow}}\quad \exists (x_{\nu_n})$ konvergens részsorozata.
		
		Legyen\quad  $\overline{x}:=\lim(x_{\nu_n})\in[a,b].$ (indirekt módon lehetett bizonyítani)
		
		\[\left.\begin{gathered}
		f\in C\{\overline{x} \}\\
		x_{\nu_n}\to\overline{x}
		\end{gathered}\right\}\quad \overset{\text{átviteli}}{\underset{\text{elv}}{\Longrightarrow}}\quad \underbrace{f(x_{\nu_n})}_{y_{\nu_n}}\longrightarrow f(\overline{x}) \quad (\text{emiatt: }\ref{second_lecture_reference}) \]
		Viszont:
		\[ y_n\to y_0, \quad y_{\nu_n}\to y_0(=f(x_0)) \]
		Ez pedig ellentmondás. \quad $\blacksquare$
		\item \textbf{A folytonosság és a derivált kapcsolata.}
		
		Tegyük fel, hogy $f\in\R\to\R, \quad a\in\Int\mathcal{D}_f$.
		\begin{enumerate}
			\item $f\in D\{a\}\quad \Rightarrow \quad f\in C\{a\}$.
			\item $f\in D\{a\}\quad \not\Leftarrow \quad f\in C\{a\}$.
		\end{enumerate}
		\textbf{Bizonyítás:}
		
		\fbox{$\Rightarrow$}
		\[ f\in D\{a\}\quad \Rightarrow\quad \lim_{x\to a}(f(x)-f(a))=\lim_{x\to a}\left(\frac{f(x)-f(a)}{x- a}\cdot(x-a)\right)=f'(a)\cdot 0=0\quad \blacksquare \]
		\fbox{$\not\Leftarrow$} abs$\notin D\{0\}:$
		\[\lim_{x\to0}\frac{|x|-|0|}{x}=\left\{
		\begin{gathered}
			1,\quad x>0\\
			-1,\quad x<0
		\end{gathered}\right.\quad  \Rightarrow\quad \nexists\lim_{x\to 0}\frac{|x|-|0|}{x}\quad \Rightarrow\quad \text{abs}\notin\text{D}\{0\}.\quad \blacksquare \]
		\item \textbf{ A deriválhatóság ekvivalens átfogalmazása lineáris közelítéssel.}
		
		Tegyük fel, hogy $f\in\R\to\R,\quad a\in\Int\mathcal{D}_f$
		\[ f\in D\{a\}\quad \Leftrightarrow\quad 
		\left\{\begin{gathered}
		\exists A\in\R\quad \text{és}\quad \exists \varepsilon:\quad \mathcal{D}_f\to\R,\quad \lim_a\varepsilon=0\\
		f(x)-f(a)=A(x-a)+\varepsilon(x)(x-a)\quad (x\in\mathcal{D}_f)
		\end{gathered}\right. \]
		$A=f'(a)$.
		\medskip
		
		\textit{Bizonyítás:}
		
		\fbox{$\Rightarrow$}
		\[ f\in D\{a\}\quad \Rightarrow\quad \lim_{x\to a}\frac{f(x)-f(a)}{x-a}=f'(a)\quad \Leftrightarrow\quad \lim_{x\to a}\underbrace{\left(\frac{f(x)-f(a)}{x-a}-f'(a)\right)}_{=:\varepsilon(x)}=0 \]
		Így: \quad $\lim_a\varepsilon=0$, és 
		\[f(x)-f(a)=f'(a)(x-a)+\varepsilon(x)(x-a)\quad (x\in\mathcal{D}_f)\checkmark  \]
		\fbox{$\Leftarrow$} Tegyük fel, hogy $\exists A\in\R,\quad \exists \varepsilon:\quad \mathcal{D}_f\to\R,\quad \lim_a\varepsilon=0:$
		\[ f(x)-f(a)=A(x-a)+\varepsilon(x)(x-a)\quad \overset{x\not=a}{\Rightarrow}\quad \underbrace{\frac{f(x)-f(a)}{x-a}}_{\overset{x\to a}{\longrightarrow}f'(a)}=\underbrace{A+\varepsilon(x)}_{\overset{x\to a}{\longrightarrow}A} \]
		\[ \Rightarrow f'(a)=A\quad \blacksquare \]
		\item \textbf{A szorzatfüggvény deriválása.}
		
		Tegyük fel, hogy $f,g:\R\to\R, \quad f,g\in D\{a\},\quad a\in \Int(\mathcal{D}_f\cap\mathcal{D}_g).$
		
		Ekkor:
		\[f,g\in D\{a\}\quad \text{és}\quad (f\cdot g)'(a)=f'(a)g(a)+f(a)g'(a)\]
		\textbf{Bizonyítás:} 
		\[ \frac{(fg)(x)-(fg)(a)}{x-a}=\frac{f(x)g(x)-f(a)g(a)}{x-a}\quad \overset{-f(a)\cdot g(x)}{\underset{+f(a)\cdot g(x)}{=}}\quad \underbrace{\frac{f(x)-f(a)}{x-a}}_{\overset{x\to a}{\longrightarrow}f'(a)}\cdot g(x)+f(a)\underbrace{\frac{g(x)-g(a)}{x-a}}_{\overset{x\to a}{\longrightarrow}g'(a)} \]
		$g(x)\quad \underset{x\to a}{\longrightarrow}\quad g(a)\quad \Rightarrow\quad $ mivel folytonos, és\quad  $x\to a$, Ui.: $g\in D\{a\}\quad \Rightarrow\quad  g\in C\{a\}\quad \blacksquare$
		
		\item \textbf{A hányadosfüggvény deriválása.}
		
		Tegyük fel, hogy $f,g:\R\to\R, \quad f,g\in D\{a\}, \quad a\in \Int(\mathcal{D}_f\cap\mathcal{D}_g),\quad g(a)\not=0.$
		
		Ekkor:
		\[ \frac{f}{g}\in D\{a\}\quad \text{és}\quad \left(\frac{f}{g}\right)'(a)=\frac{f'(a)\cdot g(a)-f(a)\cdot g'(a)}{g^2(a)} \]
		\textbf{Bizonyítás:} Közös ötlet: $\displaystyle \frac{f(x)-f(a)}{x-a}$ és $\displaystyle \frac{g(x)-g(a)}{x-a}$-t behozni.
		
		\begin{enumerate}
			\item Igazoljuk: \quad $a\in\Int\mathcal{D}_{\frac{f}{g}}$.
			
			Valóban: $g\in D\{a\}\quad \Rightarrow\quad g\in C\{a\},$ de $g(a)\not=0\quad \Rightarrow$
			\[ \exists K(a):\quad g(x)\not=0\quad (\forall x\in K(a)) \]
			$\Rightarrow a\in \Int \mathcal{D}_{\frac{f}{g}}$.
			\item 
			
			\[\frac{\left(\frac{f}{g}\right)(x)-\left(\frac{f}{g}\right)(a)}{x-a}=\frac{\left(\frac{f(x)}{g(x)}\right)-\left(\frac{f(a)}{g(a)}\right)}{x-a}=\frac{1}{g(a)\cdot g(x)}\cdot\frac{f(x)\cdot g(a)-f(a)\cdot g(x)}{x-a}\quad \overset{-f(a)g(a)}{\underset{+f(a)g(a)}{=}}\]\[\quad \frac{1}{g(a)g(x)}\cdot\left(\underbrace{\frac{f(x)-f(a)}{x-a}}_{\overset{x\to a}{\longrightarrow} f'(a)}\cdot g(a)-f(a)\cdot\underbrace{\frac{g(x)-g(a)}{x-a}}_{\overset{x\to a}{\longrightarrow}g'(a)}\right) \]
			$g(x)\overset{x\to a}{\longrightarrow }g(a)\not=0,$ mert $g\in C\{a\}.\quad \blacksquare$
			{\Large Lehetséges, hogy itt hiányzik egy kis rész. (előadáson ennyi hangzott el)}
		\end{enumerate}
		\item \textbf{A lokális szélsőértékre vonatkozó elsőrendű szükséges feltétel.}
		
		Tegyük fel, hogy $f\in\R\to\R,$ és 
		
		\[\left.\begin{gathered}
		f\in D\{a\}\quad a\in\Int D_f\\
		f\text{-nek } a\text{-ban lokális szélső értéke van}
		\end{gathered}\right\}\quad \Rightarrow\quad f'(a)=0\]
		
		\textbf{Bizonyítás:} Lokális maximumra:
		Tekintsük
		\[ \frac{f(x)-f(a)}{x-a} \]
		törtet. Ha $x>a$
		\[ \frac{\overbrace{f(x)-f(a)}^{\leq 0}}{\underbrace{x-a}_{>0}}\leq 0\quad \overset{f\in D\{a\}}{\Rightarrow}\quad \lim_{x\to a+0}\frac{f(x)-f(a)}{x-a}=f'_+(a)=f'(a)<0 \]
		Ha $x<a$
		\[ \frac{\overbrace{f(x)-f(a)}^{\leq 0}}{\underbrace{x-a}_{<0}}\geq 0 \]
		$f\in D\{a\}\quad \Rightarrow\quad f'(a)\geq0$
		Tehát: $f'(a)\leq0\quad $és\quad $f'(a)\geq0\quad  \Rightarrow\quad f'(a)=0.\quad \blacksquare$
		\item \textbf{A Rolle-féle középértéktétel.}
		
		\[\left.\begin{gathered}
		f:[a,b]\to\R\\
		f\in C[a,b]\\
		f\in D(a,b)\\
		f(a)=f(b)
		\end{gathered}\right\}\quad \Rightarrow\quad \begin{gathered}
		\xi\in (a,b)\\
		f'(\xi)=0
		\end{gathered}
		\]
		\textbf{Bizonyítás:} $f\in C[a,b]\quad \overset{\text{Weier.}}{\underset{\text{tétel}}{\Longrightarrow}}\quad$
		$ \exists \alpha , \beta\in[a,b]:\quad $
		\[ f(\alpha):=\min_{[a,b]}f:=m \]
		\[ f(\beta):= \max_{[a,b]}f:=M \]
		\begin{enumerate}
			\item eset: $f\equiv$ áll. $(m=M)\quad \Rightarrow\quad f'\equiv 0$
			\item eset: $f\not\equiv$ áll.$\quad \Rightarrow\quad m\not=M$ és $ m<M$
		\end{enumerate}
		Ha $m\not\equiv f(a)=f(b)\quad \Rightarrow\quad \alpha\in(a,b)$
		Ekkor $f(\alpha)$: abszolút minimum és $f(\alpha)$ lokális minimum is.
		\[ f'(\alpha)=0,\quad \xi=\alpha\quad \text{,,jó''} \]
		Ha $m=f(a)=f(b)\quad \blacksquare$
		{\Large Lehetséges, hogy itt hiányzik egy kis rész. (előadáson ennyi hangzott el)}
		\item \textbf{A Lagrange-féle középértéktétel.}
		
		
		\[ \left.\begin{gathered}
		f:[a,b]\to\R\\
		f\in C[a,b]\\
		f\in D(a,b)
		\end{gathered}\right\}\quad \Rightarrow\quad \begin{gathered}
		\exists \xi\in(a,b)\\
		f'(\xi)=\frac{f(b)-f(a)}{b-a}
		\end{gathered} \]
		\textbf{Bizonyítás:} A szelő egyenlete:
		\[ y=h_{a,b}(x)=\frac{f(b)-f(a)}{b-a}(x-a)+f(a) \]
		Legyen: \[ F(x):=f(x)-h_{a,b}(x)\quad (x\in[a,b]) \]
		$F$-re a Rolle feltételei teljesülnek (ellenőrizni kell!)
		\[ \Rightarrow \exists\xi\in(a,b):\quad F'(\xi)=f'(\xi)-\frac{f(b)-f(a)}{b-a}=0.\quad \blacksquare \]
	\end{enumerate}
\end{document}