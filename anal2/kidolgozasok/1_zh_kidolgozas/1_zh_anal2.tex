\documentclass[a4paper,11.5pt]{article}
\usepackage[textwidth=170mm, textheight=230mm, inner=20mm, top=20mm, bottom=30mm]{geometry}
\usepackage[normalem]{ulem}
\usepackage[utf8]{inputenc}
\usepackage[T1]{fontenc}
\PassOptionsToPackage{defaults=hu-min}{magyar.ldf}
\usepackage[magyar]{babel}
\usepackage{amsmath, amsthm,amssymb,paralist,array, ellipsis, graphicx}
%\usepackage{marvosym}

\makeatletter
\renewcommand*{\mathellipsis}{%
	\mathinner{%
		\kern\ellipsisbeforegap%
		{\ldotp}\kern\ellipsisgap%
		{\ldotp}\kern\ellipsisgap%
		{\ldotp}\kern\ellipsisaftergap%
	}%
}
\renewcommand*{\dotsb@}{%
	\mathinner{%
		\kern\ellipsisbeforegap%
		{\cdotp}\kern\ellipsisgap%
		{\cdotp}\kern\ellipsisgap%
		{\cdotp}\kern\ellipsisaftergap%
	}%
}
\renewcommand*{\@cdots}{%
	\mathinner{%
		\kern\ellipsisbeforegap%
		{\cdotp}\kern\ellipsisgap%
		{\cdotp}\kern\ellipsisgap%
		{\cdotp}\kern\ellipsisaftergap%
	}%
}
\renewcommand*{\ellipsis@default}{%
	\ellipsis@before
	\kern\ellipsisbeforegap
	.\kern\ellipsisgap
	.\kern\ellipsisgap
	.\kern\ellipsisgap
	\ellipsis@after\relax}
\renewcommand*{\ellipsis@centered}{%
	\ellipsis@before
	\kern\ellipsisbeforegap
	.\kern\ellipsisgap
	.\kern\ellipsisgap
	.\kern\ellipsisaftergap
	\ellipsis@after\relax}
\AtBeginDocument{%
	\DeclareRobustCommand*{\dots}{%
		\ifmmode\@xp\mdots@\else\@xp\textellipsis\fi}}
\def\ellipsisgap{.1em}
\def\ellipsisbeforegap{.05em}
\def\ellipsisaftergap{.05em}
\makeatother

\usepackage{hyperref}
\hypersetup{
	colorlinks = true	
}
\DeclareMathOperator{\Int}{int}
\DeclareMathOperator{\tg}{tg}
\DeclareMathOperator{\Th}{th}
\DeclareMathOperator{\sh}{sh}
\DeclareMathOperator{\ch}{ch}

\begin{document}
	%%%%%%%%%%%RÖVIDÍTÉSEK%%%%%%%%%%
	\setlength\parindent{0pt}
	\def\s{\hspace{0.2mm}\vphantom{\beta}}
	\def\Z{\mathbb{Z}}
	\def\Q{\mathbb{Q}}
	\def\R{\mathbb{R}}
	\def\C{\mathbb{C}}
	\def\N{\mathbb{N}}
	\def\Rn{\mathbb{R}^{n}}
	\def\Ra{\overline{\mathbb{R}}}
	\def\sume{\displaystyle\sum_{n=1}^{+\infty}}
	\def\sumn{\displaystyle\sum_{n=0}^{+\infty}}
	\def\biz{\emph{Bizonyítás:\ }}
	\def\narrow{\underset{n\rightarrow+\infty}{\longrightarrow}}
	\def\limn{\displaystyle\lim_{n\to +\infty}}
	\def\limx{\displaystyle\lim_{x\to +\infty}}
	
	\theoremstyle{definition}
	\newtheorem{theorem}{Tétel}[subsection] % reset theorem numbering for each chapter
	
	\theoremstyle{definition}
	\newtheorem{definition}[theorem]{Definíció} % definition numbers are dependent on theorem numbers
	\newtheorem{example}[theorem]{Példa} % same for example numbers
	\newtheorem{task}[theorem]{Feladat} % same for example numbers
	\newtheorem{note}[theorem]{Megjegyzés} % same for example numbers
	\newtheorem{revision}[theorem]{Emlékeztető} % same for example numbers
	%%%%%%%%%%%%%%%%%%%%%%%%%%%%%%%%%
	\begin{center}
		{\LARGE \textbf{Analízis II.}}
		
		{\large \textbf{Gyakorlati óra jegyzet}}
		
		5. óra
	\end{center}
	A jegyzetet \textsc{Umann} Kristóf készítette Dr. \textsc{Szili} László gyakorlatán. (\today)
	
	Tantárgyi honlap: \url{http://numanal.inf.elte.hu/~szili/Oktatas/An2_BSc_2016/index_An2_2016.htm}
	\begin{task}
		A definíció alapján határozza meg 
		\[ \lim_{x\to1}\frac{x^2+6x-7}{x^3-x^2+x-1} \]
		határértékét!
		\medskip
		
		\textit{Megoldás:} 
		\[ f(x):=\frac{x^2+6x-7}{x^3-x^2+x-1} \]
		\[ f(x)=\frac{x^2+6x-7}{x^3-x^2+x-1}=\frac{(x-1)(x+7)}{(x-1)(x^2+1)}=\frac{x+7}{x^2+1} \]
		\textit{sejtés:} $\displaystyle \lim_{x\to1}f(x)=\frac{8}{2}=4$
		\smallskip
		
		
		\textbf{Bizonyítás:} Igazoljuk:
		\[ \forall\varepsilon>0,\quad \delta>0,\quad |x-1|<\delta:\quad |f(x)-4|<\varepsilon. \]
		\[ \left|\frac{x+7}{x^2+1}-4\right| = \left|\frac{-4x^2+x+3}{x^2+1}\right|=\left|\frac{4x^2-x+4}{x^2+1}\right|=|x-1|\cdot\left|\frac{4x-3}{x^2+1}\right| \]	
	\end{task}
\end{document}