\documentclass[a4paper,11.5pt]{article}
\usepackage[textwidth=170mm, textheight=230mm, inner=20mm, top=20mm, bottom=30mm]{geometry}
\usepackage[normalem]{ulem}
\usepackage[utf8]{inputenc}
\usepackage[T1]{fontenc}
\PassOptionsToPackage{defaults=hu-min}{magyar.ldf}
\usepackage[magyar]{babel}
\usepackage{amsmath, amsthm,amssymb,paralist,array, ellipsis, graphicx}
%\usepackage{marvosym}

\makeatletter
\renewcommand*{\mathellipsis}{%
	\mathinner{%
		\kern\ellipsisbeforegap%
		{\ldotp}\kern\ellipsisgap%
		{\ldotp}\kern\ellipsisgap%
		{\ldotp}\kern\ellipsisaftergap%
	}%
}
\renewcommand*{\dotsb@}{%
	\mathinner{%
		\kern\ellipsisbeforegap%
		{\cdotp}\kern\ellipsisgap%
		{\cdotp}\kern\ellipsisgap%
		{\cdotp}\kern\ellipsisaftergap%
	}%
}
\renewcommand*{\@cdots}{%
	\mathinner{%
		\kern\ellipsisbeforegap%
		{\cdotp}\kern\ellipsisgap%
		{\cdotp}\kern\ellipsisgap%
		{\cdotp}\kern\ellipsisaftergap%
	}%
}
\renewcommand*{\ellipsis@default}{%
	\ellipsis@before
	\kern\ellipsisbeforegap
	.\kern\ellipsisgap
	.\kern\ellipsisgap
	.\kern\ellipsisgap
	\ellipsis@after\relax}
\renewcommand*{\ellipsis@centered}{%
	\ellipsis@before
	\kern\ellipsisbeforegap
	.\kern\ellipsisgap
	.\kern\ellipsisgap
	.\kern\ellipsisaftergap
	\ellipsis@after\relax}
\AtBeginDocument{%
	\DeclareRobustCommand*{\dots}{%
		\ifmmode\@xp\mdots@\else\@xp\textellipsis\fi}}
\def\ellipsisgap{.1em}
\def\ellipsisbeforegap{.05em}
\def\ellipsisaftergap{.05em}
\makeatother

\usepackage{hyperref}
\hypersetup{
	colorlinks = true	
}
\DeclareMathOperator{\Int}{int}
\DeclareMathOperator{\tg}{tg}
\DeclareMathOperator{\ctg}{ctg}
\DeclareMathOperator{\Th}{th}
\DeclareMathOperator{\sh}{sh}
\DeclareMathOperator{\ch}{ch}
\DeclareMathOperator{\arc}{arc}
\DeclareMathOperator{\arctg}{arc tg}
\DeclareMathOperator{\arcctg}{arc ctg}

\begin{document}
	%%%%%%%%%%%RÖVIDÍTÉSEK%%%%%%%%%%
	\setlength\parindent{0pt}
	\def\s{\hspace{0.2mm}\vphantom{\beta}}
	\def\Z{\mathbb{Z}}
	\def\Q{\mathbb{Q}}
	\def\R{\mathbb{R}}
	\def\C{\mathbb{C}}
	\def\N{\mathbb{N}}
	\def\Rn{\mathbb{R}^{n}}
	\def\Ra{\overline{\mathbb{R}}}
	\def\sume{\displaystyle\sum_{n=1}^{+\infty}}
	\def\sumn{\displaystyle\sum_{n=0}^{+\infty}}
	\def\biz{\emph{Bizonyítás:\ }}
	\def\narrow{\underset{n\rightarrow+\infty}{\longrightarrow}}
	\def\limn{\displaystyle\lim_{n\to +\infty}}
	\def\limx{\displaystyle\lim_{x\to +\infty}}
	
	\theoremstyle{definition}
	\newtheorem{theorem}{Tétel}[subsection] % reset theorem numbering for each chapter
	
	\theoremstyle{definition}
	\newtheorem{definition}[theorem]{Definíció} % definition numbers are dependent on theorem numbers
	\newtheorem{example}[theorem]{Példa} % same for example numbers
	\newtheorem{task}[theorem]{Feladat} % same for example numbers
	\newtheorem{note}[theorem]{Megjegyzés} % same for example numbers
	\newtheorem{revision}[theorem]{Emlékeztető} % same for example numbers
	%%%%%%%%%%%%%%%%%%%%%%%%%%%%%%%%%
	\begin{center}
		{\LARGE \textbf{Analízis II.}}
		
		{\large \textbf{Gyakorlati óra jegyzet}}
		
		9. óra
	\end{center}
	A jegyzetet \textsc{Umann} Kristóf készítette Dr. \textsc{Szili} László gyakorlatán. (\today)
	
	Tantárgyi honlap: \url{http://numanal.inf.elte.hu/~szili/Oktatas/An2_BSc_2016/index_An2_2016.htm}
	\section{Aszimptoták.}
	Van-e aszimptotája a következő függvényeknek ($\pm\infty$)-ben?
	\begin{task}
		\[ f(x):=x^4+x^3\quad (x\in\R) \]
		\textit{Megoldás:}
		
		Csak akkor létezik, ha a két határérték létezik és véges. Kezdjük $(+\infty)$-ben:
		\[ \lim_{x\to+\infty}\frac{f(x)}{x}=\lim_{x\to+\infty}(x^3+x^2)=+\infty \]
		Ennek a függvénynek nincs aszimptotája $(+\infty)$-ben.
		
		Tekintsük most ($-\infty)$ben:
		\[ \lim_{x\to-\infty}\frac{f(x)}{x}=\lim_{x\to-\infty}(x^3+x^2)=-\infty \]
		\[ \left(x^3+x^2=x^3\left(1+\frac{1}{x}\right)\right) \]
	\end{task}
	\begin{task}
		\[ f(x):=x-2\arc\tg x\quad (x\in\R) \]
		\textit{Megoldás:}
		
		\fbox{$(+\infty):$}
		\[ \lim_{x\to+\infty}\frac{f(x)}{x}=\lim_{x\to+\infty}\left(1-2\cdot\frac{\overbrace{\arc\tg x}^{\text{véges}}}{\underbrace{x}_{\text{végtelen}}}\right) \quad \overset{\lim_{x\to+\infty}\arc\tg x = \frac{\pi}{2}}{=}\quad  1 =: A\in\R \]
		Így létezik is a határérték, és ez véges.
		\[ \lim_{x\to+\infty}(f(x)-Ax)=\lim_{x\to+\infty}[(x-2\cdot\arc\tg x)-x]=\lim_{x\to+\infty}(-2\arc\tg x)=-\pi=B\in\R \]
		$\Rightarrow$ \quad Van aszimptota, és ez pedig a:
		\[ y=Ax+B=x-\pi \]
		egyenes.
		
		\fbox{$(-\infty):$}
		\[ \lim_{x\to-\infty}\frac{f(x)}{x}=\lim_{x\to-\infty}\left(1-2\cdot\frac{\arc\tg x}{x}\right) = 1 =:A\in\R \]
		Fontos megjegyezni az alábbi nevezetes határértéket:
		\fbox{$\displaystyle \lim_{x\to-\infty}\arc\tg x = -\frac{\pi}{2}$}
		\[ \lim_{x\to-\infty}(f(x)-Ax)=\lim_{x\to-\infty}(-2\arc\tg x)=\pi \]
		$\Rightarrow$\quad Létezik aszimptota, és ez a
		\[ y=x+\pi \]
		egyenes.
	\end{task}
	\section{Teljes függvény vizsgálat}
	\begin{task}
		\[ f(x):=x^4-4x^3+10\quad (x\in\R) \]
		\textit{Megoldás:}
		
		Kezdeti vizsgálat: $f\in D^2\checkmark$
		\begin{note}
			Középiskolában függvénytranszformációval dolgoztunk, ami egy negyedfokú függvényél aligha lehet célravezető.
		\end{note}
		\begin{enumerate}
			\item \textbf{Monotonitási intervallumok:}
			
			Deriváljunk, hogy meg tudjuk határozni a monotonitási intervallumokat!
			\[ f'(x)=4x^3-12x^2=4x^2(x-3)\quad (x\in\R) \]
			$$f'(x)\overset{?}{\overset{>}{\underset{=}{<}}}\quad \Leftrightarrow\quad 4x^2(x-3)\overset{>}{\underset{<}{=}}0\quad \Leftrightarrow\quad x-3\overset{>}{\underset{<}{=}}0 $$
			Így
			\[ f'(x)<0,\quad \text{ha}\quad x<3\quad \Rightarrow\quad f\downarrow (-\infty,3) \quad (*) \]
			\[ f'(x)>0,\quad \text{ha}\quad x>3\quad \Rightarrow\quad f\uparrow (3,+\infty)\quad (\#) \]
			\item \textbf{Lokális szélső értékek:}
			
			Elsőrendű szükséges feltételt alkalmazzuk!
			\[ f'(x)=4x^2(x-3)=0\quad \Rightarrow\quad x_1=0;\quad x_2=3. \]
			E két pontban lehet lokális szélső érték (ezek $f$ stacionárius pontjai).
			
			Most alkalmazzuk az elégséges feltételt (első- és másodrendű).
			
			\fbox{$x_1=0:$}
			\[ (*)\Rightarrow\quad x_1 \text{\quad nem lokális szélső érték hely.} \]
			\fbox{$x_2=3:$}
			\[ (\#) \text{\ és\ } (*)\Rightarrow\quad x_2 \text{\quad lokális minimum hely.} \]
			(alt. megoldási módszer) Elsőrendű elégséges feltétel: mivel a derivált függvény előjelet vált, ebből is következik hogy $x_2$ lokális minimum lesz.
			
			(alt. megoldási módszer) További megoldási módszer lehet a másodrendű elégséges feltétel, ahol $f''(3)$-at kéne vizsgálni.
			\item \textbf{Konvexitási intervallumok, inflexió:}
			
			\[ f''(x)=12x^2-24x=12x(x-2)\quad (x\in\R) \]
			Konvexitásra van szükséges és elégséges tételünk.
			\[ f''(x)\overset{?}{\overset{>}{\underset{<}{=}}}0\quad \Leftrightarrow\quad 12x(x-2)\overset{>}{\underset{<}{=}}0\quad \Leftrightarrow\quad x(x-2)\overset{>}{\underset{<}{=}}0 \]
			\[ f''(x)>0,\quad \text{ha}\quad x<0\quad \Rightarrow\quad f \text{\quad szigorúan konvex}\quad (-\infty,0)\text{-n} \]
			\[ f''(x)<0,\quad \text{ha}\quad 0<x<2\quad \Rightarrow\quad f \text{\quad szigorúan konkáv}\quad (0,2)\text{-n} \]
			Ebből tudjuk, hogy $x_1=0$ egy inflexiós hely.
			\[ f''(x)>0,\quad \text{ha}\quad x>2\quad \Rightarrow\quad f \text{\quad szigorúan konkáv}\quad (2,+\infty)\text{-n} \]
			Így $x_3:=2$ egy inflexiós pont.
			\item \textbf{Határértékek:}
			
			$(\pm\infty)$-ben kell vizsgálnunk őket. ($\mathcal{D}_f'\setminus\mathcal{D}_f=\{\pm\infty\}$)
			\[ \lim_{x\to\pm\infty}(x^4-4x^3+10)=+\infty \]
			Most ebben az esetben a $+$ és $-$ eseteket nem kellett külön kezelni. (Jegyezzük meg, hogy polinomoknál célszerű alkalmazni a nagyságrendi becslést)
			\item \textbf{Aszimptoták:}
			\[ \lim_{x\to+\infty}\frac{f(x)}{x}=\lim_{x\to\pm\infty}\left(x^3-4x^2+\frac{10}{x}\right)=\pm\infty \]
			$\Rightarrow\quad $ nincs aszimptota ($\pm\infty$)-ben.
			\item \textbf{Függvény képe:}
			
			Behelyettesítünk, és megállapítjuk hogy
			\[ f(0)=10;\quad f(2)=-6;\quad f(3)=-17 \]
			Ábrázolása házi feladat, mert \LaTeX ben picit nehéz :D
			
			Rajzhoz megjegyzések:
			\begin{compactenum}
				\item 0-ig konvex (szig mon csökken)
				\item 0 és 2 között szigorúan konkáv (továbbra is szig mon csökken)
				\item 2 után a függvény konvex lesz (innentól szig mon nő)
			\end{compactenum}
		\end{enumerate}
	\end{task}
	\begin{note}
		Zérushelyek csak akkor írandóak le, ha nagyon nyilvánvalóak.
	\end{note}
	\begin{note}
		Lehet táblázatot készíteni az eredményeink összefoglalására.
	\end{note}
	\begin{task}
		\[ f(x):=\frac{x^3+x}{x^2-1}\quad (x\in\R\setminus\{-1,1\}) \]
		\textit{Megoldás:}
		
		Kezdeti vizsgálatok: $f\in D^2\checkmark$
		\begin{enumerate}
			\item \textbf{Paritás:}
			
			$f$ páratlan, mert:
			\[ f(-x)=-f(x)\quad \Leftrightarrow\quad \frac{(-x)^3+(-x)}{(-x^2)-1}=-\frac{x^3+x}{x^2-1} \]
			$\Rightarrow\quad $ Jegyezzük meg, páratlanságából adódik hogy origóra szimmetrikus\quad $\Rightarrow$\quad elég a $(0,+\infty)$ intervallumon nézni. (Páros függvény az $y$ tengelyre tükrös)
			\item \textbf{Monotonitási intervallumok:}
			
			Elegendő csupán $x>0$-n tekinteni. Azonban figyeljünk arra, hog a függvény 1-ben nincs értelmezve!
			
			\textit{Kövektező órán lesz befejezve.}
		\end{enumerate}
	\end{task}
	
\end{document}