\documentclass[a4paper,11.5pt]{article}
\usepackage[textwidth=170mm, textheight=230mm, inner=20mm, top=20mm, bottom=30mm]{geometry}
\usepackage[normalem]{ulem}
\usepackage[utf8]{inputenc}
\usepackage[T1]{fontenc}
\PassOptionsToPackage{defaults=hu-min}{magyar.ldf}
\usepackage[magyar]{babel}
\usepackage{amsmath, amsthm,amssymb,paralist,array, ellipsis, graphicx}
%\usepackage{marvosym}

\makeatletter
\renewcommand*{\mathellipsis}{%
	\mathinner{%
		\kern\ellipsisbeforegap%
		{\ldotp}\kern\ellipsisgap%
		{\ldotp}\kern\ellipsisgap%
		{\ldotp}\kern\ellipsisaftergap%
	}%
}
\renewcommand*{\dotsb@}{%
	\mathinner{%
		\kern\ellipsisbeforegap%
		{\cdotp}\kern\ellipsisgap%
		{\cdotp}\kern\ellipsisgap%
		{\cdotp}\kern\ellipsisaftergap%
	}%
}
\renewcommand*{\@cdots}{%
	\mathinner{%
		\kern\ellipsisbeforegap%
		{\cdotp}\kern\ellipsisgap%
		{\cdotp}\kern\ellipsisgap%
		{\cdotp}\kern\ellipsisaftergap%
	}%
}
\renewcommand*{\ellipsis@default}{%
	\ellipsis@before
	\kern\ellipsisbeforegap
	.\kern\ellipsisgap
	.\kern\ellipsisgap
	.\kern\ellipsisgap
	\ellipsis@after\relax}
\renewcommand*{\ellipsis@centered}{%
	\ellipsis@before
	\kern\ellipsisbeforegap
	.\kern\ellipsisgap
	.\kern\ellipsisgap
	.\kern\ellipsisaftergap
	\ellipsis@after\relax}
\AtBeginDocument{%
	\DeclareRobustCommand*{\dots}{%
		\ifmmode\@xp\mdots@\else\@xp\textellipsis\fi}}
\def\ellipsisgap{.1em}
\def\ellipsisbeforegap{.05em}
\def\ellipsisaftergap{.05em}
\makeatother

\usepackage{hyperref}
\hypersetup{
	colorlinks = true	
}
\DeclareMathOperator{\Int}{int}
\DeclareMathOperator{\tg}{tg}
\DeclareMathOperator{\ctg}{ctg}
\DeclareMathOperator{\Th}{th}
\DeclareMathOperator{\sh}{sh}
\DeclareMathOperator{\ch}{ch}
\DeclareMathOperator{\arc}{arc}
\DeclareMathOperator{\arctg}{arc tg}
\DeclareMathOperator{\arcctg}{arc ctg}

\begin{document}
	%%%%%%%%%%%RÖVIDÍTÉSEK%%%%%%%%%%
	\setlength\parindent{0pt}
	\def\s{\hspace{0.2mm}\vphantom{\beta}}
	\def\Z{\mathbb{Z}}
	\def\Q{\mathbb{Q}}
	\def\R{\mathbb{R}}
	\def\C{\mathbb{C}}
	\def\N{\mathbb{N}}
	\def\Rn{\mathbb{R}^{n}}
	\def\Ra{\overline{\mathbb{R}}}
	\def\sume{\displaystyle\sum_{n=1}^{+\infty}}
	\def\sumn{\displaystyle\sum_{n=0}^{+\infty}}
	\def\biz{\emph{Bizonyítás:\ }}
	\def\narrow{\underset{n\rightarrow+\infty}{\longrightarrow}}
	\def\limn{\displaystyle\lim_{n\to +\infty}}
	\def\limx{\displaystyle\lim_{x\to +\infty}}
	
	\theoremstyle{definition}
	\newtheorem{theorem}{Tétel}[subsection] % reset theorem numbering for each chapter
	
	\theoremstyle{definition}
	\newtheorem{definition}[theorem]{Definíció} % definition numbers are dependent on theorem numbers
	\newtheorem{example}[theorem]{Példa} % same for example numbers
	\newtheorem{task}[theorem]{Feladat} % same for example numbers
	\newtheorem{note}[theorem]{Megjegyzés} % same for example numbers
	\newtheorem{revision}[theorem]{Emlékeztető} % same for example numbers
	%%%%%%%%%%%%%%%%%%%%%%%%%%%%%%%%%
	\begin{center}
		{\LARGE \textbf{Analízis II.}}
		
		{\large \textbf{Gyakorlati óra jegyzet}}
		
		8. óra
	\end{center}
	A jegyzetet \textsc{Umann} Kristóf készítette Dr. \textsc{Szili} László gyakorlatán. (\today)
	
	Tantárgyi honlap: \url{http://numanal.inf.elte.hu/~szili/Oktatas/An2_BSc_2016/index_An2_2016.htm}
	\section{Elemi függvények.}
	\subsection{Trigonometrikus függvények.}
	\begin{note}
		Érdemes a már ismert ($\sin, \cos, \tg$) függvények nevezetesebb értékeit tudni.
	\end{note}
	\begin{task}
		\begin{enumerate}
			\item $\arc\sin\frac{1}{2}=\alpha\in\left[-\frac{\pi}{2},\frac{\pi}{2}\right]:\quad \sin \alpha=\frac{1}{2} \quad \Rightarrow\quad \alpha = \frac{\pi}{6}.$
			\item $\arc\cos\left(-\frac{\sqrt{2}}{2}\right)=\alpha\in[0,\pi]:\quad \cos\alpha=-\frac{\sqrt{2}}{2}\quad \Rightarrow\quad \alpha=\frac{3}{4}\pi$
			\item $\arc\tg1=\alpha\in\left(-\frac{\pi}{2},\frac{\pi}{2}\right):\quad \tg\alpha=1\quad \Rightarrow\quad \alpha = \frac{\pi}{4}$
			\item $\arc\ctg\sqrt{3}=\alpha\in(0,\pi):\quad \ctg\alpha=\sqrt{3}\quad \Rightarrow\quad \alpha=\frac{\pi}{6}$
		\end{enumerate}
	\end{task}
	\begin{task} Bizonyítsuk be:
		\[ \arc\sin x+\arc\cos x=\frac{\pi}{2}\quad (x\in[-1,1]) \]
		\textit{Bizonyítás:}
		\[ \arc\sin x=y_1\in\left[-\frac{\pi}{2},\frac{\pi}{2}\right]:\quad \sin y_1=x \]
		\[ \arc\cos x=y_2\in\left[1,\pi\right]:\quad \cos y_2=x \]
		$x=\cos y_2=$\fbox{$\sin y_1 = \cos\left(\frac{\pi}{2}-y_1\right)$}$\checkmark$
		
		Mivel $y_2\in[0,\pi]$, megállapítható hogy $\frac{\pi}{2}-y_1\in[0,\pi]$, ugyanis
		\[y_1\in\left[-\frac{\pi}{2},\frac{\pi}{2}\right]\quad \Leftrightarrow\quad 0\leq \frac{\pi}{2}-y\leq\pi\quad \Leftrightarrow\quad y_1\leq \frac{\pi}{2}, \quad y_1\geq-\frac{\pi}{2} \]
		\[ \Rightarrow y_2=\frac{\pi}{2}-y_1\quad \Rightarrow\quad \arc\cos x=\frac{\pi}{2}-\arc\sin x \]
	\end{task}
	\begin{task}
		\[ \arc\tg x +\arc \ctg x = \frac{\pi}{2}\quad (x\in\R) \]
		\textit{Bizonyítás:}
		
		\[\arc\tg x = y_1\in\left(-\frac{\pi}{2},\frac{\pi}{2}\right)\quad  \Leftrightarrow\quad \tg y_1=x \]
		\[\arc\ctg x = y_2\in\left(0,\pi\right)\quad  \Leftrightarrow\quad \ctg y_2=x \]
		\[ x=\ctg y_2=\tg y_1= \]
		Közben állapítsuk, meg, hogy $\ctg y_2=\th \left(-\frac{\pi}{2}-y_2\right)=\tg y_1\quad (y_1\in\left(-\frac{\pi}{2},\frac{\pi}{2}\right)$
		
		Ha
		\[ y_2\in(0,\pi)\quad \Rightarrow\quad \frac{\pi}{2}-y_2\in\left(-\frac{\pi}{2},\frac{\pi}{2}\right)\quad \Rightarrow\quad \frac{\pi}{2}-y_2=y_1\quad \blacksquare \]
	\end{task}
	\begin{task}
		\[ \arc\sin(\sin x)=\left\{\begin{gathered}
			x,\quad x\in\left[-\frac{\pi}{2},\frac{\pi}{2}\right]\\
			\pi-x,\quad x\in\left[\frac{\pi}{2},\frac{3}{2}\pi \right]
		\end{gathered}\right. \]
		$2\pi$ szerint periodikus lesz.$\checkmark\quad \Rightarrow\quad \left[-\frac{\pi}{2},\frac{3}{2}\pi\right]$
			
		\textit{Megoldás:}
		
		Ha \[x\in\left[-\frac{\pi}{2},\frac{\pi}{2}\right]:\quad \arc\sin(\sin x)=\alpha\in\left[-\frac{\pi}{2},\frac{\pi}{2}\right]:\quad \sin\alpha=\sin x\quad \Rightarrow\quad \alpha = x\]
		Ha \[x\in\left[\frac{\pi}{2},\frac{3}{2}\pi\right]:\quad \arc\sin(\sin x)=y\in\left[-\frac{\pi}{2},\frac{\pi}{2}\right]:\quad \sin y=\sin x= \sin(\pi-x) \]
		Ekkor, ha $x\in\left[\frac{\pi}{2},\frac{3}{2}\pi\right]\quad \Rightarrow\quad \pi-x\in\left[-\frac{\pi}{2},\frac{\pi}{2}\right]$.
		\[ \Rightarrow y=\pi-x. \]
	\end{task}
	\section{Konvexitás.}
	\begin{task}
		$f:=\exp$.
		\[ f''(x)=e^x>0\quad (\forall x\in\R)\quad \Rightarrow\quad f\quad \text{szig konvex $\R$-en.} \]
	\end{task}
	\begin{task}
		$f:=\ln$.
		\[ f'(x)=\frac{1}{x};\quad f''(x)=-\frac{1}{x^2}<0\quad (x>0)\quad \Rightarrow\quad f\quad \text{szig. konkáv $(0,+\infty)$-n.} \]
	\end{task}
	\begin{task} Hatványfüggvények: $h_\alpha(x)=x^\alpha\quad (x>0, \alpha\in\R).$
		\[ h'_\alpha(x)=\alpha\cdot x^{\alpha-1}\quad \Rightarrow\quad \text{monoton.} \]
		\[ h''_\alpha(x)=\alpha(\alpha-1)\cdot x^{\alpha-2}\]
		$h''_\alpha(x)>1,$ ha $\alpha>0\quad \Rightarrow\quad h_\alpha$ szig konvex $(0,+\infty)$-n.
		
		$h''_\alpha(x)>0,$ ha $\alpha<0\quad \Rightarrow\quad h_\alpha$ szig konvex $(0,+\infty)$-n.
		
		$h''_\alpha(x)<0,$ ha $0<\alpha<1\quad \Rightarrow\quad h_\alpha$ szig konkáv $(0,+\infty)$-n.
	\end{task}
	\subsection{Egyenlőtlenségek igazolása.}
	\begin{task}
		\[ \left(\frac{x+y}{2}\right)^n<\frac{x^n+y^n}{2},\quad 1<n\in\N;\quad x,y>0,\quad x\not=y. \]
		\textit{Megoldás:}
		
		\[ f(x):=x^n\quad (x>0);\quad n=1,2,\ldots \]
		Ez a fv szigorúan konvex $(0,+\infty)$-en. CSAK AKKOR:
		\[ \forall x,y\in(0,+\infty),\quad \forall\lambda\in(0,1)\quad f(\lambda x+(1-\lambda)y)<\lambda f(x)+(1-\lambda)f(y) \]
		\[ \lambda= \frac{1}{2}:\quad f\left(\frac{x+y}{2}\right)\leq\frac{f(x)+f(y)}{2}=\frac{x^n+y^n}{2} \]
	\end{task}
	\begin{task}
		\[ e^{\frac{x+y}{2}}<\frac{e^x+e^y}{2}\quad (x,y\in\R;\quad x\not=y) \]
		
		\textit{Megoldás:}
		\[ f(x):=e^x\quad (x\in\R) \]
		\[ f''(x)=e^x>0\quad (x\in\R)\quad \Rightarrow\quad f\quad \text{szig konvex $\R$-en}\quad \Leftrightarrow\quad \forall x,y\in\R,\quad x<y;\quad \forall \lambda\in(0,1) \]
		\[ f(\lambda x+(1-\lambda)y)<\lambda f(x)+(1-\lambda)f(y) \]
		$\lambda=\frac{1}{2}\quad \blacksquare$
	\end{task}
	\begin{task}
		\[ f(x):=2x^3-21x^2+36x\quad (x\in\R) \]
		Hol konvex?
		
		\textit{Megoldás:} Ha 
		\[ f''(x)=(6x^2-42x+36)'=12x-42>0\quad \Leftrightarrow\quad x>\frac{42}{12}=\frac{7}{2}\quad \Rightarrow\quad f \text{\quad szig konvex $\left(\frac{7}{2},+\infty\right)$-en} \]
		
		Ha 
		\[ f''(x)=(6x^2-42x+36)'=12x-42>0\quad \Leftrightarrow\quad x<\frac{42}{12}=\frac{7}{2}\quad \Rightarrow\quad f \text{\quad szig konkáv $\left(-\infty,\frac{7}{2}\right)$-en} \]
	\end{task}
	\begin{task}
		\[ f(x)=x+\sin x\quad (x\in\R) \]
		Hol konvex?
		
		\textit{Megoldás:}
		
		\[ f''(x)=(1+\cos x)'=-\sin x\quad (x\in\R) \]
		\[ f''(x)\quad \overset{>}{\underset{<}{=}}\quad 0\quad \Leftrightarrow\quad -\sin x \quad \overset{>}{\underset{<}{=}}\quad 0  \]
		\[ \sin x >0\quad \Leftrightarrow\quad 0+2k\pi<x<\pi+2k\pi\quad (k\in\Z) \]
	\end{task}
\end{document}