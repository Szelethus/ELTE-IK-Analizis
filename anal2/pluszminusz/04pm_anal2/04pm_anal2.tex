\documentclass[a4paper,11.5pt]{article}
\usepackage[textwidth=170mm, textheight=230mm, inner=20mm, top=20mm, bottom=30mm]{geometry}
\usepackage[normalem]{ulem}
\usepackage[utf8]{inputenc}
\usepackage[T1]{fontenc}
\PassOptionsToPackage{defaults=hu-min}{magyar.ldf}
\usepackage[magyar]{babel}
\usepackage{amsmath, amsthm,amssymb,paralist,array, ellipsis, graphicx}
%\usepackage{marvosym}

\makeatletter
\renewcommand*{\mathellipsis}{%
	\mathinner{%
		\kern\ellipsisbeforegap%
		{\ldotp}\kern\ellipsisgap%
		{\ldotp}\kern\ellipsisgap%
		{\ldotp}\kern\ellipsisaftergap%
	}%
}
\renewcommand*{\dotsb@}{%
	\mathinner{%
		\kern\ellipsisbeforegap%
		{\cdotp}\kern\ellipsisgap%
		{\cdotp}\kern\ellipsisgap%
		{\cdotp}\kern\ellipsisaftergap%
	}%
}
\renewcommand*{\@cdots}{%
	\mathinner{%
		\kern\ellipsisbeforegap%
		{\cdotp}\kern\ellipsisgap%
		{\cdotp}\kern\ellipsisgap%
		{\cdotp}\kern\ellipsisaftergap%
	}%
}
\renewcommand*{\ellipsis@default}{%
	\ellipsis@before
	\kern\ellipsisbeforegap
	.\kern\ellipsisgap
	.\kern\ellipsisgap
	.\kern\ellipsisgap
	\ellipsis@after\relax}
\renewcommand*{\ellipsis@centered}{%
	\ellipsis@before
	\kern\ellipsisbeforegap
	.\kern\ellipsisgap
	.\kern\ellipsisgap
	.\kern\ellipsisaftergap
	\ellipsis@after\relax}
\AtBeginDocument{%
	\DeclareRobustCommand*{\dots}{%
		\ifmmode\@xp\mdots@\else\@xp\textellipsis\fi}}
\def\ellipsisgap{.1em}
\def\ellipsisbeforegap{.05em}
\def\ellipsisaftergap{.05em}
\makeatother

\usepackage{hyperref}

\usepackage{hyperref}
\hypersetup{
	colorlinks = true	
}

\DeclareMathOperator{\Int}{int}
\DeclareMathOperator{\tg}{tg}
\DeclareMathOperator{\Th}{th}
\DeclareMathOperator{\sh}{sh}
\DeclareMathOperator{\ch}{ch}

\begin{document}
	%%%%%%%%%%%RÖVIDÍTÉSEK%%%%%%%%%%
	\setlength\parindent{0pt}
	\def\s{\hspace{0.2mm}\vphantom{\beta}}
	\def\Z{\mathbb{Z}}
	\def\Q{\mathbb{Q}}
	\def\R{\mathbb{R}}
	\def\C{\mathbb{C}}
	\def\N{\mathbb{N}}
	\def\Ra{\overline{\mathbb{R}}}
	
	\def\sume{\displaystyle\sum_{n=1}^{+\infty}}
	\def\sumn{\displaystyle\sum_{n=0}^{+\infty}}
	
	\def\narrow{\underset{n\rightarrow+\infty}{\longrightarrow}}
	\def\limn{\displaystyle\lim_{n\to +\infty}}
	\def\limx{\displaystyle\lim_{x\to +\infty}}
	
	\theoremstyle{definition}
	\newtheorem{theorem}{Tétel}[subsection] 
	
	\theoremstyle{definition}
	\newtheorem{definition}[theorem]{Definíció} 
	\newtheorem{example}[theorem]{Példa} 
	\newtheorem{task}[theorem]{Feladat} 
	\newtheorem{note}[theorem]{Megjegyzés}
	%%%%%%%%%%%%%%%%%%%%%%%%%%%%%%%%%%%%%%%%%%%%%%%%%%%%%%%%%%%%%%%%%%%%%
	\begin{center}
		{\LARGE \textbf{Analízis II.}}
		
		{\large \textbf{+/$-$ kidolgozás}}
		
		5-6. óra
	\end{center}
	A jegyzet \textsc{Umann} Kristóf kidolgozásaiból készült, Dr. \textsc{Szili} László előadása alapján. (\today)
	
	Gyakorlathoz pdf: \url{http://numanal.inf.elte.hu/~szili/Oktatas/An2_BSc_2016/An2_gyak_2016_osz.pdf}
	\begin{enumerate}
		\item \textbf{Mi a \textit{belső pont} definíciója?}
		
		\textbf{Válasz:} 
		$0\not=A\subset \R$ halmaz \textbf{belső pontja} $a\in A$, ha
		\[  \exists K(a):\quad K(a)\subset A. \]
		Jele:\quad $\Int A:=\{ a\in A\ | \ a $ belső pontja $A$-nak \}
		\item \textbf{Mikor mondja azt, hogy egy $f:\R\to\R$ függvény \textit{differenciálható} valamely pontban?}	
		
		\textbf{Válasz:}
		$f\in\R\to\R, \quad a\in\Int\mathcal{D}_f$.\quad  $f$ \textbf{differenciálható}, vagy \textbf{deriválható} az $a$ pontban, ha
		\[ \exists\text{\quad és véges\quad }\lim_{h\to0}\frac{f(a+h)-f(a)}{h}=:f'(a)\quad \text{határérték.} \]
		$f'(a)$: $f$ \textbf{deriváltja}, vagy \textbf{differenciálhányadosa} $a$ pontban.
		Jelöljük így is: $f\in D\{a\}$.
		\item \textbf{Mi a kapcsolat a pontbeli differenciálhatóság és a folytonosság között?}
		
		\textbf{Válasz:} Tegyük fel, hogy $f\in\R\to\R, \quad a\in\Int\mathcal{D}_f$.
		\begin{enumerate}
			\item $f\in D\{a\}\quad \Rightarrow \quad f\in C\{a\}$.
			\item $f\in C\{a\}\quad \not\Rightarrow \quad f\in D\{a\}$.
		\end{enumerate}
		
		\item\textbf{Mi a \textit{jobb oldali derivált} definíciója?}
		
		\textbf{Válasz:} $f\in\R\to\R,\quad a\in\mathcal{D}_f$, és tegyük fel, hogy $\exists\delta>0:\quad [a,a+\delta)\subset\mathcal{D}_f.$ Ha létezik és véges a $\displaystyle \lim_{x\to a+0}\frac{f(x)-f(a)}{x-a}$ hátérérték, akkor az $f$ függvény \textbf{jobbról deriválható} az $a$-ban.
		\[\displaystyle \lim_{x\to a+0}\frac{f(x)-f(a)}{x-a} =:f'_+(a) \quad \text{az $f$ jobb oldali deriváltja az $a$-ban.} \]
		
		\item\textbf{Milyen ekvivalens átfogalmazást ismer a pontbeli deriválhatóságra a lineáris közelítéssel?}
		
		\textbf{Válasz:}
		Tegyük fel, hogy $f\in\R\to\R,\quad a\in\Int\mathcal{D}_f$
		\[ f\in D\{a\}\quad \Leftrightarrow\quad 
		\left\{\begin{gathered}
		\exists A\in\R\quad \text{és}\quad \exists \varepsilon:\quad \mathcal{D}_f\to\R,\quad \lim_a\varepsilon=0\\
		f(x)-f(a)=A(x-a)+\varepsilon(x)(x-a)\quad (x\in\mathcal{D}_f)
		\end{gathered}\right. \]
		$A=f'(a)$.
		
		\item\textbf{Mi az \textit{érintő} definíciója?}
		
		\textbf{Válasz:}
		$f:\R\to\R$ függvény grafikonjának van érintője, az $(a,f(a))$ pontban, ha $f\in D\{a\}$.
		
		\smallskip
		A grafikon $(a,f(a))$-béli \textbf{érintője} az
		\[ y=f'(a)(x-a)+f(a) \]
		egyenletű egyenes.
		
		\item \textbf{Milyen tételt ismer két függvény szorzatának valamely pontbeli differenciálhatóságáról és a deriváltjáról?}
		
		\textbf{Válasz:} 
		Tegyük fel, hogy $f,g:\R\to\R, \quad f,g\in D\{a\},\quad a\in \Int(\mathcal{D}_f\cap\mathcal{D}_g).$
		 \[f\cdot g\in D\{a\}\quad \text{ és} \quad (f\cdot g)'(a)=f'(a)\cdot g(a)+f(a)\cdot g'(a)\]
		
		\item\textbf{Milyen tételt ismer két függvény hányadosának valamely pontbeli differenciálhatóságáról és a deriváltjáról?}
		
		\textbf{Válasz:}
		Tegyük fel, hogy $f,g:\R\to\R, \quad f,g\in D\{a\},\quad a\in \Int(\mathcal{D}_f\cap\mathcal{D}_g),\quad g(a)\not=0\Rightarrow$
		\[ \frac{f}{g}\in D\{a\}\quad \text{és}\quad \left(\frac{f}{g}\right)'(a)=\frac{f'(a)\cdot g(a)-f(a)\cdot g'(a)}{g^2(a)}. \]
		\item\textbf{Milyen tételt ismer két függvény kompozíciójának valamely pontbeli differenciálhatóságáról és a deriváltjáról?}
		
		\textbf{Válasz:}
		Tegyük fel, hogy $f,g\in\R\to\R,\quad a\in\Int\mathcal{D}_g$ és
		
		\[\left.\begin{gathered}
		\mathcal{R}_g\subset\mathcal{D}_f\\
		g\in D\{a\}\\
		f\in D\{g(a) \}
		\end{gathered}\right\}\quad \Rightarrow\quad \begin{gathered}
		f\circ g\in D\{a\},\\
		(f\circ g)'(a)=f'(g(a))\cdot g'(a)
		\end{gathered}\]
		\item\textbf{Írja fel az $\exp_a\quad (a\in\R, a>0)$ függvény deriváltját valamely helyen.}
		
		\textbf{Válasz:}
		Az $\exp_a$ függvény: \quad $(a^x=e^{x\cdot\ln a})$
		\quad $\forall x\in\R,\quad \exp_a\in D\{x\}$
		\begin{center}
			\fbox{$(a^x)'=a^x\ln a \quad (x\in\R)$}
		\end{center}
		\item\textbf{Írja fel az $\log_a\quad (a\in\R, 0<a\not=1)$ függvény deriváltját valamely helyen.}
		
		\textbf{Válasz:}
		$\log_a$ függvény, $0<a$ és $a\not=1, \quad  \forall x\in(0,+\infty),\quad \log_a\in D\{x\} $
		\begin{center}
			\fbox{$\displaystyle \log_a'x=\frac{1}{x\cdot\ln a}\quad (x\in(0,+\infty))$}
		\end{center}
	\end{enumerate}
\end{document}