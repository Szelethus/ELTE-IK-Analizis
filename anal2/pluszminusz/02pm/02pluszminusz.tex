\documentclass[a4paper,11.5pt]{article}
\usepackage[textwidth=170mm, textheight=230mm, inner=20mm, top=20mm, bottom=30mm]{geometry}
\usepackage[normalem]{ulem}
\usepackage[utf8]{inputenc}
\usepackage[T1]{fontenc}
\PassOptionsToPackage{defaults=hu-min}{magyar.ldf}
\usepackage[magyar]{babel}
\usepackage{amsmath, amsthm,amssymb,paralist,array, ellipsis, graphicx}
%\usepackage{marvosym}

\makeatletter
\renewcommand*{\mathellipsis}{%
	\mathinner{%
		\kern\ellipsisbeforegap%
		{\ldotp}\kern\ellipsisgap%
		{\ldotp}\kern\ellipsisgap%
		{\ldotp}\kern\ellipsisaftergap%
	}%
}
\renewcommand*{\dotsb@}{%
	\mathinner{%
		\kern\ellipsisbeforegap%
		{\cdotp}\kern\ellipsisgap%
		{\cdotp}\kern\ellipsisgap%
		{\cdotp}\kern\ellipsisaftergap%
	}%
}
\renewcommand*{\@cdots}{%
	\mathinner{%
		\kern\ellipsisbeforegap%
		{\cdotp}\kern\ellipsisgap%
		{\cdotp}\kern\ellipsisgap%
		{\cdotp}\kern\ellipsisaftergap%
	}%
}
\renewcommand*{\ellipsis@default}{%
	\ellipsis@before
	\kern\ellipsisbeforegap
	.\kern\ellipsisgap
	.\kern\ellipsisgap
	.\kern\ellipsisgap
	\ellipsis@after\relax}
\renewcommand*{\ellipsis@centered}{%
	\ellipsis@before
	\kern\ellipsisbeforegap
	.\kern\ellipsisgap
	.\kern\ellipsisgap
	.\kern\ellipsisaftergap
	\ellipsis@after\relax}
\AtBeginDocument{%
	\DeclareRobustCommand*{\dots}{%
		\ifmmode\@xp\mdots@\else\@xp\textellipsis\fi}}
\def\ellipsisgap{.1em}
\def\ellipsisbeforegap{.05em}
\def\ellipsisaftergap{.05em}
\makeatother

\usepackage{hyperref}

\usepackage{hyperref}
\hypersetup{
	colorlinks = true	
}

\begin{document}
	%%%%%%%%%%%RÖVIDÍTÉSEK%%%%%%%%%%
	\setlength\parindent{0pt}
	\def\s{\hspace{0.2mm}\vphantom{\beta}}
	\def\Z{\mathbb{Z}}
	\def\Q{\mathbb{Q}}
	\def\R{\mathbb{R}}
	\def\C{\mathbb{C}}
	\def\N{\mathbb{N}}
	\def\Ra{\overline{\mathbb{R}}}
	
	\def\sume{\displaystyle\sum_{n=1}^{+\infty}}
	\def\sumn{\displaystyle\sum_{n=0}^{+\infty}}
	
	\def\narrow{\underset{n\rightarrow+\infty}{\longrightarrow}}
	\def\limn{\displaystyle\lim_{n\to +\infty}}
	\def\limx{\displaystyle\lim_{x\to +\infty}}
	
	\theoremstyle{definition}
	\newtheorem{theorem}{Tétel}[subsection] 
	
	\theoremstyle{definition}
	\newtheorem{definition}[theorem]{Definíció} 
	\newtheorem{example}[theorem]{Példa} 
	\newtheorem{task}[theorem]{Feladat} 
	\newtheorem{note}[theorem]{Megjegyzés}
	%%%%%%%%%%%%%%%%%%%%%%%%%%%%%%%%%%%%%%%%%%%%%%%%%%%%%%%%%%%%%%%%%%%%%
	\begin{center}
		{\LARGE \textbf{Analízis II.}}
		
		{\large \textbf{+/$-$ kidolgozás}}
		
		3. óra
	\end{center}
	A jegyzet \textsc{Umann} Kristóf kidolgozásaiból készült, Dr. \textsc{Szili} László előadása alapján. (\today)
	
	Gyakorlathoz pdf: \url{http://numanal.inf.elte.hu/~szili/Oktatas/An2_BSc_2016/An2_gyak_2016_osz.pdf}
	\begin{enumerate}
		\item \textbf{Definiálja egy  $f\in\R\to\R$ függvény pontbeli folytonosságát.}
		
		\textbf{Válasz:} Az $f\in\R\to\R$ az $a\in\mathcal{D'}_f$ pontban folytonos, ha 
		\[ \forall \varepsilon>0, \quad \exists \delta>0, \quad \forall x\in\mathcal{D}_f,\quad |x-a|<\delta:\quad |f(x)-f(a)|<\varepsilon. \]
		Jelölése: $f\in C(a)$.
		
		\item \textbf{Mi a kapcsolat a pontbeli folytonosság és a határérték között?}
		
		\textbf{Válasz:} Ha $a\in \mathcal{D}_f\cap\mathcal{D}'_f,$ akkor 
		\[f\in C\{a\} \quad \Leftrightarrow\quad  \exists \lim_af \text{ és }\lim_af=f(a).  \]
		
		\item \textbf{Milyen tételt ismer hatványsor összegfüggvényének a folytonosságáról?}
		
		\textbf{Válasz:}
			Hatványsor összegfüggvénye a konvergenciahalmaz minden
			\begin{enumerate}
				\item pontjában folytonos
				\item Az $\exp, \sin, \cos,$ sh, ch, $\forall\R$-beli pontban folytonos.
			\end{enumerate}
		
		\item \textbf{Hogyan szól a folytonosságra vonatkozó átviteli elv?}
		
		\textbf{Válasz:} Tegyük fel, hogy $f\in\R\to\R;\quad  a\in\mathcal{D}_f$.
		
		\[ f\in C\{a\}\quad \Leftrightarrow\quad  \forall(x_n):  \N\to\mathcal{D}_f,\quad  \lim(x_n) = a \]
		esetén
		
		\[ \limn f(x_n) = f(a). \]
		
		\item \textbf{Fogalmazza meg a hányadosfüggvény folytonosságára vonatkozó tételt.}
		
		\textbf{Válasz:} Tegyük fel, hogy $f,g\in\R\to\R, \quad f,g\in C\{a\}.$ Ekkor: 
		\[\frac{f}{g} \quad (g(a)\not=0) \quad \in C\{a\}.\]
		
		\item \textbf{Milyen tételt ismer az összetett függvény pontbeli folytonosságáról?}
		
		\textbf{Válasz:}Tegyük fel, hogy $f,g\in\R\to\R, \quad f,g\in C\{a\}.$ Ekkor, ha 
		
		\[\mathcal{R}_g\subset\mathcal{D}_f, \quad g\in C\{a\}, \quad f\in C\{g(a)\}\quad \Rightarrow\quad  f\circ g\in C\{a\}\]
		
		\item \textbf{Mit jelent az, hogy egy függvény jobbról folytonos egy pontban?}
		
		\textbf{Válasz:} 	Legyen $f\in\R\to\R$ és $a\in\mathcal{D}_f$. Az $f$ függvény  jobbról folytonos $a$-ban, ha
		
		\[\forall \varepsilon>0, \quad \exists \delta>0,\quad  \forall x\in\mathcal{D}_f,\quad  a\leq x< a + \delta \text{ \quad esetén\quad  } |f(x)-f(a)|<\varepsilon  \]
		
		\item \textbf{Mit tud mondani a korlátos és zárt $[a,b] \subset R$\quad intervallumon folytonos
			függvény értékkészletéről?}
		
		\textbf{Válasz:} 	Legyen $f\in\R\to\R$ és $a\in\mathcal{D}_f$. Az $f$ függvény  jobbról folytonos $a$-ban, ha
		
		\[\forall \varepsilon>0, \quad \exists \delta>0,\quad  \forall x\in\mathcal{D}_f,\quad  a\leq x< a + \delta \text{ \quad esetén\quad  } |f(x)-f(a)|<\varepsilon  \]
		
		\item \textbf{Hogyan szól a Weierstrass-tétel?}
		
		\textbf{Válasz:} Tegyük fel, hogy:
		\[ \left.
		\begin{gathered} 
		f: [a,b]\to\R \\
		\text{folytonos } [a,b] 
		\end{gathered}
		\right\} \Rightarrow
		\begin{gathered}
		\text{$f$-nek $\exists$ absz. szélsőértéke, azaz\quad }\exists \alpha, \beta, \in [a,b]: \\
		f(x) \leq f(\alpha) \\
		f(\beta) \leq f(x)
		\end{gathered}\quad (x\in[a,b])\]		
	\end{enumerate}
\end{document}