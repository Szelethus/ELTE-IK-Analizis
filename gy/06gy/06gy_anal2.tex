\documentclass[a4paper,11.5pt]{article}
\usepackage[textwidth=170mm, textheight=230mm, inner=20mm, top=20mm, bottom=30mm]{geometry}
\usepackage[normalem]{ulem}
\usepackage[utf8]{inputenc}
\usepackage[T1]{fontenc}
\PassOptionsToPackage{defaults=hu-min}{magyar.ldf}
\usepackage[magyar]{babel}
\usepackage{amsmath, amsthm,amssymb,paralist,array, ellipsis, graphicx}
%\usepackage{marvosym}

\makeatletter
\renewcommand*{\mathellipsis}{%
	\mathinner{%
		\kern\ellipsisbeforegap%
		{\ldotp}\kern\ellipsisgap%
		{\ldotp}\kern\ellipsisgap%
		{\ldotp}\kern\ellipsisaftergap%
	}%
}
\renewcommand*{\dotsb@}{%
	\mathinner{%
		\kern\ellipsisbeforegap%
		{\cdotp}\kern\ellipsisgap%
		{\cdotp}\kern\ellipsisgap%
		{\cdotp}\kern\ellipsisaftergap%
	}%
}
\renewcommand*{\@cdots}{%
	\mathinner{%
		\kern\ellipsisbeforegap%
		{\cdotp}\kern\ellipsisgap%
		{\cdotp}\kern\ellipsisgap%
		{\cdotp}\kern\ellipsisaftergap%
	}%
}
\renewcommand*{\ellipsis@default}{%
	\ellipsis@before
	\kern\ellipsisbeforegap
	.\kern\ellipsisgap
	.\kern\ellipsisgap
	.\kern\ellipsisgap
	\ellipsis@after\relax}
\renewcommand*{\ellipsis@centered}{%
	\ellipsis@before
	\kern\ellipsisbeforegap
	.\kern\ellipsisgap
	.\kern\ellipsisgap
	.\kern\ellipsisaftergap
	\ellipsis@after\relax}
\AtBeginDocument{%
	\DeclareRobustCommand*{\dots}{%
		\ifmmode\@xp\mdots@\else\@xp\textellipsis\fi}}
\def\ellipsisgap{.1em}
\def\ellipsisbeforegap{.05em}
\def\ellipsisaftergap{.05em}
\makeatother

\usepackage{hyperref}
\hypersetup{
	colorlinks = true	
}
\DeclareMathOperator{\Int}{int}
\DeclareMathOperator{\tg}{tg}
\DeclareMathOperator{\Th}{th}
\DeclareMathOperator{\sh}{sh}
\DeclareMathOperator{\ch}{ch}

\begin{document}
	%%%%%%%%%%%RÖVIDÍTÉSEK%%%%%%%%%%
	\setlength\parindent{0pt}
	\def\s{\hspace{0.2mm}\vphantom{\beta}}
	\def\Z{\mathbb{Z}}
	\def\Q{\mathbb{Q}}
	\def\R{\mathbb{R}}
	\def\C{\mathbb{C}}
	\def\N{\mathbb{N}}
	\def\Rn{\mathbb{R}^{n}}
	\def\Ra{\overline{\mathbb{R}}}
	\def\sume{\displaystyle\sum_{n=1}^{+\infty}}
	\def\sumn{\displaystyle\sum_{n=0}^{+\infty}}
	\def\biz{\emph{Bizonyítás:\ }}
	\def\narrow{\underset{n\rightarrow+\infty}{\longrightarrow}}
	\def\limn{\displaystyle\lim_{n\to +\infty}}
	\def\limx{\displaystyle\lim_{x\to +\infty}}
	
	\theoremstyle{definition}
	\newtheorem{theorem}{Tétel}[subsection] % reset theorem numbering for each chapter
	
	\theoremstyle{definition}
	\newtheorem{definition}[theorem]{Definíció} % definition numbers are dependent on theorem numbers
	\newtheorem{example}[theorem]{Példa} % same for example numbers
	\newtheorem{task}[theorem]{Feladat} % same for example numbers
	\newtheorem{note}[theorem]{Megjegyzés} % same for example numbers
	\newtheorem{revision}[theorem]{Emlékeztető} % same for example numbers
	%%%%%%%%%%%%%%%%%%%%%%%%%%%%%%%%%
	\begin{center}
		{\LARGE \textbf{Analízis II.}}
		
		{\large \textbf{Gyakorlati óra jegyzet}}
		
		6. óra
	\end{center}
	A jegyzetet \textsc{Umann} Kristóf készítette Dr. \textsc{Szili} László gyakorlatán. (\today)
	
	Tantárgyi honlap: \url{http://numanal.inf.elte.hu/~szili/Oktatas/An2_BSc_2016/index_An2_2016.htm}
	
	\section{Inverz deriválása.}
	\begin{revision}
		Rajz + tétel
		\[ f:\quad \exists f^{-1},\quad f'(a)\not=0 \]
		\[ b:=f(a),\quad (a=f^{-1}(b))\quad \Rightarrow\quad f^{-1}\in D\{b\},\quad (f^{-1})'(b)=\frac{1}{f'(a)}=\frac{1}{f'(f^{-1}(b))}= \]
		\[ =\left(\frac{1}{f'\circ f^{-1}}\right)(b) \]
		\begin{center}
			\fbox{$\displaystyle f^{-1}=\frac{1}{f'\circ f^{-1}}$}
		\end{center}
	\end{revision}
	\begin{task}
		Tegyük efl, hogy $f\in D(\R)$ és $f'=f$. Bizonyítsuk be:
		\[ \exists c\in\R:\quad f(x)=ce^x\quad (x\in\R) \]
		
		\textit{Megoldás:} Tudjuk: $(e^x)'=e^x\quad (x\in\R)$. Ötlet: Tekintsük a következő függvényt:
		\[ \frac{f(x)}{e^x}\quad (x\in\R)\quad \Rightarrow\quad \frac{f(x)}{e^x}\in D(\R) \]
		\[ \frac{f(x)}{e^x}=\frac{f'(x)\cdot e^x-f(x)\cdot e^x}{(e^x)^2}=0\quad (\forall x\in\R) \]
		\[ \Rightarrow\quad \exists c\in\R:\quad \frac{f(x)}{e^x}=c\quad (\forall x\in\R).\quad \blacksquare \]
	\end{task}
	\begin{task}
		Bizonyítsuk be: \[f(x):=x^3+x\quad (x\in\R)\] invertálható, továbbá $f^{-1}\in D$, továbbá \[ f(^{-1})'(2)=? \]
		\begin{note}
			Nem jelölhetjük a függvény változóját $x$-el $f$ és $f^{-1}$ esetében is. Erre figyelnünk kell.
		\end{note}
		\textit{Megoldás:} $f'(x)=3x^2+1>0\quad (\forall x\in \R)\quad \Rightarrow$
		\[ f \uparrow \ \R\text{-en}\quad \Rightarrow \quad \exists f^{-1} \]
		\[ \forall x\in\R:\quad f'(x)\not=0\quad \Rightarrow\quad y=f(x)\in\mathcal{R}_f=\mathcal{D}_{f^{-1}}\quad \Rightarrow f^{-1}\in D(y) \]
		\[ (f^{-1})'(y)=\frac{1}{f'(x)}=\frac{1}{f'(f^{-1}(y))},\quad y\in\mathcal{D}_{f^{-1}};\quad x=f^{-1}(y) \]
		\[ y=2=f(x)=x^3+x \]
		Észrevétel: $x=1$ megoldás, sőt, mivel $f\uparrow$, nincs is több megoldás.
		\[ f(^{-1})'(2)=\frac{1}{f'(1)}=\frac{1}{(3x^2+1)_{x=1}}=\frac{1}{4}.\quad \blacksquare \]
	\end{task}
	\begin{task}
		\[ f(x):=x+e^x\quad (x\in\R) \]
		Bizonyítsuk be, hogy $\exists f^{-1},\quad f^{-1}\in D^2,\quad (f^{-1})(1)=?$
		\medskip
		
		\textit{Megoldás:} $f'(x)='+e^x>0\quad (\forall x\in\R)$
		\[ \Rightarrow f\uparrow \R\text{-en}\quad \Rightarrow\quad \exists f^{-1} \]
		Inverz deriválhatósága:
		\[ f\in D(\R),\quad f'(x)\not=0\quad (\forall x\in\R) \]
		\[ \forall y=f(x)\in\mathcal{R}_f=\mathcal{D}_{f^{-1}}:\quad f^{-1}\in D\{y\}\quad \text{és}\quad (f^{-1})'(y)=\frac{1}{f'(x)}=\frac{1}{f'(f^{-1}(y))} \]
		\[ \Rightarrow\quad (f^{-1})'=\frac{1}{f'\circ f^{-1}} \]
		$f^{-1}\overset{?}{\in}D^2\checkmark$
		\[ (f^{-1})''=((f^{-1})')'=\left(\frac{1}{f'\circ f^{-1}}\right)=-\frac{1}{(f'\circ f^{-1})^2}\cdot(f'\circ f^{-1})= -\frac{f''\circ f^{-1}\cdot(f^{-1})^{-1}}{(f'\circ f^{-1})^2}=\frac{f''\circ f^{-1}}{(f'\circ f^{-1})^3} \]
		\[ (f^{-1})''(y)=-\frac{(f''\circ f^{-1})(y)}{((f'\circ f^{-1})(y))^3}=\frac{f''(f^{-1}(y))}{f'(f^{-1}(y))^3}\quad \overset{x=f^{-1}(y)}{=}\quad -\frac{f''(x)}{(f'(x))^3} \]
		$y=f(x)=x+e^x$. Ha $y=1$, akkor:
		\[ 1=x+e^x \]
		$\Rightarrow x=0$ gyök és $f\uparrow$ az egyetlen.
		\[(f^{-1})''(1)=-\frac{f''(0)}{(f'(0))^3}=-\frac{1}{8} \]
		\[ f'(x)=1+e^x,\quad f'(0)=2,\quad f''(x)=e^x,\quad f''(0)=1 \]
	\end{task}
	\begin{task}
		Bizonyítsa be:
		\[ f(x):=x^5+10x+3=0 \]
		$\exists!$ megoldás $\R$-en.
		
		\textit{Megoldás:} Bolzano-t, $f\in C(\R)\checkmark$ Keresni kell egy olyan intervalumot, melyben különbözik a függvényérték. a 0 egy triviális kiindulási pont:
		\[ f(0)=3>0;\quad f(-1)=-8<0 \]
		\[ \Rightarrow\quad \exists \xi\in(-1,0):\quad f(\xi)=0 \]
		$\xi$ az egyetlen, ugyanis $f'(x)=5x^4+10>0\quad (\forall x\in\R)$
		\[ \Rightarrow f\uparrow \R\text{-en}.\quad \blacksquare \]
	\end{task}
	\textbf{Házi feladat:} Igazoljuk hogy $\exists!$ megoldása $\R$-en ennek az egyenletnek:
	\[ e^x=1+x. \]
	\begin{task}
		Igazoljuk:
		\[ x-\frac{x^2}{2}<\ln(x+1)<x\quad (x>0) \]
		\textit{Megoldás:} 
		\begin{enumerate}
			\item \[\ln(x+1)<x\quad \Leftrightarrow\quad 0<\overbrace{x-\ln(x+1)}^{f(x)} \]
			\[ f(x):=x-\ln(1+x)\quad (x>-1) \]
			$f\in D,\quad f'(x)=1-\frac{1}{x+1}\cdot1=\frac{x}{x+1}>0,$\quad ha $x>0$, továbbá
			\[ f(0)=0\quad \Rightarrow\quad 0<x-\ln(x+1)\quad (x>0). \]
			\item \[ x-\frac{x^2}{2}<\ln(x+1)\quad (x>0)\quad \Rightarrow\quad 0<\overbrace{\ln(x+1)-\left(x-\frac{x^2}{2}\right)}^{g(x)} \]
			\[ g'(x)=\frac{1}{x+1}-1+x=\frac{1-x-1+x^2+x}{x+1}>0\quad (\forall x>0).\quad \blacksquare \]
		\end{enumerate}
	\end{task}
	\textbf{Házi feladat:} 7 gyak anyaga: 4ad
\end{document}