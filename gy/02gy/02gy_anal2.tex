\documentclass[a4paper,11.5pt]{article}
\usepackage[textwidth=170mm, textheight=230mm, inner=20mm, top=20mm, bottom=30mm]{geometry}
\usepackage[normalem]{ulem}
\usepackage[utf8]{inputenc}
\usepackage[T1]{fontenc}
\PassOptionsToPackage{defaults=hu-min}{magyar.ldf}
\usepackage[magyar]{babel}
\usepackage{amsmath, amsthm,amssymb,paralist,array, ellipsis, graphicx}
%\usepackage{marvosym}

\makeatletter
\renewcommand*{\mathellipsis}{%
	\mathinner{%
		\kern\ellipsisbeforegap%
		{\ldotp}\kern\ellipsisgap%
		{\ldotp}\kern\ellipsisgap%
		{\ldotp}\kern\ellipsisaftergap%
	}%
}
\renewcommand*{\dotsb@}{%
	\mathinner{%
		\kern\ellipsisbeforegap%
		{\cdotp}\kern\ellipsisgap%
		{\cdotp}\kern\ellipsisgap%
		{\cdotp}\kern\ellipsisaftergap%
	}%
}
\renewcommand*{\@cdots}{%
	\mathinner{%
		\kern\ellipsisbeforegap%
		{\cdotp}\kern\ellipsisgap%
		{\cdotp}\kern\ellipsisgap%
		{\cdotp}\kern\ellipsisaftergap%
	}%
}
\renewcommand*{\ellipsis@default}{%
	\ellipsis@before
	\kern\ellipsisbeforegap
	.\kern\ellipsisgap
	.\kern\ellipsisgap
	.\kern\ellipsisgap
	\ellipsis@after\relax}
\renewcommand*{\ellipsis@centered}{%
	\ellipsis@before
	\kern\ellipsisbeforegap
	.\kern\ellipsisgap
	.\kern\ellipsisgap
	.\kern\ellipsisaftergap
	\ellipsis@after\relax}
\AtBeginDocument{%
	\DeclareRobustCommand*{\dots}{%
		\ifmmode\@xp\mdots@\else\@xp\textellipsis\fi}}
\def\ellipsisgap{.1em}
\def\ellipsisbeforegap{.05em}
\def\ellipsisaftergap{.05em}
\makeatother

\usepackage{hyperref}
\hypersetup{
	colorlinks = true	
}

\begin{document}
	%%%%%%%%%%%RÖVIDÍTÉSEK%%%%%%%%%%
	\setlength\parindent{0pt}
	\def\s{\hspace{0.2mm}\vphantom{\beta}}
	\def\Z{\mathbb{Z}}
	\def\Q{\mathbb{Q}}
	\def\R{\mathbb{R}}
	\def\C{\mathbb{C}}
	\def\N{\mathbb{N}}
	\def\Rn{\mathbb{R}^{n}}
	\def\Ra{\overline{\mathbb{R}}}
	\def\sume{\displaystyle\sum_{n=1}^{+\infty}}
	\def\sumn{\displaystyle\sum_{n=0}^{+\infty}}
	\def\biz{\emph{Bizonyítás:\ }}
	\def\narrow{\underset{n\rightarrow+\infty}{\longrightarrow}}
	\def\limn{\displaystyle\lim_{n\to +\infty}}
	\def\limx{\displaystyle\lim_{x\to +\infty}}
	
	\theoremstyle{definition}
	\newtheorem{theorem}{Tétel}[subsection] % reset theorem numbering for each chapter
	
	\theoremstyle{definition}
	\newtheorem{definition}[theorem]{Definíció} % definition numbers are dependent on theorem numbers
	\newtheorem{example}[theorem]{Példa} % same for example numbers
	\newtheorem{task}[theorem]{Feladat} % same for example numbers
	\newtheorem{note}[theorem]{Megjegyzés} % same for example numbers
	\newtheorem{revision}[theorem]{Emlékeztető} % same for example numbers
	%%%%%%%%%%%%%%%%%%%%%%%%%%%%%%%%%
	\begin{center}
		{\LARGE \textbf{Analízis II.}}
		
		{\large \textbf{Gyakorlati óra jegyzet}}
		
		2. óra
	\end{center}
	A jegyzetet \textsc{Umann} Kristóf készítette Dr. \textsc{Szili} László gyakorlatán. (\today)
	
	Tantárgyi honlap: \url{http://numanal.inf.elte.hu/~szili/Oktatas/An2_BSc_2016/index_An2_2016.htm}
	\bigskip
	
	Következő órára az első 8 definíció fog kelleni.
	\section{Kritikus határértékek kiszámolása.}
	\begin{note}
		Átalakítjuk nem kritikus határértkre, ahol a tételeinket tudjuk alkalmazni.
	\end{note}
	\subsection{,,Szorzatra bontás'' technikája}
	\begin{task}
		 ($m,n=1,2,\ldots$)
		
		\[ \lim_{x\to1}\frac{x^n-1}{x^m-1}=\lim_{x\to1}\frac{(x-1)(x^{n-1}+x^{n-2}+\ldots+1)}{(x-1)(x^{m-1}+\ldots+1)}=\frac{n}{m} \]
		
		Ez egy kritikus határérték: $\frac{0}{0}$
	\end{task}
	\begin{task}
		\[ \lim_{x\to2}\frac{x^2-5x+6}{x^2-7x+10}\overset{\frac{0}{0}}{=}\lim_{x\to2}\frac{(x-2)(x-3)}{(x-2)(x-5)}=\lim_{x\to2}\frac{(x-3)}{(x-5)}=\frac{-1}{-3}=\frac{1}{3} \]
	\end{task}
	\subsection{,,Leosztás'' technikája}
	\begin{task}
		\[ \lim_{x\to+\infty}\left( -3x^2+2x-1 \right)\overset{(-\infty)+(+\infty)}{=}\lim_{x\to+\infty}x^3\cdot\overbrace{\left(-3+\frac{2}{x}-\frac{1}{x^3}\right)}^{\text{tart }-3\text{-hoz}} =-\infty \]
	\end{task}
	\begin{note}
		Polinom határértéke ($\infty$)-ben (lásd gy. 3)
	\end{note}
	\begin{task}
		\[ \lim_{x\to+\infty}\frac{x^2-3x+2}{x^3-7x^2+5x-1}=\lim_{x\to+\infty}\frac{\frac{1}{x}-\frac{2}{x^2}+\frac{2}{x^3}}{1-\frac{7}{x}+\frac{5}{x^2}-\frac{1}{x^3}}=  \frac{0}{1}=0 \]
	\end{task}
	\begin{task}
		\[ \lim_{x\to+\infty}\frac{x^3+2x^2+11x+2}{x^2+3x+2}\overset{\frac{+\infty}{+\infty}}{=}\lim_{x\to+\infty}\frac{x+2+\frac{11}{x}+\frac{2}{x^2}}{1+\frac{3}{x}+\frac{2}{x^2}}=\frac{+\infty}{1}=+\infty \]
	\end{task}
	\begin{task}
		\[ \lim_{x\to+\infty}\frac{2x^3+3x^2+23}{-3x^3-5x^2+31x+1}=\overset{\frac{+\infty}{-\infty}}{=}\lim_{x\to+\infty}\frac{2+\frac{3}{x}+\frac{23}{x^3}}{-3-\frac{5}{x}+\frac{31}{x^2}+\frac{1}{x^3}}=\frac{2}{-3}=-\frac{2}{3} \]
	\end{task}
	\subsection{,,gyöktelenítés''}
	\begin{task}
		\[ \lim_{x\to0} \frac{\sqrt{1+x}-1}{x}\overset{\frac{0}{0}}{=} \lim_{x\to}\frac{\sqrt{1+x}-1}{x}\cdot\frac{\sqrt{1+x}+1}{\sqrt{1+x}+1}=\lim_{x\to0}\frac{x}{x\left(\sqrt{1+x}+1\right)}=\frac{1}{2} \]
	\end{task}
	\begin{task}
		\[ \lim_{x\to0}\frac{\sqrt{1+x}-\sqrt{1-x^2}}{\sqrt{1+x}-1}\overset{\frac{0}{0}}{=}=\sqrt{1+x}\cdot\frac{1-\sqrt{1-x}}{\sqrt{1+x}-1}\cdot\frac{\sqrt{1+x}+1}{\sqrt{1+x}+1}= \]
		\[ =\sqrt{1+x}\frac{\sqrt{1+x}+1}{x}\cdot(1-\sqrt{1-x})\cdot\frac{1+\sqrt{1-x}}{1+\sqrt{1-x}}=\sqrt{1+x}\frac{\sqrt{1+x}+1}{x}\cdot\frac{x}{1+\sqrt{1-x}}\underset{x\to0}{\longrightarrow}1\cdot2\cdot\frac{1}{2}=1 \]
	\end{task}
	\begin{task}($n=2,3\ldots$)
		\[ \lim_{x\to0}\frac{\sqrt[n]{1+x}-1}{x}\overset{\frac{0}{0}}{=}\frac{\sqrt[n]{1+x}-1}{x}\cdot\frac{(\sqrt[n]{1+x})^{n-1}+\ldots+1 }{(\sqrt[n]{1+x})^{n-1}+\ldots+1 } =\frac{x}{\left((\sqrt[n]{1+x})^{n-1}+(\sqrt[n]{1+x})^{n-2}+\ldots+1\right)} \underset{x\to0}{\longrightarrow}\frac{1}{n} \]
		Itt alkalmaztuk azt, hogy $a^n-b^n=(a-b)(a^{n-1}+a^{n-2}b+\ldots+b^{n-1})$.
	\end{task}
	\subsection{,,$\displaystyle \frac{\sin x}{x}$''-es}
	\begin{note}
		Hatványsorral megmutattuk, hogy $\displaystyle \lim_{x\to0}\frac{\sin \alpha x}{\alpha x} = 1\quad \forall x\in\R$.
	\end{note}
	\begin{task}$(a,b\in\R, \quad a\not=b)$
		\[ \lim_{x\to0}\frac{\sin ax}{\sin bx}\overset{\frac{0}{0}}{=} \lim_{x\to0}\overbrace{\frac{\sin ax}{ax}}^{\text{tart 0-hoz}}\cdot\overbrace{\frac{1}{\frac{\sin bx}{bx}}}^{\text{tart 1-hez.}}\cdot\frac{a}{b}=\frac{a}{b} \]
	\end{task}
	\begin{task}
		\[ \lim_{x\to0}\frac{1-\cos x}{x^2}\overset{\frac{0}{0}}{=}\lim_{x\to0}\frac{1-\cos x}{x^2}\quad \overset{\text{TRÜKK}}{\cdot}\quad \frac{1+\cos x}{1+\cos x}=\lim_{x\to0}\frac{1-\cos^2x}{x^2(1+\cos x)}=\lim_{x\to 0}\frac{\sin^2x}{x^2}\cdot\frac{1}{1+\cos x}=\]\[=\lim_{x\to0}\left(\frac{\sin x}{x}\right)^2\cdot\frac{1}{1+\cos x}=\frac{1}{2} \]
	\end{task}
	\begin{note}Nevezetes határérték (zh-n nem bizonyítandó)
		\[ \lim_{x\to0}\frac{1-\cos x}{x^2}=\frac{1}{2} \]
	\end{note}
	\begin{task}
		\[ \lim_{x\to0}\frac{1-\cos x}{x}\overset{\frac{0}{0}}{=}\lim_{x\to0}\frac{1-\cos x}{x}\cdot\frac{1+\cos x}{1+\cos x}=\lim_{x\to0}\underbrace{\frac{\sin x}{x}}_{1}\cdot\underbrace{\sin x}_{0}\underbrace{\frac{1}{1+\cos x}}_{\frac{1}{2}}=0 \]
	\end{task}
	\begin{task}
		\[ \lim_{x\to0}\frac{\tan x-\sin x}{x^3}\overset{\frac{0}{0}}{=}\frac{\frac{\sin x}{\cos x}-\sin x}{x^3}=\sin x\frac{1-\cos x}{x^3\cdot \cos x}=\underbrace{\frac{\sin x}{x}}_1\cdot\underbrace{\frac{1-\cos x}{x^2}}_{\frac{1}{2}}\cdot\underbrace{\frac{1}{\cos x}}_1\rightarrow\frac{1}{2} \]
	\end{task}
	\subsection{,,Hatványsorral''.}
	\begin{revision}
		\[ e^x:=\sumn\frac{x^n}{n!}=1+\frac{x}{1!}+\frac{x^2}{2!}+\frac{x^3}{3!}+\ldots\quad (x\in\R) \]
		\[ \sin x:=x-\frac{x^3}{3!}+\frac{x^5}{5!}-\frac{x^7}{7!}+\ldots\quad (x\in\R) \]
		\[ \cos x :=1-\frac{x^2}{2!}+\frac{x^4}{4!}-\frac{x^6}{6!}+\ldots\quad (x\in\R) \]
		\[ \text{sh } x:=x+\frac{x^3}{3!}+\frac{x^5}{5!}+\frac{x^7}{7!}+\ldots\quad (x\in\R) \]
		\[ \text{ch } x :=1+\frac{x^2}{2!}+\frac{x^4}{4!}+\frac{x^6}{6!}+\ldots\quad (x\in\R)  \]
	\end{revision}
	\begin{note}
		Nagyon érdemes ezeket az alakokat kibontva is megjegyezni.
	\end{note}
	\begin{revision}
		Hatványsor összegfüggvényének határértéke.
	\end{revision}
	\begin{revision}
		($a$ középpontú hatványsor)\quad $\displaystyle \sum_{n=0}\alpha_n(x-a)^n$
	\end{revision}
	\begin{task}
		\[ \lim_{x\to0}\frac{1-\cos x}{x^2}=\lim_{x\to0}\frac{1-\left(1-\frac{x^2}{2!}+\frac{x^4}{4!}-\frac{x^6}{6!}+\ldots\right)}{x^2}=\lim_{n\to0}\frac{\frac{x^2}{2!}-\frac{x^4}{4!}+\frac{x^6}{6!}-\ldots}{x^2}=\lim_{x\to0}\underbrace{\left(\frac{1}{2}-\frac{x^2}{4!}+\frac{x^4}{6!}-\ldots\right)}_{\R\text{-en konvergens (H.F.)}}=\frac{1}{2} \]
	\end{task}
	\begin{task}
		\[ \lim_{x\to+\infty}e^x=\lim_{n\to+\infty}\left(1+x+\frac{x^2}{2!}+\frac{x^3}{3!}+\ldots\right)=+\infty, \]
		Ugyanis:
		\[ e^x>x\quad \forall x>1. \quad x\to+\infty \quad \Rightarrow\quad e^x\to+\infty \]
		Itt a minoráns kritériumot alkalmaztuk.
	\end{task}
	\begin{note}
		\[ \lim_{x\to+\infty}e^x=+\infty \]
	\end{note}
	\begin{task}
		\[ \lim_{n\to-\infty}e^x=? \]
		\[ e^x=\frac{1}{e^{-x}}\underset{x\to-\infty}{\longrightarrow}\frac{1}{+\infty}=0 \]
		\[ \lim_{x\to-\infty}e^x=0. \]
	\end{task}
	Házi feladat: 6c, 3. oldalon.
\end{document}