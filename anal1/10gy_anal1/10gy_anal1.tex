\documentclass[a4paper,11.5pt]{article}

\usepackage[textwidth=135mm]{geometry}

\usepackage[utf8]{inputenc}

\usepackage[T1]{fontenc}

\PassOptionsToPackage{defaults=hu-min}{magyar.ldf}

\usepackage[magyar]{babel}

\usepackage{amsmath,amssymb,paralist}

\usepackage{ellipsis}

\makeatletter
\renewcommand*{\mathellipsis}{%
	\mathinner{%
		\kern\ellipsisbeforegap%
		{\ldotp}\kern\ellipsisgap%
		{\ldotp}\kern\ellipsisgap%
		{\ldotp}\kern\ellipsisaftergap%
	}%
}
\renewcommand*{\dotsb@}{%
	\mathinner{%
		\kern\ellipsisbeforegap%
		{\cdotp}\kern\ellipsisgap%
		{\cdotp}\kern\ellipsisgap%
		{\cdotp}\kern\ellipsisaftergap%
	}%
}
\renewcommand*{\@cdots}{%
	\mathinner{%
		\kern\ellipsisbeforegap%
		{\cdotp}\kern\ellipsisgap%
		{\cdotp}\kern\ellipsisgap%
		{\cdotp}\kern\ellipsisaftergap%
	}%
}
\renewcommand*{\ellipsis@default}{%
	\ellipsis@before
	\kern\ellipsisbeforegap
	.\kern\ellipsisgap
	.\kern\ellipsisgap
	.\kern\ellipsisgap
	\ellipsis@after\relax}
\renewcommand*{\ellipsis@centered}{%
	\ellipsis@before
	\kern\ellipsisbeforegap
	.\kern\ellipsisgap
	.\kern\ellipsisgap
	.\kern\ellipsisaftergap
	\ellipsis@after\relax}
\AtBeginDocument{%
	\DeclareRobustCommand*{\dots}{%
		\ifmmode\@xp\mdots@\else\@xp\textellipsis\fi}}
\def\ellipsisgap{.1em}
\def\ellipsisbeforegap{.05em}
\def\ellipsisaftergap{.05em}
\makeatother



\begin{document}
	
	%%%%%%%%%%%RÖVIDÍTÉSEK%%%%%%%%%%
	
	%\def\frac{\displaystyle\frac}
	\def\a{\textbf{a}}
	\def\b{\textbf{b}}
	\def\N{\hskip 10 true mm}
	\def\a{\textbf{a}}
	\def\b{\textbf{b}}
	\def\c{\textbf{c}}
	\def\d{\textbf{d}}
	\def\e{\textbf{e}}
	\def\x{\textbf{x}}
	\def\gg{$\gamma$}
	\def\vi{\textbf{i}}
	\def\jj{\textbf{j}}
	\def\kk{\textbf{k}}
	\def\fh{\overrightarrow}
	\def\l{\lambda}
	\def\m{\mu}
	\def\v{\textbf{v}}
	\def\0{\textbf{0}}
	\def\s{\hspace{0.2mm}\vphantom{\beta}}
	\def\Z{\mathbb{Z}}
	\def\Q{\mathbb{Q}}
	\def\R{\mathbb{R}}
	\def\C{\mathbb{C}}
	\def\N{\mathbb{N}}
	\def\Rn{\mathbb{R}^{n}}
	\def\biz{\emph{Bizonyítás: }}
	\def\sume{\sum_{n=1}^{+\infty}}
	\def\sumn{\sum_{n=0}^{+\infty}}
	
	%%%%%%%%%%%%%%%%%%%%%%%%%%%%%%%%%
	
	\begin{center}
		\textbf{11. heti analízis gyakorlat}
	\end{center}
	
	\begin{enumerate}
	
	\item \textbf{Feladat}:
	
	$\displaystyle\sume nq^n=?$
	
		\medskip
	Megoldás:
	\begin{center}
		$q+2q^2+3q^3+\ldots$
	\end{center}
		\medskip
	1. lehetőség:
	
	\medskip
	\begin{tabbing}
000000000000000000000000000000\=000000000\kill
		\>$q+$\=$q^2+q^3+\ldots$
		\\
			\>\>$q^2+$\=$q^3+\ldots$
		\\
		 \>\>\>$q^3+\ldots$
	\end{tabbing}
	
	
	\medskip
	2. lehetőség
	\begin{center}
		
	$s_n=q+2q^2+q^3+\ldots+Nq^N$
	
	\end{center}
	Észrevétel: 
	\begin{center}
		
	$qs_n=q^2+2q^3+\ldots+(N-1)q^N+Nq^{N+1}$
	
	$S_N-qS_N=q+q^+\ldots+q^N=Nq^{N+1}$
	
	\end{center}
	\medskip
	\item \textbf{Feladat}:
	
	Konvergencia vizsgálat.
	\begin{enumerate}
		\item \label{1}$\sum \frac{1}{2n-1}$
		\item \label{2}$\sum \frac{1}{1+n^2}$
		\item \label{3}$\sum\frac{1}{\sqrt{n(n+1)}}$
		\item \label{4}$\sum \frac{1}{\sqrt{n(n^+1)}}$
	\end{enumerate}
	
	\bigskip
	Összehasonlító kritérium
	
	$\exists N \in \N: 0\leq a_{n+1} \leq a_n$
	\begin{itemize}
		\item $\sum b_n$ konvergens	 $\Rightarrow \sum a_n$ is komnvergens
		\item $\sum a_n$ divergens $\Rightarrow \sum b_n$ is divergens.
	\end{itemize}
	
	
	
	\medskip
	\ref{1}. \[\frac{1}{1}+\frac{1}{3}+\frac{1}{5}+\ldots\]
	
	\medskip
	Sejtés: divergens.
	
	Biz: alsó becslés kell.
	
	\begin{center}
		\[\frac{1}{1}+\frac{1}{3}+\frac{1}{5}+\ldots\]
	\[\frac{1}{3}\stackrel{\displaystyle>}{<}\frac{1}{4} ; \quad \frac{1}{5}\geqq \frac{1}{6}\]
	\[1+\frac{1}{3}+\frac{1}{5}+ \ldots \geqq 1+\frac{1}{4}+\frac{1}{6}+\frac{1}{8}\ldots\geqq \frac{1}{2}+\frac{1}{4}+\frac{1}{6}\ldots=\frac{1}{2}\left(1+\frac{1}{2}+\frac{1}{3}+\ldots \right)\]
	
	\end{center}
	\medskip
	\ref{2}. Sejtéshez: 
	\[\displaystyle\frac{1}{1+n^2} \sim \displaystyle\frac{1}{n^2} \displaystyle\frac{1}{n^2} \Rightarrow \sum \frac{1}{n^2}\text{ konv. } \]
	
	\medskip
	\textit{Sejtés}: $\displaystyle\sum \frac{1}{1+n^2}$. konv.
	
	Bizonyítás: (felső becslés kell!)
	
	\[0\leq \frac{1}{1+n^2} \leqq \frac{1}{n^2} \quad \forall n=1,2,\ldots\]
	
	és $\sum \frac{1}{n^2}$ konv. $\Rightarrow \sum \frac{1}{n^2}$ konv.


\medskip
	\ref{3}. Sejtéshez:
	
	\[ \frac{1}{\sqrt{n(n+1)^2}} \sim \frac{1}{(n+1)^2}=\frac{1}{n+1}; \quad \sum \frac{1}{n} \text{ div} .\]
	
	\medskip
	\textit{Sejtés:} div.
	
	\[\frac{1}{\sqrt{n(n+1)}} \geqq \frac{1}{\sqrt{(n+1)(n+1)}}= \frac{1}{n+1} \quad (n=1,2, \ldots)\]
	
	\medskip
	és $\sum \frac{1}{n+1}$ div.  $\Rightarrow$ az eredeti is divergens.
	
	\ref{4}. Sejtéshez: \[\frac{1}{\sqrt{n(n^2+1)}} \sim \frac{1}{\sqrt{n\cdot n^2}}=\frac{1}{n^{\frac{3}{2}}}\]
	
	$\displaystyle\sum \frac{1}{n^{\frac{3}{2}}}$ konv., ui. $\frac{3}{2}> 1.$
	\end{enumerate}
	\bigskip
	\textbf{Gyök és hányadoskritérium.}
	
	\begin{enumerate}
	\bigskip
	\item \textbf{Feladat}: $\displaystyle\sum \frac{1}{n!}$
	
	\medskip
	Megoldás:
	
	(Hányadoskritérium): \[\frac{\displaystyle\frac{1}{(n+1)!}}{\displaystyle\frac{1}{n!}}=\frac{n!}{(n+1)!}=\frac{1}{n}\rightarrow 0<1 \Rightarrow\text{ a sor konvergens.}\]
	
	\medskip 
	Megj.: $\displaystyle\sumn \frac{1}{n!}=e$
	
	\bigskip
	\item\textbf{Feladat}: $\displaystyle\sum \left(\frac{1}{2}+\frac{1}{n}\right)^n$
	
	\medskip
	Megoldás: (gyk.):  \[\sqrt[n]{(\frac{1}{2}+\frac{1}{n})^n}=\frac{1}{2}+\frac{1}{n}\longrightarrow \frac{1}{2}<1 \Rightarrow\text{ a sor konv.}\]
	
	\bigskip
	
	\item\textbf{Feladat}: $\displaystyle\sum \frac{2^nn!}{n^n}$
	
	\medskip
	Megoldás: (HK)
	
	\[\frac{\displaystyle\frac{2^{n+1}}{(n+1)^{n+1}}}{\displaystyle\frac{2^nn!}{n^n}}=2\cdot(n+1)\cdot\frac{n^n}{(n+1)^{n+1}}=2\cdot\left(\frac{n}{n+1}\right)^n=2\cdot\frac{1}{(\frac{n+1}{n})^n}=2\cdot \frac{1}{(1+\frac{1}{n})^n} \longrightarrow 2e.\] 
	
	\bigskip
	MJ. (GYK)
	
	\[\sqrt[n]{\frac{2^nn!}{n^n}}=2\cdot\frac{\sqrt[n]{n!}}{n} \longrightarrow \frac{2}{e}.\]
	
		\bigskip
	\item\textbf{Feladat}: $\displaystyle\sum \frac{n^2}{2^n+3^n}$
	
	\medskip
	Megoldás: (GYK)
	
	\[\sqrt[n]{\frac{n^2}{2^n+3^n}}=\frac{\sqrt[n]{n})^2}{\sqrt[n]{2^n+3^n}}= \frac{(\sqrt[n]{n})^2}{\sqrt[n]{3^n\left(\frac{2}{3}\right)^n+1}}=\frac{\sqrt[n]{n}^2}{3\cdot\sqrt[n]{\left(\frac{2}{3}\right)^n+1}} \longrightarrow \frac{1}{3}\]
	
	$\Rightarrow $ a sor konvergens.
	
	(e)
	
	\[\sume (\frac{n}{n+1})^{n^2+n+1}\]
	
	Megoldás:
	(GYK)
	
	\[\sqrt[n]{\left(\frac{n}{n+1}\right)^{n^2+n+1}}=\left(\frac{n}{n+1}\right)^{n+1+\frac{1}{n}}=\left(\frac{n}{n+1}\right)^n\cdot\sqrt[n]{\left(\frac{n}{n+1}\right)} \longrightarrow \frac{1}{e},\] így a sor konvergens.
	
	
	
	\bigskip
\end{enumerate}
	\textbf{Paraméteres feladatok}
	\begin{enumerate}
	\item \textbf{Feladat}: $x\geqq0$ konv. $\displaystyle\sumn(\frac{\sqrt{x}}{2}-1)^n$ és mi az összeg?
	
	Megoldás: geometriai sor, \[q=\displaystyle\frac{\sqrt{x}}{2}-1\text{
	konv }\Leftrightarrow |q|=|\displaystyle\frac{\sqrt{x}}{2}-1|<1 \Leftrightarrow -1<\frac{\sqrt{x}}{2}-1<1 \Leftrightarrow \ldots x\ldots.\]
	
	Az összeg:
	
	\[\sumn (\frac{\sqrt{x}}{2}-1)^n=\frac{1}{1-(\frac{\sqrt{x}}{2}-1)}=\ldots\]
	
	\bigskip
	\item \textbf{Feladat}: $? x\in \R$ konv. $\displaystyle\sum \frac{x^{2n}}{1+x^{4n}}$
	
	\medskip
	Megoldás: Legyen $\displaystyle|x|\overset{>}{\underset{<}{=}}1$. \quad Ha $|x|<1$, \quad $\Rightarrow x^{4n}=(x^4)^n \longrightarrow 0.$
	
	\[\frac{x^{2n}}{1+x^{4n}}\sim x^{2n}, \quad  \sum x^{2n}\text{ konv.}\]
	
	\medskip
	\textit{Sejtés}: A sor konv ha $|x|<1.$
	
	Bizonyítás: 
	
	\[0\leq \frac{x^{2n}}{1+x^{4n}}\leq x^{2n}\quad\text{ és }\quad\sum x^{2n}\text{ konv.}\]
	
	\medskip Ha $|x|>1:$
	
	Sejtéshez: \[\frac{x^{2n}}{1+x^{4n}}\sim  \frac{x^{2n}}{x^{4n}}=\frac{1}{x^{2n}}=(\frac{1}{x^2})^n\]
	
	$\left(\text{mivel }\displaystyle\frac{1}{x^2}<1 \right)$
	
	\medskip
	\textit{Sejtés}: konv, ha $|x|>1$.
	
	Bizonyítás: \[\frac{x^{2n}}{1+x^{3n}}\leqq\frac{x^{2n}}{x^{4n}}= (\frac{1}{x^2})^n\]
	
	Mivel $|x|>1 \Rightarrow\displaystyle \frac{1}{x^2}<1, \quad \sum (\frac{1}{x^2})^n$ konv., az eredeti is konv.
	
	\[|x|=1; \quad \sum \frac{x^{2n}}{1+x^{4n}}= \sum\frac{1}{1+1}\text{ div.}\]
	
	\bigskip
	\item \textbf{Feladat}: $? x$-re konv.
	\begin{center}
		
	$\displaystyle\sum\frac{(-1)^n}{2n-1}(\frac{1-x}{1+x})^n$
	
	\end{center}
	\medskip (GYK)-mal. [AF.]
	
	Útmutatás: $\left(\text{legyen }\displaystyle\frac{(-1)^n}{2n-1}\left(\frac{1-x}{1+x}\right)^n = a_n\right)$ 
	\begin{center}
		
	$\displaystyle\sqrt[n]{|a_n|}=\left|\frac{1-x}{1+x}\right|\cdot\frac{1}{\sqrt[n]{2n-1}}\longrightarrow \left|\frac{1-x}{1+x}\right|$
	
	\end{center}
	\medskip
	\[\left|\frac{1-x}{1+x}\right|\overset{>}{\underset{<}{=}}1.\]
	
\end{enumerate}
	
\end{document}