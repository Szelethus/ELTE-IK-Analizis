\documentclass[a4paper,11.5pt]{article}

\usepackage[textwidth=135mm]{geometry}

\usepackage[utf8]{inputenc}

\usepackage[T1]{fontenc}

\PassOptionsToPackage{defaults=hu-min}{magyar.ldf}

\usepackage[magyar]{babel}

\usepackage{amsmath,amssymb,paralist,array, marvosym, ellipsis }

\makeatletter
\renewcommand*{\mathellipsis}{%
	\mathinner{%
		\kern\ellipsisbeforegap%
		{\ldotp}\kern\ellipsisgap%
		{\ldotp}\kern\ellipsisgap%
		{\ldotp}\kern\ellipsisaftergap%
	}%
}
\renewcommand*{\dotsb@}{%
	\mathinner{%
		\kern\ellipsisbeforegap%
		{\cdotp}\kern\ellipsisgap%
		{\cdotp}\kern\ellipsisgap%
		{\cdotp}\kern\ellipsisaftergap%
	}%
}
\renewcommand*{\@cdots}{%
	\mathinner{%
		\kern\ellipsisbeforegap%
		{\cdotp}\kern\ellipsisgap%
		{\cdotp}\kern\ellipsisgap%
		{\cdotp}\kern\ellipsisaftergap%
	}%
}
\renewcommand*{\ellipsis@default}{%
	\ellipsis@before
	\kern\ellipsisbeforegap
	.\kern\ellipsisgap
	.\kern\ellipsisgap
	.\kern\ellipsisgap
	\ellipsis@after\relax}
\renewcommand*{\ellipsis@centered}{%
	\ellipsis@before
	\kern\ellipsisbeforegap
	.\kern\ellipsisgap
	.\kern\ellipsisgap
	.\kern\ellipsisaftergap
	\ellipsis@after\relax}
\AtBeginDocument{%
	\DeclareRobustCommand*{\dots}{%
		\ifmmode\@xp\mdots@\else\@xp\textellipsis\fi}}
\def\ellipsisgap{.1em}
\def\ellipsisbeforegap{.05em}
\def\ellipsisaftergap{.05em}
\makeatother

\begin{document}

	%%%%%%%%%%%RÖVIDÍTÉSEK%%%%%%%%%%

	%\setlength\parindent{0pt}

	\def\a{\textbf{a}}

	\def\b{\textbf{b}}

	\def\N{\hskip 10 true mm}

	\def\a{\textbf{a}}

	\def\b{\textbf{b}}

	\def\c{\textbf{c}}

	\def\d{\textbf{d}}

	\def\e{\textbf{e}}

	\def\gg{$\gamma$}

	\def\vi{\textbf{i}}

	\def\jj{\textbf{j}}

	\def\kk{\textbf{k}}

	\def\fh{\overrightarrow}

	\def\l{\lambda}

	\def\m{\mu}

	\def\v{\textbf{v}}

	\def\0{\textbf{0}}

	\def\s{\hspace{0.2mm}\vphantom{\beta}}

	\def\Z{\mathbb{Z}}
	\def\Q{\mathbb{Q}}
	\def\R{\mathbb{R}}
	\def\C{\mathbb{C}}
	\def\N{\mathbb{N}}
	\def\Rn{\mathbb{R}^{n}}
	\def\Ra{\overline{\mathbb{R}}}
	\def\sume{\displaystyle\sum_{n=1}^{+\infty}}
	\def\sumn{\displaystyle\sum_{n=0}^{+\infty}}
	\def\biz{\emph{Bizonyítás:\ }}
	\def\narrow{\underset{n\rightarrow+\infty}{\longrightarrow}}
	\def\limn{\displaystyle\lim_{n\to +\infty}}
	%%%%%%%%%%%%%%%%%%%%%%%%%%%%%%%%%
	
	\begin{center}
		\textbf{Analízis 2. zh tételek}
	\end{center}
	
	\noindent A jegyzetet \textsc{Umann Kristóf} készítette \textsc{Bajári Lúcia}, \textsc{Gecse Viktória} és \\ \textsc{Csonka Szilvia} jegyzete alapján. 
	
	\begin{enumerate}
		\item \textbf{A geometriai sor konvergenciája.}
		
		A $\sum q^n$ sor $(q \in \R)$ konvergens $\Leftrightarrow |q|<1,$ és ekkor \[\sumn q^n=1+q+q^2+ \ldots = \frac{1}{1-q}.\]
		
		\biz 
		\[ s_n=1+q+q^2+ \ldots + q^n \overset{q \not=1}{=} \frac{1-q^{n+1}}{1-q} \]
		\[ [a^{n+1}-b^{n+1}=(a-b)(a^n+a^{n-1}b+\ldots+ab^{n-1}+b^n)] \]
		\[ q=1: \quad s_n=n+1 \]
		\[(s_n)\quad\text{konv}\quad \Leftrightarrow |q|<1\quad \text{ui.} \quad q^n \rightarrow 0 \Rightarrow s_n \underset{n\rightarrow+\infty}{\longrightarrow}\frac{1}{1-q}.\quad \blacksquare\]
		
		\item \textbf{A teleszkópikus sor konvergenciája.}
		
		A $\displaystyle\sum_{n=1} \frac{1}{n(n+1)}$ sor konvergens, és \[ \sume \frac{1}{n(n+1)}=1. \]
		
		\biz \[s_n=\frac{1}{1\cdot2}+\frac{1}{2\cdot3}+\frac{1}{3\cdot4}+\ldots+\frac{1}{n(n+1)}\]
		ÖTLET: $\displaystyle \frac{1}{k(k+1)}=\frac{1}{k}-\frac{1}{k+1}$
		
		\[ \left(\frac{1}{1}-\frac{1}{2}\right)+ \left( \frac{1}{2}-\frac{1}{3} \right)+ \left(\frac{1}{3}-\frac{1}{4}\right)+\ldots+\left(\frac{1}{n}-\frac{1}{n+1}\right)=1-\frac{1}{n+1} \underset{n\rightarrow+\infty}{\longrightarrow} 1\]
		
		\[ \lim(s_n)=1=\sume \frac{1}{n(n+1)}. \quad \blacksquare\]

		\item \textbf{A harmonikus sor divergens.}
		
		$\displaystyle \sum \frac{1}{n}$ (harmonikus) sor divergens.
		
		\biz $n \in \N, k$-t válasszuk meg így: $2^k<n\leq 2^{k+1}$.
		\textsl{}
		TRÜKK: \[\displaystyle s_n=1+\frac{1}{2}+\left( \frac{1}{3}+ \frac{1}{4} \right)+ \left( \frac{1}{5}+ \ldots +\frac{1}{8} \right)+ \left(\frac{1}{9}+\ldots+\frac{1}{16}\right)+ \ldots \] \[ \ldots+\underbrace{\left(\frac{1}{2^i+1}+\frac{1}{2^i+2}+\ldots+\frac{1}{2^i+2^i}\right)}_{\geq\displaystyle\frac{2^i}{2^k+2^i}=\frac{1}{2}}+ \ldots + \frac{1}{n} \quad (i\in\N)\]
		
		\[ \Rightarrow s_n \geq \frac{k}{2} \Rightarrow\quad \text{ha } n\rightarrow+\infty \Rightarrow \quad k \rightarrow+\infty \Rightarrow \quad s_n \rightarrow +\infty \]
		
		$\Rightarrow (s_n)$ divergens.\quad $\blacksquare$
		
		\item \textbf{A végtelen sorokra vonatkozó Cauchy-féle konvergenciakritérium.}
		
		\[ \sum a_n \quad \text{sor konvergens } \Leftrightarrow \left\{\begin{gathered}
		\forall \varepsilon >0,\quad  \exists n_0,  \in \N, \quad\forall m,n \in \N:\quad m> n \geq n_0\\
		|a_{n+1}+a_{n+2}+\ldots+a_m|<\varepsilon.
		\end{gathered} \right.
		\]
		\biz A $\sum a_n$ sor konvergens $\Leftrightarrow (s_n)$ konvergens $\underset{\text{sorozatokra}}{\underset{\text{kritérium}}{\overset{\text{Cauchy-féle}}{\Longleftrightarrow}}}$ $(s_n)$ Cauchy-sorozat.
		\[ \Leftrightarrow \forall\varepsilon>0:\quad \exists n_0\in \N, \quad\forall m>n\geq n_0: \quad |s_m-s_n|= |(a_0+\ldots+a_m)-(a_0+\ldots+a_n)|=\] \[=|a_{n+1}+a_{n+2}+\ldots+a_m|<\varepsilon. \quad\blacksquare \]
		
		\item \textbf{Végtelen sorok konvergenciájának szükséges feltétele.}
		
		Ha $\sum a_n$ sor konvergens $\Rightarrow \lim(a_n)=0$.
		
		\biz $\sum(a_n)$ konvergens $\underset{\text{sorokra}}{\underset{\text{kritérium}}{\overset{\text{Cauchy-féle}}{\Longrightarrow}}} \forall \varepsilon >0: \quad \exists n_0 \in \N, \quad\forall m>n\geq n_0: \quad |a_{n+1}+\ldots+a_m|<\varepsilon.$
		
		Legyen $m=n+1 \Rightarrow |a_{n+1}|<\varepsilon. \quad \forall n\geq n_0 \Rightarrow \lim(a_n)=0. \quad\blacksquare$
		
		\item\textbf{A nemnegatív tagú sorok konvergenciájára vonatkozó tétel.}
		
		$\sum a_n $ nemnegatív tagú sor konvergens $\Leftrightarrow (s_n)$ korlátos sorozat.
		
		\biz $\sum a_n $ sor konvergens $\Leftrightarrow (s_n)$ konvergens.
		
		De: $(s_n)\nearrow,$ ami konvergens $\Leftrightarrow (s_n)$ korlátos. $\blacksquare$
		
		\item \textbf{Végtelen sorokra vonatkozó öszehasonlító kritériumok.}
		
		Tegyük fel, hogy $(a_n), (b_n)$ sorozatokra: \quad\[\exists N\in \N \quad \forall n \in \N; \quad n\geq N: \quad 0\leq a_n \leq b_n.\]
		
		Ekkor: 
		\begin{enumerate}
			\item Majoráns kritérium:
			
			Ha $\sum b_n$ konvergens $\Rightarrow \sum a_n$ is konvergens.
			
			\item Minoráns kritérium
			
			Ha $\sum a_n$ divergens $\Rightarrow \sum b_n$ divergens.
		\end{enumerate}
		
		\biz 
		\begin{enumerate}
			\item $\left.
			\begin{gathered}
				s_n^a:=a_N+a_{N+1}+\ldots +a_n\\
				s_n^b:=b_N+b_{N+1}+\ldots+b_n
			\end{gathered}\right\}n\geq N$
			
			Ha $\sum b_n$ konvergens $\overset{(s_n^b\nearrow)}{\Longrightarrow}(s_n^b)$ korlátos $\Rightarrow (s_n^a)$ is korlátos, $\nearrow\Rightarrow\displaystyle\sum_{n=N}a_n$ konvergens $\Rightarrow \sum a_n$ is konvergens. \quad$\blacksquare$
		\end{enumerate}
		
		\item \textbf{A Cauchy-féle gyökkritérium.}
		
		Tegyük fel, hogy a $\sum a_n$ sorra $\exists\displaystyle\lim_{n\to +\infty} \sqrt[n]{|a_n|}=:A\in \Ra.$
		
		Ekkor:
		\begin{enumerate}
			\item $0 \leq A <1$ esetén a $\sum a_n$ sor abszolút konvergens, tehát konvergens is.
			\item $A>1$ esetén a $\sum a_n$ sor divergens.
			\item $A=1$ esetén a $\sum a_n$ sor lehet konvergens is és divergens is (a kritérium nem használható).
		\end{enumerate}
		
		\biz Tegyük fel, hogy $0\leq A<1$. Ekkor $\exists q:\quad A<q<1$.
		
		\[\displaystyle\lim_{n\to +\infty}\sqrt[n]{|a_n|}=A \Rightarrow q\text{-hoz} \quad  \exists n_0 \in \N: \quad\forall n \geq n_0: \quad\underbrace{\sqrt[n]{|a_n|}}_{\geq 0}<q.\]
		
		\[ \forall n\geq n_0: \quad|a_n|\leq q^n,\quad  \sum_{n=1}q^n \quad\text{konvergens, mert}\quad 0<q<1 \quad \text{(geometriai sor)} \]
		
		$\displaystyle\sum_{n=1} q^n \quad\text{konvergens} \quad\overset{\text{majoráns}}{\underset{\text{kritérium}}{\Longrightarrow}} \sum |a_n| $ konvergens, azaz $\displaystyle\sum a_n$ abszolút konvergens.\\
		
		Tegyük fel, hogy $A>1$. Ekkor $\exists q:\quad 1<q<A$.
		
		\[ \lim(\sqrt[n]{|a_n|})=A \Rightarrow 1<q\text{-hoz}\quad \exists n_0 \in \N: \quad\forall n \geq n_0: \quad\underbrace{\sqrt[n]{|a_n|}}_{\geq 0}>q \]
		
		$\Rightarrow |a_n|>q^n \quad (n\geq n_0) \Rightarrow \lim(|a_n|)=+\infty,$ azaz $(a_n)$ nem 0-sorozat. $\overset{\text{szükséges}}{\underset{\text{feltétel}}{\Longrightarrow}} \sum a_n$ divergens.\\
		
		Tegyük fel, A=1. 
		
		$\displaystyle\sum \frac{1}{n} $ harmonikus sor divergens, de $\displaystyle\lim_{n\to+\infty}\sqrt[n]{\frac{1}{n}}=\lim_{n\to+\infty}\frac{1}{\sqrt[n]{n}}=1.$
		
		$\displaystyle\sum\frac{1}{n^2}$ konvergens és $\displaystyle\lim_{n\to+\infty}\frac{1}{\sqrt[n]{n^2}}=\lim_{n\to+\infty}\left(\frac{1}{\sqrt[n]{n}}\right)^2=1.\quad\blacksquare$
		\pagebreak
		\item \textbf{A D'Alembert-féle hányados-kritérium.} 
		%https://scontent-fra3-1.xx.fbcdn.net/v/t34.0-12/13149945_10207526109769892_1787002454_n.png?oh=3c2735d49ab18e9c4c8c294d4817152c&oe=5730BD87
		
		Tegyük fel, hogy a $\sum a_n$ sorra $a_n\not=0~~(n\in \N)$:
		\[ \exists\lim_{n\to +\infty}\frac{|a_{n+1}|}{|a_n|} =: A\in \Ra .\]
		Ekkor:
		\begin{enumerate}
			\item $0 \leq A <1 \Rightarrow \sum a_n$ sor abszolút konvergens, tehát konvergens is.
			\item $A>1 \Rightarrow \sum a_n$ divergens.
			\item $A=1 \Rightarrow \sum a_n$ lehet konvergens és divergens is.
		\end{enumerate}
		
		\biz Tegyük fel, hogy $0\leq A < 1$. Ekkor $\exists q: A<q<1$, és
		
		\[\limn \frac{|a_{n+1}|}{|a_n|}=A \Rightarrow q\text{-hoz}\quad\exists n_0 \in \N, \quad \forall n\geq n_0: \quad \frac{|a_{n+1}|}{|a_n|}<q.\]
		
		Legyen $n>n_0$:\quad\[|a_{n+1}|<q\cdot|a_n|\overset{\frac{|a_{n}|}{|a_{n-1}|}<q}{<} q^2|a_{n-1}|< \ldots < q^{n+1-n_0}|a_{n_0}|=\]\[ =\overbrace{|a_{n_0}|\cdot q^{1-n_0}}^{=:c}q^n=c\cdot q^n \Rightarrow |a_{n+1}|< c\cdot q^n \quad(\forall n \geq n_0).\]
		
		Mivel: $0<q<1, \displaystyle\sum_{n=n_0}q^n$ konvergens $\overset{\text{majoráns}}{\underset{\text{kritérium}}{\Longrightarrow}}\sum |a_n|$ konvergens, azaz $\sum a_n$ abszolút konvergens.\\
		
		Tegyük fel, hogy A>1. Ekkor $\exists q: 1<q<A$, és
		
		\[\limn \frac{|a_{n+1}|}{|a_n|}=A \Rightarrow q\text{-hoz}\quad\exists n_0 \in \N, \quad \forall n\geq n_0: \quad \frac{|a_{n+1}|}{|a_n|}>q.\]
		\[n>n_0:\quad|a_{n+1}|>q\cdot|a_n|> q^2|a_{n-1}|> \ldots > q^{n+1-n_0}|a_{n_0}|\]
		
		$ \overset{q>1}{\Longrightarrow}\lim(|a_{n+1}|)=+\infty,$ azaz $(a_n)$ nem 0-sorozat $\Rightarrow \sum a_n$ divergens.\\
		
		Tegyük fel, hogy $A=1$.
		
		\[ \sum\frac{1}{n}\quad \text{divergens, de }\quad \lim\left(\frac{\frac{1}{n+1}}{\frac{1}{n}}\right)=\lim\left(\frac{n}{n+1}\right)=\lim\left(\frac{1}{1+\frac{1}{n}}\right)=1 \]
		
		\[ \sum\frac{1}{n^2}\quad \text{konvergens, de }\quad \limn\left(\frac{\frac{1}{n+1}}{\frac{1}{n}}\right)^2=\limn\left(\frac{1}{1+\frac{1}{n}}\right)^2=1. \quad\blacksquare \]
		\pagebreak
		\item \textbf{Leibniz-típusú sorok konvergenciája.}
		
		Tegyük fel, hogy $ \forall n \in \N : 0 \leq a_{n+1}\leq a_{n}$. Ekkor $\displaystyle\sum_{n=1} (-1)^{n+1} a_n$ Leibniz-típusú sor, és
		
		\begin{enumerate}
			\item Konvergencia: $\displaystyle\sum_{n=1} (-1)^{n+1} a_n$ konvergens $\Leftrightarrow \lim(a_n)=0$.
			
			\smallskip
			\item Hibabecslés: tegyük fel, hogy $\displaystyle\sum_{n=1} (-1)^{n+1} a_n$ konvergens és \\ $A:=\displaystyle \sum_{n=1}^{+\infty} (-1)^{n+1} a_n$.\quad 	Ekkor:
			\[|A-s_n|=\left|A-\displaystyle \sum_{k=1}^{n} (-1)^{k+1} a_k\right| \leq a_n~~ (\forall n \in \N).\]
		\end{enumerate}
		
		\biz 
		\begin{enumerate}
			\item (konvergencia)
			%\begin{enumerate}
			
		 \fbox{$\Rightarrow:$}
			\[ \sum (-1)^{n+1}a_n\quad \text{ konvergens } \quad\overset{\text{szükséges}}{\underset{\text{feltétel}}{\Longrightarrow}}\quad\lim\left((-1)^{n+1}a_n\right)=0 \Rightarrow \lim(a_n)=0. \]
			\fbox{$\Leftarrow:$}
			
			Igazolnunk kell: $\displaystyle\sum_{n=1}(-1)^{n+1}a_n=a_1-a_2+a_3-\ldots$ konvergens.
			\[ s_n=\sum_{k=1}^n(-1)^{k+1}a_k=a_1-a_2+a_3-\ldots\pm a_n \]
			
			Igazoljuk 
			\begin{enumerate}
				\item $(s_{2n+1})\searrow$
				\[ s_1=a_1\geq s_1-\underbrace{(a_2-a_3)}_{\geq 0}=a_1-a_2+a_3=s_3 \]
				\[ \geq s_3-(a_4-a_5)=a_1-a_2+a_3-a_4+a_5=s_5\geq s_5\geq s_7 \geq\ldots\geq s_{2n+1} \]
				\item $(s_{2n})\nearrow$
				\[ s_2=a_1-a_2\leq s_2+\underbrace{(a_3-a_4)}_{\geq 0}=a_1-a_2+a_3-a_4=s_4\leq s_6\leq\ldots\leq s_{2n} \]
			\end{enumerate}
			$(s_{2n})$ és $(s_{2n+1})$ korlátosak is, ui.: \[s_2\leq s_{2n}=s_{2n-1}-a_{2n}\leq s_{2n-1} \leq s_1 \overset{\text{monoton}}{\underset{\text{korlátos}}{\Longrightarrow}} \quad\text{konvergens:}\quad
			\left\{\begin{gathered}
				\exists \alpha:=\lim(s_{2n})\\				
				\exists \beta:=\lim(s_{2n+1})
			\end{gathered}\right. \]
			\[ \begin{matrix}
				s_{2n}&=&s_{2n-1}&-&a_{2n}& (n\in \N)\\
				\downarrow&&\downarrow&&\downarrow&\\
				\alpha&&\beta&&0&
			\end{matrix} \]
			\[\Rightarrow \alpha=\beta=\lim(s_n) \Rightarrow\sume(-1)^{n+1}a_n \quad\text{konvergens.} \]
		%\end{enumerate}
		\item (hibabecslés)
		\[ s_{2n}\leq\alpha= A\leq s_{2n+1} \]
		\[ |s_{2n}-A|\leq s_{2n+1}-s_{2n}=a_{2n+1}\leq a_{2n} \]
		\[ |s_{2n+1}-A|\leq s_{2n+1}-s_{2n}=a_{2n+1} \]
		$\Rightarrow \forall n\in \N: \quad |A-s_n|\leq a_n. \quad \blacksquare$
		\end{enumerate}
		
		\item \textbf{Számok tizedestört alakban való előállítása.}
		
		Ha $\alpha\in[0;1]$, akkor $\displaystyle\exists (a_n):\quad \N^+\to\{0,1,2,\ldots,9\}: \quad \alpha=\sume \frac{a_n}{10^n}$
		
		\biz
		\begin{enumerate}[1. lépés:]
			\item $[0;1]$-at 10 egyenlő részre osztjuk
			
			\[\Rightarrow\quad \exists a_1\in\{0,1,2,\ldots,9\}:\quad \alpha \in I_1=\left[\frac{a_1}{10};\frac{a_1+1}{10}\right]\]
			
			\item $I_1$-et 10 egyenlő részre osztjuk
			\[\Rightarrow\quad\exists a_2\in\{0,1,2,\ldots,9\}:\quad \alpha \in I_2=\left[\frac{a_1}{10}+\frac{a_2}{10^2};\frac{a_1}{10}+\frac{a_2+1}{10^2}\right]\]
			\begin{center}
				$\vdots$
			\end{center}
			
			\item[$n$. lépés:] Felosztjuk $I_{n-1}$-et 10 egyenlő részre \quad $\Rightarrow \quad \exists a_n\in\{0,1,\ldots,9\}.$
			
			\[\alpha\in I_n=\left[\frac{a_1}{10}+\frac{a_2}{10^2}+\ldots+\frac{a_n}{10^n};\frac{a_1}{10}+\frac{a_2}{10^2}+\ldots+\frac{a_{n}+1}{10^n}\right],\]
			azaz
			\[\underbrace{\frac{a_1}{10}+\frac{a_2}{10^2}+\ldots+\frac{a_n}{10^n}}_{s_n}\leq \alpha \leq \underbrace{\frac{a_1}{10}+\frac{a_2}{10^2}+\ldots+\frac{a_n}{10^n}}_{s_n}+\frac{1}{10^n}\]
			\[s_n\leq\alpha\leq s_n+\frac{1}{10^n}\quad \forall n=1,2,\ldots\]
			\[\Rightarrow|\alpha-s_n|\leq\frac{1}{10^n}\to0\quad \Rightarrow\quad \lim(s_n)=\alpha=\sume \frac{a_n}{10^n}\quad \blacksquare\]
		\end{enumerate}
		
		\item\textbf{Abszolút konvergens sorok átrendezése.}
		
		Ha a $\sum a_n$ sor abszolút konvegens, és $(p_n): \N \to \N$ tetszőleges bijekció, akkor a $\sum a_{p_n}$ abszolút konvergens, és $\sumn a_n=\sumn a_{p_n}$
		
		\biz Legyen $(p_n):\N\to\N$ tetszőleges permutáció.
		\[s_n=\sum_{k=0}^na_k,\quad\sigma_n:=\sum_{k=0}^na_{p_k}\]
		\begin{enumerate}
			\item Igazoljuk:\quad  \textit{a} $\displaystyle\sum a_{p_n}$ \textit{sor abszolút konvergens (tehát konvergens is)}
			
			A $\displaystyle\left(\sum_{k=0}^n|a_{p_k}|, n\in \N\right)$ sorozat $\nearrow$ és felülről korlátos, mert 
			\[\sum_{k=0}^n|a_{p_k}|=|a_{p_0}|+\ldots+|a_{p_n}|\leq\sum_{k=0}^{+\infty}|a_k|=K\overset{\sum a_k \text{ abszolút}}{\underset{\text{konvergens}}{<}}+\infty\quad (\forall n \in \N)\]
			$\Rightarrow\displaystyle\sum|a_{p_k}|$ abszolút konvergens, azaz $\displaystyle\sum a_{p_k}$ abszolút konvergens.
			
			\item Igazoljuk$:\quad  \displaystyle\sumn a_n=\sumn a_{p_k}$.
			
			Legyen $A=\sumn a_n$, azaz $s_n\to A$.
			
			Legyen $\varepsilon>0$ tetszőlegesen rögzített szám. Mivel $\displaystyle\sum|a_n|$ konvergens $\overset{\text{Cauchy}}{\underset{\text{kritérium}}{\Longrightarrow}}$
			\[\varepsilon>0\text{-hoz}\quad \exists N\in \N,\quad \forall m\geq N:\quad |a_N|+|a_{N+1}|+\ldots+|a_m|<\varepsilon.\]
		\end{enumerate}
		Tekintsük a $(a_n)$ sorozat első $N+1$ tagját.
		
		Ekkor \[\exists N_1\in\N\quad \forall n\geq N_1\text{-re}\quad \sigma_n-s_n=\underbrace{(a_{p_0}+\ldots+a_{p_n})-(a_0+a_1+\ldots+a_n)}_{a_0;a_1;\ldots;a_N\text{-ek kiesnek, ha $N_1$ elég nagy}}=\]
		\[=\sum_{k>N}^n\pm a_k \quad \Rightarrow\quad |\sigma_n-s_n|\leq\sum_{k>N}^n|a_k|<\varepsilon\quad \forall n\geq N_1\quad \Rightarrow\]
		\[\sigma_n - s_n\narrow 0\]
		De:\[ \sigma_n=\sigma_n-s_n+s_n\narrow0+A\Rightarrow \sigma_n\narrow A,\quad  \text{azaz}\quad \sumn a_{p_n}=A\quad \blacksquare \]
		
		\item \textbf{Abszolút konvergens sorok szorzására vonatkozó Cauchy-tétel.}
		
		Ha a $\sum a_n$ és $\sum b_n$ sorok mindegyike abszolút konvergens, akkor 
		\begin{enumerate}
			\item a téglénysorozat $(\sum t_n)$ is abszolút konvergens,
			\item a $\sum c_n$ Cauchy-szorzat is abszolút konvergens,
			\item az összes $a_ib_j \quad (i,j=0,1,\ldots)$ szorzatokból tetszés szerinti sorrendben és csoportosítással képzett $\displaystyle\sum_{n=0} d_n$ sor is abszolút konvergens és $\sumn d_n=\sumn t_n=\sumn c_n=\left(\sumn a_n\right)\cdot\left(\sumn b_n\right).$
		\end{enumerate}
		
		\biz \textit{c)} 
		\[ A_N:=\sum_{n=0}^N|a_n|\rightarrow A;\quad B_N:=\sum_{n=0}^N|b_n|\rightarrow B \]
		Tekintsük a $\sum d_n$ sort, ahol $\sum a_ib_j$. 
		
		Legyen $I$ a maximális $i$ index $d_0,\ldots d_N$-ben, és $J$ a maximális $j$ index $d_0,\ldots d_N$-ben.
		\[ \sum_{n=0}^N|d_n|\leq \left(\sum_{n=0}^I|a_n|\right)\left(\sum_{n=0}^J|b_n|\right)\leq A\cdot B \quad(\forall N\in \N) \]
		
		$\Rightarrow \sum|d_n| \text{\quad konvergens, azaz}\quad \sum d_n \quad\text{abszolút konvergens.}$
		
		Tehát $\sum t_n; \quad\sum c_n$ \quad is abszolút konvergens.
		
		Azonban: $\sumn t_n=\left(\sumn a_n\right)\left(\sumn b_n\right)$
		
		Viszont $\sum t_n$ \quad abszolút konvergens\quad$\Rightarrow$\quad tetszőleges módon átrendezhető és csoportosítható az összes megváltoztatása nélkül.
		
		$\sum d_n, \sum c_n$ is megkapható $\sum t_n$-ből alkalmas átrendezéssel, csoportosítással. \quad$\blacksquare$
		
		\item \textbf{Hatványsorok konvergenciahalmazára vonatkozó, a konvergencia sugarát meghatározó tétel.}
		
		Tetszőleges $\sumn \alpha_n(x-a)^n \quad (x\in\R)$ h.s. konvergencia halmazára (KH) a következő három eset egyike teljesül:
		\begin{enumerate}
			\item $\exists! 0<R<+\infty: $ \quad a h.s. $\left\{\begin{gathered}
				\forall x:\quad |x-a|<R\text{\quad abszolút konvergens}\\
				\forall x:\quad |x-a|>R\text{\quad divergens}
			\end{gathered}\right.$
			\item a h.s. csak az $x=a$-ban konvergens $(R:=0)$
			\item a h.s. $\forall x \in \R$ esetén konvergens. $(R:=+\infty)$
		\end{enumerate}
		($R: $ a h.s. konvergencia sugara)
		
		\biz Feltehető $a=0$, azaz $\sumn \alpha_nx^n \quad (x\in\R)$, mert ha $a\not=0:$
		
		$ \Rightarrow y:=x-a$-val \quad $\sumn \alpha_ny^n$.
		
		\begin{itemize}[~~~~~~~~~]
			\item \textbf{Segédtétel:} Tegyük fel, hogy $\sum\alpha_nx^n$ h.s. konvergens egy $x_0\not=0$ pontban. Ekkor $\forall |x|<|x_0|$ esetén $\sum\alpha_nx^n$ abszolút konvergens.
			
			\biz $\sum\alpha_nx_0^n$ konvergens $\overset{\text{a konvergencia}}{\underset{\text{feltétel}}{\underset{\text{szükséges}}{\Longrightarrow}}}\lim(\alpha_nx^n_0)=0$ 
			
			$\Rightarrow(\alpha_nx^n_0)$ korlátos, azaz $\exists M>0:\quad |\alpha_nx^n_0|\leq M<+\infty \quad (\forall n\in\N)$
			
			Legyen $|x|<|x_0|$.
			
			\[ \left|\alpha_nx^n\right|=\left|\alpha_nx^n_0\right|\cdot\left|\frac{x}{x_0}\right|^n\leq M\cdot\left|\frac{x}{x_0}\right|^n \quad (\forall n \in \N) \]
			
			Mivel $|x|<|x_0|\Rightarrow\left|\displaystyle\frac{x}{x_0}\right|<1$ és $\displaystyle\sum M\cdot \left|\frac{x}{x_0}\right|^n$ geometriai sor konvergens $\overset{\text{majoráns}}{\underset{\text{kritérium}}{\Longrightarrow}}\displaystyle\sum|\alpha_nx^n|$ konvergens, azaz 
			
			\[ \sum\alpha_nx^n\quad \text{abszolút konvergens.}\quad \blacksquare \]
		\end{itemize}
		Tekintsük a $\displaystyle\sum\alpha_nx^n$ h.s.-t. $ \displaystyle\text{Ez}\quad  x=0\text{~-ban konvergens }  \quad \Rightarrow 0 \in KH(\sum\alpha_nx^n) \Rightarrow \exists \sup KH(\sum\alpha_nx^n):=R\in\Ra,\quad\text{ sőt, }\quad R>0.$
		
		A következő esetek lehetnek:
		\begin{enumerate}
			\item $0<R<+\infty:$
			\begin{itemize}
				\item Legyen $|x|<R \overset{\text{sup.}}{\underset{\text{defíciója}}{\Longrightarrow}} \exists x_0: |x|<x_0\leq R$\quad  és\quad $ \sum\alpha_nx^n_0$ \quad konvergens $\overset{\text{Segéd-}}{\underset{\text{tétel}}{\Longrightarrow}} \sum\alpha_nx^n$ abszolút konvergens.\\
				
				\item Legyen $|x|>R \Rightarrow \exists x_0:\quad  R<x_0<|x|:\quad \sum\alpha_nx^n_0$\quad divergens\quad $\overset{\text{Segéd-}}{\underset{\text{tétel}}{\Longrightarrow}} \sum \alpha_nx^n$ \quad divergens.\\
				
				(ui.: ha $\sum \alpha_nx^n$ konvergens lenne $\overset{\text{Segéd-}}{\underset{\text{tétel}}{\Longrightarrow}} \sum \alpha_n x_0^n$ konvergens {\Large\Lightning})
				
				\item Egyetlen ilyen tulajdonságú $R$ létezik (indirekt)
			\end{itemize}
			\item $R=0:$
			
			$ \displaystyle\sum\alpha_nx^n$ \quad konvergens $x=0$-ban. De $\forall |x|>0$ helyen divergens, ui.: \quad $|x|>0$ rögzített \quad $\Rightarrow\quad \exists x_0: \quad 0<x_0<|x| \quad$ és\quad $\displaystyle\sum\alpha_nx^n_0$\quad divergens $\Rightarrow \alpha_nx^n$ is divergens.
			
			\item $R=+\infty$:
			
			Ekkor $\sum\alpha_nx^n$ sor $\forall x \in \R$ esetén konvergens. ui.:
			\[ x\in \R \quad \text{tetszőleges}\quad \Rightarrow \exists x_0: \quad |x|<|x_0|\quad \text{és}\quad \sum\alpha_nx^n_0 \quad \text{konvergens}\Rightarrow\]\[ \Rightarrow\sum\alpha_nx^n\quad \text{abszolút konvergens.} \quad \blacksquare\]
		\end{enumerate}
		
		\item \textbf{Cauchy-Hadamard tétel.}
		
		Tekintsük a $\sumn \alpha_n(x-a)^n$ hatványsort, és tegyük fel hogy
		
		$\displaystyle\exists\lim_{n\to+\infty}\left(\sqrt[n]{|\alpha_n|}\right)=:A\in\Ra$.
		 
		 Legyen:
		 
		 \[ R:=\left\{
		 \begin{gathered}
			 \frac{1}{A}, \quad \text{ha}\quad 0<A<+\infty\\
			 0, \quad \text{ha}\quad A=+\infty\\
			 +\infty \quad \text{ha}\quad A=0
		 \end{gathered}\right., \]
		 a h.s. konvergencia sugara.
		 
		 Ekkor:\begin{enumerate}
			\item Ha $ 0<R<+\infty:\quad$ a h.s.
			$\left\{\begin{gathered}
				\quad\forall x: \quad |x-a|<R \quad \text{abszolút konvergens,}\\
				\quad\forall x: \quad |x-a|>R \quad \text{divergens,}
			\end{gathered}\right. $
			\item Ha $R:=0$, a h.s. csak az $x=a$-ban konvergens,
			\item Ha $R:=+\infty$, a h.s. $\forall x \in \R$\quad esetén konvervens.
		\end{enumerate}
		
		\biz Alkalmazzuk a $\sum\alpha_n(x-a)^n$ sorra a gyökkritériumot:
		\begin{enumerate}
			\item $0<A<+\infty$:
			\[\lim_{n\to+\infty}\sqrt[n]{|\alpha_n(x-a)^n|}=|x-a|\cdot\lim\sqrt[n]{|\alpha_n|}
			\left\{\begin{gathered}
				<1\quad \text{abszolút konvergens}\\
				>1\quad \text{divergens}
			\end{gathered}\right.\]
			\item $A=+\infty$:
			
			$ x=a$-ban konvergens. $x\not=a \Rightarrow |x-a|\cdot A=+\infty>1 \Rightarrow x$-ben a sor divergens.
			\item $A=0:$
			
			 $ \forall x\in\R:\quad |x-a|\cdot A=0<1$ abszolút konvergens.\quad $\blacksquare$
			
		\end{enumerate}
		
		\item \textbf{Függvények határértékének egyértelműsége.}
		
		A határérték egyértelmű.
		
		\biz: Tegyük fel, hogy $\displaystyle\lim_af=A\in\Ra, \quad \lim_af=B\in \Ra$ \quad és\quad  $A\not=B$
		
		\[\Rightarrow\left\{\begin{gathered}
			\forall \varepsilon>0, \quad \exists\sigma_1>0; \quad \forall x \in K_{\sigma_1}(a)\backslash\{a\}\cap \mathcal{D}_f:\quad f(x)\in K_{\varepsilon}(A)\\
			\forall \varepsilon>0, \quad \exists\sigma_1>0; \quad \forall x \in K_{\sigma_2}(a)\backslash\{a\}\cap \mathcal{D}_f:\quad f(x)\in K_{\varepsilon}(B)
		\end{gathered}\right.
		 \]
		$\displaystyle A\not=B\quad \Rightarrow\quad \exists\varepsilon>0:\quad K_\varepsilon(A)\cap K_\varepsilon(B)=\emptyset. \left(\text{pl. }A,B \in \R \quad \varepsilon<\displaystyle\frac{|A-B|}{2}\right)$
		
		Tekintsünk egy ilyen $\varepsilon$-t és legyen $\sigma:=\min\{\sigma_1,\sigma_2\}.$
		
		$\Rightarrow \forall x \in K_\sigma(a)\backslash\{a\}\cap\mathcal{D}_f:\quad f(x)\in K_\varepsilon(A)\cap K_\varepsilon(B)=\emptyset$\quad  {\LARGE\Lightning}\quad $\blacksquare$
		
		\item \textbf{A határértékre vonatkozó átviteli elv.}
		\[f\in\R\rightarrow\R, \quad a\in\mathcal{D}_f'.\text{ Ekkor }\lim_af=A\in\Ra\quad \Leftrightarrow^{(*)}\quad \forall(x_n):\quad \N\to\mathcal{D}_f\backslash\{a\},\]
		amire
		\[\lim_{n\to+\infty}(x_n)=a;\quad \lim_{n\to+\infty}f(x_n)=A\]
		
		\biz $\Rightarrow:$
		
		Tegyük fel, hogy $\displaystyle\lim_af=A\in\Ra\quad \Rightarrow$
		\[ \forall \varepsilon>0,\quad \exists \sigma>0,\quad \forall x \in \mathcal{D}_f\cap(K_\sigma(a)\backslash\{a\}:\quad f(x)\in K_\varepsilon(A) \]
		Legyen $\displaystyle(x_n):\quad \N\to\mathcal{D}_f\backslash\{a\},\quad  \lim_{n\to+\infty}x_n=a$ \quad tetszőleges sorozat.
		\[\varepsilon>0 \quad \text{tetszőleges,}\quad \sigma>0:\quad \exists n_0\in\N,\quad \forall n\geq n_0:\quad x_n\in K_\sigma(a),\]
		azaz \[\forall\varepsilon>0,\quad  \exists n_0\in\N,\quad \forall n\geq n_0: f(x_n)\in K_\varepsilon(A)\quad \Rightarrow\quad \lim_{n\to+\infty}f(x_n)=A.\]
		$\Leftarrow:$ (Indirekt)
		
		Tegyük fel, hogy (*) (jobb oldal) teljesül, de $\displaystyle\lim_af\not=A$.
		\[ \exists \varepsilon>0,\quad \forall \sigma>0,\quad \exists x_\sigma \in \mathcal{D}_f\cap(K_\sigma(a)\backslash\{a\}):\quad f(x)\not\in K_\varepsilon(A) \]
		Legyen
		\[\sigma=\frac{1}{n}\quad (n=1,2,\ldots):\quad \exists x_n\in\mathcal{D}_f\cap(K_{\frac{1}{n}}(a)\backslash\{a\}):\quad f(x_n)\not\in K_\varepsilon(A)\]
		\[\Rightarrow\quad x_n\in K_{\frac{1}{n}}(a)\quad (n=1,2,\ldots)\quad \Rightarrow\quad x_n\underset{n\to+\infty}{\longrightarrow} a\]
		\[f(x_n)\not\in K_\varepsilon(A)\quad \Rightarrow\quad \lim_{n\to+\infty}f(x_n)\not=A  \quad \text{{\LARGE\Lightning}}\quad \blacksquare\]
		
	\end{enumerate}
	
	Külön köszönet még nekik: \textsc{Qian Lívia}, \textsc{Pintér Arianna}, \textsc{Hoang László}, \textsc{Kovács Bence}, \textsc{Foltán Dániel}, \textsc{Majer Firderika}, \textsc{Veress Marcell}, \textsc{Benics Balázs} és mégegyszer \textsc{Csonka Szilvia}, \textsc{Gecse Viktória}, \textsc{Bajári Lúcia} a jegyzet javításáért.
\end{document}