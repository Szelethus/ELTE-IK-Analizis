\documentclass[a4paper,12pt]{article}

\usepackage[top=30mm]{geometry}

\usepackage[utf8]{inputenc}

\usepackage[T1]{fontenc}

\PassOptionsToPackage{defaults=hu-min}{magyar.ldf}

\usepackage[magyar]{babel}

\usepackage{amsmath,amssymb,paralist,floatflt}

\begin{document}
	\pagestyle{empty}
	\def\Z{\mathbb{Z}}
	\def\Q{\mathbb{Q}}
	\def\R{\mathbb{R}}
	\def\Ra{\overline{\mathbb{R}}}
	\def\C{\mathbb{C}}
	\def\N{\mathbb{N}}
	\def\a{\textbf{Állítás: }}
	\def\t{\textbf{Tétel: }}
	\def\k{\textbf{Következmény: }}
	\def\d{\textbf{Definíció: }}
	\def\m{\emph{Megjegyzés: }}
	\def\p{\textsl{Példa: }}
	\def\b{\emph{Bizonyítás: }}
	
	\begin{center}
		\textbf{9. heti analízis $+/-$ra}
	\end{center}
	
	\emph{A továbbiakban legyen $s_n$ a sor $n$-edik részletösszege.}
	
	\bigskip
	\begin{compactenum}
		
		\item Mit jelent az, hogy a $\sum a_n$ végtelen sor \emph{konvergens}, és hogyan értelmezzük az \emph{összegét}?
		
		\emph{Válasz: } A $\sum a_n$ sor konvergens, ha az $(s_n)$ sorozat konvergens (véges a hatérértéke). Ekkor a $\lim(s_n)$ számot a $\sum a_n$ végtelen sor összegének nevezzük és így jelöljük: \[\sum_{n=0}^{+\infty} a_n:=\lim(s_n). \]
		
		\bigskip 
		\item Milyen tételt ismer $q \in \R$ esetén a $\sum_{n=0} q^n $ geometriai sor konvergenciájáról?
		
		\emph{Válasz: } A $\displaystyle \sum q^n$ sor $(q \in \R) $ konvergens $\Leftrightarrow |q| <1$ és ekkor 
		\[\sum_{n=0}^{+\infty} q^n=1+q+q^2+\ldots = \frac{1}{1-q}. \]
		
		\bigskip 
		\item Mi a \emph{teleszkopikus sor} és mi az összege?
		
		\emph{Válasz: } A $\sum_{n=1} \frac{1}{n(n+1)}$ sor konvergens, és
		\[ \sum_{n=1}^{+\infty}\frac{1}{n(n+1)}=1 \]
		
		\bigskip 
		\item Milyen állítást ismer a $\sum\frac{1}{n^\alpha}$ hiperharmonikus sor konvergenciájával kapcsolatban?
		
		\emph{Válasz: } 
		\begin{center}
			$\displaystyle \sum_{n=1}\frac{1}{n^\alpha} \left\{\begin{gathered}
			\text{konvergens, ha } \alpha>1 \\
			\text{divergens, ha } \alpha\leq 1
			\end{gathered}\right.$
		\end{center}
		
		\bigskip 
		\item Mondjon \emph{szükséges} feltételt arra nézve, hogy a $\sum a_n$ végtelen sor konvergens legyen.
		
		\emph{Válasz: } Ha $\sum a_n$ sor konvergens $\Rightarrow \lim(a_n)=0$.
		
		\bigskip 
		\item Fogalmazza meg a végtelen sorokra vonatkozó \emph{összehasonlító kritériumokat}.
		
		\emph{Válasz: } Tegyük fel, hogy $(a_n), (b_n)$ sorozatokra:
		\[\exists N \in \N, \forall n \in \N, n\ge N : 0\leq a_n \leq b_n.\]
		
		Ekkor:
		\begin{compactenum}
			\item Majoráns kritérium:
			
			Ha $\sum b_n$ konvergens $\Rightarrow \sum a_n$ is konvergens
			\item Minoráns kritérium:
			
			Ha $\sum a_n$ divergens $\Rightarrow \sum b_n$ divergens.
		\end{compactenum}
		
		\bigskip 
		\item Fogalmazza meg a végtelen sorokra vonatkozó \emph{Cauchy-féle gyökkritériumot}.
		
		\emph{Válasz:} Tegyük fel, hogy a $\sum a_n$ sorra $\exists\displaystyle\lim_{n\to +\infty} \sqrt[n]{|a_n|}=:A\in \Ra.$
		
		Ekkor:
		\begin{compactitem}
			\item $0 \leq A <1$ esetén a $\sum a_n$ sor abszolút konvergens, tehát konvergens is.
			\item $A>1$ esetén a $\sum a_n$ sor divergens.
			\item $A=1$ esetén a $\sum a_n$ sor lehet konvergens is és divergens is (a kritérium nem használható).
		\end{compactitem}
		
		\bigskip 
		\item Fogalmazza meg a végtelen sorokra vonatkozó \emph{D’Alembert-féle hányados-kritériumot}.
		
		\emph{Válasz: } Tegyük fel, hogy a $\sum a_n$ sorra $a_n\not=0~~(n\in \N)$:
		\[ \exists\lim_{n\to +\infty}\frac{|a_{n+1}|}{|a_n|} =: A\in \Ra .\]
		Ekkor:
		\begin{compactitem}
			\item $0 \leq A <1 \Rightarrow \sum a_n$ sor abszolút konvergens, tehát konvergens is.
			\item $A>1 \Rightarrow \sum a_n$ divergens.
			\item $A=1 \Rightarrow \sum a_n$ lehet konvergens és divergens is.
		\end{compactitem}
		
		\bigskip 
		\item Mik a Leibniz-típusú sorok és milyen konvergenciatételt ismer ezekkel kapcsolatban?
		
		\emph{Válasz:} Tegyük fel, hogy $ \forall n \in \N : 0 \leq a_{n+1}\leq a_{n}$. Ekkor $\sum_{n=1} (-1)^{n+1} a_n$ Leibniz-típusú sor, és
		
		\begin{compactenum}
			\item Konvergencia: $\sum_{n=1} (-1)^{n+1} a_n$ konvergens $\Leftrightarrow \lim(a_n)=0$.
			
			\smallskip
			\item Hibabecslés: tegyük fel, hogy $\sum_{n=1} (-1)^{n+1} a_n$ konvergens és \\ $A:=\displaystyle \sum_{n=1}^{+\infty} (-1)^{n+1} a_n$.
			
			Ekkor:
			\[|A-s_n|=|A-\displaystyle \sum_{k=1}^{n} (-1)^{k+1} a_k| \leq a_n~~ (\forall n \in \N).\]
		\end{compactenum}
	\end{compactenum}
	
	
\end{document}