
\documentclass[a4paper,11.5pt]{article}
\usepackage[textwidth=170mm, textheight=230mm, inner=20mm, top=20mm, bottom=30mm]{geometry}
\usepackage[normalem]{ulem}
\usepackage[utf8]{inputenc}
\usepackage[T1]{fontenc}
\PassOptionsToPackage{defaults=hu-min}{magyar.ldf}
\usepackage{pgfplots}
\pgfplotsset{compat=1.10}
\usepgfplotslibrary{fillbetween}
\usepackage[magyar]{babel}
\usepackage{amsmath, amsthm,amssymb,paralist,array, ellipsis, graphicx, float, bigints,tikz}
%\usepackage{marvosym}

\makeatletter
\renewcommand*{\mathellipsis}{%
	\mathinner{%
		\kern\ellipsisbeforegap%
		{\ldotp}\kern\ellipsisgap
		{\ldotp}\kern\ellipsisgap%
		{\ldotp}\kern\ellipsisaftergap%
	}%
}
\renewcommand*{\dotsb@}{%
	\mathinner{%
		\kern\ellipsisbeforegap%
		{\cdotp}\kern\ellipsisgap%
		{\cdotp}\kern\ellipsisgap%
		{\cdotp}\kern\ellipsisaftergap%
	}%
}
\renewcommand*{\@cdots}{%
	\mathinner{%
		\kern\ellipsisbeforegap%
		{\cdotp}\kern\ellipsisgap%
		{\cdotp}\kern\ellipsisgap%
		{\cdotp}\kern\ellipsisaftergap%
	}%
}
\renewcommand*{\ellipsis@default}{%
	\ellipsis@before
	\kern\ellipsisbeforegap
	.\kern\ellipsisgap
	.\kern\ellipsisgap
	.\kern\ellipsisgap
	\ellipsis@after\relax}
\renewcommand*{\ellipsis@centered}{%
	\ellipsis@before
	\kern\ellipsisbeforegap
	.\kern\ellipsisgap
	.\kern\ellipsisgap
	.\kern\ellipsisaftergap
	\ellipsis@after\relax}
\AtBeginDocument{%
	\DeclareRobustCommand*{\dots}{%
		\ifmmode\@xp\mdots@\else\@xp\textellipsis\fi}}
\def\ellipsisgap{.1em}
\def\ellipsisbeforegap{.05em}
\def\ellipsisaftergap{.05em}
\makeatother

\usepackage{hyperref}
\hypersetup{
	colorlinks = true	
}

\DeclareMathOperator{\Int}{int}
\DeclareMathOperator{\tg}{tg}
\DeclareMathOperator{\ctg}{ctg}
\DeclareMathOperator{\sign}{sign}
\DeclareMathOperator{\Th}{th}
\DeclareMathOperator{\sh}{sh}
\DeclareMathOperator{\ch}{ch}
\DeclareMathOperator{\arsh}{arsh}
\DeclareMathOperator{\arch}{arch}
\DeclareMathOperator{\arth}{arth}
\DeclareMathOperator{\arcth}{arcth}
\DeclareMathOperator{\grad}{grad}
\DeclareMathOperator{\arc}{arc}
\DeclareMathOperator{\arctg}{arc tg}
\DeclareMathOperator{\arcctg}{arc ctg}
\newcommand{\norm}[1]{\left\lVert#1\right\rVert}

\begin{document}
	%%%%%%%%%%%RÖVIDÍTÉSEK%%%%%%%%%%
	\setlength\parindent{0pt}
	\def\a{\textbf{a}}
	\def\b{\textbf{b}}
	\def\N{\hskip 10 true mm}
	\def\a{\textbf{a}}
	\def\b{\textbf{b}}
	\def\c{\textbf{c}}
	\def\d{\textbf{d}}
	\def\e{\textbf{e}}
	\def\gg{$\gamma$}
	\def\vi{\textbf{i}}
	\def\jj{\textbf{j}}
	\def\kk{\textbf{k}}
	\def\fh{\overrightarrow}
	\def\l{\lambda}
	\def\m{\mu}
	\def\v{\textbf{v}}
	\def\0{\textbf{0}}
	\def\s{\hspace{0.2mm}\vphantom{\beta}}
	\def\Z{\mathbb{Z}}
	\def\Q{\mathbb{Q}}
	\def\R{\mathbb{R}}
	\def\C{\mathbb{C}}
	\def\N{\mathbb{N}}
	\def\Rn{\mathbb{R}^{n}}
	\def\Ra{\overline{\mathbb{R}}}
	\def\sume{\displaystyle\sum_{n=1}^{+\infty}}
	\def\sumn{\displaystyle\sum_{n=0}^{+\infty}}
	\def\biz{\emph{Bizonyítás:\ }}
	\def\narrow{\underset{n\rightarrow+\infty}{\longrightarrow}}
	\def\limn{\displaystyle\lim_{n\to +\infty}}
	%	\def\definition{\textbf{Definíció:\ }}
	%	\def\theorem{\textbf{Tétel:\ }}
	%\def\note{\emph{Megjegyzés:\ }}
	%\def\example{\textbf{Példa:\ }} 
	
	\theoremstyle{definition}
	\newtheorem{theorem}{Tétel}[subsubsection] % reset theorem numbering for each chapter
	
	\theoremstyle{definition}
	\newtheorem{definition}[theorem]{Definíció} % definition numbers are dependent on theorem numbers
	\newtheorem{example}[theorem]{Példa} % same for example numbers
	\newtheorem{exercise}[theorem]{Házi feladat} % same for example numbers
	\newtheorem{note}[theorem]{Megjegyzés} % same for example numbers
	\newtheorem{task}[theorem]{Feladat} % same for example numbers
	\newtheorem{revision}[theorem]{Emlékeztető} % same for example numbers
	%%%%%%%%%%%%%%%%%%%%%%%%%%%%%%%%%
	\begin{center}
		{\LARGE\textbf{Analízis 3. A szakirány}}
		\smallskip

		{\Large Gyakorlati jegyzet}

		\smallskip
		1-12. óra.
	\end{center}
	A jegyzetet \textsc{Umann} Kristóf készítette \textsc{Filipp} Zoltán István gyakorlatán. (Utoljára frissítve: \today)	\tableofcontents
	\section{Információk a gyakorlattal kapcsolatban}
	\begin{compactitem}
		\item emial: filipp@numanal.inf.elte.hu
		\item szoba: 2-316A
		\item Telefonszám: 06-70-332-01-41
		\item Cím: \url{numanal/~filipp}, analízis 3, azon belül A.
		\item várható majd kötelezően beadandó integrálszámítás
		\item 8:30 kezdés
		\item 7. héten első zh, 13. hét második zh
		\item egy csütörtök el fog maradni, de pótolva lesz
		\item Konzi: szerda 9-10
	\end{compactitem}
	Ami a honlapon található:
	\begin{compactitem}
		\item Gyakorlati anyag
		\item Gyemidovics tankönyv (III. fejezet) (és ennek eredményei külön fájlban)
		\item Bolyai sorozat (integrálszámítás és többváltozós analízis)
		\item Ajánlottak mint Szili analízis feladat gyűjteménye és integráltáblázata.
		\item Károlyi Katalin
	\end{compactitem}
	\section{Integrálszámítás}
	\subsection{Bevezető}
	\begin{revision}
		(Primitív függvény)\quad Legyen $\emptyset\not=I\subset\R$ nyílt intervallum, és $f:I\to\R$ fv. Ha 
		\[\exists F:I\to\R, \quad F\in D\quad  \text{és}\quad  F'(x)=f(x) \quad (\forall x\in I),\]
		akkor azt mondjuk hogy $F$ az $f$ egy primitív függvénye.
	\end{revision}
	Nem minden függvény rendelkezik primitív függvénnyel, pl.: $\sign(x)$, mert az 1/2-et nem veszi fel sehol.
	\begin{note}
	$f(x)=x^2,\quad x\in I:=\R;\quad \Rightarrow\quad F(x)=\frac{1}{3}x^3\quad (x\in\R)$, mert $\left(\frac{1}{3}x^3\right)'=\frac{1}{3}3x^2=x^2$.
	
	Ezek is primitív függvényei $f$-nek $(x\in\R)$:
	\[ F_1(x):=\frac{1}{3}x^3+1,\quad  F_{34}(x):=\frac{1}{3}x^3+34 \]
	Általánoságban
	\[ F(x):=\frac{1}{3}x^3+c\quad (c, x\in\R) \]
	\end{note}
	Két primitív függvény konstansban térhet csak el. Ez alapján megállapítható, hogy ha létezik primitív függvény, végtelen sok lehet belőlük.
	\begin{note}
		Ha van $f$-nek primitív függvénye, akkor
		\[  \int f(x)\,dx:=\{F:I\to\R~|~F\in D \text{\quad és\quad } F'=f \} =F(x) + c \]
		az $f$ primitív függvényeinek a halmaza, vagy az $f$ határozatlan integrálja
	\end{note}
	Mindig meg kell adnunk az $I$ intervallumot.
	
	Néha az is kérdés lehet, egy függvény melyik pontban tűnik el, azaz nem mindig a 0ra vagyunk kíváncsiak.
	\begin{task}
		Adjuk meg azt az $F$-et, melyre $F(1)=2$
	\end{task}
	%TODO ábra 1
	\begin{revision}
		$f:$idő$\to$távolság
		
		Ilyenkor mit jelent $f'$? az a sebesség, míg $f''$ a gyorsulás. Ott gyakran vannak olyan feltételek, hogy a kezdősebesség legyen 2 ($f(1)=2$)
		
		Ehhez a fenti feladatot példaképp véve $\frac{1}{3}^3+c=2$ egyenletet kell megoldani $\quad \Rightarrow\quad c=\frac{5}{3}$.
	\end{revision}
	Tehát:
	\[ F(x)=\frac{x^3}{3}+\frac{5}{3} \]
	\begin{task}$(x\in\R$)
		\[\int \cos x\,dx=\sin x+c\quad (c\in\R) \]
	\end{task}
	\begin{task}
		\[ (\arccos\tan x)' =\frac{1}{1+x^2}\quad \Rightarrow\quad \int\frac{1}{1+x^2}\,dx=\arctan x + c\quad (x,c\in \R) \]
	\end{task}
	\begin{task} $x\in(-1,1)$
		\[ \int\frac{1}{\sqrt{1-x^2}}\,dx=\arc\sin x + c \quad (c\in \R) \]
	\end{task}
	\subsection{Alapintegrálok és ezekre vezethető típusok}
	Szokás úgy is hívni hogy antiderivált, mert az annak az ellentettje.
	\begin{task}$(x\in\R)$
		\[\int(6x^3-2x+1)\,dx\]
	\end{task}
	Emlékezzünk vissza a műveleti tételekre. Az integrál lineáris, így az integráltat integráltak összegére bonthatjuk és a konstansokat is ki tudjuk \textit{,,cuppantani''}:
	\[ 6\cdot\int x^3\,dx-2\cdot\int x\,dx+\int1\,dx=6\cdot\frac{x^4}{4}-2\cdot\frac{x^2}{2}+x+c\quad (c\in\R) \]
	\begin{theorem}
		Általános integrálfüggvény:
		\[ \int x^\alpha\,dx=\frac{x^{\alpha+1}}{\alpha+1}+c\quad (x>0,\quad  c,\alpha\in\R) \]
		Ha $\alpha\not=-1$.
	\end{theorem}
	\begin{task}Határozzuk meg az $f(x):=\frac{1}{x}$ integráltját. Ezt esetekre kell bontani (szétválasztani az értelmezési tartományt intervallumokra), legyen az első $x>0$. Bár $\R\setminus\{0\}$ is jó választás értelmezési tartománynak, de az nem intervallum.
		\[ \int\frac{1}{x}\,dx=\ln x+c\quad (c\in\R) \]
		Második eset $x\in(-\infty,0)=:I$:
		\[ \int\frac{1}{x}\,dx=\ln(-x)+c\quad (c\in\R),\quad \text{ui.}\quad (\ln(-x))'=\frac{1}{-x}(-x)'=-\frac{1}{x}\cdot(-1)=\frac{1}{x} \]
	\end{task}
	\begin{note}
		az előző két feladat összefoglalható úgy, hogy 
		\[ \int\frac{1}{x}\,dx=\ln|x|+c\quad (c\in\R, x\in(-\infty,0)\quad \text{vagy}\quad (0,+\infty) \]
	\end{note}
	\begin{task}$x>0$
		\[ \int\sqrt{x\sqrt{x\sqrt{x}}}\,dx=\int x^\frac{1}{2}\cdot x^\frac{1}{4}\cdot x^\frac{1}{8}\, dx=\int x^{\frac{1}{2}+\frac{1}{4}+\frac{1}{8}}\,dx=\int x^{\frac{7}{8}}\,dx=\frac{x^{\frac{7}{8}+1}}{\frac{7}{8}+1}+c=\frac{8}{15}\cdot\sqrt[8]{x^{15}}+c\quad (c\in\R) \]
	\end{task}
	\begin{note}
		Ha szorzat van csinálj összeget, ha összeg van cisnálj szorzatot
	\end{note}
	\begin{task}
		\[ \int\frac{(x+1)^2}{\sqrt{x}}\,dx=\int\frac{x^2+2x+1}{x^\frac{1}{2}}\,dx=\int\frac{x^2}{x^{\frac{1}{2}}}\,dx+2\int\frac{x}{x^\frac{1}{2}}\,dx+\int\frac{1}{x^\frac{1}{2}}\,dx=\int x^\frac{3}{2}\,dx+2\int x^\frac{1}{2}\,dx+\int x^{-\frac{1}{2}}\,dx= \]
		\[=\frac{x^{\frac{3}{2}+1}}{\frac{3}{2}+1}+2\frac{x^{\frac{1}{2}+1}}{\frac{1}{2}+1}x+\frac{x^{-\frac{1}{2}+1}}{-\frac{1}{2}+1}+c\quad (c\in\R) =\frac{2}{5}\cdot\sqrt{x^5}+\frac{4}{3}\sqrt{x^3}+2\sqrt{x}+c \]
	\end{task}
	\begin{center}
		\textit{,,Integrálni úgy kell hogy nézed, nézed, nézed, és aztán rájössz.''}
		
		/Filipp/
	\end{center}
	\begin{note}
		A fenti módszer hívjuk az összegre bontás módszerének.
	\end{note}
	\begin{task}$(x\in(-1,1))$
		\[ \int\left(2x+(1-x^2)^{-\frac{1}{2}}\right)\,dx=2\int x\,dx+\int\frac{1}{\sqrt{1-x^2}}\,dx=2\frac{x^2}{2}+\arc\sin x+c\quad (c\in\R) \]
	\end{task}
	\begin{task}
		$x\in\R$
		\[ \int\frac{x^2}{x^2+1}\,dx=\int \frac{x^2+1-1}{x^2+1}\,dx=\int1\,dx-\int\frac{1}{x^2+1}\,dx=x-\arc\tg x+c\quad (c\in\R) \]
	\end{task}
	\begin{task}$x\in\left(-\frac{\pi}{2},\frac{\pi}{2}\right)$
		\[ \int \tg^2 x\,dx=\int\frac{\sin^2 x}{\cos^2x}\,dx=\int\frac{1-\cos^2x}{\cos^2x}\,dx=\int\frac{1}{\cos^2x}\,dx-\int1\,dx=\tg x-x+c\quad (c\in\R) \]
	\end{task}
	\subsection{Lineáris helyettesítés szabálya}
	\begin{task}$x\in\R$
		\[ \int\cos(2x)\,dx=\sin(2x)+c= \]
		Ellenőrizzünk:
		\[ (\sin(2x))'=\cos(2x)\cdot2 \]
		Azaz korrigálnunk kell, le kell még osztani 2-vel.
		\[ =\frac{\sin(2x)}{2}+c\quad (c\in\R) \]
	\end{task}
	\begin{revision}
		\[ \int\cos x\,dx=\sin x+c \]
	\end{revision}
	\begin{task}$x\in\R$
		\[ \int\cos(2-3x)\,dx=\frac{\sin(2-3x)}{-3}+c \]
	\end{task}
	Általában, a lineáris helyettesítés szabálya:
	\begin{enumerate}
		\item Feltesszük, hogy $\exists F:\quad \int f(x)\,dx=F(x)+c$
		\item $\int f(ax+b)\,dx=\frac{F(ax+b)}{a}+c\quad (\forall a,b\in\R, a\not=0)$
		\item Csak ha lineáris!
	\end{enumerate}
	\begin{task}$x\in\R$
		\[ \int e^{5x+4}\,dx=\frac{e^{5x+4}}{5}+c\quad (c\in\R) \]
	\end{task}
	\begin{task}
		\[ \int\frac{1}{\sqrt{1-3x^2}}\,dx=\quad \left(|x|<\frac{1}{3}\right) \quad =\int\frac{1}{\sqrt{1-(\sqrt{3}x)^2}}\,dx=\frac{\arc\sin(\sqrt{3}x)}{\sqrt{3}}+c\quad (c\in\R) \]
	\end{task}
	\begin{task}
		\[ \int\frac{2}{3+2x^2}\,dx=\frac{2}{3}\int\frac{1}{1+\frac{2}{3}x^2}\,dx=\frac{2}{3}\int\frac{1}{1+\left(\sqrt{\frac{2}{3}}x\right)^2}\,dx=\frac{2}{3}\frac{\arc\tg\left(\sqrt{\frac{2}{3}}x\right)}{\sqrt{\frac{2}{3}}}+c\quad (c\in\R) \]
	\end{task}
	\begin{task}
		\[\int\sin^2x\,dx=\int\frac{1-\cos2x}{2}\,dx=\frac{1}{2}\left(\int1\,dx-\int\cos2x\,dx\right)=\frac{1}{2}x-\frac{1}{2}\cdot\frac{\sin(2x)}{2}+c\quad (c\in\R) \]
	\end{task}
	\begin{note}
		\[ \begin{cases}
			\sin^2x+cos^2x=1\\
			\cos^2x-\sin^2x=\cos2x\\
			2\cos x\sin x=\sin2x
		\end{cases} \]
		A következő két összefüggést \textit{linearizálható formulák}nak nevezzük.
		\begin{center}
			\fbox{$\cos^2 x=\frac{1+\cos2x}{2}$}\quad \fbox{$\sin^2x=\frac{1-\cos2x}{2}$}
		\end{center}
	\end{note}
	\begin{task} $x\in(-\pi,\pi)$
		\[\int\frac{1}{1+\cos x}\,dx=\text{(félszögre térés)}=\int\frac{1}{\sin^2\frac{x}{2}+\cos^2\frac{x}{2}+\cos^2\frac{x}{2}-\sin^2\frac{x}{2}}\,dx=\frac{1}{2}\cdot\int\frac{1}{\cos^2\frac{x}{2}}\,dx=\frac{1}{2}\frac{\tg \frac{x}{2}}{\frac{1}{2}}+c\quad (c\in\R) \]
	\end{task}
	\begin{exercise}$x\in\left(-\frac{\pi}{2},\frac{\pi}{2}\right)$
		\[ \int\frac{\cos^2x-2}{1+\cos2x}\,dx=? \]
	\end{exercise}
	\subsubsection{$\int\frac{f'}{f}\,dx$ típusú feladatok}
	\begin{task}
		\[ \int\frac{f'(x)}{f(x)}\,dx=\ln |f(x)|+c \quad (c\in\R, \quad x\in I)\]
	\end{task}
	\begin{task}$(x\in\R)$
		\[ \int\frac{x}{x^2+8}\,dx\quad \overset{f(x):=x^2+8}{\underset{f'(x)=2x}{=}}\quad\frac{1}{2}\int\frac{2x}{x^2+8}\,dx=\frac{1}{2}\int\frac{(x^2+8)'}{x^2+8}\,dx=\frac{1}{2}\ln(x^2+8)+c\quad (c\in\R)  \]
	\end{task}
	\begin{task}Ha $x\in(1;+\infty)$:
		\[ \int\frac{1}{x\ln x}\,dx= \int\frac{1}{x}\cdot\frac{1}{\ln x}\,dx\quad \overset{f(x):=\ln x}{\underset{f'(x)=\frac{1}{x}}{=}}\quad \int\frac{(\ln x)'}{\ln x}\,dx=\ln|\ln x|+c=\ln(\ln(x))+c\quad (c\in\R) \]
		Ha $x\in(0,1)$:
		\[ \int\frac{(\ln x)'}{\ln x}\,dx=\ln(-\ln x)+c\quad (c\in\R) \]
	\end{task}
	\begin{task}$x\in\left(-\frac{\pi}{2},\frac{\pi}{2}\right)$:
		\[ \int\tg x\,dx=\int\frac{\sin x}{\cos x}\,dx\quad \overset{f(x):=\cos x}{\underset{f'(x)=-\sin x}{=}}\quad -\int\frac{(\cos x)'}{\cos x}\,dx=-\ln|\cos x|+c=-\ln(\cos x)+c\quad (c\in\R) \]
	\end{task}
	\subsubsection{$\int f'(x)\cdot f^\alpha(x)\,dx$ típusú feladatok}
	\begin{task} $(\alpha\in\R\setminus\{-1\})\quad (x\in I)$
		\[ \int f'(x)\cdot f^\alpha(x)\,dx=\frac{f^{\alpha+1}(x)}{\alpha+1}+c\quad (c\in\R) \]
		\[ \int 1\cdot x^\alpha\,dx=\frac{x^{\alpha+1}}{\alpha+1}+c\quad (c\in\R)\quad \longrightarrow\quad \int f'(x)f^\alpha(x)\,dx=\frac{f^{\alpha+1}(x)}{\alpha+1}+c \quad (c\in\R) \]
	\end{task}
	\begin{task}
		$(x\in\R)$
		\[ \int x^2(3x^3+4)^{2017}\,dx\quad \overset{\alpha:=2017}{\underset{f(x):=3x^3+4}{\underset{f'(x)=9x^2}{=}}}\quad \frac{1}{9}\int 9x^2(3x^3+4)^{2017}\,dx=\frac{1}{9}\int(3x^3+4)'(3x^3+4)\,dx=\]
		\[=\frac{1}{9}\frac{(3x^3+4)^{2018}}{2018}+c\quad (c\in\R) \]
		\begin{note}
			Itt kapóra jött az  $x^2$. Ha nem így lenne, akkor 2017re kéne hatványozni, szétszedni binomiális tétellel, stb.
		\end{note}
	\end{task}
	\begin{task}$(x\in\R)$
		\[\int e^x (1-e^x)^{300}\,dx\quad \overset{\alpha=300}{\underset{f(x):=1-e^x}{\underset{f'(x)=-e^x}{=}}}\quad -\int(1-e^x)'\cdot(1-e^x)^{300}\,dx=-\frac{(1-e^x)^{301}}{301}+c\quad (c\in\R) \]
	\end{task}
	\begin{task}
		$x\in\left(0,\frac{\pi}{2}\right)$
		\[ \int\frac{\cos2x}{\sqrt{\sin x+\cos x}}\,dx = \int \frac{\cos^2 x-\sin^2 x}{\sqrt{\cos x+\sin x}}\,dx=\int\frac{(\cos x-\sin x)(\cos x+\sin x)}{\sqrt{\cos x+\sin x}}\,dx=\]
		\[=\int(\cos x-\sin x)(\cos x+\sin x)^{\frac{1}{2}}\,dx\quad \overset{\alpha:=\frac{1}{2}}{\underset{f(x):=\cos x+\sin x}{\underset{f'(x)=-\sin x+\cos y}{=}}}\quad \int(\sin x+\cos x)'(\sin x+\cos x)^{\frac{1}{2}}\,dx=\]
		\[=\frac{(\sin x + \cos x)^{\frac{3}{2}}}{\frac{3}{2}}+c\quad (c\in\R) \]
		\begin{note}
			Ez már ZH szintű feladat.
		\end{note}
		\begin{center}
			\textit{,,Én is szívesen négyzetre emelném a fizetésemet''}
			\smallskip
			
			/Filipp Zoltán István/
		\end{center}
	\end{task}
	\subsubsection{$n,m\in\Z:\quad \int\sin^n x\cos^mx\,dx$\quad \text{(feladat altípus)}\quad}
	\begin{note}
		Ez a megoldási módszer fő gondolatmenetét a $\sin$ és a $\cos$ függvények közötti egyszerű váltás adja, pl.:\, $(\cos x)'=-\sin x$ \,és\, $(\sin x)'=\cos x$, \,valamint \,$1=\sin^2x + \cos^2x$.
	\end{note}
	\begin{task}$x\in\R$
		\[ \int \sin^3x\cdot\cos^5x\,dx=\int\sin x\cdot\sin^2x\cdot\cos^5x\,dx=-\int(\cos x)'\cdot\overbrace{(1-\cos^2 x)}^{\sin^2x}\cdot\cos^5x\,dx=\]
		\[=-\int(\cos x)'\cos^5x\,dx+\int(\cos x)'\cos^7x\,dx=-\frac{\cos^6x}{6}+\frac{\cos^8}{8}+c\quad (c\in\R) \]
	\end{task}
	\begin{note}
		Ha $n$ vagy $m$ páratlan, akkor örülünk, és 
		\begin{enumerate}
			\item Vesszük a kisebbik páratlan hatvánnyal rendelkező tagot (pl.: $\sin^5\cos^7x$ esetében $\sin^5x$),
			\item Leválasztunk belőle 1-et (pl.: $\sin^5x = \sin x\cdot\sin^4 x)$,
			\item Ez lesz a nagyobb hatvánnyal rendelkező tag deriváltja (pl.: $(\cos x)'\sin^4 x$).
		\end{enumerate}
		
	\end{note}
	\begin{task}$x\in\R$
		\[ \int\cos^3x\,dx=\int\cos x\cdot\cos^2x\,dx=\int(\sin x)'\cdot(1-\sin^2x)\,dx=\int(\sin x)'\,dx-\int(\sin x)'(\sin x)^2\,dx=\]
		\[=\sin x-\frac{\sin^3x}{3}+c\quad (c\in\R) \]
	\end{task}
	\begin{exercise} Tipp: Linearizálás 
		\[ \int(\sin^5x\cos^{10}x)\,dx\quad (x\in\R) \]
		\textit{Megoldás:}
		\[ \int(\sin^5x\cos^{10}x)\,dx=\int(\sin x\cdot\sin^4x\cdot\cos^{10}x)\,dx=-\int\left((\cos x)'(1-\cos^2x)^2\cos^{10}x\right)\,dx=\]
		\[=-\int(\cos x)'(\cos^4x-2\cos^2x+1)\cos^{10}x\,dx=-\int(\cos x)'\cos^{14}x\,dx+2\cdot\int(\cos x)'\cos^{12}x\,dx-\int(\cos x)'\cos^{10}x\,dx= \]
		\[ -\frac{\cos^{15}x}{15}+2\cdot\frac{\cos^{13}x}{13}-\frac{\cos^{11}x}{11}+c\quad (c\in\R) \]
	\end{exercise}
	\begin{task}$x\in\R$
		\[ \int\sin^2x\cos^4x\,dx=\int(1-\cos^2x)\cos^4 x\,dx=\int\cos^4 x\,dx-\underbrace{\int\cos^6x\,dx}_{\text{HF}}=\]
		\[\int\cos^4x\,dx=\int\left(\frac{1+\cos2x}{2}\right)^2\,dx=\frac{1}{4}\int(1+2\cos2x+\cos^22x)\,dx=\frac{1}{4}\left(x+\sin2x+\int\cos^2(2x)\right)\,dx=\]
		\[=\frac{x}{4}+\frac{\sin2x}{4}+\frac{1}{4}\int\frac{1+\cos4x}{2}\,dx=\frac{x}{4}+\frac{\sin2x}{4}+\frac{x}{8}+\frac{1}{8}\cdot\frac{\sin4x}{4}+c\quad (c\in\R) \]
		\begin{center}
			\fbox{$\displaystyle \cos^2=\frac{1+\cos2x}{2}$}\\
			\fbox{$\displaystyle \sin^2=\frac{1-\cos2x}{2}$}
		\end{center}
	\end{task}
	\begin{exercise}
		\def\fracHeight{\vphantom{\frac{3^2}{2}}}
		\[\int\cos^6x\,dx=\int\left(\frac{1+\cos2x}{2}\right)^3\,dx=\frac{1}{8}\int(1+3\cos2x+3\cos^22x+\cos^32x)\,dx=\]
		\[= \frac{1}{8}\left(\overbrace{x\fracHeight}^{\int1\,dx}+\overbrace{\frac{3}{2}\sin2x\fracHeight}^{\int3\cos2x\,dx}+ \overbrace{3\left(\frac{x}{2}+\frac{\sin4x}{8}\right)\fracHeight}^{\int3\cos^22x\,dx}+ \overbrace{\sin2x-\frac{\sin^32x}{3}}^{\int\cos^32x\,dx}\right)+c\quad (c\in\R)  \]
		A számoláshoz a gyakorlat korábbi eredményeit is felhasználtam.
	\end{exercise}
	\begin{task}$x\in(0,\pi)$
		\[\int\frac{1}{\sin x}\,dx\quad \overset{\sin2\alpha=2\sin\alpha\cos\alpha}{\underset{\text{félszögre térés}}{=}}\quad \int\frac{\sin^2\frac{x}{2}+\cos^2\frac{x}{2}}{2\sin\frac{x}{2}\cos\frac{x}{2}}\,dx=\frac{1}{2}\int\frac{\sin\frac{x}{2}}{\cos\frac{x}{2}}\,dx+\frac{1}{2}\int\frac{\cos\frac{x}{2}}{\sin\frac{x}{2}}\,dx=\]
		\[=-\int\frac{(\cos\frac{x}{2})'}{\cos\frac{x}{2}}\,dx+\int\frac{(\sin\frac{x}{2})'}{\sin\frac{x}{2}}\,dx=-\ln\left(\cos\frac{x}{2}\right)+\ln\left(\sin\frac{x}{2}\right)+c=\ln\left(\tg\frac{x}{2}\right)+c\quad (c\in\R) \]
	\end{task}
	\begin{exercise}
		\[ \int\frac{\sin^2x}{\cos^4x}\,dx\quad \left(x\in\left(0,\frac{\pi}{2}\right)\right) \]
		\textit{Megoldás:}
		\[ \int\frac{\sin^2x}{\cos^4x}\,dx=\int\frac{1}{\cos^2x}\cdot\tg^2x\,dx= \int(\tg x)'\tg^2x\,dx=\frac{\tg^3x}{3}+c\quad (c\in\R) \]
	\end{exercise}
	\begin{exercise}
		\[ \int\sin^4x\cos^4x\,dx \quad (x\in\R) \]
		\textit{Megoldás:}
		\[ \int\sin^4x\cos^4x\,dx=\int(1-\cos^2x)^2\cos^4x\,dx=\int(1-2\cos^2x+\cos^4x)\cos^4x\,dx=\int(\cos^4x-2\cos^6x+\cos^8x)\,dx=\]
		\[ =\overbrace{\int\cos^4x\,dx}^{\text{ismert}}-\overbrace{\int2\cos^6x\,dx}^{\text{ismert}}+\int\cos^8x\,dx \]
		Határozzuk meg az ismeretlen tagot:
		\[ \int\cos^8x\,dx=\int\left(\frac{1+\cos2x}{2}\right)^4\,dx=\frac{1}{16}\int(1+2\cos2x+\cos^22x)^2\,dx=\]
		\[=\frac{1}{16}\int\left(1+4\cos^22x+\cos^42x+4\cos2x+4\cos^32x+2\cos^22x\right)\,dx=\]
		\[=\frac{1}{16}\int\left(1+4\cos2x+6\cos^22x+4\cos^32x+\cos^42x\right)\,dx=\]
		\def\fracHeight{\vphantom{\frac{3^2}{2}}}
		\[= \frac{1}{16}\left(\overbrace{x\fracHeight}^{\int1\,dx}+\overbrace{\frac{4}{2}\sin2x\fracHeight}^{\int4\cos2x\,dx}+ \overbrace{6\left(\frac{x}{2}+\frac{\sin4x}{8}\right)\fracHeight}^{\int6\cos^22x\,dx}+ \overbrace{4\left(\sin2x-\frac{\sin^32x}{3}\right)}^{\int4\cos^32x\,dx}+\overbrace{\frac{2x}{4}+\frac{\sin4x}{4}+\frac{2x}{8}+\frac{\sin8x}{32}\fracHeight}^{\int\cos^42x\,dx}\right)+c\quad (c\in\R)  \]
	\end{exercise}
	\subsection{Helyettesítés szabálya}
	\begin{revision}
		Tegyük fel, hogy $\emptyset\not=\int f(x)\,dx=F(x)+c\quad (c\in\R, x\in I)$, és tegyük fel, hogy\\ $g:J\to I, \quad g\in D.$ Ekkor
		\[ \int f(g(x))\cdot g'(x)\,dx=F(g(x))+c\quad (c\in\R) \]
		Ugyanis
		\[ (F(g(x)))'=F'(g(x))\cdot g'(x)=f(g(x))\cdot g'(x)\checkmark \]
	\end{revision}
	\begin{task}$x\in\R$
		\[ \int x\cos(x^2)\,dx\quad \overset{g'(x):=2x}{\overset{g(x):=x^2}{\underset{f(x):=\cos x}{=}}}\quad \frac{1}{2}\int 2x\cos(x^2)\,dx=\frac{1}{2}\int(x^2)'\cdot\cos(x^2)\,dx=\frac{1}{2}\sin(x^2)+c\quad (c\in\R) \]
	\end{task}
	\begin{note}
		Rövid jelölés: 
		\[ \int x\cos(x^2)\,dx\quad \overset{x^2=:t\,\in\,(0;+\infty)}{\underset{(x^2)'\,dx=(t)'\,dt}{\underset{2x\,dx=1\,dt}{=}}}\quad\int\frac{1}{2}\cdot\cos t\,dt\big|_{t=x^2}=\frac{1}{2}\sin t +c\big|_{t=x^2}=\frac{1}{2}\sin(x^2)+c  \]
	\end{note}
	\begin{task}$x\in(0,\frac{\pi}{2})$
		\[ \int\frac{1+\tg^2x}{1+\tg x}\,dx \]
		Legyen  \quad $\tg x=:t\in(0;+\infty).$ \quad Ez alapján $\quad  x=\arc\tg t\quad \Rightarrow\quad (x)'\,dx=(\arc\tg t)'\,dt$
		\[ \text{\fbox{$\displaystyle dx=\frac{1}{1+t^2}\,dt$}} \quad \left(\text{ui.:\quad }(\arc\tg t)'=\frac{1}{1+t^2}.\right)\]
		\[ \Rightarrow\quad \int\frac{1+t^2}{1+t}\cdot\frac{1}{1+t^2}\,dt=\int\frac{1}{1+t}\,dt=\int\frac{(t+1)'}{t+1}\,dt=\ln(t+1)+c\quad (c\in\R) \]
		Visszírva
		\[ \int\frac{1+\tg^2 x}{1+\tg x}\,dx=\ln(1+\tg x)+c\quad (c\in\R) \]
		Vagy:
		\[ 1+\tg^2 x=1+\frac{\sin^2x}{\cos^2x}=\frac{1}{\cos^2x};\quad \Rightarrow\quad \int\frac{1}{\cos^2x}\cdot\frac{1}{1+\tg x}\,dx=\int\frac{(1+\tg x)'}{1+\tg x}\,dx=\ln(1+\tg x)+c\quad (c\in\R) \]
	\end{task}
	\begin{task}$x\in\R$
		\[ \int\frac{e^{2x}}{1+e^x}\,dx \]
		Legyen \[e^x=:t\in(0,+\infty)\quad \Rightarrow\quad x=\ln t\quad \Rightarrow\quad (x)'\,dx=(\ln t)'\,dt\quad \text{\fbox{$\displaystyle dx=\frac{1}{t}\,dt$}} \]
		Az ,,új integrál'':\quad \[\displaystyle \int\frac{t^2}{1+t}\cdot\frac{1}{t}\,dt=\int\frac{t}{1+t}\,dt\int\frac{t+1-1}{t+1}\,dt=\int\left(1-\frac{1}{1+t}\right)\,dt=t-\ln(1+t)+c \]
		,,Vissza'':
		\[ \int\frac{e^x}{1+e^x}\,dx=e^x-\ln(1+e^x)+c\quad (c\in\R) \]
	\end{task}
	HF: 30 darab feladattípus, $\int\frac{f'}{f}$,\quad  $\int f'f^\alpha$, \quad $\int(\sin^4x\cos^4x),\quad \int f\circ g\cdot g'=...$.
	\begin{center}
		\bigskip
		
		\textit{,,Tanár úr, használhatok más jelölést? A $g(t)$-t annyira nem szeretem.''}
		\smallskip
		
		/Tóth Péter/
		\bigskip
		
	\end{center}
	\begin{note}
		Mindig a nagyobb hatvánnyal rendelkező változóhoz érdemes új változót rendelni.
	\end{note}
	\subsection{Parciális integrálás}
	\begin{revision}
		\[ (f\cdot g)'=f'\cdot g + f\cdot g'\quad \Rightarrow\quad \int(f\cdot g)'=\int f'\cdot g + \int f\cdot g'\quad \Rightarrow\quad f\cdot g=\int f'\cdot g + \int f\cdot g'\]
		\begin{center}	
			\fbox{$\displaystyle \int f'\cdot g=f\cdot g - \int f\cdot g'$}
		\end{center}
		Ezt hívjuk a parciális integrálás tételének.
	\end{revision}
	\begin{task}$x\in\R$
		\[ \int \overbrace{x\vphantom{e^2}}^{g}\overbrace{e^{2x}}^{f'}\,dx\quad \underset{g'(x):=1}{\overset{f':=e^{2x}}{\overset{g:=x}{\underset{f(x):=\frac{e^{2x}}{2}}{=}}}}
		\quad \overbrace{x\cdot\frac{e^{2x}}{2}}^{f\cdot g}-\overbrace{\int\frac{e^{2x}}{2}\cdot1\,dx}^{\int f\cdot g'}=\frac{1}{2}\cdot xe^{2x}-\frac{1}{2}\int e^{2x}\,dx=\frac{1}{2}xe^{2x}-\frac{1}{2}\frac{e^{2x}}{2}+c\quad (c\in\R) \]
	\end{task}
	
	\begin{center}
		\textit{,,Cuppantsuk ki a konstans tagot!''}
		\smallskip
		
		/Filipp/
	\end{center}
	
	\begin{note}
		Rövid jelölés: ($f$ és $g$ nem lesz jelölve többet)
		\[ \int xe^{2x}\,dx\quad \overset{\text{p.i.}}{=}\int x\cdot\left( \frac{e^{2x}}{2}\right)'=x\cdot\frac{e^{2x}}{2}-\int(x)'\cdot\frac{e^{2x}}{2}\,dx=\ldots \]
		Jelölje ,,p.i.'' a parciális integrálást.
	\end{note}
	\begin{center}
		\textit{,,Fogd a deriválás kukkert. Amelyik szemed nyitott az adott formulán azt deriválod, amelyiken csukott, az hagyod. Aztán egyik szem lecsuk, másik kinyit.''}
		
		\smallskip
		/Filipp/
	\end{center}
	\begin{note}
		Ez a módszer hatásos lehet, de nem minden esetben. Ha nem jól osztjuk a szerepeket (nem a megfelelő függvényt választjuk $f$-nek vagy $g$-nek), lehet hogy egyáltalán nem jutunk előrébb. E példában ha a két függvényt fordítva választjuk meg, nem jutottunk volna előrébb semmivel.
	\end{note}
	\begin{task}$x\in\R$
		\[ \int \overbrace{x^2}^{g}\overbrace{\sin(3x)}^{f'}\,dx \quad \overset{\text{p.i.}}{=}\quad  \int x^2\left(-\frac{\cos3x}{3}\right)'\,dx=-x^2\cdot\frac{\cos3x}{3}-\int\left(x^2\right)'\cdot\left(-\frac{\cos3x}{3}\right)\,dx=\]
		\[=\left(-\frac{1}{3}\right)\cdot x^2\cos3x+\frac{2}{3}\cdot\int \overbrace{x}^{g}\overbrace{\cos3x}^{f'}\,dx=\left(-\frac{1}{3}\right)\cdot x^2\cos3x+\frac{2}{3}\int x\cdot\left( \frac{\sin3x}{3}\right)'\,dx\quad \overset{\text{p.i.}}{=} \]
		\[  \left(-\frac{1}{3}\right)\cdot x^2\cos3x+\frac{2}{3}\left(x\cdot\frac{\sin3x}{3}-\int(x)'\cdot\frac{\sin3x}{3}\,dx\right)=\left(-\frac{1}{3}\right)\cdot x^2\cos3x+\frac{2}{9}\cdot x\sin3x-\frac{2}{9}\cdot\int\sin3x\,dx=\]
		\[=\left(-\frac{1}{3}\right) \cdot x^2\cos3x+\frac{2}{9}\cdot
		x\sin3x+\frac{2}{9}\cdot\frac{\cos3x}{3} + c\quad (c\in\R) \]
		Itt egyértelműen érdemesebb volt $\sin$-t választani.
	\end{task}
	\begin{note}
		Célszerű mindig részletesen kiírni a számolásokat, mert ha valamit elrontunk, részpontot se kapunk ZH-ban.
	\end{note}
	\begin{note}
		Ez a módszer nagyon jól működik hasonló típusoknál:
		\[ \int P(x)\cdot
		\left.\begin{cases}
			\sin(\alpha x+\beta)\\
			\cos(\alpha x+\beta)\\
			e^{\alpha x+\beta}\\
			\sh(\alpha x+\beta)\\
			\ch(\alpha x+\beta)\\
		\end{cases}\right\}\,dx \]
		Ahol $P(x)$ egy valós polinom.
	\end{note}
	\begin{exercise}$x\in\R$
		\[ \int(x^2+2x)\cdot\cos x\,dx \]
		\textit{Megoldás:}
		\[ \int(x^2+2x)\cdot(\sin x)'\,dx = \sin x \cdot(x^2+2x)-\int\sin x\cdot(2x + 2)\,dx= \sin x \cdot(x^2+2x)-\int(-\cos x)'\cdot(2x + 2)\,dx= \]
		\[ =\sin x\cdot(x^2+2x)-\left((-\cos x)(2x+2)-\int-\cos x\cdot2\,dx\right)=\sin x\cdot(x^2+2x)+\cos x(2x+2)-2\sin x + c=\quad (c\in\R) \]
		\[ =\sin x\left(x^2+2x -2 \right)+2\cos x(x+1) + c\quad (c\in\R) \]
	\end{exercise}
	\begin{exercise}$x\in\R$
		\[ \int(3x+8)\cdot e^{2x-1}\,dx\]
		\textit{Megoldás:}
		\[\int(3x+8)\left(\frac{e^{2x-1}}{2}\right)'\,dx=(3x+8)\cdot\frac{e^{2x-1}}{2}-\int3\frac{e^{2x-1}}{2}\,dx=(3x+8)\frac{e^{2x-1}}{2}-\frac{3}{2}\cdot\int e^{2x-1}\,dx= \]
		\[ =(3x+8)\frac{e^{2x-1}}{2}-\frac{3}{2}\cdot\frac{e^{2x-1}}{2}+c=\frac{e^{2x-1}}{2}\left((3x+8)-\frac{3}{2}\right)+c\quad (c\in\R) \]
	\end{exercise}
	\begin{exercise}$x\in\R$
		\[ \int x\cdot\sin(1-x)\,dx \]
		\textit{Megoldás:}
		\[ -\int x\sin(x-1)\,dx=-\int x\cdot(-\cos(x-1))'\,dx=x\cdot(\cos(x-1))+\int-\cos(x-1)\,dx=\]
		\[=x\cos(x-1)-\sin(x-1)+c\quad (c\in\R) \]
	\end{exercise}
	\begin{exercise}$x\in\R$
		\[ \int x\cdot\ch(5x)\,dx \]
		\textit{Megoldás:}
		\[ \int x\cdot\left(\frac{\sh(5x)}{5}\right)'\,dx=x\cdot\frac{\sh(5x)}{5}-\int\frac{\sh(5x)}{5}\,dx=x\cdot\frac{\sh(5x)}{5}-\frac{\ch(5x)}{25}+c\quad (c\in\R) \]
	\end{exercise}
	\begin{exercise}$x\in\R$
		\[ \int(x^2+x-3)\sh(8x)\,dx \]
		\textit{Megoldás:}
		\[ \int(x^2+x-3)\left(\frac{\ch(8x)}{8}\right)'\,dx=(x^2+x-3)\left(\frac{\ch(8x)}{8}\right)-\int(2x+1)\frac{\ch(8x)}{8}\,dx=\]
		\[=(x^2+x-3)\left(\frac{\ch(8x)}{8}\right)-\int(2x+1)\left(\frac{\sh(8x)}{64}\right)'\,dx=\]
		\[=(x^2+x-3)\left(\frac{\ch(8x)}{8}\right)-\left((2x+1)\left(\frac{\sh(8x)}{64}\right)-\int2\cdot\frac{\sh(8x)}{64}\,dx\right)=\]
		\[=(x^2+x-3)\left(\frac{\ch(8x)}{8}\right)-(2x+1)\left(\frac{\sh(8x)}{64}\right)+2\left(\frac{\ch(8x)}{512}\right)+c= \]
		\[ =\left(\frac{32\cdot\ch(8x)}{256}\right)(x^2+x-3)+\frac{\ch(8x)}{256}-\left(\frac{\sh(8x)}{64}\right)(2x+1)+c\quad (c\in\R) \]
		\[ =\left(\frac{\ch(8x)}{256}\right)(32x^2+32x-95)-\left(\frac{\sh(8x)}{64}\right)(2x+1)+c\quad (c\in\R) \]
	\end{exercise}
	\begin{exercise} $x\in\R$\quad (javallott a linearizálás)
		\[ \int x\cdot\cos^2x\,dx \]
		\textit{Megoldás:}
		\[ \int x\cdot\frac{1+\cos2x}{2}\,dx=\frac{1}{2}\int x\cdot\left(x+\frac{\sin(2x)}{2}\right)'\,dx=\frac{1}{2}\left(x\cdot\left(x+\frac{\sin(2x)}{2}\right)-\int x+\frac{\sin(2x)}{2}\,dx\right)= \]
		\[ =\frac{1}{2}\left(x^2+\frac{x\sin(2x)}{2}-\frac{x^2}{2}+\frac{\cos(2x)}{4}\right)+c\quad (c\in\R) \]
	\end{exercise}
	\begin{task}
		Olyan típusú integrálokat veszünk most, melyek tartalmaznak inverz függvényeket is. Legyen $x>0$ és:
		\[ \int \ln x\,dx=\int1\cdot\ln x\,dx=\int (x)'\ln x\,dx\quad \overset{\text{p.i.}}{=}\quad x\cdot\ln x-\int x\cdot(\ln x)'\,dx=\]
		\[=x\cdot\ln x-\int x\frac{1}{x}\,dx=x\ln x-x+c\quad (c\in\R) \]
	\end{task}
	\begin{task}
		\[ \int \arc\sin x\,dx=\int 1\cdot\arc\sin x\,dx\quad \overset{\text{p.i.}}{=}\quad x\arc\sin x-\int x(\arc\sin x)'\,dx=x\arc\sin x-\text{\fbox{$\displaystyle \int x\cdot\frac{1}{\sqrt{1-x^2}}$}}\,dx= \]
		\[= x\arc\sin x+\frac{1}{2}\cdot\int\underbrace{(-2x)(1-x^2)^{-\frac{1}{2}}}_{f'\cdot f^\alpha}\,dx=x\arc\sin x+\frac{1}{2}\int(1-x^2)'(1-x^2)^{-\frac{1}{2}}\,dx=\]
		\[=x\arc\sin x+\frac{1}{2}\cdot\frac{(1-x^2)^{\frac{1}{2}}}{\frac{1}{2}}+c=x\arc\sin x+\sqrt{1-x^2}+c \]
	\end{task}
	\begin{task}
		\[ \int\arc\tg(3x)\,dx=\int(x)'\arc\tg(3x)\,dx =x \arc\tg3x-\int x\cdot\frac{1}{1+9x^2}\cdot3\,dx=x\arc\tg3x-3\cdot\int\underbrace{\frac{x}{1+9x^2}}_{\frac{f'}{f}}\,dx=\]
		\[\quad \overset{(1+9x^2)'=18x}{=}\quad x\arc\tg3x-\frac{3}{18}\int\frac{(1+9x^2)'}{1+9x^2}\,dx=x\arc\tg3x-\frac{1}{6}\ln(1+9x^2)+c\quad (c\in\R) \]
	\end{task}
	\begin{task}$x>-1$
		\[ \int x\ln(x+1)\,dx=\int \left(\frac{x^2}{2}\right)'\cdot\ln(x+1)\,dx\quad \overset{\text{p.i.}}{=}\quad \frac{x^2}{2}\ln(x+1)-\int\frac{x^2}{2}\cdot\frac{1}{x+1}\,dx\]
		\[=\frac{x^2}{2}\ln(x+1)-\frac{1}{2}\underbrace{\int\frac{x^2}{x+1}\,dx}=\]
		Határozzuk meg a kiemelt részt.
		\[\frac{x^2}{x+1}=\frac{x^2-1+1}{x+1}=\frac{(x-1)(x+1)+1}{x+1}=x-1+\frac{1}{x+1}\]
		\[  \Updownarrow\]
		\[ \int\left(x-1+\frac{1}{x+1}\right)\,dx=\frac{x^2}{2}-x+\ln(x+1)+c \quad (c\in\R)\]
		Visszatérve:
		\[ =\frac{x^2}{2}\cdot\ln(x+1)-\frac{x^2}{4}+\frac{x}{2}-\frac{1}{2}\ln(x+1)+c\quad (c\in\R) \]
		Azért célszerű ehhez hasonló példákban a polinomot választani $f'$-nak, mert így eltűnik a logaritmus.
	\end{task}
	\subsubsection{Parciális integrálás + egyenlet}
	\begin{task}$x\in\R$
		\[ \int e^x\sin x\,dx\quad \overset{\text{p.i.}}{=}\quad e^x\sin x-\int (e^x)'\cdot\cos x\,dx=e^x\sin x-\left(e^x\cos x-\int e^x(-\sin x)\,dx\right)=\]
		\[=e^x\sin x-e^x\cos x-\int e^x\sin x\,dx \]
		Kaptunk egy egyenletet az ismeretlen integrálra. Rendezzük át, fejezzük ki, és oldjuk meg.
		\[ 2\int e^x\sin x\,dx=e^x\cdot(\sin x-\cos x)\quad \Leftrightarrow\quad \int e^x\sin x\,dx=\frac{1}{2}\cdot e^x(\sin x-\cos x)+c\quad (c\in\R) \]
		Mindig ugyanazt kell kiválasztani, ha parciálisan intergálunk!
	\end{task}
	\begin{task}$x\in(-1,1)$
		\[ \int\sqrt{1-x^2}\,dx=\int1\cdot\sqrt{1-x^2}\,dx=x\cdot\sqrt{1-x^2}-\int x\cdot\frac{1}{2\sqrt{1-x^2}}(-2x)\,dx=x\cdot\sqrt{1-x^2}+\text{\fbox{$\displaystyle \int\frac{x^2}{\sqrt{1-x^2}}\,dx$}}= \]
		\[=x\cdot\sqrt{1-x^2}+\int\frac{x^2+1-1}{\sqrt{1-x^2}}\,dx=x\cdot\sqrt{1-x^2}+\int\frac{1}{\sqrt{1-x^2}}\,dx-\int\frac{1-x^2}{\sqrt{1-x^2}}\,dx=\]
		\[=x\cdot\sqrt{1-x^2}+\arc\sin x-\int\sqrt{1-x^2}\,dx\quad \Rightarrow \]
		Ismét egyenletet kaptunk.
		\[ \int\sqrt{1-x^2}\,dx=\frac{1}{2} \cdot x\cdot\sqrt{1-x^2}+\frac{1}{2}\arc\sin x+c\quad (c\in\R) \]
	\end{task}
	\begin{exercise}$x\in\R$
		\[ \int\sqrt{x^2+1}\,dx \]
		\textit{Megoldás:}
		\[ \int(x)'\sqrt{x^2+1}\,dx=x\cdot\sqrt{x^2+1}-\int x\cdot\frac{1}{2}\cdot\frac{2x}{\sqrt{x^2+1}}\,dx=x\cdot\sqrt{x^2+1}-\int\frac{x^2+1-1}{\sqrt{x^2+1}}\,dx=\]
		\[=x\cdot\sqrt{x^2+1}-\int\sqrt{x^2+1}-\frac{1}{\sqrt{x^2+1}}\,dx=x\cdot\sqrt{x^2+1}-\arsh x - \int\sqrt{x^2+1}\,dx \]
		Ez alapján az egyenlet:
		\[ \int\sqrt{x^2+1}=\frac{1}{2}\left(x\cdot\sqrt{x^2+1}-\arsh x \right)+c\quad (c\in\R) \]
	\end{exercise}
	\begin{exercise}$x\in\R$
		\[ \int x\arc\tg x\,dx \]
		\textit{Megoldás:}
		\[ \int \left(\frac{x^2}{2}\right)'\arc\tg x\,dx=\frac{x^2}{2}\arc\tg x-\int\frac{x^2}{2}\cdot\frac{1}{1+x^2}\,dx=\frac{x^2}{2}\arc\tg x-\frac{1}{2}\int\frac{x^2+1-1}{1+x^2}\,dx= \]
		\[ =\frac{x^2}{2}\arc\tg x-\frac{1}{2}\int 1-\frac{1}{1+x^2}\,dx=\frac{x^2}{2}\arc\tg x-\frac{1}{2}\cdot x+\frac{1}{2}\cdot\arc\tg x+c\quad (c\in\R) \]
	\end{exercise}
	\begin{exercise}
		$x\in\R$
		\[ \int x\arc\sin x\,dx \]
		\textit{Megoldás:}
		\[ \int\left(\frac{x^2}{2}\right)'\arc\sin x\,dx=\frac{x^2}{2}\arc\sin x-\int\frac{x^2}{2}\cdot\frac{1}{\sqrt{1-x^2}}\,dx=\frac{x^2}{2}\arc\sin x + \frac{1}{2}\int\frac{-x^2+1-1}{\sqrt{1-x^2}}\,dx= \]
		\[ =\frac{x^2}{2}\arc\sin x+ \frac{1}{2}\int\sqrt{1-x^2}-\frac{1}{\sqrt{1-x^2}}\,dx=\frac{x^2}{2}\arc\sin x-\frac{1}{2}\arc\sin x+\frac{1}{2}\int\sqrt{1-x^2}\,dx=\]
		Határozzuk meg $\int\sqrt{1-x^2}\,dx$-et. Ezt már korábban meghatároztuk, azonban később meglátjuk, hogy parciális integrálás segítségével is számolható (ld. \ref{sqrt(1-x^2)}. feladat).
		\[ \int\sqrt{1-x^2}\,dx=\frac{\arcsin\left(x\right)+x\sqrt{1-x^2}}{2}+c\quad (c\in\R) \]
		Visszatérve:
		\[ =\arc\sin x\cdot\left(\frac{x^2-1}{2}\right)+\frac{\arcsin\left(x\right)+x\sqrt{1-x^2}}{4}+c=\frac{\left(2x^2-1\right)\arcsin\left(x\right)+x\sqrt{1-x^2}}{4}+c\quad (c\in\R) \]
	\end{exercise}
	\begin{exercise}
		$x\in\R$
		\[ \int\ln(1+x^2)\,dx\]
		\textit{Megoldás:}
		\[\int(x)'\ln(1+x^2)\,dx=x\ln(1+x^2)-\int x\cdot\frac{1}{1+x^2}\cdot 2x\,dx=x\ln(1+x^2)-2\int\frac{x^2-1+1}{1+x^2}\,dx= \]
		\[ =x\ln(1+x^2)-2\int1-\frac{1}{1+x^2}\,dx=x\ln(1+x^2)-2x+2\arc\tg x + c\quad (c\in\R) \]
	\end{exercise}
	\begin{exercise}
		\[ \int \sh x\cos x\,dx \]
		\textit{Megoldás:}
		\[ \int(\ch x)'\cos x\,dx=\ch x\cos x-\int\ch x(-\sin x)\,dx=\ch x\cos x +\int(\sh x)'\sin x\,dx=\]
		\[=\ch x\cos x+\left(\sh x\sin x-\int\sh x\cos x\,dx\right)\]
		
		\[\int \sh x\cos x\,dx=\frac{1}{2}\left(\ch x\cos x-\sh x\sin x\right) \]
	\end{exercise}
	\begin{exercise}
		\[ \int e^x\cos^2x\,dx \]
		\textit{Megoldás:}
		\[ \int (e^x)'\cos^2x\,dx=e^x\cos^2x-\int e^x2\cos x(-\sin x)\,dx=e^x\cos^2x+\int e^x(2\cos x\sin x)\,dx=\]
		\[=e^x\cos^2x+\int (e^x)'\sin2x\,dx=e^x\cos^2x+e^x\sin2x-\int e^x\cdot2\cdot\cos2x\,dx=\]
		\[ =e^x\cos^2x+e^x\sin2x-4\cdot\int e^x\frac{\cos2x}{2}\,dx= \]
		Határozzuk meg $\int e^x\frac{\cos2x}{2}$-t.
		\[\int e^x\frac{\cos2x+1-1}{2}\,dx=\int e^x\left(\cos^2x-\frac{1}{2}\right)\,dx=\int e^x\cos^2x\,dx-\frac{1}{2}\int e^x\,dx=\int e^x\cos^2x\,dx-\frac{1}{2}\cdot e^x  \]
		Visszatérve:
		\[ =e^x\cos^2x+e^x\sin2x+2\cdot e^x-4\cdot\int e^x\cos^2x\,dx \]
		Így az egyenlet:
		\[ \int e^x\cos^2x\,dx=\frac{1}{5}\left(e^x\cos^2x+e^x\sin2x+2\cdot e^x \right)+c=\frac{e^x(\cos^2x+\sin2x+2)}{5}+c\quad (c\in\R) \]
	\end{exercise}
	\subsection{Integrálás helyettesítéssel}
	\begin{revision}
		\[ \int f(x)\,dx\quad \overset{x:=g(t)}{=}\quad \int f(g(t))\cdot g'(t)\,dt\big|_{t=g^{-1}(x)} \]
	\end{revision}
	\begin{task}$x\in(-1,1)$ \label{sqrt(1-x^2)}
		\[\int\sqrt{1-x^2}\,dx \quad\]
		Szükségünk lesz számos új változóra. Legyen:
		\[ f(x):=\sqrt{1-x^2} \]
		\[ g(t) := \sin t := x,\quad g^{-1}(t) = \arc\sin t,\quad g'(t) = \cos t \]
		$g^{-1}$ létezik, hisz $g$ bijektív. Ezek alapján $t\in\left(-\frac{\pi}{2},\frac{\pi}{2} \right)$. Visszatérve:
		%\[\overset{t\in\left(-\frac{\pi}{2};\frac{\pi}{2}\right)}{\overset{x=\sin t=:g(t)}{\underset{g:\left(-\frac{\pi}{2};\frac{\pi}{2}\right)}{\underset{\text{bijekció}\checkmark}{\underset{f(x):=\sqrt{1-x^2}}{\underset{g'(t)=\cos t}{\underset{g^{-1}(x)=\arc\sin x}{=}}}}}}}\]
		\[\int\sqrt{1-\sin^2t}\cdot\cos t\,dt\big|_{t=\arc\sin x}=\int\sqrt{\cos^2t}\cos t\,dt\big|_{t=\arc\sin x} \]
		Az új integrál:
		\[ \int|\cos t|\cdot\cos t\,dt\quad \overset{t\in\left(-\frac{\pi}{2};\frac{\pi}{2}\right)}{=}\quad \int\cos^2t\,dt=\int\frac{1+\cos2t}{2}\,dt=\frac{t}{2}+\frac{\sin2t}{4}+c \overset{!}{=} \frac{t}{2}+\frac{2\cdot\sin t\cdot\sqrt{1-\sin^2t}}{4}+c \]
		Visszahelyettesítve:
		\[ \int\sqrt{1-x^2}\,dx=\frac{\arc\sin x}{2}+\frac{x\sqrt{1-x^2}}{2}+c\quad (c\in\R) \] 
	\end{task}
	
	\begin{exercise}A következő feladatot az $x=:\sin t$ helyettesítéssel javallott megoldani. 
		
		Legyen $x\in(0,1)$, és:
		\[ \int\frac{1}{x\cdot\sqrt{1-x^2}}\,dx \]
		\textit{Megoldás:} Legyen
		\[ f(x):=\frac{1}{x\cdot\sqrt{1-x^2}} \]
		\[ g(t) := \sin t := x,\quad g^{-1}(t) = \arc\sin t,\quad g'(t) = \cos t \]
		$g^{-1}$ létezik, hisz $g$ bijektív. Ezek alapján $t\in\left(0,\frac{\pi}{2} \right)$. Visszatérve:
		
		\[ \int\frac{1}{\sin t\cdot\sqrt{1-\sin^2t}}\cos t\,dt=\int\frac{\cos t}{\sin t|\cos t|}\,dt\quad \overset{t\in\left(0;\frac{\pi}{2}\right)}{=}\quad\int\frac{1}{\sin t}\,dt=\quad \overset{\sin2\alpha=2\sin\alpha\cos\alpha}{\underset{\text{félszögre térés}}{=}}\quad \int\frac{\sin^2\frac{t}{2}+\cos^2\frac{t}{2}}{2\sin\frac{t}{2}\cos\frac{t}{2}}\,dt=\]
		\[\frac{1}{2}\int\frac{\sin\frac{t}{2}}{\cos\frac{t}{2}}\,dt+\frac{1}{2}\int\frac{\cos\frac{t}{2}}{\sin\frac{t}{2}}\,dt=\int\frac{(\sin\frac{t}{2})'}{\sin\frac{t}{2}}\,dt-\int\frac{(\cos\frac{t}{2})'}{\cos\frac{t}{2}}\,dt=\ln\left(\sin\frac{t}{2}\right)-\ln\left(\cos\frac{t}{2}\right)+c=\ln\left(\tg\frac{t}{2}\right)+c\quad (c\in\R) \]
		Helyettesítsük vissza:
		\[\int\frac{1}{x\cdot\sqrt{1-x^2}}\,dx=\ln\left(\tg\left(\frac{\arc\sin x}{2}\right)\right)+c\quad (c\in\R) \]
	\end{exercise}
	\begin{note}
		Rövid jelölés:
		\[ \int\sqrt{1-x^2}\,dx\quad \overset{x=\sin t}{\underset{(x)'\,dx=(\sin t)'\,dt}{\underset{dx=\cos t\,dt}{=}}}\int\sqrt{1-\sin^2 t}\cos t\,dt\big|_{t=\arc\sin x} \]
	\end{note}
	\begin{exercise}
		30 integrálfeladat, 15 parciálisan, 15 helyettesítéssel (első és második szabállyal ez utóbbit vegyesen)
	\end{exercise}
	%\section{}
	\bigskip
	
	\begin{task}
		\[ \int\sqrt{1+x^2}\,dx  \]
		Vezessünk be új változót.
		\[ g(t) := \sh t := x,\quad  t\in\R \]
		\[ (x)'\,dx=(\sh t)'\,dt\quad \Rightarrow\quad dx=\ch t\,dt \]
		\[ t = \arsh x \]
		Visszatérve:
		\[ \int\sqrt{1+\sh^2 t}\ch t\,dt=\int\sqrt{\ch^2t}\cdot\ch t\,dt=\int|\overbrace{\ch x}^{\text{poz.}}|\cdot\ch t\,dt=\int\ch^2 t\,dt=\int\left(\frac{e^t+e^{-t}}{2}\right)^2\,dt=\]
		\[= \int\left(\frac{e^{2t}+2e^te^{-t}+e^{-2t}}{4}\right)= \int\left(\frac{\frac{e^{2t}+e^{-2t}}{2}+1}{2}\right)\,dt=\int\frac{\ch(2t)+1}{2}\,dt=\frac{t}{2}+\frac{1}{4}\cdot\sh(2t)+c=\]
		Bár az integráljelek eltűntek, ha ezen a ponton helyettesítenénk vissza, túl ,,ronda'' alakot kapnánk, így érdemes továbbalakítani.
		\[=\frac{t}{2}+\frac{1}{4}\cdot\frac{e^{2t}-e^{-2t}}{2}+c=\frac{t}{2}+\frac{1}{4}\cdot\frac{(e^{t})^2-(e^{-t})^2}{2}+c=\frac{t}{2}+\frac{1}{2}\cdot\frac{e^t-e^{-t}}{2}\cdot\frac{e^t+e^{-t}}{2}= \frac{t}{2}+\frac{1}{2}\sh t\ch t +c=\]
		\[=\frac{t}{2}+\frac{1}{2}\sh t\cdot\sqrt{1+\sh^2t}+c \]
		\[ \Rightarrow\int\sqrt{1+x^2}\,dx=\frac{\arsh x}{2}+\frac{1}{2}\cdot x\cdot \sqrt{1+x^2}+c \]
	\end{task}
	\begin{revision}\
		
		\begin{center}
			\text{\fbox{$\displaystyle \ch x=\frac{e^x+e^{-x}}{2}$}}
			\quad \text{\fbox{$\displaystyle \sh x=\frac{e^x-e^{-x}}{2}$}}
			\quad \text{\fbox{$\displaystyle \ch^2x-\sh^2x=1$}}
		\end{center}
		A $\ch$ grafikonját szokás \textit{láncgörbének} hívni, mert mert ha egy lánc két végét fogjuk, akkor mindig egy $\ch$ függvényt vesz fel az alakja.
		
		Az inverzekre is megállapítható pár azonosság:
		\begin{center}
			\text{\fbox{$\displaystyle \sh^{-1}x=\arsh x=\ln(x+\sqrt{x^2+1})$}}
			\quad \text{\fbox{$\displaystyle \ch^{-1}(x)=\arch(x)=$}}
		\end{center}
		Valamint a $\sin$ és $\cos$ függvényekhez hasonló azonosságok is megállapíthatóak.
		\begin{center}
			\text{\fbox{$\displaystyle \ch^2 x=\frac{1+\ch2x}{2}$}}
			\quad \text{\fbox{$\displaystyle \sh(2x)=2\sh x\ch x$}}
		\end{center}
	\end{revision}
	\begin{note}
		Ezeket levezettük, így egyből felhasználhatóak, nem kell zh-ban őket levezetni.
	\end{note}
	\begin{exercise}
		\[ \int\sqrt{x^2-1}\,dx \]
		\begin{enumerate}
			\item $x<-1,\quad x=-\ch t \quad (t<0)$
			\item $x>1,\quad x=\ch t \quad (t>0)$
		\end{enumerate}
		\textit{Megoldás \fbox{$x>1$}:} Adjuk meg a helyettesítést.
		\[ g(t):=\ch t:=x, \quad t\in\R \]
		\[ g^{-1}(t)=\arch x,\quad x\,dx=\ch t\,dt\quad \Leftrightarrow\quad (x)'\,dx=(\ch t)'\,dt\quad \Leftrightarrow\quad dx=\sh t\,dt \]
		Az ,,új'' integrál:
		\[ \int\sqrt{\ch^2t-1}\cdot\sh t\,dt=\int\sqrt{\sh^2t}\cdot\sh t\,dt\quad \overset{x>1}{=}\quad \int\sh^2t\,dt=\int\left(\frac{e^t-e^{-t}}{2}\right)^2\,dt=\]
		\[=\int\frac{e^{2t}-2\cdot e^{t}\cdot e^{-t}+e^{-2t} }{4}\,dt=\int\frac{\frac{e^{2t}+e^{-2t}}{2}-1}{2}\,dt=\int\frac{\ch2t-1}{2}\,dt=\frac{1}{4}\sh2t-\frac{t}{2}+c=\]
		\[=\frac{1}{2}\sh t\cdot\sqrt{1+\sh^2 t}-\frac{t}{2}+c\quad (c\in\R) \]
		Helyettesítsünk vissza:
		\[ \int\sqrt{x^2-1}\,dx=\frac{1}{2}\cdot x\cdot\sqrt{1+x^2}-\frac{\arch x}{2}+c\quad (c\in\R) \]
		\textit{Megoldás \fbox{$x<-1$}:} Adjuk meg a helyettesítést:
		\[ g(t):=-\ch t:=x, \quad t\in\R \]
		\[ g^{-1}(t)=-\arch x,\quad x\,dx=-\ch t\,dt\quad \Leftrightarrow\quad (x)'\,dx=(-\ch t)'\,dt\quad \Leftrightarrow\quad dx=-\sh t\,dt \]
		Innen könnyen látható, hogy a fenti megoldási módszertől alig eltérően ugyanarrra a végeredményre jutunk.
	\end{exercise}
	\begin{note}
		$\int\sqrt{ax^2+bx+c}\,dx$ típusoknál teljes négyzetté alakítás után lineáris helyettesítés javallott az alábbiak egyikébe:
		\begin{enumerate}
			\item $\sqrt{1+x^2}\,dx$
			\item $\int\sqrt{1-x^2}\,dx$
			\item $\int\sqrt{x^2-1}\,dx$
		\end{enumerate}
	\end{note}
	\begin{exercise}
		Fejezzük be a következő feladatot:
		\[ \int\sqrt{4-9x^2-6x}\,dx%\quad x\in\left(-\frac{1}{3}-\frac{\sqrt{5}}{3};\ -\frac{1}{3}+\frac{\sqrt{5}}{3}\right) 
		\]
		Alakítsuk át a gyök alatt található kifejezést.
		\[ 4-9x^2-6x=4-(9x^2+6x)=4-((3x+1)^2-1)=5-(3x+1)^2=5\cdot\left(1-\frac{(3x+1)^2}{5}\right)=5\cdot\left(1-\left(\frac{3x+1}{\sqrt{5}}\right)^2\right)\]
		Visszahelyettesítve:
		\[\Rightarrow\quad \sqrt{5}\cdot\int\sqrt{1-\left(\frac{3x+1}{\sqrt{5}}\right)^2}\,dx \]
		Javallott az $\displaystyle g(t):=\sin t:=\frac{3x+1}{\sqrt{5}}=$ helyettesítés, mellyel
		\[ \left(\frac{3x+1}{\sqrt{5}}\right)'=(\sin t)'\quad  \Leftrightarrow\quad \frac{3}{\sqrt{5}}\,dx=\cos t\,dt\quad \Leftrightarrow\quad dx=\frac{\sqrt{5}}{3}\cos t\,dt. \]
		\textit{A feladat befejezése:}	Határozzuk meg $x$-et $t$ függvényében.
		\[ \sin t=\frac{3x+1}{\sqrt 5}\quad \Leftrightarrow\quad t = \arc\sin\left(\frac{3x+1}{\sqrt{5}}\right) \]
		
		A helyettesítéssel kapott ,,új'' integrál:
		\[ \sqrt{5}\cdot\int\sqrt{1-\sin^2t}\cdot\frac{\sqrt{5}}{3}\cos t\,dt=\frac{5}{3}\cdot\int|\cos t|\cos t\,dt\]
		Tegyük fel, hogy $\cos t$ nemnegatív.
		\[ \frac{5}{3}\cdot\int\cos^2 t\,dt=\frac{5}{3}\cdot\int\frac{\cos2t+1}{2}\,dt=\frac{5}{3}\cdot\left(\frac{t}{2}+\frac{\sin2t}{4}\right)+c=\frac{5}{3}\cdot\left(\frac{t}{2}+\frac{2\sin t\cos t}{4}\right)+c= \]
		\[ =\frac{5}{3}\cdot\left(\frac{t}{2}+\frac{\sin t\cdot\sqrt{(1-\sin^2t)}}{2}\right)+c \quad (c\in\R)\]
		Visszahelyettesítve:
		\[ \int\sqrt{4-9x^2-6x}\,dx=\frac{5}{3}\cdot\left(\frac{\arc\sin\left(\frac{3x+1}{\sqrt 5}\right)}{2}+\frac{\frac{3x+1}{\sqrt{5}}\cdot\sqrt{1-\left(\frac{3x+1}{\sqrt{5}}\right)^2}}{2}\right)+c=\]
		\[ =\frac{5\cdot\arc\sin\left(\frac{3x+1}{\sqrt 5}\right)+\sqrt{5}\cdot(3x+1)\cdot\sqrt{1-\left(\frac{3x+1}{\sqrt{5}}\right)^2}}{6}+c=\frac{5\cdot\arc\sin\left(\frac{3x+1}{\sqrt 5}\right)+(3x+1)\cdot\sqrt{5-(3x+1)^2}}{6}+c \]
		Ahol $c\in\R$.
	\end{exercise}
	\subsection{Racionális törtfüggvények integrálása}
	\subsubsection{Elemi törtek integrálása}
	Ebben a fejezetben minden példa 1-1 altípusra mutat rá.
	\begin{example}
		$(a,b,x\in\R,\quad a\not=0;\quad 1\leq n\in\N)$
		\[ \int\frac{1}{(ax+b)^n}\,dx=? \]
	\end{example}
	\begin{task}
		\[ \int\frac{1}{(2x+1)^7}\,dx=\frac{1}{2}\cdot\int(2x+1)'\cdot(2x+1)^{-7}\,dx=\frac{1}{2}\cdot\frac{(2x+1)^{-6}}{-6}+c\quad (c\in\R) \]
	\end{task}
	\begin{task}
		\[ \int\frac{1}{(3x-5)}\,dx=\frac{1}{3}\cdot\int\frac{(3x-5)'}{3x-5}\,dx=\frac{1}{3}\ln|\overbrace{3x-5}^{x>\frac{5}{3}}|+c=\frac{1}{3}\ln(3x+5)+x\quad (c\in\R) \]
	\end{task}
	\begin{example}$(a,b,c,d,e,f,x\in\R,\quad a\not=0)$
		\[ \int\frac{ex+f}{ax^2+bx+c}\,dx \]
		A megoldási módszer változhat a diszkrimináns paritásától függően.
	\end{example}
	\begin{task}$x\in(-4;2)$
		\[ \int\frac{3x+1}{x^2+2x-8}\,dx= \]
		Első lépés, nevezőt alakítsuk szorzattá, kihasználván azt, hogy a diszkriminánsa pozitív.
		\[ ax^2+bx+c=(x-x_1)(x-x_2)\quad \Leftrightarrow\quad x^2+2x-8=(x-2)(x+4) \]
		\[ f(x):=\frac{3x+1}{(x-2)(x+4)}=\frac{A}{x-2}+\frac{B}{x+4} \]
		Végezzük el a parciális törtre bontást.
		\[ (x-2)(x+4)\quad \Leftrightarrow\quad 3x+1=A(x+4)+B(x-2)\quad \Leftrightarrow\quad 3x+1=(A+B)x+4A-2B \]
		Mindkét oldal $x$-nek polinomja ezért egyenlőség pontosan akkor teljesül ha a megfelelő fokszámú tafok együtthatói megegyeznek. (A módszer neve: ,,egyenlő együtthatók módszere'')
		\begin{align*}
			x^1\quad  \text{együtthatói:}& \quad 3=A+B\\
			x^0\quad  \text{együtthatói:}& \quad 1=4A-2B
		\end{align*}
		Ez alapján:
		\[ A=\frac{7}{6}\quad \text{és}\quad B=\frac{11}{6} \]
		\[ \Rightarrow\int f(x)\,dx=\int\left(\frac{7}{6}\cdot\frac{1}{x-2}+\frac{11}{6}\cdot\frac{1}{x+4}\right)\,dx=\frac{7}{6}\int\frac{1}{x-2}\,dx+\frac{11}{6}\cdot\int\frac{1}{x+4}\,dx=\frac{7}{6}\ln|x-2|+\frac{11}{6}\ln|x+4|+c=\]
		\[\overset{x\in(-4;2)}{=}\quad \frac{7}{6}\cdot\ln(-x+2)+\frac{11}{6}\ln(x+4)+c\quad (c\in\R) \]
	\end{task}
	\begin{center}
		\textit{,,[\dots]tegyük fel hogy parciális törtfelbontó vagy''}
		
		\smallskip
		/Filipp/
	\end{center}
	\begin{task}$x\in\R$
		\[\int\frac{3x+1}{x^2-x+1}\,dx= \]
		Mivel a nevező diszkriminánsa negatív, nem léteznek valós gyökei, és az előző feladatban látott megoldási módszer nem működik.
		
		Határozzuk meg a nevező deriváltját.
		\[ (x^2-x+1)'=2x-1 \]
		Visszatérve:
		\[ =\frac{3}{2}\cdot\int\frac{2x+\frac{2}{3}}{x^2-x+1}\,dx=\frac{3}{2}\cdot\int\frac{2x-1+\frac{2}{3}+1}{x^2-x+1}\,dx=\frac{3}{2}\cdot\int\frac{2x-1}{x^2-x+1}\,dx+\frac{3}{2}\cdot\frac{5}{3}\cdot\overbrace{\int\frac{1}{x^2-x+1}\,dx}^{=:I(x)}= \] 
		\[ =\frac{3}{2}\ln(\overbrace{x^2-x+1}^{\substack{\text{nincs valós gyök,}\\\text{ poz. a főegyüttható,}\\\text{ így biztosan poz.}}})+\frac{5}{2}I(x) \]
		Ahol:
		\[ I(x)=\int\frac{1}{x^2-x+1}\,dx= \]
		Próbáljuk átalakítani a nevezőt úgy, hogy egy \label{husi}
		\begin{center}
			\textit{,,$\arc\tg$-re éhes alakra hozzuk''}
			\smallskip
			
			/Filipp/
		\end{center}
		a törtet.
		\[ x^2-x+1=\left(x-\frac{1}{2}\right)^2-\frac{1}{4}+1=\left(x-\frac{1}{2}\right)^2+\frac{3}{4}= \frac{3}{4}\cdot\left[1+\frac{4\cdot\left(x-\frac{1}{2}\right)^2}{3}\right]=\frac{3}{4}\left[1+\frac{(2x-1)^2}{3}\right]=\]
		\[=\frac{3}{4}\cdot\left[1+\left(\frac{2x-1}{\sqrt{3}}\right)^2\right] \]
		Visszatérve:
		\[ =\frac{4}{3}\int\frac{1}{1+\left(\frac{2x-1}{\sqrt{3}}\right)^2}\,dx=\frac{4}{3} \cdot\frac{\arctg\frac{2x-1}{\sqrt{3}}}{\frac{2}{\sqrt{3}}}+c \]
		Visszaírva az eredeti integrálba:
		\[ \frac{3}{2}\cdot\ln(x^2-x+1)+\frac{5}{2}\cdot\frac{4}{3}\cdot\frac{\sqrt{3}}{2}\cdot\arc\tg\frac{2x-1}{\sqrt{3}}+c\quad (c\in\R) \]
	\end{task}
	\begin{note}
		Ezt a típust $\frac{f'}{f}$ + $\arc\tg$-re visszavezetésnek hívjuk
	\end{note}
	\begin{task}$x>-1$
		\[ \int\frac{x^4+3x^3+x^2+1}{x^3+1}\,dx= \]
		Ha a számláló foka nagyobb mint a nevezőé, polinom osztást szokás alkalmazni első lépésben, azaz
		\[ \int\frac{P(x)}{Q(x)}\,dx\quad \text{ha}\quad \deg P\geq\deg Q\quad \Rightarrow\quad \text{polinomosztás} \]
		$(x^4+3x^3+x^2+1):(x^3+1)=x+3$ és a maradék $x^2-x-2$
		\[ \int\left(x+3+\underbrace{\frac{x^2-x-2}{x^3+1}}_{\text{valódi tört}}\right)\,dx=\frac{x^2}{2}+3x+J(x)= \]
		ahol $J(x)=\int\frac{x^2-x-2}{x^3+1}\,dx$.
		\[ =\int\frac{x^2-x-2}{(x+1)\underbrace{(x^2-x+1)}_{D<0!}}\,dx=\int\left(\frac{A}{x+1}+\frac{Bx+C}{x^2-x+1}\right)\,dx= \]
		Végezzük el a törte bontást:
		\[ x^2-x-2=A(x^2-x+1)+(Bx+C)(x+1) \]
		\[ x^2-x-2=(A+B)x^2+(-A+B+C)x+A+C \]
		\vspace{-7mm}
		\begin{align*}
			x^2 \quad \text{együtthatója:}&\hspace{7.8mm} 1=A+B\\
			x^1 \quad \text{együtthatója:}& \quad -1=-A+B+C\\
			x^0 \quad \text{együtthatója:}&\quad  -2=A+C
		\end{align*}
		Megállítható hogy $A=0,\quad B=1,\quad C=-2$.
		\[=\int\frac{x-2}{x^2-x+1}\,dx=\frac{1}{2}\cdot\int\frac{2x-1-3}{x^2-x+1}\,dx=\frac{1}{2}\cdot\int\frac{2x-1}{x^2-x+1}\,dx-\frac{3}{2}\cdot\overbrace{\int\frac{1}{x^2-x+1}\,dx}^{=:I(x)}= \frac{1}{2}\ln(x^2-x+1)-\frac{3}{2}I(x) \]
		ld. előző példa a befejezésért.
	\end{task}
	\begin{exercise}$x>1$
		\[\int\frac{x^4+3x^3-x^2+8}{x^3-1}\,dx  \] 
		\textit{Megoldás:}
		Osszuk le a nevezőt a számlálóval.
		\[ (x^4+3x^3-x^2+8):(x^3-1)=(x^3-1)\overbrace{(x+3)}^{\text{hányados}}+\overbrace{(-x^2+x+11)}^{\text{maradék}} \]
		\[\int\frac{(x^3-1)(x+3)+(-x^2+x+11)}{x^3-1}\,dx=\int\left(x+3+\frac{-x^2+x+11}{x^3-1}\right)\,dx=\frac{x^2}{2}+3x+I(x) \]
		\[ I(x)=\int\frac{-x^2+x+11}{x^3-1}\,dx=\int\frac{-x^2+x+11}{(x-1)(x^2+x+1)}\,dx=\int\left(\frac{A}{x-1}+\frac{Bx+C}{x^2+x+1}\right)\,dx= \]
		Végezzük el a parciális törtre bontást egyenlő együtthatók módszerével.
		\[ -x^2+x+11=A(x^2+x+1)+(Bx+C)(x-1) \]
		\[ -x^2+x+11=(B+A)x^2+(A+C-B)x+(A-C) \]
		
		\vspace{-7mm}
		\begin{align*}
			x^2 \quad \text{együtthatója:}&\quad          -1=B+A\\
			x^1 \quad \text{együtthatója:}&\hspace{7.8mm}  1=A+C-B\\
			x^0 \quad \text{együtthatója:}&\hspace{6.1mm}  11=A-C
		\end{align*}
		Ez alapján $A= \frac{11}{3},\quad B=-\frac{14}{3},\quad C=-\frac{22}{3}.$
		\[=\int\left(\frac{11}{3}\cdot\frac{1}{x-1}+\left(-\frac{1}{3}\right)\cdot\frac{14x+22}{x^2+x+1}\right)\,dx=\frac{11}{3}\cdot\ln|x-1|-\frac{7}{3}\cdot\int\frac{2x+1+\frac{22}{7}-1}{x^2+x+1}\,dx= \]
		\[ =\frac{11}{3}\cdot\ln|x-1|-\frac{7}{3}\cdot\ln|x^2+x+1|-\frac{7}{3}\cdot\int\frac{\frac{22}{7}-1}{x^2+x+1}\,dx=\frac{11}{3}\cdot\ln|x-1|-\frac{7}{3}\cdot\ln|x^2+x+1|-\frac{7}{3}\cdot\frac{15}{7}\cdot\int\frac{1}{x^2+x+1}\,dx \]
		Határozzuk meg $\int\frac{1}{x^2+x+1}\,dx$-t. Ehhez próbáljunk meg egy $\arc\tg$ deriváltjához megfelelő alakot előállítani.
		\[ x^2+x+1=\frac{3}{4}\left[1+\left(\frac{2x+1}{\sqrt{3}}\right)^2\right] \]
		Részletesebb levezetés a \ref{husi} feladatban található.
		\[ \int\frac{1}{x^2+x+1}\,dx=\frac{4}{3}\cdot\int\frac{1}{1+\left(\frac{2x+1}{\sqrt{3}}\right)^2}\,dx=\frac{4}{3}\cdot\frac{\sqrt{3}\cdot\arc\tg\left(\frac{2x+1}{\sqrt{3}}\right)}{2}+c\quad (c\in\R)  \]
		Így a feladat megoldása:
		\[ \int\frac{x^4+3x^3-x^2+8}{x^3-1}\,dx =\frac{x^2}{2}+3x+ \frac{11}{3}\cdot\ln|x-1|-\frac{7}{3}\cdot\ln|x^2+x+1|-5\cdot\frac{4}{3}\cdot\frac{\sqrt{3}\cdot\arc\tg\left(\frac{2x+1}{\sqrt{3}}\right)}{2}+c= \]
		\[=\frac{x^2}{2}+3x+ \frac{1}{3}\cdot\ln\left(\frac{|x-1|^{11}}{|x^2+x+1|^7}\right)-\frac{10\sqrt{3}}{3}\cdot\arc\tg\left(\frac{2x+1}{\sqrt{3}}\right)+c\quad (c\in\R)\]
	\end{exercise}
	\begin{task}$x>3$
		\[ \int\frac{2x-1}{(x+2)(x-3)^2}\,dx=\int\left(\frac{A}{x+2}+\frac{B}{x-3}+\frac{C}{(x-3)^2}\right)\,dx \]
		\[ \Rightarrow\quad 2x-1=A(x-3)^2+B(x+2)(x-3)+C(x+2) \]
		HF: Megoldás egyenlő együtthatókkal.
		
		A következő megoldási módszert ,,értékadás''-nak hívjuk.
		\[ x=3\quad \Rightarrow\quad 5=5C\quad \Rightarrow \quad C=1 \]
		\[ x=-2\quad \Rightarrow\quad -5=25A\quad \Rightarrow\quad A=-\frac{1}{5} \]
		\[ x=4\quad \Rightarrow\quad 7=-\frac{1}{5}+6B+6\quad \Rightarrow\quad B=\frac{1}{5} \]
		\[ \Rightarrow\quad -\frac{1}{5}\int\frac{1}{x+2}\,dx+\frac{1}{5}\int\frac{1}{x-3}\,dx+\int\frac{1}{(x-3)^2}\,dx=-\frac{1}{5}\ln(x+2)+\frac{1}{5}\ln(x-3)-\frac{1}{x-3}+c= \]
		\[ = \frac{1}{5}\ln\frac{x-3}{x+2}-\frac{1}{x-9}+c\quad (c\in\R) \]
	\end{task}
	Házi feladat: 10 db. racionális tört integrál. (Gyemidovicsban 1866. feladattól)
	%\section{}
	\bigskip
	
	Egy darab elemi törttípus maradt meg, amit nem vettünk.
	\begin{example}$(x\in\R)$
		\[ \int\frac{1}{(ax^2+bx+c)^n}\,dx \]
		Ahol a diszkrimináns negatív, és $2\leq n\in\N$. Ezeket hogyan számolhatnánk? Ha egy alkalmas helyettesítéssel egy ilyen alakra tudjuk hozni:
		\[ I_n(x):=\int\frac{1}{(1+x^2)^n}\,dx, \]
		akkor meg tudunk adni $n$-re egy rekurziót, mellyel meg tudjuk határozni ezeket az integrálokat is. Vezessük ezt le. $n=1$-re:
		\[I_1(x)=\arc\tg x+c\quad (c\in\R) \]
		Határozzuk meg az $n.$ elemet.
		\[ I_{n-1}(x)=\int(x)'\cdot(1+x^2)^{-n+1}\,dx\quad \overset{\text{p.i.}}{=}\quad x\cdot\frac{1}{(1+x^2)^{n-1}}-\int \left(x\cdot(-n+1)\cdot(1+x^2)^{-n}\cdot2x\right)\,dx \]
		\[ \Rightarrow\quad I_{n-1}(x)=\frac{x}{(1+x^2)^{n-1}}+2(n-1)\int\frac{x^2}{(1+x^2)^n}\,dx=\frac{x}{(1+x^2)^{n-1}}+2(n-1)\int\frac{x^2+1-1}{(1+x^2)^n}\,dx= \]
		\[ \Rightarrow\quad I_{n-1}(x)=\frac{x}{(1+x^2)^{n-1}}+2(n-1)\cdot I_{n-1}(x)-2(n-1)\cdot I_n(x) \]
		\[ \Rightarrow\quad I_n(x)=\frac{1}{2(n-1)}\cdot\frac{x}{(1+x^2)^{n-1}}+\frac{2n-3}{2(n-1)}\cdot I_{n-1}(x) \]
		Ezzel kaptunk egy rekurzív integrál sorozatot, mellyel magasabb hatványokat is könnyen számolhatunk.
	\end{example}
	\begin{note}
		Speciális esetben, ha $n=2$:
		\[ I_2(x)=\int\frac{1}{(1+x^2)^2}\,dx=\frac{1}{2}\cdot\frac{x}{1+x^2}+\frac{1}{2}\arc\tg x+c\quad (c\in\R) \]
	\end{note}
	\begin{exercise}
		\[ I_3(x)=\int\frac{1}{(1+x^2)^3}\,dx \]
		\textit{Levezetés:}
		\[ I_3(x)=\frac{1}{4}\cdot\frac{x}{(1+x^2)^2}+\frac{3}{4}\cdot I_2(x) =\frac{1}{4}\cdot\frac{x}{(1+x^2)^2}+\frac{3}{4}\cdot \left(\frac{1}{2}\cdot\frac{x}{1+x^2}+\frac{1}{2}\arc\tg x\right)+c\quad (c\in\R) \]
	\end{exercise}
	\begin{note}
		Nem érdemes a formulát megjegyezni, érdemesebb a négyzetre vonatkozót észben tartani, és abból a köbösre vonatkozót levezetni.
	\end{note}
	\begin{note}
		Várhatóan minden feladat legfeljebb a második hatványra vonatkozó alakot fogja számon kérni.
	\end{note}
	\begin{exercise}
		\[ \int\frac{3x+1}{(x^2+4x+5)^2}\,dx=\frac{3}{2}\int\frac{2x+\frac{2}{3}}{(x^2+4x+5)^2}\,dx=\frac{3}{2}\int\frac{2x+4+\frac{2}{3}-4}{(x^2+4x+5)^2}\,dx=\]
		\[=\frac{3}{2}\int\frac{2x+4}{(x^2+4x+5)^2}\,dx+\frac{3}{2}\int\frac{\frac{2}{3}-4}{(x^2+4x+5)^2}\,dx=\frac{3}{2}\int(x^2+4x+5)'(x^2+4x+5)^{-2}\,dx-\frac{3}{2}\cdot\frac{10}{3}\cdot\overbrace{\int\frac{1}{(x^2+4x+5)^2}\,dx}^{=J(x)}=\]
		\[=\frac{3}{2}\cdot\frac{x^2+4x+5}{-1}-5\cdot J(x), \]
		Határozzuk meg $J(x)$-et.
		\[J(x)=\int\frac{1}{(1+(x+2)^2)^2}\,dx\quad \overset{x+2=t}{\underset{dx=dt}{=}}\quad \int\frac{1}{(1+t^2)^2}\,dt\]
		\textit{A feladat befejezése:}
		\[ J(x) = \int\frac{1}{(1+t^2)^2}=\frac{1}{2}\cdot\frac{t}{1+t^2}+\frac{1}{2}\arc\tg t+c\quad (c\in\R) \]
		Visszahelyettesítve:
		\[\int\frac{3x+1}{(x^2+4x+5)^2}\,dx=\frac{3}{2}\cdot\frac{x^2+4x+5}{-1}-5\cdot\left(\frac{1}{2}\cdot\frac{x+2}{1+(x+2)^2}+\frac{1}{2}\arc\tg (x+2)\right)+c = \]
		\[ =-\frac{3(x^2+4x+5)}{2}-\frac{5(x+2)}{2+2(x+2)^2}-\frac{5}{2}\arc\tg(x+2) + c=\]
		\[=\frac{5\arc\tg(x+2)-3(x^2+4x+5)}{2}-\frac{5x+10}{2x^2+8x+10} + c \quad (c\in\R)\]
	\end{exercise}
	\begin{note}
		Egy másfajta rekurzió is megállapítható, a korábbihoz hasonlóan.
		\[ \int\frac{1}{(1+x^2)^n}\,dx= \]
		Vezessünk be egy jelölést.
		\[ \tg t:=x,\quad t\in\left(-\frac{\pi}{2};\frac{\pi}{2}\right) \]
		\[ dx=\frac{1}{\cos^2t}dt,\quad t=\arc\tg x \]
		Visszatérve:
		\[ =\int\frac{1}{\frac{1}{\cos^{2n}t}}\cdot\frac{1}{\cos^2t}\,dt=\int\cos^{2n-2}t\,dt \]
	\end{note}
	\begin{exercise}
		Rekurzió levezetése.
	\end{exercise}
	\begin{note}
		\[ 1+\tg^2=\frac{1}{\cos^2t} \]
	\end{note}
	\begin{note}
		A parciális törtekre bontás algoritmusa szerint a törtekre bontás után minden olyan nevezőhöz, melynek nincs valós gyöke, egy elsőfokú számlálót kell meghatároznunk.
	\end{note}
	\begin{task}
		\[ \frac{x+1}{(x+3)^2(x^2+1)^2}=\frac{A}{x+3}+\frac{B}{(x+3)^2}+\frac{Cx+B}{x^2+1}+\frac{Ex+F}{(x^2+1)^2} \]
	\end{task}
	\begin{exercise}
		\[ \frac{3x^2+1}{x^2(x^2+x+1)}=\frac{A}{x}+\frac{B}{x^2}+\frac{Cx+D}{x^2+x+1} \]
		\textit{A tört integráljának meghatározása:}
		\[ \int\frac{3x^2+1}{x^2(x^2+x+1)}\,dx=\int \left(\frac{A}{x}+\frac{B}{x^2}+\frac{Cx+D}{x^2+x+1}\right)\,dx= \]
		Határozzuk meg $A,B,C,D\in\R$-t \textit{egyenlő együtthatók} módszerével.
		\begin{align*}
			3x^2+1 \quad &=\quad  A\cdot x(x^2+x+1)+B\cdot(x^2+x+1)+(Cx+D)\cdot x^2=\\
						 &=\quad  Ax^3+Ax^2+Ax+Bx^2+Bx+B+Cx^3+Dx^2=\\
						 &=\quad  (A+C)x^3+(A+B+D)x^2+(A+B)x+B
		\end{align*}
		Ez alapján rendere $A = -1, \quad B = 1, \quad C = 1, \quad D = 3$.
		\[ =\int \left(-\frac{1}{x}+\frac{1}{x^2}+\frac{x+3}{x^2+x+1}\right)\,dx=-\ln|x|-\frac{1}{x}+I(x), \]
		ahol $I(x)=\int\frac{x+3}{x^2+x+1}\,dx$.
		\[ I(x)=\int\frac{x+3}{x^2+x+1}\,dx=\frac{1}{2}\cdot\int\frac{2x+1+5}{x^2+x+1}\,dx=\frac{1}{2}\cdot\int\frac{(x^2+x+1)'}{x^2+x+1}\,dx+\frac{5}{2}\cdot\int\frac{1}{x^2+x+1}\,dx= \]
		Hozzuk az utolsó törtet egy \textit{,,$\arc\tg$-re éhes''} alakra.
		\[ x^2+x+1=\left(x+\frac{1}{2}\right)^2+\frac{3}{4}=\frac{3}{4}\cdot\left[1+\frac{4\left(x+\frac{1}{2}\right)^2}{3}\right]=\frac{3}{4}\cdot\left[1+\left(\frac{2x+1}{\sqrt{3}}\right)^2\right] \]
		Visszatérve:
		\[=\frac{1}{2}\cdot\ln|x^2+x+1|+\frac{5}{2}\cdot\frac{4}{3}\cdot\int\frac{1}{1+\left(\frac{2x+1}{\sqrt{3}}\right)^2}\,dx=\frac{1}{2}\cdot\ln|x^2+x+1|+\frac{10}{3}\cdot\frac{\arc\tg\left(\frac{2x+1}{\sqrt{3}}\right)}{\frac{2}{\sqrt{3}}}+c=\]
		\[=\frac{1}{2}\cdot\ln|x^2+x+1|+\frac{5\cdot\arc\tg\left(\frac{2x+1}{\sqrt{3}}\right)}{\sqrt{3}}+c\quad (c\in\R) \]
		Az eredeti integrál így:
		\[ \int\frac{3x^2+1}{x^2(x^2+x+1)}\,dx=-\ln|x|-\frac{1}{x}+\frac{1}{2}\cdot\ln|x^2+x+1|+\frac{5\cdot\arc\tg\left(\frac{2x+1}{\sqrt{3}}\right)}{\sqrt{3}}+c=\]
		\[= \frac{1}{2}\cdot\left(\ln\frac{|x^2+x+1|}{|x|^2}\right)-\frac{1}{x}+\frac{5\cdot\arc\tg\left(\frac{2x+1}{\sqrt{3}}\right)}{\sqrt{3}}+c \quad (c\in\R) \]
	\end{exercise}
	\begin{exercise}
		\[ \frac{1}{x^4-1}=\frac{1}{(x^2-1)(x^2+1)}=\frac{1}{(x-1)(x+1)(x^2+1)}=\frac{A}{x-1}+\frac{B}{x+1}+\frac{Cx+D}{x^2+1} \]
		\textit{A fenti tört integráltja:}
		\[\int\frac{1}{(x-1)(x+1)(x^2+1)}\,dx=\int\left(\frac{A}{x-1}+\frac{B}{x+1}+\frac{Cx+D}{x^2+1}\right)\,dx= \]
		Határozzuk meg $A,B,C,D\in\R$-t \textit{egyenlő együtthatók} és \textit{értékadás} módszerek keverékével.
		\[ 1=A(x+1)(x^2+1)+B(x-1)(x^2+1)+(Cx+D)\overbrace{(x-1)(x+1)}^{x^2-1} \]
		$A$ és $B$ kényelmesen meghatározható \textit{értékadás}sal.
		\begin{align*}
			x=-1\quad \Rightarrow&\quad 1=B\cdot(-2)\cdot2\quad \Rightarrow\quad B=-\frac{1}{4}\\
			x=1\quad \Rightarrow&\quad 	1=A\cdot 2\cdot 2\quad \Rightarrow\quad A=\frac{1}{4}		
		\end{align*}
		A kényelmetlen számolások elkerülése végett $C$-t és $D$-t \textit{egyenlő együtthatók} segítségével határozzuk meg.
		\begin{align*}
			1\quad &=\quad A(x+1)(x^2+1)+B(x-1)(x^2+1)+(Cx+D)(x^2-1)\\
				   &=\quad Ax^3+Ax^2+Ax+A+Bx^3-Bx^2+Bx-B+Cx^3-Cx+Dx^2-D\\
				   &=\quad (A+B+C)x^3+(A-B+D)x^2+(A+B-C)x+A-B-D
		\end{align*}
		Kiolvasható, hogy $0=A-B+D$, amiből következik $D=-\frac{1}{2}$, valamint $0=A+B-C$-ből következik $C=0$. Térjünk vissza:
		\[ =\frac{1}{4}\cdot\int\frac{1}{x-1}\,dx-\frac{1}{4}\cdot\int\frac{1}{x+1}\,dx-\frac{1}{2}\int\frac{1}{x^2+1}\,dx=\frac{1}{4}\ln|x-1|-\frac{1}{4}\ln|x+1|-\frac{1}{2}\arc\tg(x)+c\quad (c\in\R) \]
	\end{exercise}
	\begin{exercise}$(x\in\R)$
		\[ \int\frac{1}{1+x^4}\,dx \]
		\textit{Megoldás:} 
		\[ \int\frac{1}{(1+x^2)^2-2x^2}\,dx=\int\frac{1}{(x^2+\sqrt{2}x+1)(x^2-\sqrt{2}x+1) }\,dx=\int\left(\frac{Ax+B}{x^2+\sqrt{2}x+1}+\frac{Cx+D}{x^2-\sqrt{2}x+1}\right)\,dx= \]
		Határozzuk meg $A,B,C,D\in\R$-t \textit{egyenlő együtthatók} módszerével.
		\begin{align*}
			1 \quad =&\quad (Ax+B)(x^2-\sqrt{2}x+1)+(Cx+D)(x^2+\sqrt{2}x+1)=\\
					=&\quad Ax^3+Bx^2-A\cdot\sqrt{2}x^2-B\cdot\sqrt{2}x+Ax+B+\\
					&\quad +Cx^3+Dx^2+C\cdot\sqrt{2}x^2+D\cdot\sqrt{2}x+Cx+D=\\ 
					=&\quad (A+C)x^3+(B-A\cdot\sqrt{2}+D+C\cdot\sqrt{2})x^2+\\
					&\quad +(A-B\cdot\sqrt{2}+C+D\cdot\sqrt{2})x+B+D
 		\end{align*}
 		Azaz, az átláthatóság kedvéért:
 		\begin{align*}
	 		x^3 \quad \text{együtthatója:}&\quad  0=A+C\\
	 		x^2 \quad \text{együtthatója:}&\quad  0=B-A\cdot\sqrt{2}+D+C\cdot\sqrt{2}\\
	 		x^1 \quad \text{együtthatója:}&\quad  0=A-B\cdot\sqrt{2}+C+D\cdot\sqrt{2}\\
	 		x^0 \quad \text{együtthatója:}&\quad  1=B+D
 		\end{align*}
 		Ezekből meghatározható rendre hogy $A=\frac{1}{2\sqrt{2}},\quad B=\frac{1}{2},\quad C=-\frac{1}{2\sqrt{2}},\quad D=\frac{1}{2}$.
 		\[=\int\left(\frac{\frac{1}{2\sqrt{2}}\cdot x+\frac{1}{2}}{x^2+\sqrt{2}x+1}+\frac{-\frac{1}{2\sqrt{2}}\cdot x+\frac{1}{2}}{x^2-\sqrt{2}x+1}\right)\,dx=\frac{1}{2\sqrt{2}}\cdot\overbrace{\int\frac{x+\sqrt{2}}{x^2+\sqrt{2}x+1}\,dx}^{=:I(x)}-\frac{1}{2\sqrt{2}}\cdot\overbrace{\int\frac{x-\sqrt{2}}{x^2-\sqrt{2}x+1}\,dx}^{=:J(x)}= \]
 		Egyesével haladván, határozzuk meg $I(x)$-et.
 		\[ I(x)=\int\frac{x+\sqrt{2}}{x^2+\sqrt{2}x+1}\,dx=\frac{1}{2}\cdot\int\frac{2x+2\sqrt{2}}{x^2+\sqrt{2}x+1}\,dx=\frac{1}{2}\cdot\int\frac{(x^2+\sqrt{2}x+1)'+\sqrt{2}}{x^2+\sqrt{2}x+1}\,dx=\]
 		\[=\frac{1}{2}\cdot\ln|\overbrace{x^2+\sqrt{2}x+1}^{>0}|+\frac{\sqrt{2}}{2}\cdot\int\frac{1}{x^2+\sqrt{2}x+1}\,dx= \]
 		Hozzuk a nevezőt egy \textit{,,$\arc\tg$-re éhes alakra''}:
 		\[ x^2+\sqrt{2}x+1=\left(x+\frac{\sqrt{2}}{2}\right)^2+1-\frac{2}{4}=\frac{1}{2}\cdot\left[1+2\left(x+\frac{\sqrt{2}}{2}\right)^2\right]=\frac{1}{2}\cdot\left[1+\left(\sqrt{2}x+1\right)^2\right] \]
 		Visszatérve:
 		\[ =\frac{1}{2}\cdot\ln(x^2+\sqrt{2}x+1)+\sqrt{2}\cdot\int\frac{1}{1+\left(\sqrt{2}x+1\right)^2}\,dx=\frac{1}{2}\cdot\ln(x^2+\sqrt{2}x+1)+\frac{\sqrt{2}}{\sqrt{2}}\cdot\arc\tg(\sqrt{2}x+1)+c\quad (c\in\R) \]
 		\[ I(x)=\frac{1}{2}\cdot\ln(x^2+\sqrt{2}x+1)+\arc\tg(\sqrt{2}x+1)+c\quad (c\in\R) \]
 		Határozzuk meg $J(x)$-et.
 		\[ J(x)=\int\frac{x-\sqrt{2}}{x^2-\sqrt{2}x+1}\,dx=\frac{1}{2}\cdot\ln|\overbrace{x^2-\sqrt{2}x+1}^{>0}|-\frac{\sqrt{2}}{2}\cdot\int\frac{1}{x^2-\sqrt{2}x+1}\,dx=\]
 		Hozzuk a nevezőt egy \textit{,,$\arc\tg$-re éhes alakra''}:
 		\[ x^2-\sqrt{2}x+1=\left(x-\frac{\sqrt{2}}{2}\right)^2+1-\frac{2}{4}=\frac{1}{2}\cdot\left[1+2\left(x-\frac{\sqrt{2}}{2}\right)^2\right]=\frac{1}{2}\cdot\left[1+\left(\sqrt{2}x-1\right)^2\right] \]
 		Visszatérve:
 		\[=\frac{1}{2}\cdot\ln(x^2-\sqrt{2}x+1)-\sqrt{2}\cdot\int\frac{1}{1+\left(\sqrt{2}x-1\right)^2}\,dx=\frac{1}{2}\cdot\ln(x^2-\sqrt{2}x+1)+\arc\tg(\sqrt{2}x-1)+c\quad (c\in\R) \]
 		\[ J(x)=\frac{1}{2}\cdot\ln(x^2-\sqrt{2}x+1)+\arc\tg(\sqrt{2}x-1)+c\quad (c\in\R) \]
 		Rakjuk össze az eddigi eredményeinket:
 		\[ \int\frac{1}{1+x^4}\,dx =\]
 		\[=\frac{1}{2\sqrt{2}}\cdot\left(\frac{1}{2}\cdot\ln(x^2+\sqrt{2}x+1)+\arc\tg(\sqrt{2}x+1)\right)-\frac{1}{2\sqrt{2}}\cdot\left(\frac{1}{2}\cdot\ln(x^2-\sqrt{2}x+1)+\arc\tg(\sqrt{2}x-1)\right)+c= \]
 		\[=\frac{\frac{1}{2}\cdot\ln(x^2+\sqrt{2}x+1)+\arc\tg(\sqrt{2}x+1)-\frac{1}{2}\cdot\ln(x^2-\sqrt{2}x+1)-\arc\tg(\sqrt{2}x-1)}{2\sqrt{2}}+c\quad (c\in\R) \]
% 		
% 		\[ =\frac{1}{2\sqrt{2}}\cdot I(x)-\frac{1}{2\sqrt{2}}\cdot\int\frac{x+\sqrt{2}-\sqrt{2}-\sqrt{2}}{x^2-\sqrt{2}x+1}\,dx=\]
% 		\[ =\frac{1}{2\sqrt{2}}\cdot I(x)-\frac{1}{2\sqrt{2}}\cdot\left(\int\frac{x+\sqrt{2}}{x^2+\sqrt{2}x+1}\,dx-\int\frac{2\sqrt{2}}{x^2+\sqrt{2}x+1}\,dx\right)=\]
% 		\[ =\frac{1}{2\sqrt{2}}\cdot I(x)-\frac{1}{2\sqrt{2}}\cdot I(x)+\frac{1}{2\sqrt{2}}\cdot\int\frac{2\sqrt{2}}{x^2+\sqrt{2}x+1}\,dx= \int\frac{1}{x^2+\sqrt{2}x+1}\,dx \]
	\end{exercise}
	\begin{exercise}$(x\in\R)$
		\[ \int\frac{x}{1+x^4}\,dx \]
		\textit{Megoldás:} 
		\[ \R_0^+\ni t:=x^2,\quad x=\sqrt{t},\quad dx=\frac{1}{2\sqrt{t}}\,dt \]
		Helyettesítsünk be:
		\[ \int\frac{x}{1+x^4}\,dx=\int\frac{\sqrt{t}}{1+t^2}\cdot\frac{1}{2\sqrt{t}}\,dt=\int\frac{1}{2(1+t^2)}\,dt=\frac{1}{2}\cdot\int\frac{1}{1+t^2}\,dt=\frac{1}{2}\cdot\arc\tg(t)+c\quad (c\in\R) \]
		Az eredeti integrál így:
		\[ \int\frac{x}{1+x^4}\,dx=\frac{1}{2}\cdot\arc\tg(x^2)+c\quad (c\in\R) \]
	\end{exercise}
	\subsubsection{Racionális törtre vezethető helyettesítések}
	\begin{example}
		\[ \int R(e^x)\,dx \]
		Ahol $R$ egy racionális törtfüggvény.
		Megoldási módszer ezen típusokhoz az alábbi új változó bevezetés:
		\[ t:=e^x,\quad t>0,\quad x=\ln t,\quad dx=\frac{1}{t}\,dt \]
	\end{example}
	\begin{exercise}$(x>1)$
		\[ \int\frac{1}{e^{2x}-4}\,dx \]
		Használjunk egy behelyettesítést.
		\[ t:=e^{2x},\quad t>0 \]
		\[ x=\frac{1}{2}\ln t,\quad dx=\frac{1}{2t}\,dt \]
		Visszatérve:
		\[ \int\frac{1}{e^{2x}-4}\,dx=\int\frac{1}{t-4}\cdot\frac{1}{2t}\,dt=\frac{1}{2}\cdot\int\frac{A}{t}+\frac{B}{t-4}\,dt \]
		\textit{A feladat befejezése:} Határozzuk meg $A,B\in\R$-t.
		\[ 1=A(t-4)+Bt\quad \Leftrightarrow\quad 1=(A+B)t-4A \]
		Ez alapján $A=-\frac{1}{4},\quad B=\frac{1}{4}$.
		\[ \frac{1}{8}\cdot\int\frac{1}{t-4}\,dt-\frac{1}{8}\cdot\int\frac{1}{t}\,dt=\frac{1}{4}\cdot\ln|t-4|-\frac{1}{4}\cdot\ln|t|+c\quad (c\in\R) \]
		Az eredeti integrál:
		\[ \int\frac{1}{e^{2x}-4}\,dx=\frac{1}{8}\cdot\ln|e^{2x}-4|-\frac{1}{8}\cdot\ln|e^{2x}|+c=\frac{1}{8}\cdot\ln|e^{2x}-4|-\frac{1}{4}\cdot x+c\quad (c\in\R) \]
	\end{exercise}
	\begin{note}
		Megállapítható, hogy $e^x=t$ helyettesítéssel 3 törtre kéne bontani.
	\end{note}
	\begin{task}$x\in\R$
		\[ \int\frac{e^{3x}}{e^x+2}\,dx \]
		Használjuk a fenti behelyettesítést.
		\[ t:=e^x,\quad t>0,\quad x=\ln t,\quad dx=\frac{1}{t}\,dt \]
		Visszatérve:
		\[ \int\frac{e^{3x}}{e^x+2}\,dx=\int\frac{t^3}{t+2}\frac{1}{t}\,dt=\int\frac{t^2}{t+2}\,dt= \]
		Végezzünk el egy polinomosztást: \quad $t^2 : (t+2)=(t-2)(t+2)+4$.
		\[ =\int t-2+\frac{4}{t+2}\,dt=\frac{t^2}{2}-2t+4\cdot\ln(t+2)+c\quad (c\in\R) \]
		Így az eredeti integrál:
		\[ \int\frac{e^{3x}}{e^x+2}\,dx=\frac{e^{2x}}{2}-2e^x+4\cdot\ln(e^x+2)+c\quad (c\in\R) \]
	\end{task}
	\begin{exercise}$x\in\R$
		\[ \int\frac{e^x+4}{e^{2x}+4e^x+3}\,dx \]
		Helyettesítsünk be.
		\[ e^x=t\quad dx=\frac{1}{t}\,dt \]
		Visszatérve:
		\[ \int\frac{e^x+4}{e^{2x}+4e^x+3}\,dx=\int\frac{t+4}{t^2+4t+3}\cdot\frac{1}{t}\,dt \]
		\textit{A feladat befejezése:} Megállapítható, hogy $t^2+4t+3$ diszkriminánsa nemnegatív.
		\[ \int\frac{t+4}{t^2+4t+3}\cdot\frac{1}{t}\,dt=\int\frac{t+4}{(t+3)(t+1)\cdot t}\,dt=\int\left(\frac{A}{t+3}+\frac{B}{t+1}+\frac{C}{t}\right)\,dt= \]
		Határozzuk meg $A,B,C\in\R$-t \textit{értékadás} segítségével.
		\[ t+4=A(t+1) t+B(t+3) t+C(t+3)(t+1) \]
		\vspace{-7mm}
		\begin{align*}
			t=0\quad \Rightarrow&\quad 4=3C\quad \Rightarrow\quad C=\frac{4}{3}\\
			t=-1\quad \Rightarrow&\quad 3=-2B\quad \Rightarrow\quad B=-\frac{3}{2}\\
			t=-3\quad \Rightarrow&\quad 1=6A\quad \Rightarrow\quad A=\frac{1}{6}
		\end{align*}
		Ezalapján:
		\[ =\frac{1}{6}\cdot\int\frac{1}{t+3}\,dx-\frac{3}{2}\cdot\int\frac{1}{t+1}\,dx+\frac{4}{3}\cdot\int\frac{1}{t}\,dt=\frac{\ln|t+3|}{6}-\frac{3\cdot\ln|t+1|}{2}+\frac{4\cdot\ln|t|}{3}+c\quad (c\in\R) \]
		Határozzuk meg az eredeti integrált:
		\[ \int\frac{e^x+4}{e^{2x}+4e^x+3}\,dx=\frac{\ln|e^x+3|}{6}-\frac{3\cdot\ln|e^x+1|}{2}+\frac{4x}{3}+c \]
	\end{exercise}
	\begin{note}
		Ezen típusokból egy tuti elő fog fordulni egy a zh-ban.
	\end{note}
	\begin{example}
		\[ \int R\left(x;\sqrt[n]{\frac{a^x+b}{cx+d}}\right)\,dx= \]
		Módszer:
		\[ \sqrt[n]{\frac{ax+b}{cx+d}}=:t \]
	\end{example}
	\begin{task}$x>\frac{3}{2}$
		\[ \int x\cdot\sqrt{5{x}-3}\,dx \]
		Vezessünk be egy új változót:
		\[ t:=\sqrt{5x-3},\quad x=\frac{t^2+3}{5},\quad dx=\frac{2t}{5}\,dt \]
		Visszatérve:
		\[ \int x\sqrt{5x-3}=\int\frac{t^2+3}{5}\cdot t\cdot\frac{2t}{5}\,dt=\frac{2}{25}\cdot\int(t^4+3t^2)\,dt=\frac{2}{25}\cdot\frac{t^5}{5}+\frac{2}{25}t^3+c \]
		Eredeti integrál:
		\[ \frac{2}{125}(\sqrt{5x-3})^5+\frac{2}{25}(\sqrt{5x-3})^3+c\quad (c\in\R) \]
	\end{task}
	\begin{exercise}$x\in(3;+\infty)$
		\[ \int\sqrt{\frac{x-3}{x-1}}\,dx \]
		Új változó:
		\[ t:=\sqrt{\frac{x-3}{x-1}},\quad t>0,\quad x=\frac{t^2-3}{t^2-1} \]
		\[ dx=\frac{2t\cdot(t^2-1)-(t^2-3)\cdot2t}{(t^2-1)^2}\,dt=\frac{4t}{(t^2-1)^2}\,dt \]
		Visszatérve:
		\[ \int t\cdot\frac{4t}{(t^2-1)^2}\,dt=4\cdot\int\frac{t^2}{(t-1)^2(t+1)^2}\,dt=4\cdot\int\left(\frac{A}{t-1}+\frac{B}{(t-1)^2}+\frac{C}{t+1}+\frac{D}{(t+1)^2}\right)\,dt \]
		\textit{Megoldás:} Határozzuk meg $A,B,C,D\in\R$-t.
		\[ t^2=A(t-1)(t+1)^2+B(t+1)^2+C(t+1)(t-1)^2+D(t-1)^2= \]
		\textit{Értékadással} 2 darab konstanst könnyen meghatározhatunk.
		\begin{align*}
			t=1\quad \Rightarrow&\quad 1=4B\quad \Rightarrow\quad B=\frac{1}{4}\\
			t=-1\quad \Rightarrow&\quad 1=4D\quad \Rightarrow\quad D=\frac{1}{4}
		\end{align*}
		A kellemetlen számolások elkerülése végett, $A$-t és $C$-t \textit{egyenlő együtthatókkal} adjuk meg.
		\begin{align*}
		t^2\quad &=\quad A(t-1)(t+1)^2+\frac{1}{4}(t+1)^2+C(t+1)(t-1)^2+\frac{1}{4}(t-1)^2=\\
				 &=\quad A(t^2-1)(t+1)+\frac{1}{4}(t+1)^2+C(t^2-1)(t-1)+\frac{1}{4}(t-1)^2=\\
				 &=\quad At^3+At^2-At-A+\frac{1}{4}(t^2+2t+1)+Ct^3-Ct^2-Ct+C+\frac{1}{4}(t^2-2t+1)=\\
				 &=\quad (A+C)t^3+\left(A-C+\frac{1}{4}+\frac{1}{4}\right)t^2+\left(-A-C+\frac{1}{2}-\frac{1}{2}\right)t+C-A+\frac{1}{4}+\frac{1}{4}=\\
				 &=\quad (A+C)t^3+\left(A-C+\frac{1}{2}\right)t^2+(-A-C)t+C-A+\frac{1}{2}
		\end{align*}
		Azaz, az átláthatóság kedvéért:
		\begin{align*}
			t^3 \quad \text{együtthatója:}&\quad  0=A+C\\
			t^2 \quad \text{együtthatója:}&\quad  1=\left(A-C+\frac{1}{2}\right)\\
			t^1 \quad \text{együtthatója:}&\quad  0=-A-C\\
			t^0 \quad \text{együtthatója:}&\quad  0=C-A+\frac{1}{2}
		\end{align*}
		Ez alapján $A=\frac{1}{4}$ és $C=-\frac{1}{4}$. A fenti hosszadalmas számolás ellenére nagyon kellemes integrált kapunk:
		\[=4\cdot\frac{1}{4}\int\frac{1}{t-1}\,dt+4\cdot\frac{1}{4}\int\frac{1}{(t-1)^2}\,dt-4\cdot\frac{1}{4}\cdot\int\frac{1}{t+1}\,dt+4\cdot\frac{1}{4}\cdot\int\frac{1}{(t+1)^2}\,dt=\]
		\[=\int\frac{1}{t-1}\,dt+\int\frac{1}{(t-1)^2}\,dt-\int\frac{1}{t+1}\,dt+\int\frac{1}{(t+1)^2}\,dt=\ln|t-1|-\frac{1}{t-1}-\ln|t+1|-\frac{1}{t+1}+c\quad (c\in\R) \]
		Az eredeti integrál így:
		\[ \int\sqrt{\frac{x-3}{x-1}}\,dx=\ln\left|\sqrt{\frac{x-3}{x-1}}-1\right|-\frac{1}{\sqrt{\frac{x-3}{x-1}}-1}-\ln\left|\sqrt{\frac{x-3}{x-1}}+1\right|-\frac{1}{\sqrt{\frac{x-3}{x-1}}+1}+c= \]
		\[=\ln\left(\sqrt{\frac{x-3}{x-1}}-1\right)-\frac{1}{\sqrt{\frac{x-3}{x-1}}-1}-\ln\left(\sqrt{\frac{x-3}{x-1}}+1\right)-\frac{1}{\sqrt{\frac{x-3}{x-1}}+1}+c\quad (c\in\R) \]
	\end{exercise}
	\begin{exercise}
		\[ \int\frac{1}{\sqrt{x}+\sqrt[3]{x}}\,dx \]
		Vezessünk be egy új változót (itt érdemes a legkisebb közös többszöröst venni a gyököknél):
		\[ \sqrt[6]{x}=t,\quad x=t^6,\quad dx=6t^5\,dt \]
		Visszatérve:
		\[ \int\frac{1}{t^3+t^2}\cdot6t^5\,dt=6\cdot\int\frac{t^3}{t+1}\,dt= \]
		\textit{A feladat befejezése:} Polinom osztás segítségével csökkentsük a a számlálóban lévő ismeretlen kitevőjét.
		\[ t^3=(t+1)(t^2-t+1)+1 \]
		Ez alapján:
		\[ =\int\left(t^2-t+1-\frac{1}{t+1}\right)\,dx=\frac{t^3}{3}-\frac{t^2}{2}+t-\ln|t+1|+c\quad (c\in\R) \]
		Így az eredeti integrál:
		\[ \int\frac{1}{\sqrt{x}+\sqrt[3]{x}}\,dt=\frac{\sqrt[6]{x}^3}{3}-\frac{\sqrt[6]{x}^2}{2}+\sqrt[6]{x}-\ln|\sqrt[6]{x}+1|+c=\frac{\sqrt{x}}{3}-\frac{\sqrt[3]{x}}{2}+\sqrt[6]{x}-\ln|\sqrt[6]{x}+1|+c\quad (c\in\R) \]
	\end{exercise}
	\begin{exercise}$x>\frac{3}{2}$
		\[ \int\frac{1}{x}\sqrt{\frac{2x-3}{x}}\,dx \]
		\textit{Megoldás:}
		\[ t:=\sqrt{\frac{2x-3}{x}},\quad x=\frac{3}{2-t^2},\quad dx=\frac{6t}{(2-t^2)^2}\,dt \]
		Helyettesítsünk be.
		\[ \int\frac{2-t^2}{3}\cdot t\cdot\frac{6t}{(2-t^2)^2}\,dt=-\int\frac{2t^2}{t^2-2}\,dt=\]
		Végezzünk el egy polinomosztást.
		\[  \]
	\end{exercise}
	\begin{exercise}
		\[ \int\sqrt{\frac{1+x}{1-x}}\,dx \]
		\textit{Megoldás:} Vezessünk be egy új változót.
		\[ t:=\sqrt{\frac{1+x}{1-x}},\quad x=\frac{t^2-1}{t^2+1},\quad dx=\frac{4t}{\left(t^2+1\right)^2}\,dt \]
		Helyettesítsünk be:
		\[ \int\sqrt{\frac{1+x}{1-x}}\,dx=\int t\cdot\frac{4t}{(t^2+1)^2}\,dt=4\cdot\int\frac{t^2}{(t^2+1)^2}\,dt=4\cdot\int\frac{t^2+1-1}{(t^2+1)^2}\,dt=4\cdot\left(\int\frac{1}{t^2+1}-\frac{1}{(t^2+1)^2}\,dt\right)= \]
		\[ =4\arc\tg t-4\cdot\int\frac{1}{(t^2+1)^2}\,dt=4\arc\tg t-4\cdot\left( \frac{1}{2}\cdot\frac{t}{1+t^2}+\frac{1}{2}\arc\tg x\right)+c =4\arc\tg t-2\cdot\frac{t}{1+t^2}-2\arc\tg x+c= \]
		\[ =2\arc\tg t-2\cdot\frac{t}{1+t^2}+c\quad (c\in\R) \]
		Helyettesítsünk vissza:
		\[ \int\sqrt{\frac{1+x}{1-x}}\,dx=2\arc\tg \sqrt{\frac{1+x}{1-x}}-2\cdot\frac{\sqrt{\frac{1+x}{1-x}}}{1+\left(\sqrt{\frac{1+x}{1-x}}	\right)^2}+c\quad (c\in\R) \]
	\end{exercise}
	\begin{exercise}$x<1$
		\[ \int\frac{x}{1+\sqrt{1-x}}\,dx \]
		\textit{Megoldás:} Vezessünk be egy új változót.
		\[ \R_0^+\ni t:=\sqrt{1-x},\quad x=1-t^2,\quad dx=-2t\,dt \]
		Helyettesítsünk be.
		\[ \int\frac{x}{1+\sqrt{1-x}}\,dx=\int\frac{1-t^2}{1+t}\cdot\left(-2t\right)\,dt=-2\cdot\int\frac{(1-t)(1+t)t}{1+t}\,dt=-2\cdot\int (1-t)t\,dt=\int 2t^2-2t\,dt= \]
		\[ =\frac{2t^3}{3}-t^2+c\quad (c\in\R) \]
		Az eredeti integrál így:
		\[ \int\frac{x}{1+\sqrt{1-x}}\,dx=\frac{2(\sqrt{1-x})^3}{3}-(1-x)+c_1=\frac{2(\sqrt{1-x})^3}{3}+x+c\quad  (c_1,c\in\R) \]
		Megállapítható, hogy mivel $(-1)$ is konstans, összevonható $c_1$-el, a fenti példában pl. a $c:=c_1-1$ választással.
	\end{exercise}
	\begin{example}
		\[ R\left(\sin x,\cos x\right)\,dt \]
		Racionális törtfüggvények $\sin, \cos$ függvényekkel.
	\end{example}
	\begin{exercise}$x\in\left(1,\pi\right)$
		\[ \int\frac{1+\sin x}{1-\cos x}\,dx \]
		A következő módszer mindenhol használható, de néha nem célszerű. Vezessünk be egy új helyettesítést:
		\[  t:=\tg \left(\frac{x}{2}\right) \]
		\[ \sin x=2\cdot\sin\frac{x}{2}\cdot\cos\frac{x}{2}=2\cdot\frac{\sin\frac{x}{2}}{\cos\frac{x}{2}}\cdot\cos^2\frac{x}{2}=2\cdot\tg\frac{x}{2}\cdot\cos^2\frac{x}{2}\quad \overset{1+\tg^2\alpha=\frac{1}{\cos^2\alpha}}{\underset{\cos^2\alpha=\frac{1}{1+\tg^2\alpha}}{=}}\quad \frac{2\tg\left(\frac{x}{2}\right)}{1+\tg^2\frac{x}{2}} \]
		Ez alapján könnyen megállapítható hogy
		\[ \sin x=\frac{2t}{1+t^2}.  \]
		Hasonlóan, $\cos$-ra is megállapíthatunk hasonlót.
		\[ \cos x=\cos^2\frac{x}{2}-\sin^2\frac{x}{2}=\cos^2\frac{x}{2}-1+\cos^2\frac{x}{2}=2\cos^2\frac{x}{2}=\frac{2}{1+\tg^2\frac{x}{2}}-1=\frac{2-1-\tg^2\frac{x}{2}}{1+\tg^2\frac{x}{2}}=\frac{1-\tg^2\frac{x}{2}}{1+\tg^2\frac{x}{2}} \]
		Azaz
		\[ \cos x=\frac{1-t^2}{1+t^2} \]
		Határozzuk meg a behelyettesítéshez szükséges utolsó információkat is.
		\[ x=2\arc\tg t,\quad dx=\frac{2}{1+t^2}\,dt \]
		Visszatérve:
		\[ \int\frac{1+\sin x}{1-\cos x}\,dx=\int\frac{1+\frac{2t}{1+t^2}}{1-\frac{1-t^2}{1+t^2}}\cdot\frac{2}{1+t^2}\,dt=\int\frac{\frac{1+t^2+2t}{1+t^2}}{\frac{1+t^2-1+t^2}{1+t^2}}\cdot\frac{2}{1+t^2}\,dt=\int\frac{t^2+1+2t}{1+t^2-1+t^2}\cdot\frac{2}{1+t^2}\,dt=\]
		\[=\int\frac{t^2+2t+1}{t^2(t^2+1)}\,dt=\int\left(\frac{A}{t}+\frac{B}{t^2}+\frac{Ct+D}{t^2+1}\right)\,dt  \] 
		\textit{A feladat befejezése:} Határozzuk meg $A,B,C,D\in\R$-t \textit{egyenlő együtthatók} módszerével.
		\begin{align*}
			t^2+2t+1\quad &=\quad A(t^2+1)t+B(t^2+1)+(Ct+D)t^2\\
						  &=\quad At^3+At+Bt^2+B+Ct^3+Dt^2\\
						  &=\quad (A+C)t^3+(B+D)t^2+At+B
		\end{align*}
		Azaz, az átláthatóság kedvéért:
		\begin{align*}
			t^3 \quad \text{együtthatója:}&\quad  0=A+C\\
			t^2 \quad \text{együtthatója:}&\quad  1=B+D\\
			t^1 \quad \text{együtthatója:}&\quad  2=A\\
			t^0 \quad \text{együtthatója:}&\quad  1=B
		\end{align*}
		Gyorsan megállapítható hogy $A=2,\quad B=1,\quad C=-2,\quad D=0$. Így:
		\[ =2\cdot\int\frac{1}{t}\,dt+\int\frac{1}{t^2}\,dt-2\cdot\int\frac{t}{t^2+1}\,dt=2\ln t-\frac{1}{t}-\int\frac{(t^2+1)'}{t^2+1}\,dt=2\ln t-\frac{1}{t}-\ln(t^2+1)+c\quad (c\in\R) \]
		Az eredeti integrál:
		\[ \int\frac{1+\sin x}{1-\cos x}\,dx=2\ln \tg \left(\frac{x}{2}\right)-\frac{1}{\tg \left(\frac{x}{2}\right)}-\ln(\tg^2\left(\frac{x}{2}\right)+1)+c\quad (c\in\R) \]
	\end{exercise}
	\begin{note}
		Ezt a módszert $\tg\left(\frac{x}{2}\right)$ módszernek hívjuk.
	\end{note}
	\begin{exercise}$x\in\left(0,\frac{\pi}{2}\right)$
		\[ \int\frac{\cos x}{\cos x+2\sin x}\]
		Osszuk le a nevezőt és a számlálót is $\cos x$-el.
		\[ \int\frac{1}{1+2\tg x}\,dx= \]
		Második módszer, ha csak $\tg$-re átírható:
		\[ t:=\tg x,\quad x=\arc\tg t,\quad dx=\frac{1}{1+t^2}\,dt \]
		Visszatérve:
		\[ =\int\frac{1}{1+2t}\cdot\frac{1}{1+t^2}\,dt \]
		A megoldás házi feladat, valamint ugyanennek a feladatnak az 1. módszerrel való megoldása is.
		
		\textit{Megoldás (2. módszer):} 
		\[ \int\frac{1}{1+2t}\cdot\frac{1}{1+t^2}\,dt=\int\frac{1}{(1+2t)(1+t^2)}\,dt=\int\left(\frac{A}{1+2t}+\frac{Bt+C}{1+t^2}\right)\,dt= \]
		\textit{Értékadással} határozzunk meg valamennyi konstanst $A,B,C\in\R$-ből.
		\[ 1=A(1+t^2)+(Bt+C)(1+2t) \]
		\[ t=\frac{1}{2}\quad \Rightarrow\quad 1=A\left(1+\frac{1}{4}\right)=A\cdot\frac{5}{4}\quad \Leftrightarrow\quad \frac{4}{5}=A \]
		\textit{Egyenlő együtthatók} módszerével határozzuk meg a maradékot.
		\begin{align*}
			1\quad &=\quad A(1+t^2)+(Bt+C)(1+2t)=\\
				   &=\quad A+At^2+Bt+2Bt^2+C+2Ct=\\
				   &=\quad (2B+A)t^2+(B+2C)t+A+C
		\end{align*}
		Azaz, az átláthatóság kedvéért:
		\begin{align*}
		t^2 \quad \text{együtthatója:}&\quad  0=2B+A\\
		t^1 \quad \text{együtthatója:}&\quad  0=B+2C\\
		t^0 \quad \text{együtthatója:}&\quad  1=A+C
		\end{align*}
		Az utolsó egyenletből $C=\frac{1}{5}$, a másodikból pedig $B=-\frac{2}{5}$ következik. Visszatérve az integrálthoz:
		\[ =\frac{4}{5}\cdot\int\frac{1}{1+2t}\,dt+\frac{1}{5}\cdot\int\frac{1-2t}{1+t^2}\,dt=\frac{4}{5}\cdot\int\frac{1}{1+2t}\,dt-\frac{1}{5}\cdot\int\frac{(1+t^2)'-1}{1+t^2}\,dt=\]
		\[=\frac{4}{5}\cdot\int\frac{1}{1+2t}\,dt-\frac{1}{5}\cdot\int\frac{(1+t^2)'}{1+t^2}\,dt+\frac{1}{5}\cdot\int\frac{1}{1+t^2}\,dt=\frac{2}{5}\ln|1+2t|-\frac{1}{5}\ln|1+t^2|+\frac{1}{5}\arctg(t)+c\quad (c\in\R) \]
	\end{exercise}
	\begin{exercise}$x\in\left(0,\frac{\pi}{2}\right)$
		\[ \int\frac{\sin x}{2\cos^2x+3\cos x}\,dx= \]
		Harmadik módszer:
		\[ t:=\cos x,\quad -\sin x\,dx=dt \]
		Visszatérve
		\[ =-\int\frac{1}{2t^2+3t}\,dt \]
		\textit{Megoldás:} 
		\[ =-\int\frac{1}{2t^2+3t}\,dt=-\int\frac{1}{t(2t+3)}\,dt=-\int\left(\frac{A}{t}+\frac{B}{2t+3}\right)\,dt \]
		Határozzuk meg $A,B\in\R$-t \textit{értékadással}.
		\[ 1=A(2t+3)+Bt \]
		\vspace{-7mm}
		\begin{align*}
			t=0\quad \Rightarrow&\quad 1=3A\quad \Rightarrow\quad A=\frac{1}{3}\\
			t=-\frac{3}{2}\quad \Rightarrow&\quad 1=-\frac{3}{2}B\quad \Rightarrow \quad B=-\frac{2}{3}
		\end{align*}
		Visszatérve:
		\[ =-\frac{1}{3}\cdot\int\frac{1}{t}\,dt+\frac{2}{3}\cdot\int\frac{1}{2t+3}\,dt=-\frac{1}{3}\ln|t|+\frac{1}{3}\ln|2t+3|+c\quad (c\in\R) \]
		Így az eredeti integrál:
		\[ \int\frac{\sin x}{2\cos^2x+3\cos x}\,dt=-\frac{1}{3}\ln|\cos x|+\frac{1}{3}\ln|2\cos x+3|+c\quad (c\in\R) \]
	\end{exercise}
	\begin{exercise}$x\in\left(0,\frac{\pi}{2}\right)$
		\[ \int\frac{\cos x}{\cos^2x+\sin^3x-1}\,dx \]
		Végezzünk egy apró átalakítást $(\cos^2x-1 = -\sin^2x)$.
		\[ \int\frac{\cos x}{\sin^3x-\sin^2x}\,dx= \]
		Negyedik módszer: Vezessünk be új változót.
		\[ t:=\sin x,\quad \cos x\,dx=dt \]
		Visszatérve:
		\[ \int\frac{dt}{t^3-t^2}=\int\frac{1}{t^2(t-1)}\,dt=\int\left(\frac{A}{t}+\frac{B}{t^2}+\frac{C}{t-1}\right)\,dt \]
		\textit{Megoldás:} Határozzuk meg $A,B,C\in\R$-t \textit{egyenlő együtthatók} módszerével
		\begin{align*}
			1\quad &=\quad At(t-1)+B(t-1)+Ct^2\\
				   &=\quad At^2-At+Bt-B+Ct^2\\
				   &=\quad (A+C)t^2+(B-A)t-B\\
		\end{align*}
		Azaz, az átláthatóság kedvéért:
		\begin{align*}
			t^2 \quad \text{együtthatója:}&\quad  0=A+C\\
			t^1 \quad \text{együtthatója:}&\quad  0=B-A\\
			t^0 \quad \text{együtthatója:}&\quad  1=-B
		\end{align*}
		Ez alapján triviálisan $B=-1$, melyből $A=-1$ és $C=1$ következik. Visszatérve:
		\[ =-\int\frac{1}{t}\,dt-\int\frac{1}{t^2}\,dt+\int\frac{1}{t-1}\,dt=-\ln|t|+\frac{1}{t}+\ln|t-1|+c\quad (c\in\R) \]
		Így az eredeti integrál:
		\[ \int\frac{\cos x}{\cos^2x+\sin^3x-1}\,dx=-\ln|\sin x|+\frac{1}{\sin x}+\ln|\sin x-1|+c\quad (c\in\R) \]
	\end{exercise}
	\begin{exercise}$x\in\left(0,\frac{\pi}{2}\right)$
		\[ \int\frac{\sin x+\cos x}{1-\sin2x} \]
		Tipp: $f'\cdot f^\alpha$
		\textit{Megoldás:}
		\[ \int\frac{\sin x+\cos x}{1-\sin2x}=\int\frac{\sin x+\cos x}{1-2\sin x\cos x} \]
		
	\end{exercise}
	Ezzel befejeztük a határozatlan integrált. HF: 10 darab beadandó házi feladat: 3 db. exponenciális helyettesítéssel, 3 darab gyökös, 4 darab trigonometrikus.
	\subsection{Határozott integrál és alkalmazásai}
	\subsubsection{Területszámítás}
		\begin{figure}[H]
			\centering
			\includegraphics[height=3cm]{kepek/01.png}
			\includegraphics[height=3cm]{kepek/02.png}
			\caption{Rendre: $T=\int_a^bf,\quad f>0$,\quad valamint\quad  $T=\int_a^bf,\quad f<0$}
		\end{figure}
		\begin{figure}[H]
			\centering
			\includegraphics[height=3cm]{kepek/03.png}
			\caption{}\label{eltolatlan-fv}
		\end{figure}
		Hogyan lehet megoldani ezt? Megoldás: eltoljuk a függvényt.
		\begin{figure}[H]
			\centering
			\includegraphics[height=2cm]{kepek/04.png}
			\caption{Ugyanaz az mint a \ref{eltolatlan-fv}. ábra, adott $c$ konstanssal eltolva.}
		\end{figure}
		Így már a terület könnyen meghatározható:
		\[ T=\int_a^b(f+c)-\int_a^b(g+c)=\int_a^b(f-g) \]
		\begin{figure}[H]
			\centering
			\includegraphics[height=3cm]{kepek/05.png}
			\caption{}
		\end{figure}
		\[ T=\int_a^c(f-g)+\int_c^d(g-f)+\int_d^b(f-g)=\int_a^b|f-g|= \]
		Megállapítható, hogy ez $f$ és $g$ 1-es metrikája.
		\[ =\rho_1(f,g),\quad (f,g\in C[a,b]) \]
		
		Szimmetriák kihasználása:
		\begin{figure}[H]
			\centering
			\includegraphics[height=3cm]{kepek/06.png}
			\caption{Elég a negyed kör területének meghatározása.}
		\end{figure}
		\[ T_{\text{kör}}=4\cdot T_{\text{negyedkör}}=4\cdot\int_0^1\sqrt{1-x^2}\,dx \]
		Szimmetria kihasználható így is:
		\begin{figure}[H]
			\centering
			\includegraphics[height=3cm]{kepek/07.png}
			\caption{Elegendő a $[0,a]$ intervallumon a függvény integráltjának kétszeresét meghatározni.}
		\end{figure}
		\[ \int_{-a}^{a}f=2\cdot\int_0^af \]
		Megállapítható és kihasználható, hogy $f(x):=x^2$ páros.
		
		
	\begin{revision}
		Newton-Leibniz tétel: Ha $f\in\R[a,b]$ és $\int f\not=0$ 
			\[ \int_a^bf(x)\,dx=F(b)-F(a)=:[F(x)]_a^b\quad (\forall F\in\int f) \]
	\end{revision}
	\begin{example}
		Mennyi az e két \textit{reláció} által határolt terület?
			
		\[\begin{cases}
			y=x-1\\
			y^2=2x+6
		\end{cases}\]
		Világos, hogy a másik \textit{reláció} nem \textit{függvény}, azonban fel tudjuk írni két függvény együtteseként.
		\[y^2=2x+6\Leftrightarrow\quad y=\pm\sqrt{2x+6}\quad \Leftrightarrow\quad x=\frac{y^2-6}{2} \]
		\begin{figure}[H]
			\centering
			\includegraphics[height=3cm]{kepek/08.png}
			\caption{}
		\end{figure}
		Most már megállapítható a függvények metszéspontjai:
		\[ (x-1)^2=2x+6\quad \Leftrightarrow\quad x_1=-1\quad \text{és\quad } x_2=5 \]
		Valamint megállapítható, hogy az $y=\pm\sqrt{2x-6}$ függvények a $-3$ pontban metszik az $x$ tengelyt.
		\smallskip
		
		Ez alapján a területet kiszámolhatjuk. A zöld területről megállapítható hogy szimmetrikus, és ezt ki is használhatjuk.
		\[ T=2\cdot\int_{-3}^{-1}\sqrt{2x+6}\,dx+\int_{-1}^{5}\left(\sqrt{2x+6}-(x-1)\right)\,dx=2\cdot\left[\frac{(2x+6)^{\frac{3}{2}}}{\frac{3}{2}\cdot2}\right]_{-3}^{-1}+\left[\frac{(2x+6)^{\frac{3}{2}}}{\frac{3}{2}\cdot2}-\frac{x^2}{2}+x\right]^{5}_{-1}=\]
		\[=\frac{2}{3}\left[4^\frac{3}{2}-0\frac{3}{2}\right]+\frac{16^\frac{3}{2}}{3}-\frac{25}{2}+5-\left(\frac{4^\frac{3}{2}}{3}-\frac{1}{2}-1\right)=18  \]
	\end{example}
	\begin{example}Határozzuk meg az ezen görbék által határolt területet!
		\[\begin{cases}
			y=x^3\\
			y^2+x^2=2\\
			y=0
		\end{cases}\]
		\begin{figure}[H]
			\centering
			\includegraphics[height=4cm]{kepek/09.png}
			\caption{}
		\end{figure}
		Metszéspontok: (megfigyelhető, hogy az első egyenletet négyzetre emeltük)
		\begin{align*}
			x^2+x^6-2=0\\
			x^6-1+x^2-1=0\\
		\end{align*}
		Külön megállapítandó:
		\[(x^2-1)(x^4+x^2+1)=0\quad \Rightarrow\quad x=\pm1 \]
		\[ x^2-1=0\quad \Rightarrow\quad x=\pm1 \]
		Számoljuk ki  területet. A körre természetes okokból nem tudunk függvényt felírni, azonban megállapítható, hogy a számunkra fontos körnegyed egyenlete $y=+\sqrt{2-x^2}$.
		\[ T=\int_0^1x^3\,dx+\int_1^{\sqrt{2}}\sqrt{2-x^2}\,dx=\left[\frac{x^4}{4}\right]_0^1+I=\frac{1}{4}+I \]
		Ahol:
		\[ I=\int_1^{\sqrt{2}}\sqrt{2-x^2}\,dx=\sqrt{2}\cdot\int_1^{\sqrt{2}}\sqrt{1-\left(\frac{x}{\sqrt{2}}\right)^2}\,dx= \]
		Vezessünk be egy új változót.
		\[ \sin t:=\frac{x}{\sqrt{2}},\quad t=\arc\sin\left(\frac{x}{\sqrt{2}}\right) \]
		\[\text{Ha}\quad x=1\quad \Rightarrow\quad t=\arc\sin\frac{1}{\sqrt{2}}=\arc\sin\frac{\sqrt{2}}{2}=\frac{\pi}{4} \]
		\[\text{Ha}\quad x=\sqrt{2}\quad \Rightarrow\quad t=\arc\sin1=\frac{\pi}{2} \]
		Visszatérve:
		\[ =\sqrt{2}\cdot\int_{\frac{\pi}{4}}^{\frac{\pi}{2}}\sqrt{1-\sin^2t}\cdot\sqrt{2}\cos t\,dt= \]
		Visszahelyettesíteni fölösleges, hisz nem primitív függvényt, hanem egy konkrét számot keresünk.
		\[ =2\cdot\int_{\frac{\pi}{4}}^{\frac{\pi}{2}}|\cos t|\cdot\cos t\,dt\quad \overset{\frac{\pi}{4}\leq t\leq \frac{\pi}{2}}{=}\quad2\cdot\int_{\frac{\pi}{4}}^{\frac{\pi}{2}}\overbrace{\cos^2t}^{\frac{1+\cos2t}{2}}\,dt=\left[t+\frac{\sin2t}{2}\right]_{\frac{\pi}{4}}^{\frac{\pi}{2}}=\frac{\pi}{2}+\frac{\sin\pi}{2}-\frac{\pi}{4}-\frac{\sin\frac{\pi}{2}}{2}=\frac{\pi}{4}-\frac{1}{2}  \]
	\end{example}
	\begin{note}
		Okkal $I$-vel, és nem $I(x)$-el jelölünk. A határozatlan integrál egy függvény, a határozott csupán egy szám.
	\end{note}
	\begin{example}
		Számítsuk ki a következő minimumot:
		\[ \min\left\{\int _0^1|x^2-c|\,dx\quad :\quad c\in\R \right\} \]
		\begin{figure}[H]
			\centering
			\includegraphics[height=3cm]{kepek/10.png}
			\caption{}
		\end{figure}
		
		Azaz, hogyan kell meghúzni az egyenest úgy, hogy a terület a legkisebb legyen? Világos, hogy elég $c\in[0,1]$ intervallumot vizsgálnunk, ui. ellenkező esetben 1-1 téglalap területével nő a terület.
		\smallskip
		
		Metszéspontok:
		\[ x^2=x\quad \Leftrightarrow\quad x=\pm\sqrt{c}\quad \Rightarrow\quad x=\sqrt{c}\in[0,1] \]
		Határozzuk meg a területet:
		\[ T(c)=\int_0^{\sqrt{c}}(c-x^2)\,dx+\int_{\sqrt{c}}^{1}(x^2-c)\,dx=\left[cx-\frac{x^3}{3}\right]_0^{\sqrt{c}}+\left[\frac{x^3}{3}-cx\right]^1_{\sqrt{c}}=c\sqrt{c}-\frac{c\sqrt{c}}{3}+\frac{1}{3}-c-\frac{c\sqrt{c}}{3}+c\sqrt{c}=\]
		\[=\frac{4}{3}c\sqrt{c}-c+\frac{1}{3}\quad c\in[0,1] \]
		Kompakt halmazon folytonos függvényt vizsgálunk, és kell hogy legyen maximum vagy minimum. Ez lehet 0 vagy 1, vagy intervallumon belül. Ha $c\in(0,1)$ akkor:
		\[T'(c)=\frac{4}{3}\cdot\frac{3}{2}\cdot\sqrt{c}-1=2\cdot\sqrt{c}-1=0\quad \Leftrightarrow\quad c=\frac{1}{4}\in(0,1)\checkmark \]
		\begin{center}		
			\begin{tabular}{l|c|c|c}
				$T'(x)$&$-$&0&$+$\\
				\hline
				$T(x)$&$\frac{1}{3}\searrow$&$\frac{1}{4}$&$\nearrow\frac{2}{3}$
			\end{tabular}
		\end{center}
		\[ T(0)=\frac{1}{3};\quad T(1)=\frac{2}{3};\quad T\left(\frac{1}{4}\right)=\frac{1}{4} \]
		Összefoglalva, a terület minimális a $c=\frac{1}{4}$ választással.
	\end{example}
	\begin{note}
		Mi ez a feladat? $f(x):=x^2, \quad x\in[0,1];\quad g(x):=c;\quad \rho_1(f,g):=\int_0^1|f-g|$
		\[ \Rightarrow\quad (C[0,1];\rho_1) \quad \text{m. tér};\quad \min\left\{\rho_1(f,g)\ | \ y=c\in\R\right\} \]
	\end{note}
	\subsubsection{Ívhossz számolás}
	\begin{note}
		$f\in C^1[a,b]$
		\begin{figure}[H]
			\centering
			\includegraphics[height=3cm]{kepek/11.png}
			\caption{}
		\end{figure}
		\vspace{-6mm}
		\[ l=\int_a^b\sqrt{1+(f'(x))^2}\,dx \]
	\end{note}
	\begin{example}
		\[ f(x):=\frac{2(x-1)^{\frac{3}{2}}}{3};\quad x\in[2,5] \]
		Megoldás:
		\[ f'(x)=\frac{2}{3}\cdot\frac{3}{2}\cdot(x-1)^\frac{1}{2}=\sqrt{x-1} \]
		\[ \Rightarrow\quad l=\int_2^5\sqrt{1+x-1}\,dx=\int_2^5\sqrt{x}\,dx=\left[\frac{x^\frac{3}{2}}{\frac{3}{2}}\right]_2^5=\frac{2}{3}\left(5\sqrt{5}-2\sqrt{2}\right) \]
	\end{example}
	\begin{exercise}
		\[ f(x):=\sqrt{x};\quad x\in[1,2] \]
		Mekkora ívének hossza 1 és 2 között?
		\begin{figure}[H]
			\centering
			\includegraphics[height=3cm]{kepek/12.png}
			\caption{}
		\end{figure}
		Megoldás:
		\[ \Rightarrow\quad l=\int_1^2\sqrt{1+\left[(\sqrt{x}')\right]^2}\,dx=\int_1^2\sqrt{1+\left(\frac{1}{2\sqrt{x}}\right)^2}\,dx=\int_1^2\sqrt{1+\frac{1}{4x}}\,dx=\frac{1}{2}\cdot\int_1^2\sqrt{\frac{4x+1}{x}}\,dx= \]
		Vezessünk be egy új változót:
		\[ t:=\sqrt{\frac{4x+1}{x}},\quad x=\frac{1}{t^2-4},\quad dx=-\frac{2t}{(t^2-4)^2}\,dt\]
		Megállapítható, hogy 
		\[ x=1\quad \Rightarrow\quad t=\sqrt{5} \]
		\[ x=2\quad \Rightarrow\quad t=\frac{3}{\sqrt{2}} \]
		Visszatérve:
		\[ =-\frac{1}{2}\cdot\int_{\sqrt{5}}^{\frac{3}{\sqrt{2}}}t\left(\frac{2t}{(t^2-4)^2}\right)\,dt\quad \overset{\int_a^bf=-\int_b^af}{=}\quad \int_{\frac{3}{\sqrt{2}}}^{\sqrt{5}}\frac{t^2}{(t-2)^2(t+2)^2}\,dt=\]
		\[=\int_{\frac{3}{\sqrt{2}}}^{\sqrt{5}}\left(\frac{A}{t-2}+\frac{B}{(t-2)^2}+\frac{C}{t+2}+\frac{D}{(t+2)^2}\right)\,dt=\]
		\textit{A feladat befejezése:} Határozzuk meg $A,B,C,D\in\R$-t.
		\begin{align*}
			t^2\quad &=\quad A(t-2)(t+2)^2+B(t+2)^2+C(t+2)(t-2)^2+D(t-2)^2=\\
					 &=\quad A(t^2-4)(t+2)+B(t+2)^2+C(t^2-4)(t-2)+D(t-2)^2=\\
					 &=\quad A(t^3+2t^2-4t-8)+B(t^2+4t+4)+C(t^3-2t^2-4t+8)+D(t^2-4t+4)=\\
					 &=\quad (A+C)t^3+(2A+B-2C+D)t^2+(-4A+4B-4C-4D)t+(-8A+4B+8C+4D)
		\end{align*}
		\textit{Értékadással} könnyen megadható pár konstans.
		\begin{align*}
		t=-2\quad \Rightarrow&\quad 4=16D\quad \Leftrightarrow\quad D=\frac{1}{4}\\
		t=2\quad \Rightarrow &\quad 4=16B\quad \Leftrightarrow\quad B=\frac{1}{4}
		\end{align*}
		Így könnyebben számolható a többi konstans \textit{egyenlő együtthatók} módszerével.
		\begin{align*}
			t^3 \quad \text{együtthatója:}\quad 0&=A+C\\
			t^2 \quad \text{együtthatója:}\quad 
												1&=2A+\frac{1}{4}-2C+\frac{1}{4}\\
											   \frac{1}{4}&=A-C\\
			t^1 \quad \text{együtthatója:}\quad 0&=-4A+1-4C-1\\
												0&=-A-C\\
			t^0 \quad \text{együtthatója:}\quad 0&=-8A+1+8C+1\\
												1&=-4A+4C
		\end{align*}
		Ez alapján megállapítható hogy $C=-\frac{1}{8}$ és $A=\frac{1}{8}$. Visszatérve:
		\[=-\frac{1}{8}\cdot\int_{\frac{3}{\sqrt{2}}}^{\sqrt{5}}\frac{1}{t-2}\,dt+\frac{1}{4}\cdot\int_{\frac{3}{\sqrt{2}}}^{\sqrt{5}}\frac{1}{(t-2)^2}\,dt+\frac{1}{8}\cdot\int_{\frac{3}{\sqrt{2}}}^{\sqrt{5}}\frac{1}{t+2}\,dt+\frac{1}{4}\cdot\int_{\frac{3}{\sqrt{2}}}^{\sqrt{5}}\frac{1	}{(t+2)^2}\,dt=\]
		\[= \frac{1}{8}\left(-\Big[\ln(t-2)\Big]^{\sqrt{5}}_{\frac{3}{\sqrt{2}}}+2\left[-\frac{1}{1-2}\right]^{\sqrt{5}}_{\frac{3}{\sqrt{2}}}+\Big[\ln(t+2)\Big]^{\sqrt{5}}_{\frac{3}{\sqrt{2}}}+2\left[-\frac{1}{t+2}\right]^{\sqrt{5}}_{\frac{3}{\sqrt{2}}} \right)\]
		Megállapítható, hogy a területe létezik.~~~:)
	\end{exercise}
	\begin{note}
		Megállapítható, hogy $x^2$-te visszavezethető a $\sqrt{x}$ függvény.
		
		\begin{figure}[H]
			\centering
			\includegraphics[height=3cm]{kepek/13.png}\quad \quad \quad 
			\includegraphics[height=3cm]{kepek/14.png}
			\caption{}
		\end{figure}
		\[ \Rightarrow\quad l=\int_1^{\sqrt{2}}\sqrt{1+4x^2}\,dx= \]
		Befejezése házi, javallott a $2x=\sh t$ helyettesítés.
	\end{note}
	\subsubsection{Forgástestek térfogata és felszíne}
	\begin{revision}
		\begin{figure}[H]
			\centering
			\includegraphics[height=3cm]{kepek/15pre.png}$\quad \quad \quad $
			\includegraphics[height=3cm]{kepek/15.png}
			\caption{}\label{rotation}
		\end{figure}
		Ha a \ref{rotation}. ábrán lévő példát megforgatjuk az $x$ tengely körül, egy testet kapunk, melynek térfogatát számolhatjuk határozott integrállal.
		\[ V=\pi\int_a^bf^2(x)\,dx\quad (f\in\R[a,b]) \]
		\[ \mathcal{F}=2\pi\int_a^bf(x)\cdot\sqrt{1+(f'(x))^2}\,dx\quad (f\in C^1[a,b]) \]
		Ahol $V$ a térfogat (\textit{volume}) és $\mathcal{F}$ a felület.
	\end{revision}
	\begin{example} Határozzuk meg $f$ függvény $x$ tengely körüli forgástestének térfogatát ($V$), felületét ($\mathcal{F}$), és $f$ függvény alatti területét ($T$).
		\[ f(x):=\sin x\quad x\in[0,\pi]\]
		\begin{figure}[H]
			\centering
			\includegraphics[height=3cm]{kepek/16.png}
			\caption{}
		\end{figure}
		\vspace{-6mm}
		\[ T=\int_0^\pi\sin x\,dx=\left[-\cos x\right]_0^\pi=1+\cos 0=2 \]
		\[ V=\pi\int_0^\pi\sin^2x\,dx=\frac{\pi}{2}\cdot\left[x-\frac{\sin2x}{2}\right]_0^\pi=\frac{\pi}{2}\cdot\left[\pi-\frac{\sin2\pi}{2}-0\right]=\frac{\pi^2}{2} \]
		Folytatván
		\[ \mathcal{F}=2\pi\cdot\int_0^\pi \sin x\cdot\sqrt{ 1+\cos^2x}\,dx= \]
		Vezessünk be egy új változót.
		\[ \sh t:=\cos x,\quad -\sin x\,dx=\ch t\,dt \]
		Visszatérve:
		\[ 2\pi\cdot\int \sin x\cdot\sqrt{ 1+\cos^2x}\,dx=2\pi\cdot\int\sqrt{1+\sh^2t}\cdot(-\ch t)\,dt \]
		\[ x=0\quad \Rightarrow\quad 1=\sh t\quad \Rightarrow\quad t=\arsh 1=\ln(\sqrt{2}+1) \]
		\[ x=\pi\quad \Rightarrow\quad -1=\sh t\quad \Rightarrow\quad t=\arsh (-1)=\ln(\sqrt{2}-1) \]
		\[ \arsh x=\ln(x+\sqrt{1+x^2}) \]
		Befejetése hf.
	\end{example}
	\begin{exercise}
		\[ f(x)=2\cdot\sqrt{1-x^2}\quad x\in[-1,1] \]
		Forgástest $V, \mathcal{F}=?$
		
		\[ y^2=2\sqrt{1-x^2}\quad \Rightarrow\quad \frac{y^2}{4}+x^2=1 \]
		\[ \frac{x^2}{a^2}+\frac{y^2}{b^2}=1 \]
	\end{exercise}
	\subsection{Összefoglaló}
	\subsubsection{$\sin, \ \cos$ azonosságok ($\forall x\in\R$)}
	\begin{align*}
		\sin^2x+\cos^2x&=1\\
		\cos^2x-\sin^2x&=\cos2x\\
		2\cos x\sin x&=\sin2x\\
		\frac{1+\cos2x}{2}&=\cos^2 x\\
		\frac{1-\cos2x}{2}&=\sin^2x
	\end{align*}
	A következő helyettesítéssel:
	\[  t:=\tg \left(\frac{x}{2}\right) \]
	Könnyen megállapítható hogy
	\[ \sin x=\frac{2t}{1+t^2}\quad \text{és}\quad  \cos x=\frac{1-t^2}{1+t^2} \]
	\subsubsection{$\sh, \ \ch$ azonosságok ($\forall x\in\R$)}
	\begin{align*}
		\frac{e^x+e^{-x}}{2}&=\ch x\\
		\frac{e^x-e^{-x}}{2}&=\sh x\\
		\ch^2x-\sh^2x&=1\\
		\frac{1+\ch2x}{2}&=\ch^2 x\\
		2\sh x\ch x&=\sh(2x)
	\end{align*}
	\subsubsection{$\ln$ azonosságok ($\forall a,b\in\R^+$)}
	\begin{align*}
		\ln(a^2)&=2\ln a\\
		\ln(a)-\ln(b)&=\ln\left(\frac{a}{b}\right)\\
		\ln(a)+\ln(b)&=\ln(a\cdot b)
	\end{align*}
	\subsubsection{Integrálazonosságok}
	Legyen $f\in\R\to\R$, $F$ legyen $f$ primitív függvénye. Legyen továbbá $f\in D$.
	\[ \int\frac{f'(x)}{f(x)}\,dx=\ln|f(x)|+c\quad (x,c\in\R) \]
	\[ \int f'(x)\cdot f^\alpha(x)\,dx=\frac{f^{\alpha+1}}{\alpha+1}+c\quad (x,c\in\R) \]
	Racionális törtfüggvény nevezőjében másodfokú irreducibilis polinom $n$-edik hatványához tartozó rekurzív formula:
	\[ I_n(x):=\int\frac{1}{(1+x^2)^n}\,dx=\frac{1}{2(n-1)}\cdot\frac{x}{(1+x^2)^{n-1}}+\frac{2n-3}{2(n-1)}\cdot I_{n-1}(x) \]
	Speciális esetben, ha $n=2$:
	\[ I_2(x):=\int\frac{1}{(1+x^2)^2}\,dx=\frac{1}{2}\cdot\frac{x}{1+x^2}+\frac{1}{2}\arc\tg x+c\quad (c\in\R) \]
	\subsubsection{Határozott integrál}
	A terület ($T$), térfogat ($x$ tengely körüli forgatáskor keletkező forgástest, $V$), és felület ($\mathcal{F}$) meghatározása $a,b\in\R$ intervallumon:
	\[ T=\int_a^bf(x)\,dx\quad (f\in\R[a,b]) \]
	\[ V=\pi\int_a^bf^2(x)\,dx\quad (f\in\R[a,b]) \]
	\[ \mathcal{F}=2\pi\int_a^bf(x)\cdot\sqrt{1+(f'(x))^2}\,dx\quad (f\in C^1[a,b]) \]
	\section{Többváltozós függvények analízise}
	\subsection{Személtetés $\R^2\to\R$ esetre}
	\begin{note}
		Példa lehet alkalmazására pl. egy sík terület adott pontjához annak hőmérsékletének hozzárendelése.
	\end{note}
	\begin{note}
		Világos, hogy a $\R^2\to\R$ függvények ábrázolásához szükséges lesz egy új, $z$ tengelyre.
	\end{note}
	\begin{task}
		\[ f(x,y):=x^2+y^2\quad ((x,y)\in\R^2)\quad \Leftrightarrow\quad z=y^2+y^2 \]
		Az ilyen függvényeket úgy fogjuk tudni megoldani, hogy a függvény képéből kimetszünk egy görbét, melynek segítségével már vizsgálható az az adott rész egyváltozós analízissel.
		
		\medskip
		Megállapítható, hogy $\forall (x,y)\in\R^2:\quad z>0$.
		\smallskip
		Szintvonalak: (vagy vízszintes metszetek)
		\begin{align*}
			z=0&\quad \Leftrightarrow\quad x^2+y^2=0\quad \Leftrightarrow\quad (x,y)=(0,0) \\
			z=1&\quad \Leftrightarrow\quad x^2+y^2=1\quad \Leftrightarrow\quad \text{1 sugarú körvonal} \\
			z=2&\quad \Leftrightarrow\quad x^2+y^2=(\sqrt{2})^2\quad \Leftrightarrow\quad \text{$\sqrt{2}$ sugarú körvonal} \\
			0<z&\quad \Leftrightarrow\quad x^2+y^2=(\sqrt{z})^2\quad \Leftrightarrow\quad \sqrt{z}\text{ sugarú körvonal} 
		\end{align*}
		\begin{figure}[H]
			\centering
			\includegraphics[height=3cm]{kepek/18.png}
			\caption{}
		\end{figure}
		Függőleges metszetek (bizonyos irányok mentén)
		\begin{align*}
			y=0&\quad \Leftrightarrow\quad \text{,,$x$ tengely''}\quad \Rightarrow\quad z=x^2=f(x,0)\quad (x\in\R) \\
			x=0&\quad \Leftrightarrow\quad \text{,,$y$ tengely''}\quad \Rightarrow\quad z=y^2=f(0,y)\quad (x\in\R) \\
			y=x&\quad \text{mentén}\quad \Rightarrow\quad x=f(x,x)=2x^2\quad (y\in\R) 
		\end{align*}
		Ezt a felületet \textit{forgás-paraboloid}nak hívjuk.
		\begin{figure}[H]
			\centering
			\includegraphics[height=3cm]{kepek/19.png}
			\caption{}
		\end{figure}
	\end{task}
	\begin{note}
		Megállapítható, hogy illeszthető $z=0$-ban érintősík.
	\end{note}
	\begin{task}
		\[ z:=f(x,y):=(\sqrt{x^2+y^2})\quad ((x,y)\in\R^2, z\geq 0) \]
		Szintvonalak:
		\[ z=0\quad \Leftrightarrow\quad \sqrt{x^2+y^2}=0\quad \Leftrightarrow\quad (x,y)=(0,0) \]
		\[ z=1\quad \Leftrightarrow\quad \sqrt{x^2+y^2}=1\quad \Leftrightarrow\quad x^2+y^2=1^2 \]
		\[ z=2\quad \Leftrightarrow\quad \sqrt{x^2+y^2}=2\quad \Leftrightarrow\quad x^2+y^2=2^2 \]
		\begin{figure}[H]
			\centering
			\includegraphics[height=3cm]{kepek/20.png}
			\caption{}
		\end{figure}
		Függőleges metszetek:
		\[ y=0\quad \Rightarrow\quad z=\sqrt{x^2}=|x|=f(x,0)\quad (x\in\R) \]
		\[ x=0\quad \Rightarrow\quad z=\sqrt{y^2}=|y|=f(0,y)\quad (y\in\R) \]
		Ez a már jól ismert \textit{kúp}.
		\begin{figure}[H]
			\centering
			\includegraphics[height=3cm]{kepek/21.png}
			\caption{}
		\end{figure}
	\end{task}
	\begin{note}
		Megálapítható, hogy $z=0$-ra nem tudunk érintősíkot illeszteni.
	\end{note}
	\begin{task}Keresük a legbővebb halmazt, ahol ez függvényként értelmezhető
		\[ f(x,y):=\sqrt{1-x^2-y^2}=z,\quad ((x,y)\in\mathcal{D}_f=?) \]
		Világos, mikor értelmezhető.
		\[ \mathcal{D}_f=\left\{(x,y)\in\R^2~|~\sqrt{x^2+y^2}\leq 1 \right\}=\left\{ (x,y)\in\R^2~|~\norm{(x,y)-(0,0)}_2\leq 1 \right\} \]
		Ezek azok a pontok, melyek az origótól 1 távolságra vannak.
		\begin{figure}[H]
			\centering
			\includegraphics[height=3cm]{kepek/22.png}
			\caption{}
		\end{figure}
		Megállapítható, hogy 1 és $-1$-en kívül nincsenek függvényértékek.
		\smallskip
		
		Síkvonalak:
		\begin{align*}
			x=0&\quad \Rightarrow\quad \sqrt{1-x^2-y^2}=0\quad \Leftrightarrow\quad x^2+y^2=1 \quad \text{(ábrán fekete körvonal)}\\
			x=1&\quad \Rightarrow\quad \sqrt{1-x^2-y^2}=1\quad \Leftrightarrow\quad x^2+y^2=0\quad \Rightarrow\quad (x,y)=(0,0) \\
			x=\frac{1}{2}&\quad \Rightarrow\quad \sqrt{1-x^2-y^2}=\frac{1}{2}\quad \Leftrightarrow\quad x^2+y^2=\frac{3}{4}=\left(\frac{a\sqrt{3}}{2}\right)^2 \quad \text{(ábrán zöld körvonal)}
		\end{align*}
		Függőleges
		\[ x=0\quad \text{mentén}\quad \Rightarrow\quad z=f(x,0)=\sqrt{1-x^2}\quad x\in[-1,1]\quad \text{ábrán piros félkörvonal} \]
		\[ y=0\quad \text{mentén}\quad \Rightarrow\quad  z=f(0,y)=\sqrt{1-y^2}\quad y\in[-1,1]\quad \text{ábrán kék félkörvonal} \]
		\begin{figure}[H]
			\centering
			\includegraphics[height=3cm]{kepek/23.png}
			\caption{}
		\end{figure}
	\end{task}
	\begin{exercise}
		\[ z:=f(x,y):=e^{-x^2-y^2}\quad (x,y)\in\R^2 \]
		Szemléltessük!
	\end{exercise}
	\begin{note}
		Irodalom:
		\begin{enumerate}
			\item Gyemidovics
			\item Bolyai sorozat (többváltozós függvények analízise)
			\item Szili László
			\item Kórolyi Katalin $\rightarrow$ honlap
		\end{enumerate}
	\end{note}
	\subsection{Határérték számítása}
	\begin{revision} (határérték definíciója és átviteli elv)
		\begin{enumerate}
			\item $f\in\R^n\to\R^m;\quad 1\leq n,m\in\N;\quad a\in\mathcal{D}_f',\quad A\in\R^m$:
			\[ \lim_{x\to a}f(x)=\lim_af=A\quad \overset{\text{def.}}{\Longleftrightarrow}\quad \forall\varepsilon>0\quad \exists\partial>0\quad \forall x\in\mathcal{D}_f\setminus\{a\}:\quad 0<||x-a||_{\R^n}<\partial\quad ||f(x)-A||_{\R^m}<\varepsilon \]
			\item (átviteli elv)\[ \forall (x_k):\N\to\R^n\setminus\{a\}\quad \text{és}\quad \lim_{k\to\infty}(x_k)=a:\quad \lim_{k\to\infty}f(x_k)=A \]
		\end{enumerate}
	\end{revision}
	\begin{task} $(x,y)\in\R^2,\quad x^2+y^2\not=0$
		\[ \lim_{(x,y)\to\underbrace{(1,2)}_{a}}\left(\frac{x^2+xy}{x+y^2}\right)=\frac{1^2+1\cdot2}{1+2^2}=\frac{3}{5} \]
	\end{task}
	Miért tehetjük meg azt, hogy itt többváltoznál is simán beírhatjuk az értékeket?
	\begin{task}
		\[ \lim_{(x,y)\to(6,3)}x\cdot y\cdot\cos(x-2y)=18\cdot\cos0=18 \]
		Ugyanis: ld. átviteli elv. Tekintsünk egy vektorsorozatot, melyre
		\[ \underbrace{(x,y)}_{\not=(6,3)}\to(6,3)\quad (n\to\infty)\quad \Leftrightarrow\quad  \]
		Ez akkor és csak akkor konvergens, ha komponensenként konvergál.
		\[\begin{cases}
			\lim(x_n)=6\\
			\lim(y_n)=3
		\end{cases} \Rightarrow\quad \exists\lim(x_n,y_n)\quad \text{és}\quad \lim(x_n,y_n)=6\cdot3\cdot1=18 \]
		Válasszunk meg egy valós sorozatot. Szerencsés, ha ez 0-hoz tart. %TODO sure?
		\[ u_n:=x_n-2y_n\to 6-2\cdot3=0\quad (n\to\infty) \]
		Legyen 
		\[ g(t):=\cos t\quad (t\in\R)\quad (g\in\R\to\R) \]
		Hatványsorozatok összege folytonos, így alkalmazható az átviteli elv
		\[ g\in C\quad \Rightarrow\quad \exists\lim_{n\to\infty}\cos(x_n-2y_n)\quad \text{és}\quad \lim_{n\to\infty}\cos(x_n-2y_n)=1 \]
		Ahol kihasználtuk, hogy $u_n\to0\quad \Rightarrow\quad \cos(u_n)\to\cos 0=1\quad (n\to\infty)$
	\end{task}
	\begin{task}
		\[ \lim_{(x,y)\to(0,0)}\frac{x^2+y^2}{\sqrt{x^2+y^2+1}-1}\quad \overset{\frac{0}{0}}{=}\quad \lim_{(x,y)\to(0,0)}\frac{(x^2+y^2)(\sqrt{x^2+y^2+1}+1)}{x^2+y^2+1-1}=\sqrt{1}+1=2 \]
	\end{task}
	\begin{task}$(x,y)\in\R^2\setminus\{(0,0)\}$
		\[ \lim_{(x,y)\to(0,0)}\overbrace{\frac{x^2}{x^2+y^2}}^{:=f(x,y)}\quad \overset{\frac{0}{0}}{=}\quad  \]
		Sajnos itt nem tudunk tovább haladni hagyományos módszerekkel. Így új módszert kell használnunk. Válasszunk meg egy konvergens sorozatot, és az átviteli elv segítségével próbáljuk meg belátni, hogy ez a határérték létezik-e.
		\smallskip
		
		Legyen a sorozatunk \[\left(\frac{1}{n},0\right),\quad \text{ekkor}\quad  \lim_{n\to\infty}\left(\frac{1}{n},0\right)\to(0,0)\quad \Rightarrow\quad f\left(\frac{1}{n},0\right)=\frac{\frac{1}{n^2}}{\frac{1}{n^2}+0^2}=1 \to1\quad (n\to\infty)\]
		
		$\quad \overset{\text{átv. elv}}{\Rightarrow}$\quad Ha $\exists\lim_{(0,0)}f=1$ \textbf{lehet} csak.
		Legyen 
		\[ \left(0,\frac{1}{n}\right)\to(0,0)\quad \text{ha}\quad (n\to\infty) \]
		\[ \Rightarrow\quad f\left(0,\frac{1}{n}\right)=\frac{0}{\frac{1}{n^2}+0^2}=0\to 0(n\to\infty) \]
		\[ \Rightarrow \text{ha}\quad \exists\lim_{(0,0)}f\quad \text{csak}\quad 0 \quad \text{lehet} \]
		Mivel két kül. határérték nem lehet, nem létezik ez a határérték.
	\end{task}
	\begin{note}
		A cél egy 0-hoz tartó sorozat megválasztása és vizsgálása.
	\end{note}
	\begin{task}
		\[ \lim_{(x,y)\to(0,0)}\frac{x^2y^2}{x^2y^2+(x-y)^2}\quad \overset{\frac{0}{0}}{=}\quad  \]
		Legyen újra
		\[ \left(\frac{1}{n},0\right)=\frac{0}{\left(\frac{1}{n}\right)^2}=0\to0 \]
		Azaz a határérték csak 0 lehet, más nem, ha létezik.
		\[ \left(0,\frac{1}{n}\right)\to(0,0)\quad (n\to\infty)\quad \text{és}\quad  f\left(0,\frac{1}{n}\right)=\frac{0}{\left(\frac{1}{n}\right)^2}=0\to 0\quad (n\to\infty) \]
		De:
%		\[ \left(\frac{1}{n},\frac{1}{n}\right) \to(0,0) (n\to\infty)\quad \text{és}\quad f\left(\frac{1}{n},\frac{1}{n}\right)=\frac{\left(\frac{1}{n}\right)^n}{left(\frac{1}{n}\right)^n+\left(\frac{1}{n}-\frac{1}{n}\right)^2}=1 \]
	\end{task}
	\begin{note}
		Mindig más irányból közelítjük az origót. Rendre $x, y$ és $z$ tengely mentén.
	\end{note}
	\begin{task}
		Bizonyítsuk be, hogy
		\begin{enumerate}
			\item \[ \exists\lim_{x\to0}\left(\lim_{y\to0}\frac{x^2y^2}{x^2y^2+(x-y^2)}\right)=\lim_{x\to0}\left(\frac{1}{x^2}\right)=\lim_{x\to0}(0)=0 \]
			\item \[ \exists\lim_{y\to0}\left(\lim_{x\to0}\frac{x^2y^2}{x^2y^2+(x-y^2)}\right)= \]
			\begin{figure}[H]
				\centering
				\includegraphics[height=3cm]{kepek/24.png}
				\caption{}
			\end{figure}
			%TODO hiányos 07
		\end{enumerate}
	\end{task}
	\begin{task}
		\[ \lim_{(x,y)\to(0,0)}\left(\frac{xy}{\sqrt{x^2+y^2}}\right) \]
		Rövid jelölés: Sorozatok helyett irányokkal dolgozunk.
		\[ y=0\quad \text{mentén}\quad \Rightarrow\quad \underbrace{f(x,y)}_{x\not=0}=\frac{x\cdot0}{\sqrt{x^2}}=\frac{0}{|x|}=0\to0\quad (\text{ha}\quad x\to0) \]
		Ha $\exists\lim_{(0,0)}f=0$ lehet csak.
		\[ x=0\quad \text{mentén}\quad \Rightarrow\quad f(0,y)=\frac{0\cdot y}{\sqrt{y^2}}=\frac{0}{|y|}=0\to 0 \]
		ha $y\to0$.
		
		\smallskip
		vagy:
		\[ y=x\quad \text{mentén}\quad \Rightarrow\quad g(x):=f(x,x)=\frac{x^2}{\sqrt{2x^2}}=\frac{x^2}{\sqrt{2}\cdot|x|}=\frac{|x|^2}{\sqrt{2}\cdot|x|}=\frac{|x|}{\sqrt{2}}\to0 \]
		Sejtés: $\lim_{(0,0)}f=0$. (lévén nem tudtuk cáfolni)
		
		Definíció alapján:
		\[ \forall\varepsilon>0\quad \exists\partial>0\quad \forall(x,y)\in\mathcal{D}_f=\R^2\setminus\{(0,0)\}:\quad 0<||(x,y)-(0,0)||_{\R^2}<\partial\quad |f(x,y)-0|<\varepsilon \]
		Legyen $\varepsilon>0$ fix; és $|f(x,y)-0|=\left|\frac{xy}{\sqrt{x^2+y^2}}\right|=\frac{|xy|}{\sqrt{x^2+y^2}}$
		
		Mi a cél? $||f(x,y)-0||\leq K||(x,y)-(0,0)||_{\R{^2}}$
		\[ \underset{|xy|=\sqrt{x^2y^2}\leq\frac{x^2+y^2}{2}}{\leq}\quad\frac{x^2+y^2}{2\sqrt{x^2+y^2}}=\frac{1}{2}\sqrt{x^2+y^2}=\frac{1}{2}||(x,y)-(0,0)||_2<\varepsilon  \]
		\[ \Leftrightarrow||(x,y)-(0,0)||_2<2\varepsilon \]
		Legyen $\partial:=2\varepsilon$ jó.
	\end{task}
	\subsection{Folytonosság}
	\begin{example}
		\[ f(x,y):=\begin{cases}
		\frac{xy}{x^2+y^2};\quad (x,y)\in\R^2\setminus\{ (0,0) \}\\
		0,,\quad (x,y)=(0,0)
		\end{cases} \]
		Ekkor:
		\begin{enumerate}
			\item $\forall y_0\in\R\Rightarrow\quad g(x):=f(x,y_0) \quad (x\in\R,\quad g\in C(\R))$
			\item $\forall x_0\in\R\quad \Rightarrow\quad h(y):=f(x_0,y)\quad (y\in\R)\quad \Rightarrow\quad g\in C(\R)$
			\item $f\notin C\{(0,0)\}$
		\end{enumerate}
		\begin{figure}[H]
			\centering
			\includegraphics[height=3cm]{kepek/25.png}
			\caption{}
		\end{figure}
		\textit{Bizonyítás:}
		
		\begin{enumerate}
			\item Ha $y_0=0\quad \Rightarrow$
			\[ g(x)=\begin{cases}
			\frac{x\cdot0}{x^2+0^2}=\frac{0}{x^2}=0\quad x\not=0\\
			0,\quad (x,0)=(0,0)\quad \Leftrightarrow\quad x=0
			\end{cases} \]
			\[ \Rightarrow\quad g(x)=0\quad (\forall x\in\R) \quad \Rightarrow\quad g\in C(\R) \]
			Ha $y_0\in\R\setminus\{ 0 \}\quad \Rightarrow$
			\[ g(x)=f(x,y_0)=\begin{cases}
			\frac{xy_0}{ x^2+y_0^2 }\quad x\in\R\\
			\text{NEM LEHETSÉGES}
			\end{cases}\]
			$\Rightarrow\quad g(x)=\frac{xy_0}{x^2+\underbrace{y_0^2}_{\not=0}}\quad (x\in\R)$ és ez folytonos (ld. racionális törtfüggvények)
			\item Házi feladat.
			\item \[ f(0,0)=0=?=\lim_{(x,y)\to(0,0)}\frac{xy}{x^2+y^2} \]
			\[ y=0\quad \Rightarrow\quad \lim_{x\to0}\left(\frac{x\cdot0}{x^2+0^2}\right)=\lim_{x\to0}\frac{0}{x^2}=\lim_{x\to0}(0)=0 \]
			Ha van $\lim$ akkor az csak 0 lehet. De:
			\[ y=x\quad \text{mentén}\quad \lim_{(x,x)\to(0,0)}\frac{x^2}{2x^2}\quad\overset{x\not=0}{=} \quad \lim{x\to0}\frac{1}{2}=\frac{1}{2} \]
			$\Rightarrow\nexists\lim_{(0,0)}f\Rightarrow\quad f\notin C\{ (0,0) \}$ 
		\end{enumerate}
	\end{example}
	\begin{task}
		Vegyünk egy olyan függvényt, mely az origón $y$ tengely körül metszve folytonos, de 0-ban mégsem
		\[ f(x,y):=\begin{cases}
			\frac{x^2y}{x^4+y^2}\quad (x,y)\not=(0,0)\\
			0,\quad (x,y)=(0,0)
		\end{cases} \]
		\begin{enumerate}
			\item $\forall$ origón átmenő $l$ egyenes ,,mentén'' $f$ folytonos ($f|_l\in C$)
			\item $f\in C\{ (0,0) \}$
		\end{enumerate}
		\textit{Megoldás:}
		\begin{figure}[H]
			\centering
			\includegraphics[height=3cm]{kepek/26.png}
			\caption{Megjegyzendő, hogy $x=0$ nem függvény.}
		\end{figure}
		\begin{enumerate}
			\item\begin{enumerate}
				\item eset: $x=0\quad \Rightarrow\quad $
				\[ f(0,y)=h(y)=\begin{cases}
					\frac{0}{y^2}\quad \text{ha}\quad y\not=0
					0\quad y=0
				\end{cases}=0\quad (\forall x\in\R)\quad \Rightarrow\quad h\in C(\R) \]
				\item eset: $m=0\quad \Rightarrow\quad l(x)=y=0$\quad ($x$-tengely)
				\[ g(x):=f(x,0)=\begin{cases}
					\frac{0}{x^4}\quad x\not=0\\
					0,\quad x=0
				\end{cases}=0\quad \Rightarrow\quad  \]
				$g=0\quad \Rightarrow\quad g\in C(\R)$
				\item eset $y=mx\quad (m\in\R\setminus\{0\})\quad \Rightarrow$
				\[ g(x):=f(x,m)=\begin{cases}
					\frac{x^2mx}{x^4+m^2x^2},\quad \text{ha}\quad (x,mx)\not=(0,0)
					0\quad \text{ha}\quad (x,\underbrace{m}_{\not=0,\quad x=0}x)=(0,0)
				\end{cases}\quad \Rightarrow\quad g(x)=\begin{cases}
					\frac{mx}{x^2+m^2}\quad x\not=0\\
					0\quad x=0
				\end{cases}\quad \Rightarrow\quad   \]
				$g\in C(\R\setminus\{0\})$\quad (rac. törtfüggvények, nevező $\not=0$). És 0-ban:
				\[ g(0)=0=?=\lim_{x\to0}\left(\frac{mx}{x^2+m^2}\right)=\frac{0}{\underbrace{m^2}_{\not=0}}=0 \]
				$g\in C\{0\}$ is.
			\end{enumerate}
			\item $f\in C\{ (0,0) \}$\quad ui.:
			\[ f(0,0)=0=?=\lim_{(x,y)\to(0,0)}\left(\frac{x^2y}{x^4+y^2}\right) \]
			Ha $y=0\quad \Rightarrow\lim_{x\to0}\frac{0}{\underbrace{x^4}_{\not=0}}=\lim_{x\to0}(0)=0$
			
			Ha $y=x\quad \Rightarrow\quad \lim_{x\to0}\left(\frac{x^3}{x^4+x^2}\right)=\lim\left(\frac{x}{x^2+1}\right)=\frac{0}{1}=0$
			
			\begin{figure}[H]
				\centering
				\includegraphics[height=3cm]{kepek/27.png}
				\caption{}
			\end{figure}
			\[ \lim_{x\to0}\frac{x^2\cdot x^2}{x^4+x^4}=\lim_{x\to0}\frac{x^4}{2x^4}=\lim_{x\to0}\frac{1}{2}=\frac{1}{2}\not=0\quad \Rightarrow\quad  \]
			$\nexists\lim_{(0,0)}\quad \Rightarrow \quad f\notin C\{(0,0)\}$
		\end{enumerate}
		\begin{note}
			Ha azt írom hogy $g\equiv0$, az azt jelenti, hogy $g$ az azonosan 0 függvény.
		\end{note}
		\subsection{Differenciálás}
		\begin{task} Mutassok meg hogy erősen deriválható $f$ az $a:=(1,2)$ pontban.
			\[ f(x,y):=2x^2+3xy-y^2 \quad ((x,y)\in\R^2) \]
			$\Rightarrow\quad f\in D\{a\}\quad \text{és}\quad f'(1,2)=?\quad \text{és}$ ellenőrzés a parciálisokkal.
			
			\textit{Megoldás:}
			\begin{revision}
				$1\leq n,m\in\N\quad f\in\R^n\to\R^m,\quad a\in \Int \mathcal{D}_f,\quad f\in D\{ a \}\quad \Leftrightarrow \quad \exists L\in\LARGE(\R^n,\R^m)$.
				\[ \lim_{h\to0} \frac{\norm{f(a,h)-f(a)-L(h)}_{\R^m}}{\norm{h}_{\R^n}}=0 \]
			\end{revision}
			\begin{revision}
				$L\in L(\R^n,\R^m)\quad \Leftrightarrow\quad \exists A\in\R^{m\times n}\quad L(h)=Ah\quad (\forall h\in\R^n)$
				 Ekkor: $f'(a)=A$ Jacobi mátrix.
			\end{revision}
			\textit{Visszatérve a feladathoz:}
			Most: $f:\R^2\to\R\quad \Rightarrow\quad f'(1,2)\in\R^{1\times 2}\equiv \R{^2}$
			Legyen: $h:=(x,y);\quad a=(1,2)$
			\[ f(a+h)-f(a)=f(x+1,y+2)-f(1,2)=2(x+1)^2+3(x+1)(y+2)-(y+2)^2-4=\]\[=2x^2+4x+2+3xy+6x+3y+6-y^2-4y-4-4=\underbrace{10x-y}_{\text{lineáris tagok}}+(\underbrace{2x^2+3xy-y^2}_{\text{maradék}}) \]
			\begin{figure}[H]
				\centering
				\includegraphics[height=3cm]{kepek/28.png}
				\caption{}
			\end{figure}
			Legyen: $L(h):=L(x,y):=10x-y=\langle(10,-1),(x,y)\rangle=A\cdot\binom{x}{y}=Ah$ ahol $A=(10,-1)\in\R^{1\times2}$
			
			Kell, hogy igaz (*)
			\[ \lim_{(x,y)\to(0,0)}=\frac{\overbrace{|2x^2+3xy-y^2|}^{\norm{.}_a}}{\norm{x,y}_{\R^2}} \]
			Tökmindegy, milyen normával dolgozunk. Itt célszerű (mint az esetek többségében is) végtelen normával dolgozni.
			\[ \overset{\text{háromszög}}{\underset{\text{egyenlőtelenség}}{\leq}}\lim_{(x,y)\to(0,0)}\frac{2|x|^2+3|x|\cdot|y|+|y|^2}{\norm{(x,y)}_\infty}=\frac{0}{0}\quad \overset{|x|,|y|\leq\norm{(x,y)}_\infty=\max\{|x|,|y|\}}{\leq}\quad \lim_{(x,y)\to(0,0)}\frac{6\cdot\norm{(x,y)}_\infty}{\norm{(x,y)}_\infty}=\]
			\[=6\cdot0=0 \]
			Mivel normák hányadosa nemnegatív, így alsó becslésünk is van.
			\[ \Rightarrow\quad (\text{közrefogás})\quad \text{a definícióbeli $\lim$ valóban } 0 \]
			$\Rightarrow \quad f\in D\{(1,2)\}$ és $f'(1,2)=(10,-1)$.
			\textit{Ellenőrzés:}
			\begin{revision}
				Ha $f\in\R^n\to\R$ típus\quad $(m=1)$\quad és\quad $f\in D\{a\}\quad \Rightarrow$
				\[ f'(a)=\grad f(a)=(\partial_1 f(a),\partial_2 f(a),\dots,\partial_nf(a)) \]
				Most: $((x,y)\in\R^2)$
				\[ \partial_1 f(\overbrace{x}^{\text{változó}},\overbrace{y}^{\text{konstans}})=4x+3y \]
				\[ \partial_2f(\underbrace{x}_{\text{fix}},\underbrace{y}_{\text{változó}})=3x-2y  \]
				$\Rightarrow\quad \exists\partial_1 f,\partial_2 f:\quad \R^2\to\R$ parciális derivált fvek és folytonosak $\R^2$-en polinomok
				
				$\overset{\text{tétel}}{\Rightarrow}\quad f\in D(\R^2)$ és 
				\[ f'(x,y)=\grad f(x,y)=(\partial_1 f(x,y), \partial_2f(x,y)) \]
				Most:
				\[ f'(x,y)=(4x+3y,3x-2y)\quad (\forall(x,y)\in\R^2) \]
				Spec:
				\[ f'(1,2)=(4\cdot1+\cdot2;3\cdot1-2\cdot2)=(10;-1) \]
			\end{revision}
		\end{task}
		\begin{note}
			Itt $\equiv$ izomorfiát jelöl.
		\end{note}
		
		\begin{task}
			\[ f(x,y)=\begin{bmatrix}
				x^3+xy\\
				x-y^2\\
				1+y
			\end{bmatrix}\quad (x,y)\in\R^2;\quad \Rightarrow\quad \exists f'(1,2)=? \]
			Ellenőrzés jacobival.
			\begin{figure}[H]
				\centering
				\includegraphics[height=3cm]{kepek/29.png}
				\caption{}
			\end{figure}
			Def: $h:=(x,y); a:=(1,2)$
			\[ f(a+h)-f(a)=f(x+1,y+2)-f(1,2)=\begin{bmatrix}
				(x+1)^2+(x+1)(y+2)\\
				x+1-(y+2)^2\\
				1+(y+2)
			\end{bmatrix}-\begin{bmatrix}
				3\\
				-3\\
				3
			\end{bmatrix}=\]
			\[=\begin{bmatrix}
				x^3+3x^2+3x+1+xy+2x+y+2-3\\
				x+1-y^2-4y-4+3\\
				y+3-3
			\end{bmatrix}=\begin{bmatrix}
			5x+y\\
			x-4y\\
			y
			\end{bmatrix}-\begin{bmatrix}
				x^3+3x^2+xy\\
				-y^2\\
				0
			\end{bmatrix} \]
			Legyen: $L(h):=L(x,y):=\begin{bmatrix}
			5x+y\\
			x-4y\\
			y
			\end{bmatrix}=\begin{bmatrix}
			5&1\\
			1&-4\\
			0&1
			\end{bmatrix}\cdot \begin{bmatrix}
			x\\
			y
			\end{bmatrix}$
			\[ 0\leq\lim_{(x,y)\to(0,0)}\frac{\norm{
					\begin{bmatrix}x^3+3x^2\\
					-y^2\\
					0\end{bmatrix}
			}_{\R^3}}{\norm{(x+y)}_{\R^2}}=\lim_{(x,y)\to(0,0)}\frac{|x^2+3x^2+xy|+|-y^2|+|0|}{\norm{(x,y)}_\infty}\leq\lim_{(x,y)\to(0,0)}\frac{|x|^3+3|x|^2+|x|\cdot|y|+|y|^2}{\norm{(x,y)}_\infty}\leq\lim_{(x,y)\to(0,0)}\frac{\norm{(x,y)}^3_\infty+5\norm{(x,y)}^2_\infty}{\norm{(x,y)}_\infty}=\lim_{(x,y)\to(0,0)}(\norm{(x,y)}^2_\infty+5\norm{(x,y)}_\infty)=0^2+5\cdot0=0\quad \Rightarrow \]
			a def-beli $\lim=0\quad \Rightarrow
			\quad$
			\[ f\in D(1,2)\quad \text{és}\quad f'(1,2)=\begin{bmatrix}
				5&1\\
				1&-4\\
				0&1
			\end{bmatrix}\in\R^{3\times 2} \]
			\textit{Ellenőrzés}
			Ha $f\in D\{(x,y)\}\quad \Rightarrow$
			\[ f'(x,y)=\begin{bmatrix}
				\grad f_1(x,y)\\
				\grad f_2(x,y)\\
				\grad f_3(x,y)
			\end{bmatrix} =\begin{bmatrix}
				\partial_1 f_1(x,y)&\partial_2 f_1(x,y)\\
				\partial_1 f_2(x,y)&\partial_2 f_2(x,y)\\
				\partial_1 f_3(x,y)&\partial_2 f_3(x,y)
			\end{bmatrix}=\begin{bmatrix}
				3x^2+y&y\\
				1&-2y\\
				0&1
			\end{bmatrix}\quad (\forall (x,y)\in\R^2) \]
			
			Enek a mátrixnak mind a 6 komponense folytonos fv (polinomok)$\quad \Rightarrow\quad f\in D(\R^2)$
		\end{task}
	\end{task}
	\subsubsection{Irány menti és parciális derivált}
	\begin{revision}
		\begin{enumerate} 
			\item $f\in\R{^n}\to\R;\quad 1\leq n\in\N;\quad a\in\int\mathcal{D}_f;\quad e\in\R^n;\quad \norm{e}_{\R^n}=1 m\quad F(t):=f(a+te),\quad t\in k_\partial(0)\quad (\partial$ alkalmas $a+te\in\mathcal{D}_f).$ Ha
			\[ \exists F'(0):=\partial_ef(a)\in\R \]
		\end{enumerate}
	\end{revision}
	\begin{task}
		\[ f(x,y):=x^2-xy+y^2\quad ((x,y)\in\R^2) \]
		Legyen továbbá $e$ az $x$ tengellyel 60 fokos szöget bezáró egységvektor.
		Ekkor:
		\begin{enumerate}
			\item $\partial_{e}f(1,1)=?$
			\item Milyen $e$ mentén lesz $\partial_{e}f(1,1)$ min ill. max?
		\end{enumerate}
		\begin{figure}[H]
			\centering
			\includegraphics[height=3cm]{kepek/30.png}
			\caption{$l(t)=a+te,\quad  t\in\R$}
		\end{figure}
		\textit{Megoldás:} $a:=(1,1), e=\left(\frac{1}{2};\frac{\sqrt{3}}{2}\right)\in\R^2$
		\[ \Rightarrow\quad \norm{e}_2=\sqrt{\frac{1}{4}+\frac{3}{4}}=2\checkmark \]
		\[ \Rightarrow \quad F(t)=f(a+te)=f((1,1)+t\left(\frac{1}{2};\frac{\sqrt{3}}{2}\right))=f\left(1+\frac{t}{2};1+\frac{t\sqrt{3}}{2}\right)=\left(1+\frac{t}{2}\right)^2-\left(1+\frac{t}{2}\right)\left(1+\frac{t\sqrt{3}}{2}\right)+\left(1+\frac{t\sqrt{3}}{2}\right)^2\]
		\[\quad \Rightarrow\quad F\in D\quad \text{és} \]
		\[ F'(t)=2\cdot\left(1+\frac{t}{2}\right)\cdot\frac{1}{2}-\frac{1}{2}\left(1+\frac{t\sqrt{3}}{2}\right)-\left(1+\frac{t}{2}\right)\frac{\sqrt{}}{2}+2\left(1+\frac{t\sqrt{3}}{2}\right)\cdot\frac{\sqrt{3}}{2}\quad (t\in\R) \]
		\[\Rightarrow\quad \text{(def)}\quad \partial_ef(1,1)=F'(0)=2\cdot\frac{1}{2}-\frac{1}{2}-\frac{\sqrt{3}}{2}+2\cdot\frac{\sqrt{3}}{2}=\frac{1}{2}+\frac{\sqrt{3}}{2}=\frac{\sqrt{3}+1}{2}\in\R \]
		\begin{figure}[H]
			\centering
			\includegraphics[height=3cm]{kepek/31.png}
			\caption{}
		\end{figure}
		
	\end{task}
	\begin{note}
		Ha nincs külön kijelentve, akkor kettes normával dolgozunk.
	\end{note}
	\begin{note}
		\begin{enumerate}
			\item Ha $f\in D\{a\}\quad \Rightarrow\quad \forall e\in\R^n,\quad \norm{e}=1$
			\[ \exists\partial_ef(a)=f'(a)\quad e=\langle f'(a),e\rangle \]
			Most: $\forall (x,y)\in\R^2$
			\begin{align*}
				\partial_1f(x,y)=2x-y\\
				\partial_2f(x,y)=-x+2y	
			\end{align*}
			Mivel $\exists\partial_1f, \partial_2f: \R^2\to\R$ és folytonosak:
			\[ \R^2\text{-en}\quad \Rightarrow\quad f\in D(\R^2) \]
			SPeciálisan:
			\[ f'(1,1)=(\partial_1f(1,1);\partial_2f(1,1))=(2\cdot1-1;-1+2\cdot1)=(1,1)=\grad f(1,1)\quad \Rightarrow\quad \exists\partial_ef(1,1)=\langle f'(1,1);e\rangle \]
			\[ =\langle(1,1),\left(\frac{1}{2};\frac{\sqrt{3}}{2}\right)=1\cdot\frac{1}{2}+1\cdot\frac{\sqrt{3}}{2}=\frac{\sqrt{3}+1}{2} \]
			\item Legyen $e:=(\cos\alpha; \sin\alpha)\quad \alpha\in[0;2\pi]$
			\[ \Rightarrow\quad \exists\partial_ef(1,1)=\langle f'(1,1);e\rangle\quad \overset{\norm{e}_2}{=}\quad \langle(1,1);(\cos\alpha,\sin\alpha)\rangle=\sin\alpha+\cos\alpha=:g(\alpha)\quad \Rightarrow \]
			$g'(\alpha)=\cos\alpha-\sin\alpha=0\quad \Leftrightarrow\quad \tg\alpha=0$
			\begin{figure}[H]
				\centering
				\includegraphics[height=3cm]{kepek/32.png}
				\caption{}
			\end{figure}
			\begin{figure}[H]
				\centering
				\includegraphics[height=3cm]{kepek/33.png}
				\caption{$e_{\min}= \left(-\frac{\sqrt{2}}{2};-\frac{\sqrt{2}}{2}\right),\quad e_{\max}= \left(\frac{\sqrt{2}}{2};\frac{\sqrt{2}}{2}\right)$}
			\end{figure}
			Véletlen, higy a max grádiens irányú? hf.
		\end{enumerate}
	\end{note}
	\begin{note}
		Legyen
		\[ f(x,y):=\begin{cases}
			1\quad (x,y)\in A\cup B\cup C\\
			0\quad \text{különben}
		\end{cases} \]
		\begin{figure}[H]
			\centering
			\includegraphics[height=3cm]{kepek/34.png}
			\caption{}
		\end{figure}
		$e$ fix.
		\begin{figure}[H]
			\centering
			\includegraphics[height=3cm]{kepek/35.png}
			\caption{}
		\end{figure}
		\[ \Rightarrow\quad \exists\partial_ef(0,0)=0\quad (\forall e\in\R^2,\quad \norm{e}=1) \]
		De:
		\[ f\notin C\{(0,0)\} \quad \text{ui.}\quad \nexists\lim_{(0,0)}f\quad \Rightarrow\quad f\notin D(0,0) \]
	\end{note}
	\begin{center}
		\textit{,,Itt áll Gizi a Himalája tetején. Azt kérdezzük Gizitől, hogy indulj el, hogy a lehető legmeredekebb pályán menjél?''}
		
		/Filipp/
	\end{center}
	\begin{task}
		\[ f(x,y)=e^x\cdot y+x\cos y\quad ((x,y)\in\R^2) \]
		$a:=(0,1);\quad u:=(1,-\sqrt{3})\quad \Rightarrow\quad \partial_ef(a)=?\quad \text{ahol}\quad e$ az $a$ irányú egységvektor (kettes normával)
		
		Egyes normával hf.
		
		\textit{Megoldás (2es norma:)} 
		%TODO 07
		\[ e:=\frac{u}{\norm{u}_2} \]
		Ha más normában szmolnánk, ahhol más norma szerint kell ozstnai is.
		\[ e=\frac{(1,-\sqrt{3})}{\norm{1,-\sqrt{3}}_2}=\frac{(1,-\sqrt{3})}{ \sqrt{1^2+3}}=\left(\frac{1}{2},-\frac{\sqrt{3}}{2}\right) \]
		\begin{align*}
			\partial_1f(x,y)=&ye^x+\cos y\\
			\partial_2f(x,y)=&e^x-x\sin y
		\end{align*}
		$((x,y)\in\R^2)$, és $\partial_1f\partial_2f:\R^2\to\R$ folytonosak
		\[ f\in D(\R^2)\quad \text{spec.}\quad f\in D\{(0,1)\}\quad \text{és}\quad \Rightarrow\quad \exists\partial_ef(0,1)=\langle f'(0,1);e\rangle=\]\[\langle \partial_1f(0,1),\partial_2f(0,1);\left(\frac{1}{2},-\frac{\sqrt{3}}{2}\right)\rangle=\frac{1}{2}(1+\cos 1)-\frac{\sqrt{3}}{2}=\frac{1-\sqrt{3}-\cos1}{2} \]
	\end{task}
	\begin{task}$f:\R^2\to\R$ úgy, hogy: \quad ($(x,y)\in\R^2)$
		\[ \begin{cases}
			\partial_1f(x,y)=x^2y\\
			\partial_2f(x,y)=1+\frac{x^3}{3}
		\end{cases} \]
		$f=?$
		
		\textit{Megoldás:} 
		\[f(x,y)=\int x^2y\,dx =y\cdot\int x^2\,dx=y\cdot\frac{x^3}{3}+c(y)\quad \text{ahol}\quad c\in\R\to\R,\quad c\in D \]
		Illetve:
		\[ \exists\partial_2f(x,y)=\partial_2\left(y\frac{x^3}{3}+c(y)\right)=\frac{x^3}{3}+x'(y)\quad \overset{\text{feltétel}}{=}\quad 1+\frac{x^3}{3}\quad \Rightarrow\quad c'(y)=1, \quad \text{azaz}\]
		\[\quad c(y)=\int1\,dy=y+c, \quad c\in\R \]
		Mivel most az $x$-es tagok kiesnek, így lesz eredmény. 
		Így a keresett függvények:
		\[ f(x,y)=y\frac{x^3}{3}+y+c\quad ((x,y)\in\R^2,\quad c\in\R) \]
	\end{task}
	\begin{note}
		Vajon csak $+c$ a derivált vége? A megoldás az hogy nem, bármilyen deriválható egyváltozós függvény mehet oda, mely $y$-tól függ.
	\end{note}
	\begin{note}
		\begin{enumerate}
			\item Legyen $g(x,y):=(x^2y;1+\frac{x^3}{3})\quad ((x,y)\in\R^2);\quad \Rightarrow\quad f:\R^2\to\R^2;$ azt kaptuk, hogy $\exists f:\R^2\to\R,\quad f\in D (?)$ és $f'(x,y)=(\partial_1f(x,y),\partial_2f(x,y))=g(x,y)$ vagy $f'=g$. EZt mondjuk, hogy $f$ a $g$ primitív függvénye.
		\end{enumerate}
	\end{note}
	\begin{task}
		\[ f(x,y):=\sqrt{|xy|}\quad ((x,y)\in\R^2) \]
		Bizonyítsuk be, hogy \begin{enumerate}
			\item $f\in C(0,0)$
			\item $\exists\partial_1f(0,0);\quad \partial_2f(0,0)=?$
			\item $f\notin D\{(0,0)\}$
		\end{enumerate}
		\begin{enumerate}
			\item 
			\[ \varepsilon>0\quad \text{fix.}\quad |f(x,y)-f(0,0)|=\sqrt{|x,y|}\quad \overset{\text{sz-m}}{\leq}\quad \frac{|x|+|y|}{2}=\frac{1}{2}\cdot\norm{(x,y)-(0,0)}_1<\varepsilon \]
			Ha $\norm{(x,y)-(0,0)}_1<2\varepsilon\quad \Rightarrow\quad \partial:=2\varepsilon$ jó a definícióhoz\quad $\Rightarrow\quad f\in\C\{(0,0)\}$
			\item \[\partial_1f(0,0)=g(0)=\lim_{x\to0}g(y)-g(0)=\lim_{x\to0}\frac{f(x,0)-f(0,0)}{x-0}=\lim_{x\to}\left(\frac{\sqrt{|x\cdot0}}{x}\right)=\lim_{x\to0}\frac{0}{x}=\lim_{x\to0}0=0 \]
			Illetve:
			\[ \partial_2f(0,0)=\lim_{y\to0}\frac{(0,y)-f(0,0)}{y-0}=\lim_{y\to0}\left(\frac{\sqrt{|0\cdot y|}}{y}\right)=0 \]
			AZaz, ha $f\in D\{(0,0)\}\quad \Rightarrow\quad f'(0,0)=(\partial_1f(0,0), \partial_2f(0,0))=(0,0)$
			A def. beli lim:
			\[ \lim_{(x,y)\to(0,0)}\frac{|f(x,y-f(0,0)-\partial_1f(0,0)-\partial_2f(0,0)}{\norm{x,y}_{\R^2}}=\lim_{(x,y)\to(0,0)}\frac{\sqrt{|xy|}}{|x|+|y|}\not=0 \]
			Ugyanis $x=y$ mentén:
			\[ \lim_{x\to0}\frac{\sqrt{|x|^2}}{2|x|}=\lim_{x\to0}\frac{|x|}{2|x|}=\frac{1}{2}\not=0\quad \Rightarrow\quad f\notin D\{(0,0)\} \]
		\end{enumerate}
	\end{task}
	%\subsection{Folytatás}
	\begin{task}
		\[ f(x,y)=\begin{cases}
			(x^2+y^2)\cdot\sin\frac{1}{x^2+y^2},\quad x^2+y^2\not=0\\
			0,\quad x^2+y^2=0
		\end{cases} \]
		Feladat:
		\begin{enumerate}
			\item $\partial_1f, \partial_2f=?$
			\item $\partial_1f, \partial_2f\notin C\{(0,0)\}$
			\item $f\in D\{(0,0)\}$
		\end{enumerate}
		Megoldás: 
		\[ \forall(x,y)\in\R^2\setminus\{(0,0)\}\Rightarrow \]
		\[ \partial_1f(x,y)=2x\cdot\sin\frac{1}{x^2+y^2}+(x^2+y^2)\cdot\cos\frac{1}{x^2+y^2}\cdot(-1)(x^2+y^2)\cdot2x=2x\cdot\sin\frac{1}{x^2+y^2}-\frac{2x}{x^2+y^2}\cdot\frac{1}{x^2+y^2} \]
		\[ \partial_1f(0,0)=\lim_{x\to0}\frac{f(x,0)-f(0,0)}{x-0}=\lim_{x\to0}\frac{x^2\cdot\sin\frac{1}{x^2}}{x}=\lim_{x\to0}(x\cdot\sin\frac{1}{x^2})=0 \]
		Összefoglalva:
		\[ \partial_1f(x,y)=\begin{cases}
			2x\sin\frac{1}{x^2+y^2}-\frac{2x}{x^2+y^2}\cos\frac{1}{x^2+y^2}\quad (x,y)\not=(0,0)\\
			0;\quad (x,y)=(0,0)
		\end{cases} \]
		Hasonlóan
		\[ \partial_2f(x,y)=\begin{cases}
			?\quad \text{HF.}\\
			0,\quad (x,y)=(0,0)
		\end{cases} \]
		\[ \partial_2f(0,0)=\lim_{y\to0}\frac{f(0,y)-f(0,0)}{y-0}=\lim_{y\to0}(y\cdot\sin\frac{1}{x^2})=0 \]
		Ha $f\in D\{(0,0)\}\quad \Rightarrow\quad f'(0,0)=\big(\partial_1f(0,0),\partial_2f(0,0)\big) =(0,0)=\grad f(0,0)$.
		(iii)
		\[ \lim_{(x,y)\to(0,0)}\frac{|f(x,y)-f(0,0)-\partial_1f(0,0)x-\partial_2f(0,0)y}{\norm{(x,y)}_{\R^2} }=\lim_{(x,y)\to(0,0)}\frac{\left|(x^2+y^2)\sin\frac{1}{x^2+y^2}\right|}{\norm{(x,y)}_{2}}= \]
		Itt célszerű $x^2+y^2$ miat kettes normát váalsztani.
		\[ \lim_{(x,y)\to(0,0)}\sqrt{x^2+y^2}\cdot\left|\sin\frac{1}{x^2+y^2}\right|=0 \]
		$\Rightarrow\quad f\in D\{(0,0)\}$ és $f'(0,0)=(0,0)$.
		
		(ii). 
		\[ \partial_1f(0,0)=0,\quad \text{elég azonban belátni hogy}\quad \lim_{(x,y)\to(0,0)}\partial_1f(x,y)\not=0 \]
		Lehet érezni, hogy a $\partial_1f(x,y)$ függvény $\frac{2x}{x^2+y^2}$ tagjával lesz baj. Ha letakarjuk $y$-t, akkor a tört $\pm\infty$-hez tart. Így tekintsük et
		\[ y=0\quad \text{mentén}\quad \Rightarrow\quad \partial_1f(x,0)=2x\sin\frac{1}{x^2}-\frac{2}{x}\cdot\cos\frac{1}{x^2} \]
		Átviteli elv: Válasszuk az $(x_n)$ sorozatot úgy, hogy $\cos\frac{1}{x^2_n}=1$ legyen:
		\[ \frac{1}{x^2_n}=2n\pi,\quad n\in\N\quad \Rightarrow\quad x_n:=\frac{1}{\sqrt{2n\pi}}\to0\quad \text{ha}\quad (n\to\infty) \]
		Ekkor:
		\[ \partial_1f(x_n,0)=\frac{2}{\sqrt{2n\pi}}\sin(2n\pi)-2\sqrt{2n\pi}\cos(2n\pi)=-2\sqrt{qn\pi}\to-\infty\quad \text{ha}\quad n\to\infty \]
		$\Rightarrow \partial_1f\notin C\{(0,0)\}$
	\end{task}
	\begin{exercise}
		$\partial_2f\notin C\{(0,0)\}$
	\end{exercise}
	\begin{note}
		Nem áganként (pl. fent is ugye a függvény kétágú) kell deriválni, ez tipikus hiba. Gondoljunk arra, hogy egy adott pont környezetét is értelmezni kell.
	\end{note}
	\begin{note}
		Tehát az erős deriválhatóság nem vonja maga után a parciálisok folytonosságát
	\end{note}
	\begin{task}
		\[ f(x,y):=\begin{cases}
			xy\cdot\frac{x^2-y^2}{x^2+y^2},\quad (x,y)\in \R^2\setminus\{(0,0)\}\\
			0,\quad (x,y)=(0,0)
		\end{cases} \]
		Lássuk be, hogy $\partial_{12}f(0,0)\not=\partial_{21}f(0,0)$
		
		\textit{Megoldás:}
		\[ \partial_{12}f(0,0)=\partial_1(\partial_2f)(0,0)=\lim_{x\to0}\frac{\partial_2f(x,0)-\partial_2f(0,0)}{x-0}=\lim_{x\to0}\frac{1}{x}\left[\lim_{y\to0}\frac{f(x,y)-f(x,0)}{y-0}-\lim_{\to0}\frac{f(0,y)-f(0,0)}{y-0}\right]= \]
		\[= \lim_{x\to0}\frac{1}{x}\left[\lim_{y\to0}\frac{xy\cdot\frac{x^2-y^2}{x^2+y^2}-0}{y}-\lim_{y\to0}\frac{0-0}{y}\right]=\lim_{x\to0}\frac{1}{x}\cdot\lim_{y\to0}\left(x\frac{x^2-y^2}{x^2+y^2}\right)=\lim_{x\to0}\left(\frac{1}{x}\cdot x\cdot\frac{x^2}{x^2}\right)=\lim_{x\to0}(1)=1 \]
	\end{task}
	\begin{exercise}
		$\partial_{21}f(0,0)=-1$
	\end{exercise}
	\subsection{Taylor-formula}
	\begin{revision}
		\[ 1\leq n\in\N;\quad f\in\R^n \to\R;\quad a\in\Int D_f;\quad s\in\N ;\quad \exists k(a):\quad \forall x\in k(a):\quad f\in D^{s+1}\{x\}\quad \Rightarrow\]
		\[\quad \forall h\in \R^n\setminus\{0\}\quad \exists\nu(0,1) \]
		\[ f(a+h)=f(a)+\sum_{k=1}^s \sum_{ \substack{i\in\N^n\\|i|=k} }\frac{\partial^if(a)}{i!}h^i+\sum_{i\in\N^n}\frac{\partial^if(a+\nu h)}{i!}h^i \]
	\end{revision}
	\begin{example}
		\[ f(x,y)=x^2y+xy^2-2xy\quad ((x,y)\in\R^2) \]
		Adjuk meg azon $a_{nk}$ együtthatókat, melekkel felírható:
		\[ f(x,y)=\sum_{n,k=0}^{+\infty}a_{nk}\cdot(x-1)^n(y+1)^k \]
		(azaz $(x-1)^n(y+1)^n$ lineáris kombinációjaként)
		\begin{figure}[H]
			\centering
			\includegraphics[height=3cm]{kepek/37.png}
			\caption{Taylor polinomok. Melyik elsőfokú függvény közelíti a legjobban $f$-t? Melyik másodfokú?}
		\end{figure}
		%TODO 01 Melyik elsőfokú függvény közelíti a legjobban? T_0f(x)=f(a)
		%TODO másodfokú T_1f(x)=f(a)+f'(a)(x-a)
		%TODO másodfokú: T_2f(x)=f(a)+f'(a)(x-a)+\frac{f''(a)}{2!}(x-a)^2
		$a:=(1,-1);\quad a+h:=(x,y)\quad \Rightarrow\quad h=(x,y)-(1,-1)=(x-1,y+1)$
		\[ |i|=i_1+i_2+\ldots+i_n\quad i!:=i_1!i_2!\cdot\ldots\cdot i_n!;\quad \partial^if(a):=\partial_{\underbrace{1\cdots1}_{i_1}}\partial_{\underbrace{2\cdots2}_{i_2}}\cdots\partial_{\underbrace{n\cdots n}_{i_n}}f(a);\quad h^i=h_1^{i_1}\cdot h_2^{i_2}\dots h_n^{i_n}\quad h\in\R^n \]
		Folytatván:
		\[ f(1,-1)=2\quad (k=0) \]
		\[ k=1\quad |i|=2 \]
		\[ i=(1;0)\to\frac{\partial_1f(1,-1)}{1!\cdot0!}(x-1)^1(y+1)^0 \]
		\[ i=(0,1)\to\frac{\partial_2f(1,-1)}{0!1!}(x-1)^0(y+1)^1 \]
		Tehát au 1. fokú tagok:
		\[ \partial_1f(1,-1)(x-1)+\partial_2f(1,-1)(y+1); \]
		\begin{align*}
			k=2&\quad \Rightarrow\quad i\in\N^2,\quad |i|=2\quad \Rightarrow\quad\\
			i=(2,0)&\quad \to\quad \frac{\partial_{11}f(1,-1)}{2!0!}(x-1)^2(y+1)^0\\
			i=(1,1)&\quad \to\quad \frac{\partial_{12}f(1,-1)}{1!1!}(x-1)^1(y+1)^1\\
			i=(0,2)&\quad \to\quad \frac{\partial_{22}f(1,-1)}{0!2!}(x-1)^0(y+1)^2
		\end{align*}
		Befejezésül:
		\[ \frac{1}{2}\partial_{11}f(1,-1)(x-1)^2+\partial_{12}f(1,-1)(x-1)(y+1)+\frac{1}{2}\partial_{22}f(1,-1)(y+1)^2 \]
		\begin{align*}
			k=3&\quad \Rightarrow\quad i\in\N^2,\quad |i|=3\quad \Rightarrow\quad\\
			i=(3,0)&\quad \to\quad \frac{\partial_{111}f(3,-1)}{2!0!}(x-1)^3(y+1)^0\\
			i=(2,1)&\quad \to\quad \frac{\partial_{112}f(1,-1)}{2!1!}(x-1)^2(y+1)^1\\
			i=(1,2)&\quad \to\quad \frac{\partial_{122}f(1,-1)}{0!2!}(x-1)^1(y+1)^2\\
			i=(0,3)&\quad \to\quad \frac{\partial_{212}f(1,-1)}{0!2!}(x-1)^1(y+1)^2
			%TODO WHAT?
		\end{align*}
		Tehát az együtthatók:
		\begin{align*}
			\partial_1f(x,y)&=2xy+y^2-2y;\quad \partial_1f(1,-1)=1\\
			\partial_2f(x,y)&=x^2+2xy-2y;\quad \partial_2f(1,-1)=-3\\
			\partial_{11}f(x,y)&=2y;\quad \partial_{11}f(1,-1)=-2\\
			\partial_{12}f(x,y)&=\partial_1(\partial_2f)(x,y)=2x+2y-2;\quad \partial_{12}f(1,-1)=-2\\
			\partial_{22}f(x,y)&=2x;\quad \partial_{22}f(1,-1)=2\\
			\partial_{111}f(x,y)&=\partial_1(\partial_{11}f)(x,y)=0\\
			\partial_{112}f(x,y)&=\partial_1(\partial_{12}f)(x,y)=2\\
			\partial_{122}f(x,y)&=\partial_1(\partial_{22}f)(x,y)=2\\
			\partial_{222}f(x,y)&=\partial_2(\partial_{22}f)(x,y)=0\\
		\end{align*}
		Ebből következik, hogy az összes $k\geq4$ esetén a $k$-adrendű parciális deriváltak mindegyike 0. Ezért, a negyedrendű hibatag 0, így a harmadrendű Taylor-polinom pontosan előállítja $f$-et.
		
		Tehát:
		\[ f(x,y)=T_3f(x,y)+0=f(1,-1)+\underbrace{\partial_1f(1,-1)(x-1)+\partial_2f(1,-1)(y+1)}_{k=1}+\]
		\[+\underbrace{\frac{1}{2}\partial_{11}f(1,-1)(x-1)^2+\partial_{12}f(1,-1)(x-1)(y+1)+\frac{1}{2}\partial_{22}f(1,-1)(y+1)^2}_{k=2}+\]
		\[+\underbrace{\frac{1}{6}\partial_{111}f(1,-1)(x-1)^3+\frac{1}{2}\partial_{112}f(1,-1)(x-1)^2(y+1)+\frac{1}{2}\partial_{122}f(1,-1)(x-1)(y+1)^2+\frac{1}{6}\partial_{222}f(1,-1)(y+1)^3}_{k=3} \]
		Tehát:
		\[ f(x,y)=2+(x-1)-3(y+1)-(x-1)^2-2(x-1)(y+1)+(y+1)^2+(x-1)^2(y+1)+(x-1)(y+1)^2 \quad (\forall(x,y)\in\R^2)\]
	\end{example}
	\begin{note}
		Az elsőfokú Taylor polinomra: ($n=2$)
		\[ T_1f(x,y)=f(a)+\partial_1f(a)(x-a_1)+\partial_2f(a)(x-a_2)=z \]
		az érinti sík egyenlete $a$-ban
		\[ z=2+(x-1)-3(y+1) \]
	\end{note}
	%\subsection{Taylor-polinomok folytatása}
	
	\begin{task}
		\[ f(x,y)=x^y+y^2\cdot\cos(x-1)\quad (x,y)\in\R^2,\quad x>0 \]
		Tekintsük az $a=(1,3)$ pontot és számítsuk ki $T_2f(x,y)$-t!
		
		\textit{Megoldás:} 
		\[ T_2f(x,y)=\overbrace{f(1,3)}^{k=0}+\overbrace{\partial_1f(1,3)(x-1)+\partial_2f(1,3)(y-3)}^{k=1}+\frac{1}{2}\partial_{11}f(1,3)\cdot(x-1)^2+\partial_{12}f(1,3)(x-1)(y-3)+\frac{1}{2}\partial_{22}f(1,3)(y-3)^2 \]
		\[ f(1,3)=1^3+9\cos0=10 \]
		Ekkor:
		\begin{align*}
			\partial_1f(x,y)=&\ y\cdot x^{y-1} + y^2\cdot(-\sin(x-1))\cdot1=y\cdot x^{y-1}-y^2\cdot\sin(x-1)\\
			\partial_2f(x,y)=&\ x^y\cdot\ln x+2y\cdot\cos(x-1)\\
			\partial_{11}f(x,y)=&\ y(y-1)x^{y-2}-y^2\cos(x-1)\\
			\partial_{12}f(x,y)=&\ y\cdot x^{y-1}\ln x + x^y\cdot\frac{1}{x}-2y\cdot\sin(x-y)=\partial_{21}f(x,y)\\
			\partial_{22}f(x,y)=&\ x^y(\ln x)^2 +2\cos(x-1)
		\end{align*}
		Így:
		\begin{align*}
			\partial_1f(x,y)=&\ 3\\
			\partial_2f(x,y)=&\ 6\\
			\partial_{11}f(x,y)=&\ 6-9=-3\\
			\partial_{12}f(x,y)=&\ 1\\
			\partial_{22}f(x,y)=&\ 2
		\end{align*}
		Ez alapján
		\[ T_2f(x,y)=10+3(x-1)+6(y-3)-\frac{3}{2}(x-1)^2+(x-1)(y-3)+(y-3)^2 \quad \big((x,y)\in\R^2\big) \]
	\end{task}
	\begin{note}\
		\[ T_1f(x,y)=10+3(x-1)+6(y-3)\quad \big((x,y)\in\R^2\big) \]
		Ez az érintő sík függvénye lesz az $(1,3)$-nak, vagy másképp:
		\[ z=10+3(x-1)+6(y-3)\quad \Leftrightarrow\quad  3x+6y-z=11. \]
		Ez utóbbi a legegyszerűbb alakja az érintő sík egyenletének.
	\end{note}
	\begin{note}
		ZH-ban maximum egy harmadrendű polinom várható, és le se kell vezetni multiindexekkel.
	\end{note}
	\begin{note}
		A formula felírása 1 pont, a deriváltak felírása szintén (darabonként), a megoldás felírása 1 pont (nem darabonként).
	\end{note}
	\begin{exercise}
		Adjuk meg a
		\[ f(x,y)=\sqrt{x^2+2y^2+3xy}\quad \big((x,y)\in(0,+\infty)\big) \]
		függvénynek az $a=(1,2)$ pontban az érintő síkjának egyenletét. (ez az elsőfokú Taylor polinom)
	\end{exercise}
	\begin{note}
		\[ f(a+h)=f(a)+\langle f'(a); h\rangle + \frac{1}{2}\langle f''(a)\cdot h, h\rangle+\nu(h)\cdot\norm{h}^2 \]
		ahol:
		\[ f\in\R^n\to\R;\quad f'(a)=\big(\partial_1 f(a),\ldots,\partial_nf(a) \big);\quad \left[ a+h=(x,y),\quad h=(x,y)-a\right] \]
		És a Hesse mátrix:
		\[
			\begin{bmatrix}
				\partial_{11}f(a)&\dots&\partial{n1}f(a)\\
				\partial_{12}f(a)&\dots&\partial{n2}f(a)\\
				\vdots&&\vdots\\
				\partial_{1n}f(a)&\dots&\partial{nn}f(a)
			\end{bmatrix} = \left[\partial_{ij}f(a) \right]^n_{i,j=1}
		\]
	\end{note}
	\subsection{Lokális szélső érték keresés}
	\begin{example}
		\[ f(x,y)=x^2+xy+y^2-5x-4y+1\quad \big((x,y)\in\R^2\big) \]
		lokális szélső érték = ?
		
		\textit{Megoldás:} Első lépésben határozzuk meg a parciális deriváltakat.
		\begin{align*}
			\partial_1f(x,y)&=2x+y-5 = 0\\
			\partial_2f(x,y)&=x+2y-4=0\\
		\end{align*}
		Ezt az egyenletrendszert oldjuk meg, így $x=2, y=1$. Stacionárius hely a $(2,1)$ pont, itt \textbf{lehet} lokális szélső értéke $f$-nek (mivel $f\in D(\R^2)$, hisz pl. $\exists \partial_1f,\partial_2f\in C(\R^2)$, így az összes lehetséges szélsőértékhely meglesz az $f'=0$ rendszerből.
		
		Második lépésben: 
		\begin{align*}
			\partial_{11}f(x,y)&=2 \\
			\partial_{12}f(x,y)&=1 \\
			\partial_{22}f(x,y)&=2 
		\end{align*} 
		$\forall(x,y)\in\R^2$ spec. $(2,1)$-ben is.
		\[ \Rightarrow\quad f(x,y)=\begin{bmatrix}
			2&1\\
			1&2
		\end{bmatrix}\quad \Rightarrow\quad f''(2,1)\begin{bmatrix}
			2&1\\
			1&2
		\end{bmatrix}\quad \Leftrightarrow\quad \varDelta_1=2>0,\quad \varDelta_2=3>0 \]
		Ahol $\varDelta_i$ az $i$-edik főminor. Mivel így ez a mátrix pozitív definit, a Sylvester tételből így következik, hogy $f''(2,1)$ lokális minimum hely, értéke $f(2,1)=6$
		%TODO 01
		\begin{figure}[H]
			\centering
			\includegraphics[height=4cm]{kepek/38.png}
			\caption{}
		\end{figure}
	\end{example}
	\begin{task}
		\[ f(x,y)=(1+e^y)\cos x-y\cdot 3^y\quad \big((x,y)\in\R^2 \big) \]
		Keressük meg a lkális szélső érték(ek)et.
		
		\textit{Megoldás:} Első lépés:
		\[\left.\begin{gathered}
			\partial_1f(x,y)=(1+e^y)(-\sin x)=0\\
			\partial_2f(x,y)=e^y\cdot\cos x-e^y-y\cdot e^y=0\\
		\end{gathered}\right\}  \]
		Ez alapján
		\[\begin{cases}
			\sin x=0\quad \Leftrightarrow\quad X_k=k\pi\quad (k\in\Z)\\
			\cos x-1-y=0\quad \Leftrightarrow\quad y_k=\cos(k\pi)-1=(-1)^k-1
		\end{cases}\]
		Ez alapján a stacionárius helyek: $\big((k\pi; (-1)^k-1)\quad k\in\Z\big)$, azaz
		\[ (2k\pi,0)\quad \text{és}\quad ((2k+1)\pi; -2) \quad (k\in\Z)\]
		\begin{figure}[H]
			\centering
			\includegraphics[height=4cm]{kepek/39.png}
			\caption{}
		\end{figure}
		Második lépés
		\begin{align*}
			\partial_{11}f(x,y)=-(1+e^y)\cos x\\
			\partial_{12}f(x,y)=-e^y\sin x\\
			\partial_{22}f(x,y)=e^y\cos x-e^y-e^y-ye^y
		\end{align*}
		\[ \Rightarrow\quad f''(x,y)=\begin{bmatrix}
			-(1+e^y)\cos x&-e^y\sin x\\
			-e^y\sin x&e^y\cos x-2e^y-ye^y
		\end{bmatrix}\quad \Rightarrow\quad \overbrace{f''(2k\pi,0)}^{k\in\Z}=\begin{bmatrix}
			-2&0\\
			0&-1
		\end{bmatrix},\quad \varDelta_1=-2>0, \quad \varDelta_2=2>0 \]
		Ez a Sylvester tétel szerint $f''(2k\pi,0)$ egy negatív definit mátrix$\quad \Rightarrow\quad \forall k\in\Z \quad (2k\pi,0)$ pontok lokális maximum helyek. Értékük: $(2k\pi,0)=(1+e^0)\cos2k\pi=2$.
		
		Illetve:
		\[f''((2k+1)\pi;-2)=\begin{bmatrix}
			1+\frac{1}{e^2}&0\\
			0&-\frac{1}{e^2}
		\end{bmatrix},\quad \varDelta_1=1+\frac{1}{e^2}>0,\quad \varDelta_2=-\frac{1}{e^2}<0 \]
		$\Rightarrow\quad f''((2k+1)\pi,2)\quad (\forall k\in\Z)$ indefinit mátrix. A másodrendű szükséges feltétel alapján ezek $((2k+1)\pi,2)\quad (k\in\Z)$ biztosan nem lokális szélső érték helyek (NYEREG PONTOK).
	\end{task}
	\begin{task}
		\[ f(x,y)=x^4+y^4-x^2-2xy-y^2\quad \big((x,y)\in\R^2\big) \]
		\textit{Megoldás:}
		\begin{align*}
			\partial_1f(x,y)=4x^3-2x-2y=0\\
			\partial_2f(x,y)=4y^3-2x-2y=0\\
		\end{align*}
		\[ 4x^3-4y^3=0\quad \Leftrightarrow\quad x^3=y^3\quad \Leftrightarrow\quad x=y \]
		\[ 4x^3-4x=0\quad \Leftrightarrow\quad 4x(x^2-1)=0 \]
		Ez alapján
		\begin{align*}
			x_1=y_1=0\\
			x_2=y_2=1\\
			x_3=y_3=-1\\
		\end{align*}
		Stacionárius helyek: $(0,0),\quad (1,1),\quad (-1,-1)$
		\begin{figure}[H]
			\centering
			\includegraphics[height=4cm]{kepek/40.png}
			\caption{}
		\end{figure}
		\begin{align*}
			\partial_{11}f(x,y)=12x^2-2
			\partial_{12}f(x,y)=-2
			\partial_{22}f(x,y)=12y^2-2
		\end{align*}
		A Hesse mátrix ez alapján
		\[ f''(x,y)=\begin{bmatrix}
			12x^2-2&-2\\
			-2&12y^2-2
		\end{bmatrix} \]
		Először vizsgáljuk meg az (1,1) helyen:
		\[ f''(1,1)=\begin{bmatrix}
			10&-2\\
			-2&10
		\end{bmatrix}\quad \delta_1=10>0,\quad \delta_2=96>0 \] 
		Pozitív definit mátrixok $\quad \Rightarrow\quad (1,1)$ és $(-1,-1)$ is minimum helyek, értékük $f(1,1)=f(-1,-1)=-2$
		
		Most tekintsük a (0,0) helyet:
		\[ f''(0,0)=\begin{bmatrix}
			-2&-2\\
			-2&-2
		\end{bmatrix}\quad \delta_1=2>0,\quad \delta_2=0 \]
		Negatív szemidefinit, itt msot nem alkalmazhatjuk a tételt, további vizsgálat kell.
		\begin{center}
			[gondolatmenet közepe]
			\smallskip
			
			\textit{,,Ha valakinek a nevében P betű van, az a tanítást és igehirdetést jelenti.''}
			
			/Filipp/
			\medskip
			
			\textit{,,Illés Zoltán''}
			
			/Husi/
		\end{center}
		Feladat folytatása a következő feladatban.
	\end{task}
	%\subsection{Folytatás}
	\begin{task}
		%TODO feladatfolytatás
		\[ f(x,y)=x^4+y^4-x^2-2xy-y^2 \]
		Megállapítottuk korábban, hogy a $(0,0)$ ponthoz kellett külön vizsgálat, lévén za általunk kapott mátrix negatív szemidefinit volt.
		\[ f(0,0)=0 \]
		
		\begin{figure}[H]
			\centering
			\includegraphics[height=4cm]{kepek/41.png}
			\caption{Ha (0,0) lokális minimum, akkor be kéne látunk, hogy a (0,0) körüli környezetben, melyben minden érték pozitív. Ha maximum, a környezetben negatívnak kell lennie. Ha egyik sem, akkor kell negatív és pozitív érték is.}
		\end{figure}
		\[  y=0\quad \text{mentén}\quad f(x,0)=x^4-x^2=x^2(x^2-1) \]
		\begin{figure}[H]
			\centering
			\includegraphics[height=4cm]{kepek/42.png}
			\caption{Ez alapján látszik, hogy a 0 körüli környezetnek mindig lesz olyan része, mely negatív értékeket vesz fel.}
		\end{figure}
		Próbáljunk keresni egy olyan részt ebben az 1 sugarú körben, melyben pozitív értékeket is felvesz a függvény.
		\[ f(x,y)=x^4+y^4-(x+y)^2 \]
		Ezt elérhetjük úgy, hogy a fenti képletből lenullázzuk a a $(x+y)^2$ tagot.
		\[ y=-x\quad \text{mentén}\quad f(x,-x)=2x^4>0\quad \forall x\in\R\setminus\{0\} \]
		\begin{figure}[H]
			\centering
			\includegraphics[height=4cm]{kepek/43.png}
			\caption{}
		\end{figure}
		Ezzel beláttuk, hogy a (0,0) pont minden környezetében találhatunk negatív és pozitív értékeket is, azaz
		
		Legyen $x_\delta:=\frac{\min\{1,\delta\}}{2}$
		\[ \forall\delta>0\quad \text{és}\quad k_\delta(0,0):\quad \overbrace{f(x_\delta,0)}^{>0}<f(0,0)=0<f\left(\frac{\delta}{2},-\frac{\delta}{2}\right)=2\cdot\frac{\delta^4}{16}=\frac{\delta^4}{8} \]
		\begin{figure}[H]
			\centering
			\includegraphics[height=4cm]{kepek/44.png}
			\caption{}
		\end{figure}
		$\Rightarrow \quad (0,0)$ nem lokélis szélső érték hely.
	\end{task}
	\begin{task}
		\[ f(x,y):=x^4+y^2\quad ((x,y)\in\R^2) \]
		Lokális szélső értéket keresünk.
		
		\textit{Megoldás:} Első lépésben:
		\begin{align*}
			\partial_1f(x,y)=4x^3=0\\
			\partial_1f(x,y)=2y=0
		\end{align*}
		Ebből követekzik hogy $(x,y)=(0,0)$.
		
		Második lépésben?
		\[ f''(x,y)=\begin{bmatrix}
			12y^2&0\\
			0&2
		\end{bmatrix}\quad ((x,y)\in\R^2)\quad \Rightarrow\quad f''(0,0)=\begin{bmatrix}
		0&0\\
		0&2
		\end{bmatrix},\quad \varDelta_1=0,\quad \varDelta_2=0 \]
		Ismét nem működnek a tételek, további vizsgálatra van szükség.
		\[ f(0,0)=0\leq x3+y^2=f(x,y)\quad (\forall(x,y)\in\R^2)\quad \Rightarrow\quad \text{abszolút minimum hely.} \]
	\end{task}
	\begin{task}
		\[ f(x,y)=x^3+y^2\quad ((x,y)\in\R^2) \]
		\textit{Megoldás:} ELső lépésben
		\begin{align*}
			\partial_1f(x,y)=3x^2=0\\
			\partial_2f(x,y)=2y=0
		\end{align*}
		Azaz $(x,y)=(0,0)$.
		
		Második lépésben:
		\[ f''(x,y)=\begin{bmatrix}
			6x&0\\
			0&2
		\end{bmatrix}\quad \Rightarrow\quad f''(0,0)=\begin{bmatrix}
			0&0\\
			0&2
		\end{bmatrix} \]
		Tételek ismét nem máködnek, tovább kell vizsgálni (0,0)-t. Kínálja magát az $y=0$ irány:
		\[ f(x,y)=x^3\quad \Rightarrow\quad f(x,0)=x^3=\begin{cases}
			>0\quad x>0\\
			<0\quad x<0
		\end{cases} \]
		$\Rightarrow\quad \forall\delta>0$ és $k_\delta(0,0)$ környezetben:
		\[ f\left(-\frac{\delta}{2},0\right)<f(0,0)=0<f\left(\frac{\delta}{2},0\right)\quad \Rightarrow\quad \text{(0,0) nem lokális szélső érték hely.} \]
		\begin{figure}[H]
			\centering
			\includegraphics[height=4cm]{kepek/45.png}
			\caption{}
		\end{figure}
	\end{task}
	\subsection{Abszolút szélső érték keresés}
	\begin{task}
		\[ f(x,y):=y(2y-3)\quad ((x,y)\in A) \]
		$A$: Az $>=x^2;\quad y=0;$ és $x=2$ egyenletű götbék által határolt korlátos és zárt. Keressünk abszolút szélső értéket $A$-n.
		\begin{figure}[H]
			\centering
			\includegraphics[height=4cm]{kepek/46.png}\quad \quad \quad 
			\includegraphics[height=4cm]{kepek/47.png}
			\caption{$A$ halmaz, így nézkez ki, egyenlőre csak egy fiktív ábra}
		\end{figure}
		Erre a problémakörre egy tételünk is van: $f\in C(A)$ és $A\subset\R^2$ korlátos és zárt halmaz ($\Leftrightarrow A$ kompakt): $\exists\min\mathcal{R}_f, \max\mathcal{R}_f$.
		
		Bontsuk szép az A halmazt belső pontokra és határpontokra. Belső pontokban tudunk deriválni, itt ha $f'=0$, akkor ott lehet szélső érték. A határpontoknál vissza kell vezetnünk egyváltozós esetre, és az 1 dimenziós technikákat vesszük elő.
		
		\medskip
		VIzsgáljuk a belső pontokat. Ha $(x,y)\in\Int A$:
		\begin{align*}
			\partial_1f(x,y)=2y=0\\
			\partial_1f(x,y)=2x=0\\
		\end{align*}
		Ebből a $\left(\frac{3}{2},0\right)$-t kapjuk, azonabn ez nem belső pont: $\left(\frac{3}{2},0\right)\notin\Int A$. Ebből már azt is tudhatjuk, hogy belső pontokban ($\Int A$) nem lehet szélső érték.
		\medskip
		
		Most vizsgáljuk a határpontokat, melyek azon pontok, melyeknek létezik olyan környezete, melyben belső és külső pont is található.
		\begin{figure}[H]
			\centering
			\includegraphics[height=4cm]{kepek/48.png}
			\caption{}
		\end{figure}
		
		$B:=\{(x,0)\in\R^2\ |\ 0\leq x\leq 2 \}$, vizsgáljuk az $f\big|_B$ függvényt!
		\[ g(x):=f(x,0)=0\quad x\in[0,2]\quad \Rightarrow\quad (x,0)\quad (x\in[0,2]) \]
		$C:=\{ (2,y)\in\R^2\ | \ 0\leq y\leq 4 \}$
		\[ h(y):=f(2,y)\quad y\in[0,4] \]
		Ha 
		\[ y\in(0,4)\quad \Rightarrow\quad h'(y)=1\not=0 \]
		$h$ függvény határai: 
		\begin{align*}
			y=0\quad \Rightarrow\quad (2,0) \quad \text{már volt.}\\
			y=4\quad \Rightarrow\quad (2,4)
		\end{align*}
		$D:=\{(x,x^2)\in\R^2\ | \ 0\leq x\leq 2\}$
		\[ \Rightarrow\quad l(x):=f(x,x^2)=x^2(2x-3)=2x^3-3x^2\quad x\in[0,2] \]
		Ha $x\in(0,2)\quad \Rightarrow\quad l'(x)=6x^2-6x=6x(x-1)$, ebből pedig $x_1=0\notin(0,2), \quad x_2=1\in(0,2)$ követekezik. Ebből megállapítható, hogy $(1,1)$ lehet szélső érték.
		
		A végpontok, bár mindegyik már megvolt:
		\begin{align*}
			x=0\quad \Rightarrow\quad (0,0)\\
			x=2\quad \Rightarrow\quad (2,4)
		\end{align*}
		
		Harmadik lépésben vessük össze az eddigi eredményeinket.
		\begin{align*}
			f(x,0)=&0\\
			f(2,4)=&4\cdot(2\cdot2-3)=4\quad \text{abszolút maximum}\\
			f(1,1)=&1\cdot(-1)=4\quad \text{abszolút minimum}\
		\end{align*}
	\end{task}
	\begin{task}
		\[ f(x,y)=x^3-3x^2-y^2 \]
		Legyen
		\[ (x,y)\in A:=\{ (x,y)\in\R^2\ | \ -1\leq x\quad \text{és}\quad x-1\leq y\leq 4 \} \]
		Az ábráról könnyen leolvasható, hogy $x\leq 5$.
		\begin{figure}[H]
			\centering
			\includegraphics[height=4cm]{kepek/49.png}
			\caption{Az $A$ halmaz}
		\end{figure}
		
		$f\in C(A)$ és $A$ kompakt (korlátos és zárt)\quad $\Rightarrow\quad \exists\min\mathcal{R}_f, \max\mathcal{R}_f$.
		
		Ha $(x,y)\in\Int A\quad \Rightarrow$
		\begin{align*}
			\partial_1f(x,y)=3x^2-6x=0\\
			\partial_2f(x,y)=-2y
		\end{align*}
		Ebből $x_1=y_1=0$, vbalamint $x_2=2,\quad y_2=0$ következik. 
		\[ (0,0)\in\Int A,\quad (2,0)\notin\Int A \]
		Így (2,0) nem lehet abszolút szélső érték, (0,0)-t pedig később kell vizsgálnunk.
		
		Most dolgozzunk  határokkal:
		\begin{figure}[H]
			\centering
			\includegraphics[height=4cm]{kepek/50.png}
			\caption{}
		\end{figure}
		
		
		$B:=\{ (-1,y)\in\R\ | \ -2\leq y\leq 4 \}$
		\[ g(y):=f(-1,y)=-3-y^2\quad y\in[-2,4] \]
		Ha $y\in(-2,y)\quad \Rightarrow\quad g'(y)=-2y=0\quad \Leftrightarrow\quad y=0\in(-2,y)\quad \Rightarrow\quad (-1,0)$.
		
		Végpontok:
		\begin{align*}
			y=-2\quad \Rightarrow\quad (-1,-2)\\
			y=4\quad \Rightarrow\quad (-1,4)
		\end{align*}
		Mind a kettő pontot meg kel majd vizsálunk.
		
		$C:=\{ (x,4)\in\R^2\ | \ -1\leq x\leq 5 \}$
		\[ l(x):=f(x,4)=x^3-3x^2-16\quad x\in[-1,5] \]
		Ha $x\in(-1,5)\quad \Rightarrow\quad l'(x)=3x^2-6x=3x(x-2)=0$, melyből $x_1=0,\quad x_2=2$,\quad $x_1,x_2\in(-1,5)$, azaz (0,4)-et és (2,4)-et is meg kell majd vizsgálni.
		
		Sarokpontok:
		\begin{align*}
			x=-1\quad \Rightarrow\quad (-1,4)\quad \text{már volt}\\
			x = 5\quad \Rightarrow\quad (5,4)
		\end{align*}
		$D:=\{ (x,x-1)\in\R^2\ | \ -1\leq x\leq 5 \}$
		\[ h(x):=f(x,x-1)=x^3-3x^2-(x-1)^2\quad x\in[-1,5] \]
		Ha $x\in(-1,5)\quad \Rightarrow h'(x)=3x^2-6x-2(x-1)\cdot1=3x^2-8x+2=0\quad \Leftrightarrow\quad x_1=\frac{4+\sqrt{10}}{3}\in(-1,5), x_2=\frac{4-\sqrt{10}}{3}\in(-1,5)$
		
		sarokpontok:
		\begin{align*}
			\left(\frac{4+\sqrt{10}}{3},\frac{1+\sqrt{10}}{3}\right)\\
			\left(\frac{4-\sqrt{10}}{3},\frac{1-\sqrt{10}}{3}\right)
		\end{align*}
		
		Vessük össze:
		\begin{figure}[H]
			\centering
			\includegraphics[height=4cm]{kepek/51.png}
			\caption{}
		\end{figure}
		\begin{align*}
			f(0,0)&=0\\
			f(-1,-2)&=-8\\
			f(-1,0)&=-4\\
			f(-1,4)&=-20\quad \text{abszolút minimum}\\
			f(0,4)&=-16\\
			f(2,4)&=-20\quad \text{abszolút minimum}\\
			f(5,4)&=34\quad \text{abszolút maximum}\\
			f\left(\frac{4+\sqrt{10}}{3},\frac{1+\sqrt{10}}{3}\right)&=\text{HF}, \in(-20,34)\\
			f\left(\frac{4-\sqrt{10}}{3},\frac{1-\sqrt{10}}{3}\right)&=\text{HF} \in(-20, 34)
		\end{align*}
	\end{task}
	\begin{note}
		Az $A$ halmaz ábrázolása 1 pont, a Weierstass tétel felírása 1 pont, a deriváltak felírása (ált.) 2 pont, a határvonalak 1-2 pontot érnek darabonkéntösszevetés 1pont
	\end{note}
\end{document}
