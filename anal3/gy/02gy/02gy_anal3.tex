\documentclass[a4paper,11.5pt]{article}
\usepackage[textwidth=170mm, textheight=230mm, inner=20mm, top=20mm, bottom=30mm]{geometry}
\usepackage[normalem]{ulem}
\usepackage[utf8]{inputenc}
\usepackage[T1]{fontenc}
\PassOptionsToPackage{defaults=hu-min}{magyar.ldf}
\usepackage{pgfplots}
\pgfplotsset{compat=1.10}
\usepgfplotslibrary{fillbetween}
\usepackage[magyar]{babel}
\usepackage{amsmath, amsthm,amssymb,paralist,array, ellipsis, graphicx, float, bigints,tikz}
%\usepackage{marvosym}

\makeatletter
\renewcommand*{\mathellipsis}{%
	\mathinner{%
		\kern\ellipsisbeforegap%
		{\ldotp}\kern\ellipsisgap
		{\ldotp}\kern\ellipsisgap%
		{\ldotp}\kern\ellipsisaftergap%
	}%
}
\renewcommand*{\dotsb@}{%
	\mathinner{%
		\kern\ellipsisbeforegap%
		{\cdotp}\kern\ellipsisgap%
		{\cdotp}\kern\ellipsisgap%
		{\cdotp}\kern\ellipsisaftergap%
	}%
}
\renewcommand*{\@cdots}{%
	\mathinner{%
		\kern\ellipsisbeforegap%
		{\cdotp}\kern\ellipsisgap%
		{\cdotp}\kern\ellipsisgap%
		{\cdotp}\kern\ellipsisaftergap%
	}%
}
\renewcommand*{\ellipsis@default}{%
	\ellipsis@before
	\kern\ellipsisbeforegap
	.\kern\ellipsisgap
	.\kern\ellipsisgap
	.\kern\ellipsisgap
	\ellipsis@after\relax}
\renewcommand*{\ellipsis@centered}{%
	\ellipsis@before
	\kern\ellipsisbeforegap
	.\kern\ellipsisgap
	.\kern\ellipsisgap
	.\kern\ellipsisaftergap
	\ellipsis@after\relax}
\AtBeginDocument{%
	\DeclareRobustCommand*{\dots}{%
		\ifmmode\@xp\mdots@\else\@xp\textellipsis\fi}}
\def\ellipsisgap{.1em}
\def\ellipsisbeforegap{.05em}
\def\ellipsisaftergap{.05em}
\makeatother

\usepackage{hyperref}
\hypersetup{
	colorlinks = true	
}

\DeclareMathOperator{\Int}{int}
\DeclareMathOperator{\tg}{tg}
\DeclareMathOperator{\ctg}{ctg}
\DeclareMathOperator{\Th}{th}
\DeclareMathOperator{\sh}{sh}
\DeclareMathOperator{\ch}{ch}
\DeclareMathOperator{\arsh}{arsh}
\DeclareMathOperator{\arch}{arch}
\DeclareMathOperator{\arth}{arth}
\DeclareMathOperator{\arcth}{arcth}
\DeclareMathOperator{\arc}{arc}
\DeclareMathOperator{\arctg}{arc tg}
\DeclareMathOperator{\arcctg}{arc ctg}

\begin{document}
	%%%%%%%%%%%RÖVIDÍTÉSEK%%%%%%%%%%
	\setlength\parindent{0pt}
	\def\a{\textbf{a}}
	\def\b{\textbf{b}}
	\def\N{\hskip 10 true mm}
	\def\a{\textbf{a}}
	\def\b{\textbf{b}}
	\def\c{\textbf{c}}
	\def\d{\textbf{d}}
	\def\e{\textbf{e}}
	\def\gg{$\gamma$}
	\def\vi{\textbf{i}}
	\def\jj{\textbf{j}}
	\def\kk{\textbf{k}}
	\def\fh{\overrightarrow}
	\def\l{\lambda}
	\def\m{\mu}
	\def\v{\textbf{v}}
	\def\0{\textbf{0}}
	\def\s{\hspace{0.2mm}\vphantom{\beta}}
	\def\Z{\mathbb{Z}}
	\def\Q{\mathbb{Q}}
	\def\R{\mathbb{R}}
	\def\C{\mathbb{C}}
	\def\N{\mathbb{N}}
	\def\Rn{\mathbb{R}^{n}}
	\def\Ra{\overline{\mathbb{R}}}
	\def\sume{\displaystyle\sum_{n=1}^{+\infty}}
	\def\sumn{\displaystyle\sum_{n=0}^{+\infty}}
	\def\biz{\emph{Bizonyítás:\ }}
	\def\narrow{\underset{n\rightarrow+\infty}{\longrightarrow}}
	\def\limn{\displaystyle\lim_{n\to +\infty}}
	%	\def\definition{\textbf{Definíció:\ }}
	%	\def\theorem{\textbf{Tétel:\ }}
	%\def\note{\emph{Megjegyzés:\ }}
	%\def\example{\textbf{Példa:\ }} 
	
	\theoremstyle{definition}
	\newtheorem{theorem}{Tétel}[subsubsection] % reset theorem numbering for each chapter
	
	\theoremstyle{definition}
	\newtheorem{definition}[theorem]{Definíció} % definition numbers are dependent on theorem numbers
	\newtheorem{example}[theorem]{Példa} % same for example numbers
	\newtheorem{exercise}[theorem]{Házi feladat} % same for example numbers
	\newtheorem{note}[theorem]{Megjegyzés} % same for example numbers
	\newtheorem{task}[theorem]{Feladat} % same for example numbers
	\newtheorem{revision}[theorem]{Emlékeztető} % same for example numbers
	%%%%%%%%%%%%%%%%%%%%%%%%%%%%%%%%%
	\begin{center}
		{\LARGE\textbf{Analízis 3. A szakirány}}
		\smallskip
		
		{\Large Gyakorlati jegyzet}
		
		\smallskip
		2. óra.
	\end{center}
	A jegyzetet \textsc{Umann} Kristóf készítette \textsc{Filipp} Zoltán István gyakorlatán. (\today)
	\bigskip
	
	%\subsection{Lineáris helyettesítés (folyt.)}
	\begin{task}
		\[ \int\frac{f'(x)}{f(x)}\,dx=\ln |f(x)|+c \quad (c\in\R, \quad x\in I)\]
	\end{task}
	\begin{task}$(x\in\R)$
		\[ \int\frac{x}{x^2+8}\,dx\quad \overset{f(x):=x^2+8}{\underset{f'(x)=2x}{=}}\quad\frac{1}{2}\int\frac{2x}{x^2+8}\,dx=\frac{1}{2}\int\frac{(x^2+8)'}{x^2+8}\,dx=\frac{1}{2}\ln(x^2+8)+c\quad (c\in\R)  \]
	\end{task}
	\begin{task}Ha $x\in(1;+\infty)$:
		\[ \int\frac{1}{x\ln x}\,dx= \int\frac{1}{x}\cdot\frac{1}{\ln x}\,dx\quad \overset{f(x):=\ln x}{\underset{f'(x)=\frac{1}{x}}{=}}\quad \int\frac{(\ln x)'}{\ln x}\,dx=\ln|\ln x|+c=\ln(\ln(x))+c\quad (c\in\R) \]
		Ha $x\in(0,1)$:
		\[ \int\frac{(\ln x)'}{\ln x}\,dx=\ln(-\ln x)+c\quad (c\in\R) \]
	\end{task}
	\begin{task}$x\in\left(-\frac{\pi}{2},\frac{\pi}{2}\right)$:
		\[ \int\tg x\,dx=\int\frac{\sin x}{\cos x}\,dx\quad \overset{f(x):=\cos x}{\underset{f'(x)=-\sin x}{=}}\quad -\int\frac{(\cos x)'}{\cos x}\,dx=-\ln|\cos x|+c=-\ln(\cos x)+c\quad (c\in\R) \]
	\end{task}
	\subsection{$\int f'(x)\cdot f^\alpha(x)\,dx$ típusú feladatok}
	\begin{task} $(\alpha\in\R\setminus\{-1\})\quad (x\in I)$
		\[ \int f'(x)\cdot f^\alpha(x)\,dx=\frac{f^{\alpha+1}(x)}{\alpha+1}+c\quad (c\in\R) \]
		\[ \int 1\cdot x^\alpha\,dx=\frac{x^{\alpha+1}}{\alpha+1}+c\quad (c\in\R)\quad \longrightarrow\quad \int f'(x)f^\alpha(x)\,dx=\frac{f^{\alpha+1}(x)}{\alpha+1}+c \quad (c\in\R) \]
	\end{task}
	\begin{task}
		$(x\in\R)$
		\[ \int x^2(3x^3+4)^{2017}\,dx\quad \overset{\alpha:=2017}{\underset{f(x):=3x^3+4}{\underset{f'(x)=9x^2}{=}}}\quad \frac{1}{9}\int 9x^2(3x^3+4)^{2017}\,dx=\frac{1}{9}\int(3x^3+4)'(3x^3+4)\,dx=\]
		\[=\frac{1}{9}\frac{(3x^3+4)^{2018}}{2018}+c\quad (c\in\R) \]
		\begin{note}
			Itt kapóra jött az  $x^2$. Ha nem így lenne, akkor 2017re kéne hatványozni, szétszedni binomiális tétellel, stb.
		\end{note}
	\end{task}
	\begin{task}$(x\in\R)$
		\[\int e^x (1-e^x)^{300}\,dx\quad \overset{\alpha=300}{\underset{f(x):=1-e^x}{\underset{f'(x)=-e^x}{=}}}\quad -\int(1-e^x)'\cdot(1-e^x)^{300}\,dx=-\frac{(1-e^x)^{301}}{301}+c\quad (c\in\R) \]
	\end{task}
	\begin{task}
		$x\in\left(0,\frac{\pi}{2}\right)$
		\[ \int\frac{\cos2x}{\sqrt{\sin x+\cos x}}\,dx = \int \frac{\cos^2 x-\sin^2 x}{\sqrt{\cos x+\sin x}}\,dx=\int\frac{(\cos x-\sin x)(\cos x+\sin x)}{\sqrt{\cos x+\sin x}}\,dx=\]
		\[=\int(\cos x-\sin x)(\cos x+\sin x)^{\frac{1}{2}}\,dx\quad \overset{\alpha:=\frac{1}{2}}{\underset{f(x):=\cos x+\sin x}{\underset{f'(x)=-\sin x+\cos y}{=}}}\quad \int(\sin x+\cos x)'(\sin x+\cos x)^{\frac{1}{2}}\,dx=\]
		\[=\frac{(\sin x + \cos x)^{\frac{3}{2}}}{\frac{3}{2}}+c\quad (c\in\R) \]
		\begin{note}
			Ez már ZH szintű feladat.
		\end{note}
		\begin{center}
			\textit{,,Én is szívesen négyzetre emelném a fizetésemet''}
			\smallskip
			
			/Filipp Zoltán István/
		\end{center}
	\end{task}
	\subsubsection{$n,m\in\Z:\quad \int\sin^n x\cos^mx\,dx$\quad \text{(feladat altípus)}\quad}
	\begin{note}
		Ez a megoldási módszer fő gondolatmenetét a $\sin$ és a $\cos$ függvények közötti egyszerű váltás adja, pl.:\, $(\cos x)'=-\sin x$ \,és\, $(\sin x)'=\cos x$, \,valamint \,$1=\sin^2x + \cos^2x$.
	\end{note}
	\begin{task}$x\in\R$
		\[ \int \sin^3x\cdot\cos^5x\,dx=\int\sin x\cdot\sin^2x\cdot\cos^5x\,dx=-\int(\cos x)'\cdot\overbrace{(1-\cos^2 x)}^{\sin^2x}\cdot\cos^5x\,dx=\]
		\[=-\int(\cos x)'\cos^5x\,dx+\int(\cos x)'\cos^7x\,dx=-\frac{\cos^6x}{6}+\frac{\cos^8}{8}+c\quad (c\in\R) \]
	\end{task}
	\begin{note}
		Ha $n$ vagy $m$ páratlan, akkor örülünk, és 
		\begin{enumerate}
			\item Vesszük a kisebbik páratlan hatvánnyal rendelkező tagot (pl.: $\sin^5\cos^7x$ esetében $\sin^5x$),
			\item Leválasztunk belőle 1-et (pl.: $\sin^5x = \sin x\cdot\sin^4 x)$,
			\item Ez lesz a nagyobb hatvánnyal rendelkező tag deriváltja (pl.: $(\cos x)'\sin^4 x$).
		\end{enumerate}
		
	\end{note}
	\begin{task}$x\in\R$
		\[ \int\cos^3x\,dx=\int\cos x\cdot\cos^2x\,dx=\int(\sin x)'\cdot(1-\sin^2x)\,dx=\int(\sin x)'\,dx-\int(\sin x)'(\sin x)^2\,dx=\]
		\[=\sin x-\frac{\sin^3x}{3}+c\quad (c\in\R) \]
	\end{task}
	\begin{exercise} Tipp: Linearizálás 
		\[ \int(\sin^5x\cos^{10}x)\,dx\quad (x\in\R) \]
		\textit{Megoldás: (nem órán hangzott el)}
		\[ \int(\sin^5x\cos^{10}x)\,dx=\int(\sin x\cdot\sin^4x\cdot\cos^{10}x)\,dx=-\int((\cos x)'(1-\cos^2x)^2\cos^{10}x)\,dx=\]
		\[=-\int(\cos x)'\cos^{10}x\,dx+\int(\cos x)'\cos^{12}x\,dx=-\frac{\cos^{11}x}{11}+\frac{\cos^{13}x}{13}+c\quad (c\in\R) \]
	\end{exercise}
	\begin{task}$x\in\R$
		\[ \int\sin^2x\cos^4x\,dx=\int(1-\cos^2x)\cos^4 x\,dx=\int\cos^4 x\,dx-\underbrace{\int\cos^6x\,dx}_{\text{HF}}=\]
		\[\int\cos^4x\,dx=\int\left(\frac{1+\cos2x}{2}\right)^2\,dx=\frac{1}{4}\int(1+2\cos2x+\cos^22x)\,dx=\frac{1}{4}\left(x+\sin2x+\int\cos^2(2x)\right)\,dx=\]
		\[=\frac{x}{4}+\frac{\sin2x}{4}+\frac{1}{4}\int\frac{1+\cos4x}{2}\,dx=\frac{x}{4}+\frac{\sin2x}{4}+\frac{x}{8}+\frac{1}{8}\cdot\frac{\sin4x}{4}+c\quad (c\in\R) \]
		\begin{center}
			\fbox{$\displaystyle \cos^2=\frac{1+\cos2x}{2}$}\\
			\fbox{$\displaystyle \sin^2=\frac{1-\cos2x}{2}$}
		\end{center}
	\end{task}
	\begin{exercise}
		\def\fracHeight{\vphantom{\frac{3^2}{2}}}
		\[\int\cos^6x\,dx=\int\left(\frac{1+\cos2x}{2}\right)^3\,dx=\frac{1}{8}\int(1+3\cos2x+3\cos^22x+\cos^32x)\,dx=\]
		\[= \frac{1}{8}\left(\overbrace{x\fracHeight}^{\int1\,dx}+\overbrace{\frac{3}{2}\sin2x\fracHeight}^{\int3\cos2x\,dx}+ \overbrace{3\left(\frac{x}{2}+\frac{\sin4x}{8}\right)\fracHeight}^{\int3\cos^22x\,dx}+ \overbrace{\sin2x-\frac{\sin^32x}{3}}^{\int\cos^32x\,dx}\right)+c\quad (c\in\R)  \]
		A számoláshoz a gyakorlat korábbi eredményeit is felhasználtam.
	\end{exercise}
	\begin{task}$x\in(0,\pi)$
		\[\int\frac{1}{\sin x}\,dx\quad \overset{\sin2\alpha=2\sin\alpha\cos\alpha}{\underset{\text{félszögre térés}}{=}}\quad \int\frac{\sin^2\frac{x}{2}+\cos^2\frac{x}{2}}{2\sin\frac{x}{2}\cos\frac{x}{2}}\,dx=\frac{1}{2}\int\frac{\sin\frac{x}{2}}{\cos\frac{x}{2}}\,dx+\frac{1}{2}\int\frac{\cos\frac{x}{2}}{\sin\frac{x}{2}}\,dx=\]
		\[=\int\frac{(\sin\frac{x}{2})'}{\sin\frac{x}{2}}-\int\frac{(\cos\frac{x}{2})'}{\cos\frac{x}{2}}=\ln\left(\sin\frac{x}{2}\right)-\ln\left(\cos\frac{x}{2}\right)+c=\ln\left(\tg\frac{x}{2}\right)+c\quad (c\in\R) \]
	\end{task}
	\begin{exercise}
		\[ \int\frac{\sin^2x}{\cos^4x}\,dx\quad \left(x\in\left(0,\frac{\pi}{2}\right)\right) \]
		\textit{Megoldás:}
		\[ \int\frac{\sin^2x}{\cos^4x}\,dx=\int\frac{1}{\cos^2x}\cdot\tg^2x\,dx= \int(\tg x)'\tg^2x\,dx=\frac{\tg^3x}{3}+c\quad (c\in\R) \]
	\end{exercise}
	\begin{exercise}
		\[ \int\sin^4x\cos^4x\,dx \quad (x\in\R) \]
		\textit{Megoldás:}
		\[ \int\sin^4x\cos^4x\,dx=\int(1-\cos^2x)^2\cos^4x\,dx=\int(1-2\cos^2x+\cos^4x)\cos^4x\,dx=\int(\cos^4x-2\cos^6x+\cos^8x)\,dx=\]
		\[ =\overbrace{\int\cos^4x\,dx}^{\text{ismert}}-\overbrace{\int2\cos^6x\,dx}^{\text{ismert}}+\int\cos^8x\,dx \]
		Határozzuk meg az ismeretlen tagot:
		\[ \int\cos^8x\,dx=\int\left(\frac{1+\cos2x}{2}\right)^4\,dx=\frac{1}{16}\int(1+2\cos2x+\cos^22x)^2\,dx=\]
		\[=\frac{1}{16}\int\left(1+4\cos^22x+\cos^42x+4\cos2x+4\cos^32x+2\cos^22x\right)\,dx=\]
		\[=\frac{1}{16}\int\left(1+4\cos2x+6\cos^22x+4\cos^32x+\cos^42x\right)\,dx=\]
		\def\fracHeight{\vphantom{\frac{3^2}{2}}}
		\[= \frac{1}{16}\left(\overbrace{x\fracHeight}^{\int1\,dx}+\overbrace{\frac{4}{2}\sin2x\fracHeight}^{\int4\cos2x\,dx}+ \overbrace{6\left(\frac{x}{2}+\frac{\sin4x}{8}\right)\fracHeight}^{\int6\cos^22x\,dx}+ \overbrace{4\left(\sin2x-\frac{\sin^32x}{3}\right)}^{\int4\cos^32x\,dx}+\overbrace{\frac{2x}{4}+\frac{\sin4x}{4}+\frac{2x}{8}+\frac{\sin8x}{32}\fracHeight}^{\int\cos^42x\,dx}\right)+c\quad (c\in\R)  \]
		Alternatív (még nem teljes) megoldás:
		\[=\int\left(\frac{1}{2}2\sin x\cos x\right)^4\,dx=\frac{1}{16}\int\left(2\sin x\cos x\right)\,dx=\frac{1}{16}\int\sin^42x\,dx= \]
	\end{exercise}
	\subsubsection{Helyettesítés szabálya}
	\begin{revision}
		Tegyük fel, hogy $\emptyset\not=\int f(x)\,dx=F(x)+c\quad (c\in\R, x\in I)$, és tegyük fel, hogy\\ $g:J\to I, \quad g\in D.$ Ekkor
		\[ \int f(g(x))\cdot g'(x)\,dx=F(g(x))+c\quad (c\in\R) \]
		Ugyanis
		\[ (F(g(x)))'=F'(g(x))\cdot g'(x)=f(g(x))\cdot g'(x)\checkmark \]
	\end{revision}
	\begin{task}$x\in\R$
		\[ \int x\cos(x^2)\,dx\quad \overset{g'(x):=2x}{\overset{g(x):=x^2}{\underset{f(x):=\cos x}{=}}}\quad \frac{1}{2}\int 2x\cos(x^2)\,dx=\frac{1}{2}\int(x^2)'\cdot\cos(x^2)\,dx=\frac{1}{2}\sin(x^2)+c\quad (c\in\R) \]
	\end{task}
	\begin{note}
		Rövid jelölés: 
		\[ \int x\cos(x^2)\,dx\quad \overset{x^2=:t\,\in\,(0;+\infty)}{\underset{(x^2)'\,dx=(t)'\,dt}{\underset{2x\,dx=1\,dt}{=}}}\quad\int\frac{1}{2}\cdot\cos t\,dt\big|_{t=x^2}=\frac{1}{2}\sin t +c\big|_{t=x^2}=\frac{1}{2}\sin(x^2)+c  \]
	\end{note}
	\begin{task}$x\in(0,\frac{\pi}{2})$
		\[ \int\frac{1+\tg^2x}{1+\tg x}\,dx \]
		Legyen  \quad $\tg x=:t\in(0;+\infty).$ \quad Ez alapján $\quad  x=\arc\tg t\quad \Rightarrow\quad (x)'\,dx=(\arc\tg t)'\,dt$
		\[ \text{\fbox{$\displaystyle dx=\frac{1}{1+t^2}\,dt$}} \quad \left(\text{ui.:\quad }(\arc\tg t)'=\frac{1}{1+t^2}.\right)\]
		\[ \Rightarrow\quad \int\frac{1+t^2}{1+t}\cdot\frac{1}{1+t^2}\,dt=\int\frac{1}{1+t}\,dt=\int\frac{(t+1)'}{t+1}\,dt=\ln(t+1)+c\quad (c\in\R) \]
		Visszírva
		\[ \int\frac{1+\tg^2 x}{1+\tg x}\,dx=\ln(1+\tg x)+c\quad (c\in\R) \]
		Vagy:
		\[ 1+\tg^2 x=1+\frac{\sin^2x}{\cos^2x}=\frac{1}{\cos^2x};\quad \Rightarrow\quad \int\frac{1}{\cos^2x}\cdot\frac{1}{1+\tg x}\,dx=\int\frac{(1+\tg x)'}{1+\tg x}\,dx=\ln(1+\tg x)+c\quad (c\in\R) \]
	\end{task}
	\begin{task}$x\in\R$
		\[ \int\frac{e^{2x}}{1+e^x}\,dx \]
		Legyen \[e^x=:t\in(0,+\infty)\quad \Rightarrow\quad x=\ln t\quad \Rightarrow\quad (x)'\,dx=(\ln t)'\,dt\quad \text{\fbox{$\displaystyle dx=\frac{1}{t}\,dt$}} \]
		Az ,,új integrál'':\quad \[\displaystyle \int\frac{t^2}{1+t}\cdot\frac{1}{t}\,dt=\int\frac{t}{1+t}\,dt\int\frac{t+1-1}{t+1}\,dt=\int\left(1-\frac{1}{1+t}\right)\,dt=t-\ln(1+t)+c \]
		,,Vissza'':
		\[ \int\frac{e^x}{1+e^x}\,dx=e^x-\ln(1+e^x)+c\quad (c\in\R) \]
	\end{task}
	HF: 30 darab feladattípus, $\int\frac{f'}{f}$,\quad  $\int f'f^\alpha$, \quad $\int(\sin^4x\cos^4x),\quad \int f\circ g\cdot g'=...$.
	\begin{center}
		\bigskip
		
		\textit{,,Tanár úr, használhatok más jelölést? A $g(t)$-t annyira nem szeretem.''}
		\smallskip
		
		/Tóth Péter/
		\bigskip
		
	\end{center}
	\begin{note}
		Mindig a nagyobb hatvánnyal rendelkező változóhoz érdemes új változót rendelni.
	\end{note}
\end{document}