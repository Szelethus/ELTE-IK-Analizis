\documentclass[a4paper,11.5pt]{article}
\usepackage[textwidth=170mm, textheight=230mm, inner=20mm, top=20mm, bottom=30mm]{geometry}
\usepackage[normalem]{ulem}
\usepackage[utf8]{inputenc}
\usepackage[T1]{fontenc}
\PassOptionsToPackage{defaults=hu-min}{magyar.ldf}
\usepackage{pgfplots}
\pgfplotsset{compat=1.10}
\usepgfplotslibrary{fillbetween}
\usepackage[magyar]{babel}
\usepackage{amsmath, amsthm,amssymb,paralist,array, ellipsis, graphicx, float, bigints,tikz}
%\usepackage{marvosym}

\makeatletter
\renewcommand*{\mathellipsis}{%
	\mathinner{%
		\kern\ellipsisbeforegap%
		{\ldotp}\kern\ellipsisgap
		{\ldotp}\kern\ellipsisgap%
		{\ldotp}\kern\ellipsisaftergap%
	}%
}
\renewcommand*{\dotsb@}{%
	\mathinner{%
		\kern\ellipsisbeforegap%
		{\cdotp}\kern\ellipsisgap%
		{\cdotp}\kern\ellipsisgap%
		{\cdotp}\kern\ellipsisaftergap%
	}%
}
\renewcommand*{\@cdots}{%
	\mathinner{%
		\kern\ellipsisbeforegap%
		{\cdotp}\kern\ellipsisgap%
		{\cdotp}\kern\ellipsisgap%
		{\cdotp}\kern\ellipsisaftergap%
	}%
}
\renewcommand*{\ellipsis@default}{%
	\ellipsis@before
	\kern\ellipsisbeforegap
	.\kern\ellipsisgap
	.\kern\ellipsisgap
	.\kern\ellipsisgap
	\ellipsis@after\relax}
\renewcommand*{\ellipsis@centered}{%
	\ellipsis@before
	\kern\ellipsisbeforegap
	.\kern\ellipsisgap
	.\kern\ellipsisgap
	.\kern\ellipsisaftergap
	\ellipsis@after\relax}
\AtBeginDocument{%
	\DeclareRobustCommand*{\dots}{%
		\ifmmode\@xp\mdots@\else\@xp\textellipsis\fi}}
\def\ellipsisgap{.1em}
\def\ellipsisbeforegap{.05em}
\def\ellipsisaftergap{.05em}
\makeatother

\usepackage{hyperref}
\hypersetup{
	colorlinks = true	
}

\DeclareMathOperator{\Int}{int}
\DeclareMathOperator{\tg}{tg}
\DeclareMathOperator{\ctg}{ctg}
\DeclareMathOperator{\Th}{th}
\DeclareMathOperator{\sh}{sh}
\DeclareMathOperator{\ch}{ch}
\DeclareMathOperator{\arsh}{arsh}
\DeclareMathOperator{\arch}{arch}
\DeclareMathOperator{\arth}{arth}
\DeclareMathOperator{\arcth}{arcth}
\DeclareMathOperator{\arc}{arc}
\DeclareMathOperator{\arctg}{arc tg}
\DeclareMathOperator{\arcctg}{arc ctg}

\begin{document}
	%%%%%%%%%%%RÖVIDÍTÉSEK%%%%%%%%%%
	\setlength\parindent{0pt}
	\def\a{\textbf{a}}
	\def\b{\textbf{b}}
	\def\N{\hskip 10 true mm}
	\def\a{\textbf{a}}
	\def\b{\textbf{b}}
	\def\c{\textbf{c}}
	\def\d{\textbf{d}}
	\def\e{\textbf{e}}
	\def\gg{$\gamma$}
	\def\vi{\textbf{i}}
	\def\jj{\textbf{j}}
	\def\kk{\textbf{k}}
	\def\fh{\overrightarrow}
	\def\l{\lambda}
	\def\m{\mu}
	\def\v{\textbf{v}}
	\def\0{\textbf{0}}
	\def\s{\hspace{0.2mm}\vphantom{\beta}}
	\def\Z{\mathbb{Z}}
	\def\Q{\mathbb{Q}}
	\def\R{\mathbb{R}}
	\def\C{\mathbb{C}}
	\def\N{\mathbb{N}}
	\def\Rn{\mathbb{R}^{n}}
	\def\Ra{\overline{\mathbb{R}}}
	\def\sume{\displaystyle\sum_{n=1}^{+\infty}}
	\def\sumn{\displaystyle\sum_{n=0}^{+\infty}}
	\def\biz{\emph{Bizonyítás:\ }}
	\def\narrow{\underset{n\rightarrow+\infty}{\longrightarrow}}
	\def\limn{\displaystyle\lim_{n\to +\infty}}
	%	\def\definition{\textbf{Definíció:\ }}
	%	\def\theorem{\textbf{Tétel:\ }}
	%\def\note{\emph{Megjegyzés:\ }}
	%\def\example{\textbf{Példa:\ }} 
	
	\theoremstyle{definition}
	\newtheorem{theorem}{Tétel}[subsubsection] % reset theorem numbering for each chapter
	
	\theoremstyle{definition}
	\newtheorem{definition}[theorem]{Definíció} % definition numbers are dependent on theorem numbers
	\newtheorem{example}[theorem]{Példa} % same for example numbers
	\newtheorem{exercise}[theorem]{Házi feladat} % same for example numbers
	\newtheorem{note}[theorem]{Megjegyzés} % same for example numbers
	\newtheorem{task}[theorem]{Feladat} % same for example numbers
	\newtheorem{revision}[theorem]{Emlékeztető} % same for example numbers
	%%%%%%%%%%%%%%%%%%%%%%%%%%%%%%%%%
	\begin{center}
		{\LARGE\textbf{Analízis 3. A szakirány}}
		\smallskip
		
		{\Large Gyakorlati jegyzet}
		
		\smallskip
		1. óra.
	\end{center}
	A jegyzetet \textsc{Umann} Kristóf készítette \textsc{Filipp} Zoltán István gyakorlatán. (\today)
	\section{Információk a gyakorlattal kapcsolatban}
	\begin{compactitem}
		\item emial: filipp@numanal.inf.elte.hu
		\item szoba: 2-316A
		\item Telefonszám: 06-70-332-01-41
		\item Cím: \url{numanal/~filipp}, analízis 3, azon belül A.
		\item várható majd kötelezően beadandó integrálszámítás
		\item 8:30 kezdés
		\item 7. héten első zh, 13. hét második zh
		\item egy csütörtök el fog maradni, de pótolva lesz
		\item Konzi: szerda 9-10
	\end{compactitem}
	Ami a honlapon található:
	\begin{compactitem}
		\item Gyakorlati anyag
		\item Gyemidovics tankönyv (III. fejezet) (és ennek eredményei külön fájlban)
		\item Bolyai sorozat (integrálszámítás és többváltozós analízis)
		\item Ajánlottak mint Szili analízis feladat gyűjteménye és integráltáblázata.
		\item Károlyi Katalin
	\end{compactitem}
	\section{Integrálszámítás}
	(első 6 hét anyaga)
	\subsection{Bevezető}
	\begin{revision}
		(Primitív függvény)\quad Legyen $\emptyset\not=I\subset\R$ nyílt intervallum, és $f:I\to\R$ fv. Ha 
		\[\exists F:I\to\R, \quad F\in D\quad  \text{és}\quad  F'(x)=f(x) \quad (\forall x\in I),\]
		akkor azt mondjuk hogy $F$ az $f$ egy primitív függvénye.
	\end{revision}
	Nem minden függvény rendelkezik primitív függvénnyel, pl.: $sign(x)$, mert az 1/2-et nem veszi fel sehol.
	\begin{note}
	$f(x)=x^2,\quad x\in I:=\R;\quad \Rightarrow\quad F(x)=\frac{1}{3}x^3\quad (x\in\R)$, mert $\left(\frac{1}{3}x^3\right)'=\frac{1}{3}3x^2=x^2$.
	
	Ezek is primitív függvényei $f$-nek $(x\in\R)$:
	\[ F_1(x):=\frac{1}{3}x^3+1,\quad  F_{34}(x):=\frac{1}{3}x^3+34 \]
	Általánoságban
	\[ F(x):=\frac{1}{3}x^3+c\quad (c, x\in\R) \]
	\end{note}
	Két primitív függvény konstansban térhet csak el. Ez alapján megállapítható, hogy ha létezik primitív függvény, végtelen sok lehet belőlük.
	\begin{note}
		Ha van $f$-nek primitív függvénye, akkor
		\[  \int f(x)\,dx:=\{F:I\to\R~|~F\in D \text{\quad és\quad } F'=f \} =F(x) + c \]
		az $f$ primitív függvényeinek a halmaza, vagy az $f$ határozatlan integrálja
	\end{note}
	Mindig meg kell adnunk az $I$ intervallumot.
	
	Néha az is kérdés lehet, egy függvény melyik pontban tűnik el, azaz nem mindig a 0ra vagyunk kíváncsiak.
	\begin{task}
		Adjuk meg azt az $F$-et, melyre $F(1)=2$
	\end{task}
	%TODO ábra 1
	\begin{revision}
		$f:$idő$\to$távolság
		
		Ilyenkor mit jelent $f'$? az a sebesség, míg $f''$ a gyorsulás. Ott gyakran vannak olyan feltételek, hogy a kezdősebesség legyen 2 ($f(1)=2$)
		
		Ehhez a fenti feladatot példaképp véve $\frac{1}{3}^3+c=2$ egyenletet kell megoldani $\quad \Rightarrow\quad c=\frac{5}{3}$.
	\end{revision}
	Tehát:
	\[ F(x)=\frac{x^3}{3}+\frac{5}{3} \]
	\begin{task}$(x\in\R$)
		\[\int \cos x\,dx=\sin x+c\quad (c\in\R) \]
	\end{task}
	\begin{task}
		\[ (\arccos\tan x)' =\frac{1}{1+x^2}\quad \Rightarrow\quad \int\frac{1}{1+x^2}\,dx=\arctan x + c\quad (x,c\in \R) \]
	\end{task}
	\begin{task} $x\in(-1,1)$
		\[ \int\frac{1}{\sqrt{1-x^2}}\,dx=\arc\sin x + c \quad (c\in \R) \]
	\end{task}
	\subsection{Alapintegrálok és ezekre vezethető típusok}
	Szokás úgy is hívni hogy antiderivált, mert az annak az ellentettje.
	\begin{task}$(x\in\R)$
		\[\int(6x^3-2x+1)\,dx\]
	\end{task}
	\begin{note}
		Mi az a $dx$? EMlékezzünk vissza mi az a határozott integrál: nagyon kis területekre osztottunk fel egy intervalumot, és így közelítettük a területet.
	\end{note}
	Emlékezzünk vissza a műveleti tételekre. Az integrál lineáris, így a konstansot ki lehet emelni
	\[ 6\cdot\int x^3\,dx-2\cdot\int x\,dx+\int1\,dx=6\cdot\frac{x^4}{4}-2\cdot\frac{x^2}{2}+x+c\quad (c\in\R) \]
	\begin{theorem}
		Általános integrálfüggvény:
		\[ \int x^\alpha\,dx=\frac{x^{\alpha+1}}{\alpha+1}+c\quad (x>0, c,\alpha\in\R) \]
		HA $\alpha\not=-1$
	\end{theorem}
	\begin{task}$x>0$. Bár $\R\setminus\{0\}$ is jó válasz, de az nem intervallum.
		\[ \int\frac{1}{x}\,dx=\ln x+c\quad (c\in\R) \]
	\end{task}
	\begin{task}($x\in(-\infty,0)=:I$
		\[ \int\frac{1}{x}\,dx=\ln(-x)+c\quad (c\in\R),\quad \text{ui.}\quad (\ln(-x))'=\frac{1}{-x}(-x)'=-\frac{1}{x}\cdot(-1)=\frac{1}{x} \]
		Logaritmus nem megoldás, mert az argumentumának pozitívnak kell lennie. ($-x$ pozitív itt)
	\end{task}
	\begin{note}
		az előző két feladat összefoglalható úgy, hogy 
		\[ \int\frac{1}{x}\,dx=\ln|x|+x\quad (c\in\R, x\in(-\infty,0)\quad \text{vagy}\quad (0,+\infty) \]
	\end{note}
	\begin{task}$x>0$
		\[ \int\sqrt{x\sqrt{x\sqrt{x}}}\,dx=\int x^\frac{1}{2}\cdot x^\frac{1}{4}\cdot x^\frac{1}{8}\, dx=\int x^{\frac{1}{2}+\frac{1}{4}+\frac{1}{8}}\,dx=\int x^{\frac{7}{8}}\,dx=\frac{x^{\frac{7}{8}+1}}{\frac{7}{8}+1}+c=\frac{8}{15}\cdot\sqrt[8]{x^{15}}+c\quad (c\in\R) \]
	\end{task}
	\begin{note}
		Ha szorzat van csinálj öszeget, ha összeg van cisnálj szorzatot
	\end{note}
	\begin{task}
		\[ \int\frac{(x+1)^2}{\sqrt{x}}\,dx=\int\frac{x^2+2x+1}{x^\frac{1}{2}}\,dx=\int\frac{x^2}{x^{\frac{1}{2}}}\,dx+2\int\frac{x}{x^\frac{1}{2}}\,dx+\int\frac{1}{x^\frac{1}{2}}\,dx=\int x^\frac{3}{2}\,dx+2\int x^\frac{1}{2}\,dx+\int x^{-\frac{1}{2}}\,dx= \]
		\[=\frac{x^{\frac{3}{2}+1}}{\frac{3}{2}+1}+2\frac{x^{\frac{1}{2}+1}}{\frac{1}{2}+1}x+\frac{x^{-\frac{1}{2}+1}}{-\frac{1}{2}+1}+c\quad (c\in\R) =\frac{2}{5}\cdot\sqrt{x^5}+\frac{4}{3}\sqrt{x^3}+2\sqrt{x}+c \]
	\end{task}
	\begin{center}
		\textit{,,Integrálni úgy kell hogy nézed, nézed, nézed, és aztán rájössz.''}
		
		/Filipp/
	\end{center}
	\begin{note}
		A fenti módszer hívjuk az összegre bontás módszerének.
	\end{note}
	\begin{task}$(x\in(-1,1))$
		\[ \int\left(2x+(1-x^2)^{-\frac{1}{2}}\right)\,dx=2\int x\,dx+\int\frac{1}{\sqrt{1-x^2}}\,dx=2\frac{x^2}{2}+\arc\sin x+c\quad (c\in\R) \]
	\end{task}
	\begin{task}
		$x\in\R$
		\[ \int\frac{x^2}{x^2+1}\,dx=\int \frac{x^2+1-1}{x^2+1}\,dx=\int1\,dx-\int\frac{1}{x^2+1}\,dx=x-\arc\tg x+c\quad (c\in\R) \]
	\end{task}
	\begin{task}$x\in\left(-\frac{\pi}{2},\frac{\pi}{2}\right)$
		\[ \int \tg^2 x\,dx=\int\frac{\sin^2 x}{\cos^2x}\,dx=\int\frac{1-\cos^2x}{\cos^2x}\,dx=\int\frac{1}{\cos^2x}\,dx-\int1\,dx=\tg x-x+c\quad (c\in\R) \]
	\end{task}
	\subsection{Lineáris helyettesítés szabálya}
	\begin{task}$x\in\R$
		\[ \int\cos(2x)\,dx=\sin(2x)+c= \]
		Ellenőrizzünk:
		\[ (\sin(2x))'=\cos(2x)\cdot2 \]
		Azaz korrigálnunk kell, le kell még osztani 2-vel.
		\[ =\frac{\sin(2x)}{2}+c\quad (c\in\R) \]
	\end{task}
	\begin{revision}
		\[ \int\cos x\,dx=\sin x+c \]
	\end{revision}
	\begin{task}$x\in\R$
		\[ \int\cos(2-3x)\,dx=\frac{\sin(2-3x)}{-3}+c \]
	\end{task}
	Általában, a lineáris helyettesítés szabálya:
	\begin{enumerate}
		\item Feltesszük, hogy $\exists F:\quad \int f(x)\,dx=F(x)+c$
		\item $\int f(ax+b)\,dx=\frac{F(ax+b)}{a}+c\quad (\forall a,b\in\R, a\not=0)$
		\item Csak ha lineáris!
	\end{enumerate}
	\begin{task}$x\in\R$
		\[ \int e^{5x+4}\,dx=\frac{e^{5x+4}}{5}+c\quad (c\in\R) \]
	\end{task}
	\begin{task}
		\[ \int\frac{1}{\sqrt{1-3x^2}}\,dx=\quad \left(|x|<\frac{1}{3}\right) \quad =\int\frac{1}{\sqrt{1-(\sqrt{3}x)^2}}\,dx=\frac{\arc\sin(\sqrt{3}x)}{\sqrt{3}}+c\quad (c\in\R) \]
	\end{task}
	\begin{task}
		\[ \int\frac{2}{3+2x^2}\,dx=\frac{2}{3}\int\frac{1}{1+\frac{2}{3}x^2}\,dx=\frac{2}{3}\int\frac{1}{1+\left(\sqrt{\frac{2}{3}}x\right)^2}\,dx=\frac{2}{3}\frac{\arc\tg\left(\sqrt{\frac{2}{3}}x\right)}{\sqrt{\frac{2}{3}}}+c\quad (c\in\R) \]
	\end{task}
	\begin{task}
		\[\int\sin^2x\,dx=\int\frac{1-\cos2x}{2}\,dx=\frac{1}{2}\left(\int1\,dx-\int\cos2x\,dx=\frac{1}{2}x-\frac{1}{2}\cdot\frac{\sin(2x)}{2}+c\quad (c\in\R)\right) \]
	\end{task}
	\begin{note}
		\[ \begin{cases}
		\sin^2x+cos^2x=1\\
		\cos^2x-\sin^2x=\cos2x
		\end{cases} \]
		\begin{center}
			$+\quad \Rightarrow\quad $\fbox{$\cos^2 x=\frac{1+\cos2x}{2}$}\quad \fbox{$\sin^2x=\frac{1-\cos2x}{2}$}
			
			$-\quad \Rightarrow\quad $ LINEARIZÁLÓ FORMULÁK
		\end{center}
	\end{note}
	\begin{task} $x\in(-\pi,\pi)$
		\[\int\frac{1}{1+\cos x}\,dx=\text{(félszögre térés)}=\int\frac{1}{\sin^2\frac{x}{2}+\cos^2\frac{x}{2}+\cos^2\frac{x}{2}-\sin^2\frac{x}{2}}\,dx=\frac{1}{2}\cdot\int\frac{1}{\cos^2\frac{x}{2}}\,dx=\frac{1}{2}\frac{\tg \frac{x}{2}}{\frac{1}{2}}+c\quad (c\in\R) \]
	\end{task}
	\textbf{Házi feladat:}
	\begin{task}$x\in\left(-\frac{\pi}{2},\frac{\pi}{2}\right)$
		\[ \int\frac{\cos^2x-2}{1+\cos2x}\,dx=? \]
	\end{task}
	Továbbá 50 db. beadandó I. és II. típusból (honlapon, 1628. feladattól 1673)
	\begin{revision}
		$\cos2x=\cos^2x-\sin^2x$
	\end{revision}
\end{document}