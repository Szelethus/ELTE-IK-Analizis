
\documentclass[a4paper,11.5pt]{article}
\usepackage[textwidth=170mm, textheight=230mm, inner=20mm, top=20mm, bottom=30mm]{geometry}
\usepackage[normalem]{ulem}
\usepackage[utf8]{inputenc}
\usepackage[T1]{fontenc}
\PassOptionsToPackage{defaults=hu-min}{magyar.ldf}
\usepackage{pgfplots}
\pgfplotsset{compat=1.10}
\usepgfplotslibrary{fillbetween}
\usepackage[magyar]{babel}
\usepackage{amsmath, amsthm,amssymb,paralist,array, ellipsis, graphicx, float, bigints,tikz}
%\usepackage{marvosym}

\makeatletter
\renewcommand*{\mathellipsis}{%
	\mathinner{%
		\kern\ellipsisbeforegap%
		{\ldotp}\kern\ellipsisgap
		{\ldotp}\kern\ellipsisgap%
		{\ldotp}\kern\ellipsisaftergap%
	}%
}
\renewcommand*{\dotsb@}{%
	\mathinner{%
		\kern\ellipsisbeforegap%
		{\cdotp}\kern\ellipsisgap%
		{\cdotp}\kern\ellipsisgap%
		{\cdotp}\kern\ellipsisaftergap%
	}%
}
\renewcommand*{\@cdots}{%
	\mathinner{%
		\kern\ellipsisbeforegap%
		{\cdotp}\kern\ellipsisgap%
		{\cdotp}\kern\ellipsisgap%
		{\cdotp}\kern\ellipsisaftergap%
	}%
}
\renewcommand*{\ellipsis@default}{%
	\ellipsis@before
	\kern\ellipsisbeforegap
	.\kern\ellipsisgap
	.\kern\ellipsisgap
	.\kern\ellipsisgap
	\ellipsis@after\relax}
\renewcommand*{\ellipsis@centered}{%
	\ellipsis@before
	\kern\ellipsisbeforegap
	.\kern\ellipsisgap
	.\kern\ellipsisgap
	.\kern\ellipsisaftergap
	\ellipsis@after\relax}
\AtBeginDocument{%
	\DeclareRobustCommand*{\dots}{%
		\ifmmode\@xp\mdots@\else\@xp\textellipsis\fi}}
\def\ellipsisgap{.1em}
\def\ellipsisbeforegap{.05em}
\def\ellipsisaftergap{.05em}
\makeatother

\usepackage{hyperref}
\hypersetup{
	colorlinks = true	
}

\DeclareMathOperator{\Int}{int}
\DeclareMathOperator{\tg}{tg}
\DeclareMathOperator{\ctg}{ctg}
\DeclareMathOperator{\sign}{sign}
\DeclareMathOperator{\Th}{th}
\DeclareMathOperator{\sh}{sh}
\DeclareMathOperator{\ch}{ch}
\DeclareMathOperator{\arsh}{arsh}
\DeclareMathOperator{\arch}{arch}
\DeclareMathOperator{\arth}{arth}
\DeclareMathOperator{\arcth}{arcth}
\DeclareMathOperator{\grad}{grad}
\DeclareMathOperator{\arc}{arc}
\DeclareMathOperator{\arctg}{arc tg}
\DeclareMathOperator{\Graph}{\Graph}
\DeclareMathOperator{\arcctg}{arc ctg}
\newcommand{\norm}[1]{\left\lVert#1\right\rVert}

\begin{document}
	%%%%%%%%%%%RÖVIDÍTÉSEK%%%%%%%%%%
	\setlength\parindent{0pt}
	\def\a{\textbf{a}}
	\def\b{\textbf{b}}
	\def\N{\hskip 10 true mm}
	\def\a{\textbf{a}}
	\def\b{\textbf{b}}
	\def\c{\textbf{c}}
	\def\d{\textbf{d}}
	\def\e{\textbf{e}}
	\def\gg{$\gamma$}
	\def\vi{\textbf{i}}
	\def\jj{\textbf{j}}
	\def\kk{\textbf{k}}
	\def\fh{\overrightarrow}
	\def\l{\lambda}
	\def\m{\mu}
	\def\v{\textbf{v}}
	\def\0{\textbf{0}}
	\def\s{\hspace{0.2mm}\vphantom{\beta}}
	\def\Z{\mathbb{Z}}
	\def\Q{\mathbb{Q}}
	\def\R{\mathbb{R}}
	\def\C{\mathbb{C}}
	\def\N{\mathbb{N}}
	\def\Rn{\mathbb{R}^{n}}
	\def\Ra{\overline{\mathbb{R}}}
	\def\sume{\displaystyle\sum_{n=1}^{+\infty}}
	\def\sumn{\displaystyle\sum_{n=0}^{+\infty}}
	\def\biz{\emph{Bizonyítás:\ }}
	\def\narrow{\underset{n\rightarrow+\infty}{\longrightarrow}}
	\def\limn{\displaystyle\lim_{n\to +\infty}}
	%	\def\definition{\textbf{Definíció:\ }}
	%	\def\theorem{\textbf{Tétel:\ }}
	%\def\note{\emph{Megjegyzés:\ }}
	%\def\example{\textbf{Példa:\ }} 
	
	\theoremstyle{definition}
	\newtheorem{theorem}{Tétel}[subsubsection]
	
	\theoremstyle{definition}
	\newtheorem{definition}[theorem]{Definíció}
	\newtheorem{example}[theorem]{Példa}
	\newtheorem{exercise}[theorem]{Házi feladat}
	\newtheorem{note}[theorem]{Megjegyzés}
	\newtheorem{task}[theorem]{Feladat}
	\newtheorem{revision}[theorem]{Emlékeztető}
	%%%%%%%%%%%%%%%%%%%%%%%%%%%%%%%%%
	\begin{center}
		{\LARGE\textbf{Az analízis alkalmazásai}}
		\smallskip

		{\Large Gyakorlati jegyzet}

		\smallskip
		5. óra.
	\end{center}
	A jegyzetet \textsc{Umann} Kristóf készítette \textsc{Kovács} Sándor gyakorlatán. (Utoljára frissítve: \today)
	
	\subsection{???}
	\begin{theorem}
		$d\in\N,\ \phi\not=\omega\subset\R^d$ nyílt, $f:\omega\to\R^d,\ f\in C^1,\ a\in\omega,\ \det[f'(a)]\not=0$:
		\begin{enumerate}
			\item $\exists a\in U\subset\varOmega$ nyílt, $\exists f(a)\in V\subset\R^d$ nyílt:
			\[ f_{\big|_U}\quad \text{injektív és}\quad f[U]=V \]
			\item $g:=\left(f_{\big|_U}\right)^{-1}\quad \Rightarrow\quad g\in C^1$ és
			\[ \forall y\in V:\quad (f^{-1})'(y)=g'(y)=[f'(f'^{-1}(y))]^{-1} \]
		\end{enumerate}
	\end{theorem}
	\begin{example}
		$f:\R^2\to\R^2,\quad f(x,y):=(x^2-y^2, 2xy)$ \[\Rightarrow\quad f(1,-1)=(0,-2),\quad f\in C^1, \det[f'(1,-1)]=\det \begin{bmatrix}
			2x&-2y\\
			2y&2x
		\end{bmatrix}_{\substack{x=1\\y=-1}}=(4x^2+4y^2)_{\substack{x=1\\y=-1}}=8\not=0\]
		\[ \Rightarrow (f^{-1})'(f(1,-1))=[f'(1,-1)]^{-1}=\begin{bmatrix}
			2&2\\
			-2&2
		\end{bmatrix}^{-1}=\frac{1}{8} \begin{bmatrix}
			2&-2\\
			2&2
		\end{bmatrix} \]
	\end{example}
	\begin{note}
		Ha $f:\R^d\to\R^d$ lineáris, azaz $\exists M\in\R^{d\times d}:\quad f(r)=Mr\quad (r\in\R^d)$, akkor $f\in C^1,\quad f'(r)=M\quad (r\in\R^d)$.
		
		Így, ha $f$ lokálisan invertálható, akkor globálisan is invertálható, és
		\[ (f^{-1})'(f(a))=[f'(a)]^{-1}=M^{-1} \]
		Ugyanakkor, ha $f:\R^d\to\R^d$ nem-lineáris, akkor lehetséges hogy $ \forall a\in\omega:\quad \det[f'(a)]\not=0,\quad \text{azaz}\quad \forall a\in\omega\text{-ban létezik lokális inverz, de $f$ nem injektív.}$ Erre elég egy ellenpéldát felhozni.
		\begin{example}
			\[ f(x,y):=(e^x\cos(y),e^x\sin(y))\quad ((x,y)\in\R^2)\quad \Rightarrow\quad f'(x,y)=\begin{bmatrix}
				e^x\cos(y)&-e^x\sin(y)\\
				e^x\sin(y)&e^x\cos(y)
			\end{bmatrix}\quad ((x,y)\in\R^2) \]
		\[ \det[f'(x,y)]=e^{2x}\cos^2(y)+e^{2x}\sin^2(y)=e^{2x}\not=0\quad ((x,y)\in\R^2) \]
		DE!:
		\[ f(0,0)=(1,0)=f(0,2\pi) \]
		\end{example}
	\end{note}
	\begin{definition}
		$m,n\in\N:\quad m<n,\quad A\in\R^{n\times n},\ A^T=A$. Azt mondjuk, hogy $A$ feltételesen pozitív (negatív) definit a teljes rangú $B\in\R^{m\times n}$ mátrixra vonatkozóan, ha
		\[ \forall 0\not=r\in\R^n:\quad (Br=0\quad \Rightarrow\quad \underbrace{Q_A(r)}_{\langle Ar,r\rangle=r^TAr}>0\ (<0)) \]
	\end{definition}
	\begin{note}\ 
		
		\begin{enumerate}
			\item Ha $Q_A$ definit, akkor feltételesen is definit.
			
			\item $rang(B)=m$
			
			\item $n=2,\ m=1:\quad A:=\begin{bmatrix}
				a&b\\
				b&c				
			\end{bmatrix},\quad B=\begin{bmatrix}
				d&e
			\end{bmatrix}$
			\[ rang(B)=1\quad \Leftrightarrow\quad d^2+e^2>0,\quad \text{pl.\ \ }e\not=0.\quad r=(x,y)\in\R^2,\quad Br=dx+ey=0\quad \Rightarrow\quad y=-\frac{d}{e}x \]
			\[ Q_A(r)=\begin{bmatrix}
				x&y
			\end{bmatrix}\cdot\begin{bmatrix}
				a&b\\
				b&c
			\end{bmatrix}\cdot\begin{bmatrix}
				x&y
			\end{bmatrix}=ax^2+2bxy+cy^2\quad \overset{y=-\frac{d}{e}x}{=}\quad ax^2+2bx\left(-\frac{d}{e}x\right)+c\left(-\frac{d}{e}x\right)^2=\]
			\[=ax^2-\frac{2bd}{e}x^2+\frac{cd^2}{e^2}x^2=\frac{ae^2-2bde+cd^2}{e^2}x^2 \]
			\[ ae^2-2bde+cd^2=-\det \begin{bmatrix}
				a&b&d\\
				b&c&e\\
				d&e&0
			\end{bmatrix}=-\det \begin{bmatrix}
				A&B^T\\
				B&0
			\end{bmatrix} \]
			$Q_A$ feltételesen pozitív (negatív) definit\quad $\Leftrightarrow\quad \det \begin{bmatrix}
				A&B^T\\
				B&0
			\end{bmatrix}<0\quad (>0)$
			
			\item $A:=\begin{bmatrix}
			1&3\\
			3&7
			\end{bmatrix},\quad B:=\begin{bmatrix}
				1&1
			\end{bmatrix}$
			\[ \Rightarrow\quad \det A=7-9=2<0\quad \Rightarrow\quad Q_A \quad \text{indefinit} \]
			\[ \det \begin{bmatrix}
				1&3&1\\
				3&7&1\\
				1&1&0
			\end{bmatrix}=\det \begin{bmatrix}
				1&3&1\\
				0&-2&-2\\
				0&-2&-1
			\end{bmatrix}=-2<0\quad \]
			$\Rightarrow\quad A$ feltételesen pozitív definit $B$-re vonatkozóan.
		\end{enumerate}
	\end{note}
	\begin{theorem}
		$m,n\in\N,\quad m<n,\quad A\in\R^{n\times n}:\quad A^T=A,\quad B\in\R^{m\times n}:\quad rang(B)=m.$
		\[ C:=\begin{bmatrix}
			A&B^T\\
			B&0
		\end{bmatrix}\in\R^{(n+m)\times(n+m)},\quad \text{ahol a 0 elemre:}\quad 0\in\R^{m\times m}. \]
		$A$ feltételesen pozitív (negatív) definit $B$-re vonatkozóan akkor és csak akkor, ha
		\[ (-1)^m C_k>0\quad [(-1)^{m+k}C_k>0]\quad (k\in{2m+1,\ldots,m+n}) \]
		(Itt ezek sarokminorok.)
	\end{theorem}
	\begin{note}\
	
		\begin{enumerate}
			\item Sok esetben $n=m+1\quad \Rightarrow\quad m+n=m+m+1=2m+1$.
			\item\[ \det C<0\quad \Rightarrow\quad \text{$A$ feltéltesen pozitív $B$-re} \]
			\[ \det C>0\quad \Rightarrow\quad \text{$A$ feltéltesen negatív $B$-re} \]
		\end{enumerate}
	\end{note}
	\begin{example}
		\[ A:=\begin{bmatrix}
			1&2&3\\
			2&4&4\\
			3&4&9
		\end{bmatrix},\quad B:=\begin{bmatrix}
			1&-1&1\\
			2&1&-1
		\end{bmatrix} \]
		$rang(B)=rang\left(\begin{bmatrix}
			1&-1&1\\
			0&3&-3
		\end{bmatrix}\right)=2$
		\[ C:=\begin{bmatrix}
			A&B^T\\
			B&0
		\end{bmatrix}=\begin{bmatrix}
			1&2&3&1&2\\
			2&4&4&-1&1\\
			3&4&9&1&-1\\
			1&-1&1&0&0\\
			2&1&-1&0&0
		\end{bmatrix} \]
		Hf.: Mutassuk meg, hogy $\det(C)=189>0$.
		
		$\Rightarrow (-1)^2\det(C)>0\quad \Rightarrow\quad A$ felételesen pozitív definit $B$-re.
	\end{example}
	\begin{exercise}
		A honlapon lévő pdf-ben a definíció szerint kiszámolt feladatokat tétel szerint is oldjuk meg.
	\end{exercise}
	\begin{task}
		$T(x,y):=4xy\quad ((x,y)\in(0,1)^2),\quad x^2+y^2=1$
		
		1. módszer: $y=\sqrt{1-x^2}\quad \rightsquigarrow\quad T(x,\sqrt{1-x^2})=4x\cdot\sqrt{1-x^2}=:\phi(x)\quad (y\in(0,1))$
		\[ \phi\in D^{\infty},\quad \phi(x)=4\sqrt{1-x^2}+4x\cdot\frac{-x}{\sqrt{q-x^2}}=\frac{4-8x^2}{\sqrt{1-x^2}}=0\quad \Leftrightarrow\quad x=\frac{1}{\sqrt{2}} \]
		\[ \phi''(x)=\frac{-4x}{\sqrt{1-x^2}}-\frac{4x}{\sqrt{1-x^2}}-4x\cdot\frac{\sqrt{1-x^2}-x\frac{-x}{\sqrt{1-x^2}}}{1-x^2}\]
		\[\Rightarrow\quad \phi''\left(\frac{1}{\sqrt{2}}\right)=-\frac{8}{\sqrt{2}}-\frac{8}{\sqrt{2}}+\frac{8}{\sqrt{2}}\left\{\frac{1}{\sqrt{2}}+\frac{1}{\sqrt{2}}\cdot\frac{1}{\sqrt{2}}\right\}=-8\sqrt{2}+4+\sqrt{2}=-7\sqrt{2}+4<0\]
		\[ \Rightarrow\quad x=\frac{1}{\sqrt{2}},\quad y=\sqrt{1-\left(\frac{1}{\sqrt{2}}\right)^2}=\frac{1}{\sqrt{2}}\quad \text{lokális maximum van.} \]
		2. módszer: $L(x,y):=4xy+\lambda(x^2+y^2-1)\quad ((x,y)\in(0,1)^2)$
		\begin{align*}
			\partial_1 L(x,y)=4y+2\lambda x&=0\\
			\partial_2L(x,y)=4x+2\lambda y&=0\\
			x^2+y^2-1&=0
		\end{align*}
		Az első két egyenletből
		\[ 2(y-x)+\lambda(x-y)=0,\quad \text{azaz}\quad (x-y)(\lambda-2)=0 \]
		Így vagy $x=y$ és $\lambda=-2$, vagy $\lambda=2$, ami ellentmondáshoz vezetne, mert így $x=-y$ teljesülne. Az első eset szerint $x=\frac{1}{\sqrt{2}},\quad y=\frac{1}{\sqrt{2}}$.
		\[ L''(x,y)=\begin{bmatrix}
			2\lambda&4\\
			4&2\lambda
		\end{bmatrix}\quad \Rightarrow\quad L''\left(\frac{1}{\sqrt{2}},\frac{1}{\sqrt{2}}\right)=\begin{bmatrix}
			-4&4\\
			4&-4
		\end{bmatrix}\quad \text{szemidefinit} \]
		\[ rang(\begin{bmatrix}
			2x&2y
		\end{bmatrix})=1\quad \Leftrightarrow\quad x^2+y^2>0 \]
		\[ C:=\begin{bmatrix}
			-4&4&\sqrt{2}\\
			4&-4&\sqrt{2}\\
			\sqrt{2}&\sqrt{2}&0
		\end{bmatrix}\quad \Rightarrow\quad \det(C)=\det \begin{bmatrix}
		-4&4&\sqrt{2}\\
		0&0&2\sqrt{2}\\
		\sqrt{2}&\sqrt{2}&0
		\end{bmatrix}=-2\sqrt{2}(-4\sqrt{2}-4\sqrt{2})=16\sqrt{2}>0\]
		$ \Rightarrow\quad \text{felt. lok. max.}$
	\end{task}
	\begin{task}
		Határozzuk meg az
		\[ f(x,y,z)=x+2y+3z\quad ((x,y)\in\R^3:\quad x^2+y^2=2,\quad y+z=1) \]
		függvény lokális szélsőértékeit!
		\[ \omega:=\{(x,y)\in\R^3: x^2+y^2=2,\quad y+z=1  \} \]
		\[ \lambda(x,y,z)=x+2y+3z ,\quad g(x,y,z):=(x^2+y^2-2,y+z-1) \]
		\[ f'(x,y,z)=\begin{bmatrix}
			2x&2y&0\\
			0&1&1
		\end{bmatrix},\quad rang(g'(x,y,z))=1\quad \Leftrightarrow\quad x^2+y^2>0 \]
		\[ L(x,y,z):=x+2y+3z+\lambda(x^2+y^2-z)+\mu(y+z-1) \]
		\begin{align*}
			\partial_1 L(x,y,z)=1+2\lambda y&=0\\
			\partial_2L(x,y,z)=2+2\lambda y&=0\\
			\partial_3L(x,y,z)=3+\mu&=0\\
			g_1(x,y,z)=x^2+y^2-2&=0\\
			g_2(x,y,z)=y+z-1&=0
		\end{align*}
	\end{task}
	%TODO finish
\end{document}