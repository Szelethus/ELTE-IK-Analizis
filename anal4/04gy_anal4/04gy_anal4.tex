
\documentclass[a4paper,11.5pt]{article}
\usepackage[textwidth=170mm, textheight=230mm, inner=20mm, top=20mm, bottom=30mm]{geometry}
\usepackage[normalem]{ulem}
\usepackage[utf8]{inputenc}
\usepackage[T1]{fontenc}
\PassOptionsToPackage{defaults=hu-min}{magyar.ldf}
\usepackage{pgfplots}
\pgfplotsset{compat=1.10}
\usepgfplotslibrary{fillbetween}
\usepackage[magyar]{babel}
\usepackage{amsmath, amsthm,amssymb,paralist,array, ellipsis, graphicx, float, bigints,tikz}
\usepackage{marvosym}
\renewcommand{\mvchr}[1]{\mbox{\mvs\symbol{#1}}} %to make \Lightning work in math mode
\makeatletter
\renewcommand*{\mathellipsis}{%
	\mathinner{%
		\kern\ellipsisbeforegap%
		{\ldotp}\kern\ellipsisgap
		{\ldotp}\kern\ellipsisgap%
		{\ldotp}\kern\ellipsisaftergap%
	}%
}
\renewcommand*{\dotsb@}{%
	\mathinner{%
		\kern\ellipsisbeforegap%
		{\cdotp}\kern\ellipsisgap%
		{\cdotp}\kern\ellipsisgap%
		{\cdotp}\kern\ellipsisaftergap%
	}%
}
\renewcommand*{\@cdots}{%
	\mathinner{%
		\kern\ellipsisbeforegap%
		{\cdotp}\kern\ellipsisgap%
		{\cdotp}\kern\ellipsisgap%
		{\cdotp}\kern\ellipsisaftergap%
	}%
}
\renewcommand*{\ellipsis@default}{%
	\ellipsis@before
	\kern\ellipsisbeforegap
	.\kern\ellipsisgap
	.\kern\ellipsisgap
	.\kern\ellipsisgap
	\ellipsis@after\relax}
\renewcommand*{\ellipsis@centered}{%
	\ellipsis@before
	\kern\ellipsisbeforegap
	.\kern\ellipsisgap
	.\kern\ellipsisgap
	.\kern\ellipsisaftergap
	\ellipsis@after\relax}
\AtBeginDocument{%
	\DeclareRobustCommand*{\dots}{%
		\ifmmode\@xp\mdots@\else\@xp\textellipsis\fi}}
\def\ellipsisgap{.1em}
\def\ellipsisbeforegap{.05em}
\def\ellipsisaftergap{.05em}
\makeatother

\usepackage{hyperref}
\hypersetup{
	colorlinks = true	
}

\DeclareMathOperator{\Int}{int}
\DeclareMathOperator{\tg}{tg}
\DeclareMathOperator{\ctg}{ctg}
\DeclareMathOperator{\sign}{sign}
\DeclareMathOperator{\Th}{th}
\DeclareMathOperator{\sh}{sh}
\DeclareMathOperator{\ch}{ch}
\DeclareMathOperator{\arsh}{arsh}
\DeclareMathOperator{\arch}{arch}
\DeclareMathOperator{\arth}{arth}
\DeclareMathOperator{\arcth}{arcth}
\DeclareMathOperator{\grad}{grad}
\DeclareMathOperator{\arc}{arc}
\DeclareMathOperator{\arctg}{arc tg}
\DeclareMathOperator{\Graph}{graph}
\DeclareMathOperator{\arcctg}{arc ctg}
\newcommand{\norm}[1]{\left\lVert#1\right\rVert}

\begin{document}
	%%%%%%%%%%%RÖVIDÍTÉSE K %%%%%%%%%%
	\setlength\parindent{0pt}
	\def\a{\textbf{a}}
	\def\b{\textbf{b}}
	\def\N{\hskip 10 true mm}
	\def\a{\textbf{a}}
	\def\b{\textbf{b}}
	\def\c{\textbf{c}}
	\def\d{\textbf{d}}
	\def\e{\textbf{e}}
	\def\gg{$\gamma$}
	\def\vi{\textbf{i}}
	\def\jj{\textbf{j}}
	\def\kk{\textbf{k}}
	\def\fh{\overrightarrow}
	\def\l{\lambda}
	\def\m{\mu}
	\def\v{\textbf{v}}
	\def\0{\textbf{0}}
	\def\s{\hspace{0.2mm}\vphantom{\beta}}
	\def\Z{\mathbb{Z}}
	\def\Q{\mathbb{Q}}
	\def\R{\mathbb{R}}
	\def\C{\mathbb{C}}
	\def\N{\mathbb{N}}
	\def\Rn{\mathbb{R}^{n}}
	\def\Ra{\overline{\mathbb{R}}}
	\def\sume{\displaystyle\sum_{n=1}^{+\infty}}
	\def\sumn{\displaystyle\sum_{n=0}^{+\infty}}
	\def\biz{\emph{Bizonyítás:\ }}
	\def\narrow{\underset{n\rightarrow+\infty}{\longrightarrow}}
	\def\limn{\displaystyle\lim_{n\to +\infty}}
	%	\def\definition{\textbf{Definíció:\ }}
	%	\def\theorem{\textbf{Tétel:\ }}
	%\def\note{\emph{Megjegyzés:\ }}
	%\def\example{\textbf{Példa:\ }} 
	
	\theoremstyle{definition}
	\newtheorem{theorem}{Tétel}[subsubsection]
	
	\theoremstyle{definition}
	\newtheorem{definition}[theorem]{Definíció}
	\newtheorem{example}[theorem]{Példa}
	\newtheorem{exercise}[theorem]{Házi feladat}
	\newtheorem{note}[theorem]{Megjegyzés}
	\newtheorem{task}[theorem]{Feladat}
	\newtheorem{revision}[theorem]{Emlékeztető}
	%%%%%%%%%%%%%%%%%%%%%%%%%%%%%%%%%
	\begin{center}
		{\LARGE\textbf{Az analízis alkalmazásai}}
		\smallskip

		{\Large Gyakorlati jegyzet}

		\smallskip
		4. óra.
	\end{center}
	A jegyzetet \textsc{Umann}  K ristóf készítette \textsc{ K ovács} Sándor gyakorlatán. (Utoljára frissítve: \today)
	
	\subsection{Implicit függvények}
	\begin{note}
		Adott $f\in\R^2\to\R$ esetén van-e olyan $\varphi\in\R\to\R$, hogy
		\[ \{ (x,\varphi(x))\in\R^2:\quad x\in\mathcal{D}_f \}=:\Graph\,\varphi=H:=\{ (x,y)\in\mathcal{D}_f:\quad f(x,y)=0 \} \]
		\[ \Graph\,\varphi=H\quad \Leftrightarrow\quad \forall x\in\mathcal{D}_f:\quad f(x,\varphi(x))=0 \]
	\end{note}
	\begin{example}
		\[ f(x,y):=x^2+y^2-1 \quad ((x,y)\in\R^2)\]
		\[ \Graph\,\varphi=H\quad \Rightarrow\quad \{ \varphi(0) \}=\{-1,1\}\quad \Lightning \]
	\end{example}
	\begin{example}
		\[ f(x,y):=x^2+y^2-1\quad ((x,y)\in\R^2,\quad y\leq 0) \]
		\[ \varphi(x):=-\sqrt{1-x^2}\quad (x\in[-1,1])\quad \text{jó} \]
	\end{example}
	\begin{theorem}
		$a,b\in\R,\quad f\in\R^2\to\R,\quad f\in C^1,\quad f(a,b)=0,\quad \partial_2f(a,b)\not=0.$ Ekkor
		\[ \exists \varepsilon>0\quad \exists\delta>0\quad \exists\varphi:(a-\varepsilon,a+\varepsilon)\to(b-\varepsilon,b+\varepsilon),\quad \varphi\in C^1: \]
		\begin{enumerate}
			\item 
			$ f(x,\varphi(x))=0\quad (|x-a|<\varepsilon) $
			\item $\partial_2f(x,\varphi(x))\not=0\quad \text{ÉS} \quad \varphi'(x)= -\frac{\partial_1f(x,\varphi(x))}{\partial_2f(x,\varphi(x))}\quad (|x-a|<\varepsilon)$
		\end{enumerate}
	\end{theorem}
	\begin{task}
		Bizonyítsuk be, hogy $\exists\varepsilon>0,\quad \exists\varphi\in\R\to\R,\quad \varphi\in D$, hogy
		\[\ln(x)+\varphi(x)e^{\varphi^2(x)}=1\quad (|x-e|<\varepsilon)\quad \varphi'(e)=?\]
		\textit{Megoldás:}
		\[ f(x,y):=\ln(x)+ye^{y^2}-1=0\quad ((x,y)\in\R^2:\quad x>0) \]
		\[ f(e,0)=0,\quad \partial_2f(e,0)=\left[e^{y^2}+2y^2e^{y^2}\right]_{\substack{x=e\\y=0}}=1\not=0 \]
		\[ \Rightarrow\quad \exists\varepsilon>0,\quad \exists\varphi:(e-\varepsilon,e+\varepsilon)\to\R,\quad \varphi\in D:\quad f(x,\varphi(x))=0\quad (|x-e|<\varepsilon) \]
		\[ \varphi'(x)\equiv -\frac{\frac{1}{x}}{e^{\varphi^2(x)}+2\varphi^2(x)e^{\varphi^2(x)}}\quad \Rightarrow\quad \varphi'(e)=-\frac{\frac{1}{e}}{1}=-\frac{1}{e} \]
	\end{task}
			
			
	\begin{task}
		Bizonyítsuk be, hogy $\exists\varepsilon>0,\quad \exists\varphi\in\R\to\R,\quad \varphi\in D$, hogy
		$$\ln(x^2+\varphi^2(x))=x\cdot \varphi(x)\quad (|x-e|<\varepsilon)\quad \varphi'(1)=?$$
		\textit{Megoldás:}
		\[ f(x,y):=\ln(x^2+y^2)-xy\quad ((x,y)\in\R^2:\quad x^2+y^2>0) \]
		\[ f(1,0)=0,\quad \partial_2f(1,0)=\left[\frac{2y}{x^2+y^2}-x\right]_{\substack{x=1\\y=0}}=-1\not=0 \]
		\[ \Rightarrow\quad \exists\varepsilon>0,\quad \exists\varphi:(1-\varepsilon,1+\varepsilon)\to\R,\quad \varphi\in D:\quad f(x,\varphi(x))=0\quad (|x-1|<\varepsilon)\quad \text{és}\] \[\varphi'(x)\equiv-\frac{\frac{2x}{x^2+\varphi^2(x)}-\varphi(x)}{\frac{2\varphi(x)}{x^2+\varphi^2(x)}-x} \quad \Rightarrow\quad \varphi'(1)\frac{2-0}{-1-1}=-1 \]
	\end{task}
			
	\begin{task}
		Bizonyítsuk be, hogy $\exists\varepsilon>0,\quad \exists\varphi\in\R\to\R,\quad \varphi\in D$, hogy
		$$ xe^{-\varphi(x)}+\varphi(x)e^x=a\in\R\quad (|x|<\varepsilon)\quad \varphi'(0)=?$$
		\textit{Megoldás:}
		\[ f(x,y):=xe^{-y}+ye^x-a\quad ((x,y)\in\R^2) \]
		\[ f(0,a)=0,\quad \partial_2f(0,a)=\left[-xe^{-y}+e^x\right]_{\substack{x=0\\y=a}}=e^0=1\not=0 \]
		\[ \Rightarrow\quad \exists\varepsilon>0,\quad \exists\varphi:(-\varepsilon,\varepsilon)\to\R,\quad \varphi\in D:\quad f(x,\varphi(x))=0\quad (|x|<\varepsilon)\quad \text{és}\]
		\[\varphi'(x)\equiv -\frac{e^{-\varphi(x)}+\varphi(x)e^x}{-xe^{-\varphi(x)}+e^x} \quad  \Rightarrow\quad \varphi'(0)=-\frac{e^{-a}+a}{1}=-(e^{-a}+a) \]
	\end{task}
			
	\begin{task}
		Bizonyítsuk be, hogy $\exists\varepsilon>0,\quad \exists\varphi\in\R\to\R,\quad \varphi\in D$, hogy
			$$x\cos(\varphi(x))=\varphi(x)\cos(x)\quad (|x|<\varepsilon),\quad \varphi'(0)=?$$
			\textit{Megoldás:}
			\[ f(x,y):=x\cos(y)-y\cos(x)\quad ((x,y)\in\R^2) \]
			\[ f(0,0)=0,\quad \partial_2f(0,0)=[-x\sin(y)-\cos(x)]_{\substack{x=0\\y=0}}=-1\not=0 \]
			\[ \Rightarrow\quad \exists\varepsilon>0,\quad \exists\varphi:(-\varepsilon,\varepsilon)\to\R,\quad y\in D:\quad f(x,\varphi(x))=0,\quad (|x|<\varepsilon)\quad \text{és}\]
			\[ \varphi'(x)\equiv -\frac{\cos(\varphi(x)+\varphi(x)\sin(x)}{-x\sin(\varphi(x))-\cos(x)}\quad  \Rightarrow\quad \varphi'(0)=-\frac{1+0}{-1}=1 \]
	\end{task}
	\begin{note}
		Megállapítható (ez utolsó feladatnál), hogy
		\[ \varphi(x):=x\quad (x\in\R)\quad \text{jó},\quad \varphi(0)=1 \]
		Ennek örömére ez a feladat nem fog szerepelni a ZH-ban.
	\end{note}
	\begin{note}
		$m,n\in\N,\quad A\subset \R^m,\quad B\subset \R^n$ nyílt halmazok,
		\[ f\in C^1(A\times B,\R^2),\quad a\in A,\quad b\in B;\quad g(r):=f(r,b)\quad (r\in A) \]
		Legyen továbbá legyen a második parciális függvény 
		\[ h(s):=f(a,s)\quad (s\in B) \]
		\[ \R^{n\times m}\ni g'(r)=:\partial_1f(r,b)=:\frac{\partial}{\partial_r}f(r,b)\quad (r\in A) \]
		\[ \R^{n\times n}\ni h'(s)=:\partial_2f(a,s)=:\frac{\partial}{\partial_s}f(a,s)\quad (s\in B) \]
	\end{note}
	\begin{theorem}
		$f(a,b)=0,\quad \det[\partial_2f(a,b)]\not=0\quad \Rightarrow\quad \exists  K (a)\subset A\quad \exists  K (b)\subset B:$
		\[ \forall r\in  K (a),\quad \exists!\varphi(r)\in  K (b):\quad f(r,\varphi(r))=0\quad \text{és}\]
		\[ \varphi\in C^1,\quad \varphi'(r)=-[\partial_2f(r,\varphi(r))]^{-1}\cdot\partial_1f(r,\varphi(r))\quad (r\in  K (a)) \]
	\end{theorem}
	\begin{task}
		\[ xy^2z^3+2x^2ye^{z-1}=0\quad \Rightarrow\quad \exists(a,b)\in\R^2\quad \exists c\in\R:\quad \exists  K (a,b)\subset\R^2 \]
		\[ \exists\varphi: K (a,b)\to\R,\quad \varphi\in D:\quad \varphi(a,b)=c\quad \text{és}\quad \forall(x,y)\in K (a,b):\quad (x,y,\varphi(x,y))\]
		\text{megoldása az egyenletnek!} 
		\[ \varphi'(a,b)=? \]
		Vajon megoldás lesz a $\ x=-1,\ y=2, z=1$ hármas?
		\[ f(x,y,z):=xy^2z^3+2x^2ye^{z-1}\quad ((x,y,z)\in\R^2): \]
		\[ f(-1,2,1)=0,\quad f\in C^1,\quad \frac{\partial}{\partial_z}f(-1,2,1)=\left[3xy^2z^2+2x^2ye^{z-1}\right]_{\substack{x=-1\\y=2\\z=1}}=-8\not=0 \]
		\[ \Rightarrow\quad \exists K (-1,2),\quad \exists\varphi: K (-1,2)\to\R,\quad \varphi\in C^1:\quad f(x,y,\varphi(x,y))=0 \]
		\[ \varphi'(-1,2)=\frac{1}{-8}\cdot\left[\frac{\partial}{\partial_{(x,y)}}f(-1,2,1)\right]=-\frac{1}{8}\left(y^2z^3+4xye^{z-1},\ 2xyz^3+2x^2e^{z-1}\right)_{\substack{x=-1\\y=2\\z=1}}=-\frac{1}{8}(-4,2)=\left(\frac{1}{2},\frac{1}{4}\right) \]
	\end{task}
	\begin{task}
		 \begin{align*}
			7x+\sin(y)+3z&=5\\
			x+e^y-2z&=9
		\end{align*}
		\[ \exists\varepsilon>0,\quad \exists\varphi=(\varphi_1,\varphi_2):(-3-\varepsilon,-3+\varepsilon)\to\R^2,\quad \varphi\in D,\quad \varphi(-3)=(2,0)\quad \text{és}\quad \forall z\in(-3-\varepsilon,-3+\varepsilon), \]
		\[(\varphi_1(z),\varphi_2(z),z)\quad \text{megoldása az egyenletnek!}\quad \varphi'(-3)=?  \]
		Megoldás:
		\[ f(x,y,z):=(7x+\sin(y)+3z-5,x+e^y-2z-9)\quad ((x,y,z)\in\R^3) \]
		\[ f\in C^1,\quad f(2,0,-3)=(0,0),\quad \det\left[\frac{\partial}{\partial_{(x,y)}}f(2,0,-3)\right]=\det \begin{bmatrix}
			7&\cos(y)\\
			1&e^y
		\end{bmatrix}_{\substack{x=2\\y=0\\z=-3}}=\det \begin{bmatrix}
			7&1\\
			1&1
		\end{bmatrix}=6\not=0 \]
		\[ \exists\varepsilon>0,\quad \exists\varphi(\varphi_1,\varphi_2):(-3-\varepsilon,-3+\varepsilon)\to\R^2,\quad \varphi\in D:\quad f(\varphi_1(z),\varphi_2(z),z)\equiv0 \]
		És:
		\[ \varphi'(-3)=-\left[\frac{\partial}{\partial(x,y)}f(2,0,-3)\right]^{-1}\cdot\frac{\partial}{\partial z}f(2,0,-3)=-\begin{bmatrix}
			7&1\\
			1&1
		\end{bmatrix}^{-1}\cdot \begin{bmatrix}
		3\\
		-2
		\end{bmatrix}=-\frac{1}{6} \begin{bmatrix}
			1&-1\\
			-1&7
		\end{bmatrix} \cdot \begin{bmatrix}
			3\\
			-2
		\end{bmatrix}=-\frac{1}{6} \begin{bmatrix}
			7\\
			-17
		\end{bmatrix}= \begin{bmatrix}
		5/6\\
		17/6
		\end{bmatrix} \]
		%TODO alsó index wtf???
	\end{task}
\end{document}